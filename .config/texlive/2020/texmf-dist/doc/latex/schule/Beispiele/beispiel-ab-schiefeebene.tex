\documentclass[a4paper]{scrartcl}
\usepackage[
    typ=ab,
    fach=Physik,
    lerngruppe=9a,
    nummer=2A,
    datumAnzeigen,
    namensfeldAnzeigen,
]{schule}

% Dieses Dokument gehört zu den Beispiel des LaTeX Paketes Schule und ist von den Autoren
% des Pakets erstellt worden.
%
% Das Dokument steht unter der Lizenz: Creative Commons by-nc-sa Version 4.0
% http://creativecommons.org/licenses/by-nc-sa/4.0/deed.de
%
% Nach dieser Lizenz darf das Dokument beliebig kopiert und bearbeitet werden,
% sofern das Folgeprodukt wiederum unter gleichen Lizenzbedingungen vertrieben
% und auf die ursprünglichen Urheber verwiesen wird.
% Eine kommerzielle Nutzung ist ausdrücklich ausgeschlossen.

\author{Andr'e Hilbig}
\title{Schiefe Ebene}
\date{2014-11-13}

\usetikzlibrary{decorations.pathreplacing}

\xsimsetup{
    aufgabe/template=schule-keinenummer,
}

\newcommand{\Hintergrund}[4][0.5cm]{
    \begin{tikzpicture}
        \pgfmathtruncatemacro{\anzahl}{(\linewidth-\pgflinewidth)/#1}
        \draw[gray,step=#1] (0,0) rectangle (\anzahl*#1,#2*#3) (0,0) grid (\anzahl*#1,#2*#3);
        #4
    \end{tikzpicture}
}
\newcommand{\schiefeEbene}[4]{
    \draw[black, very thick] (#1,#2) -- (#3,#2);
    \draw[black, very thick] (#3,#4) -- (#3,#2);
    \draw[black, very thick] (#3,#4) -- (#1,#2);
}

\begin{document}
\section*{\Titel}
\enlargethispage{1ex}
\begin{aufgabe}
    Ein Massestück mit der Masse $m=\unit[200]{g}$ soll um eine Höhe $\Delta h=\unit[10]{cm}$ angehoben werden. Die Masse wird jedoch nicht direkt gehoben, sondern über eine schiefe Ebene hochgezogen.

    \hinweis{Verwende für die Kräfte den Maßstab: $\unit[1]{N} \overset{\wedge}{=} \unit[1]{cm}$.}

    \vspace*{1ex}

    \Hintergrund{0.5cm}{0.85cm}{
        \draw[decorate, black, decoration={brace, amplitude=10pt},
            xshift=4pt, yshift=0pt] (14.5cm,11cm) -- (14.5cm,1cm) node
            [black,midway,xshift=0.7cm] {$\Delta h$};
        \draw[decorate,black,decoration={brace,amplitude=10pt},
            xshift=3pt, yshift=-4pt] (14.5cm,11cm) -- (0cm,1cm) node
            [black, midway, xshift=12pt, yshift=-16pt] {$\Delta s$};
        \schiefeEbene{0cm}{1cm}{14.5cm}{11cm}
        \draw[black, very thick, rotate=35, yshift=0.8cm,
            xshift=9.4cm] (0cm,0cm) rectangle (1cm, 1cm);
    }

    \begin{teilaufgaben}
        \teilaufgabe Konstruiere mit Hilfe eines Kräfteparallelogramms die Kraft $F_{\text{Zug}}$ mit der an  der Kiste gezogen werden muss.
        \teilaufgabe Fülle danach die unten abgedruckte Tabelle mit deinen gemessenen Werten aus und berechne das Produkt $F_{\text{Zug}}\cdot\Delta s$.

        \begin{tabular}{l|c|c|c}
            & $F_{\text{Zug}}$ in N & $\Delta s$ in cm &  $F_{\text{Zug}}\cdot\Delta s$ in Nm\\\hline
            \multirow{2}{*}{Schiefe Ebene A} &	& & \\
            &	& & \\\hline
            \multirow{2}{*}{Schiefe Ebene B} &	& & \\
            & & & \\
        \end{tabular}
    \end{teilaufgaben}

\end{aufgabe}

\end{document}