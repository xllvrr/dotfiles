\documentclass[12pt,a4paper]{scrartcl}
\usepackage[
    typ=ab,
    fach=Informatik,
    lerngruppe=EF,
]{schule}

% Dieses Dokument ist in der Zusammenarbeit von verschiedenen
% Informatikreferendaren und Informatiklehrern entstanden.
% Der Herausgeber dieses Dokuments ist die Fachgruppe Informatische
% Bildung in NRW der Gesellschaft für Informatik.
%
% Das Dokument steht unter der Lizenz: Creative Commons by-nc-sa Version 4.0
% http://creativecommons.org/licenses/by-nc-sa/4.0/deed.de
%
% Nach dieser Lizenz darf das Dokument beliebig kopiert und bearbeitet werden,
% sofern das Folgeprodukt wiederum unter gleichen Lizenzbedingungen vertrieben
% und auf die ursprünglichen Urheber verwiesen wird.
% Eine kommerzielle Nutzung ist ausdrücklich ausgeschlossen.
%
% Die Namensnennung durch einen Verweis und die Lizenzangabe der ursprünglichen
% Urheber auf den Materialien für Schülerinnen und Schüler ist erforderlich.
%
% Die Sammlung der Dokumente steht unter
% http://ddi.uni-wuppertal.de/material/materialsammlung/
% zur Verfügung.
%
% Das LaTeX-Paket zum Setzen der Dokumente der Sammlung steht
% unter  http://www.ctan.org/pkg/schule
% zur Verfügung.

\author{Fachgruppe Informatische Bildung in NRW der Gesellschaft für Informatik}

\title{Fahrkartenauskunft}

\xsimsetup{
    aufgabe/template=schule-keinenummer,
}

\begin{document}
    \section*{Problembeschreibung}
    \subsection*{Ausgangssituation}
        Das örtliche Nahverkehrsunternehmen \enquote{NahUnt} will an den Bushaltestellen Fahrscheinautomaten installieren. An dem Automaten kann der Kunde eine Entfernungszone per Knopfdruck wählen. Es gibt drei Entfernungszonen mit unterschiedlichen Preisen: 1.Zone: 1,10~\euro, 2.Zone: 1,90~\euro, 3.Zone: 4,20~\euro. In einem Display steht als erstes der Text \enquote{Bitte wählen Sie eine Entfernungszone aus}. Nach der Betätigung einer Entfernungszonentaste soll die ausgewählte Zone und der Preis angezeigt werden.
        \begin{aufgabe}
            \begin{teilaufgaben}
                \teilaufgabe Ermitteln Sie die vorkommenden Objekte und die zugehörigen Attribute und Attributwerte und notieren Sie diese mit Objektkarten.
                \teilaufgabe Erstellen Sie das Objektdiagramm.
                \teilaufgabe Fassen Sie die Objekte geeignet zu Klassen zusammen und dokumentieren diese mit  Klassenkarten.
                \teilaufgabe Erstellen Sie das Klassendiagramm.
            \end{teilaufgaben}
        \end{aufgabe}
\end{document}
