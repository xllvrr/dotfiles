\documentclass[a4paper]{scrartcl}
\usepackage[
    fach=Informatik,
    typ=ab,
    farbig,
]{schule}

\title{Aufgabentemplates}

\begin{document}
    \setzeAufgabentemplate{schule-binnen}
    \begin{aufgabe}[points=4,bonus-points=2,subtitle={Binnen}]
        Diese Aufgabe ist im Stil: Binnen
    \end{aufgabe}
    \vspace{0.5cm}

    \setzeAufgabentemplate{schule-default}
    \begin{aufgabe}[points=4,bonus-points=2,subtitle={Default}]
        Diese Aufgabe ist im Stil: Default
    \end{aufgabe}
    \vspace{0.5cm}

    \setzeAufgabentemplate{schule-keinenummer}
    \begin{aufgabe}[points=4,bonus-points=2,subtitle={Keine Nummer}]
        Diese Aufgabe ist im Stil: Keine Nummmer
    \end{aufgabe}
    \vspace{0.5cm}

    \setzeAufgabentemplate{schule-keinepunkte}
    \begin{aufgabe}[points=4,bonus-points=2,subtitle={Keine Punkte}]
        Diese Aufgabe ist im Stil: Keine Punkte
    \end{aufgabe}
   \vspace{0.5cm}

    \setzeAufgabentemplate{schule-keintitel}
    \begin{aufgabe}[points=4,bonus-points=2,subtitle={Kein Titel}]
        Diese Aufgabe ist im Stil: Kein Titel
    \end{aufgabe}
    \vspace{0.5cm}

    \setzeAufgabentemplate{schule-randpunkte}
    \begin{aufgabe}[points=4,bonus-points=2,subtitle={Randpunkte}]
        Diese Aufgabe ist im Stil: Randpunkte
    \end{aufgabe}
    \vspace{0.5cm}

    \setzeAufgabentemplate{schule-tcolorbox}
    \begin{aufgabe}[points=4,bonus-points=2,subtitle={TColorBox}]
        Diese Aufgabe ist im Stil: TColorbox
    \end{aufgabe}
\end{document}
