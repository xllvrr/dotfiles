\documentclass[a4paper,12pt]{scrartcl}
\usepackage[
    typ=ab,
    fach=Informatik,
    lerngruppe=9g,
    loesungen=seite,
    module={Symbole,Bewertung},
    farbig,
    seitennummern=autoGesamt,
]{schule}

% Dieses Dokument gehört zu den Beispiel des LaTeX Paketes Schule und ist von den Autoren
% des Pakets erstellt worden.
%
% Das Dokument steht unter der Lizenz: Creative Commons by-nc-sa Version 4.0
% http://creativecommons.org/licenses/by-nc-sa/4.0/deed.de
%
% Nach dieser Lizenz darf das Dokument beliebig kopiert und bearbeitet werden,
% sofern das Folgeprodukt wiederum unter gleichen Lizenzbedingungen vertrieben
% und auf die ursprünglichen Urheber verwiesen wird.
% Eine kommerzielle Nutzung ist ausdrücklich ausgeschlossen.

\author{Test Person}
\date{\today}
\title{Ein Arbeitsblatt}

\usepackage{todonotes}
\usepackage{blindtext}

\begin{document}

\maketitle

\listoftodos

\section*{\Titel für die Klasse \Kurs am \Datum von \Autor.}
\blindtext

\ifthenelse{\boolean{schule@farbig}}{BUNT}{SW}
\begin{tikzpicture}[pap]
\node[startstop] (s1){Los!};
\node[verzweigung, below = of s1] (v1) {Lieblingsfach Informatik?};
\node[unterprogramm, right = of v1] (up1) {\nodepart[text width=7em]{two} Pro\-gram\-mie\-re ein Spiel};
	\node[aktion, below = of up1] (a1) {führe es aus!};
\node[einausgabe, below = of v1] (ea1) {ERROR 1337};
\node[startstop, below = of ea1] (e1){Ende};

\draw[linie] (s1)--(v1);
\draw[linie] (v1)--(up1)  node[near start, above] {ja};
\draw[linie] (v1)--(ea1) node[near start, right] {nein};
\draw[linie] (up1)--(a1);

\draw[linie] (a1) |- ($(e1.north) + (0,0.5)$); 
\draw[linie] (ea1)--(e1);
\end{tikzpicture}


\begin{exercise}
    Übung, keine Aufgabe.
\end{exercise}
\begin{solution}
    Lösung für eine Übung
\end{solution}

\section*{Aufgaben}

\xsimsetup{
    aufgabe/template=schule-binnen,
}

\clearpage
\section*{Geschichtsklausur:}
    Interpretiere die Quelle, indem du\dots
    \begin{aufgabe}[points=10,subtitle=Formale Analyse,symbol=\symBleistift]
        \dots sie analysierst (Einleitung + strukturierte und detaillierte Zusammenfassung des Inhalts!),
        \begin{bearbeitungshinweis}
            Direkt für die erste Aufgabe ein Hinweis
        \end{bearbeitungshinweis}
    \end{aufgabe}
    \bearbeitungshinweisZuAufgabe{}

    \begin{aufgabe}[subtitle=Einordnung in hist. Kontext,symbol=\symSprechblase]
        \dots sie in den historischen Kontext mit besonderer Bezugnahme auf die Bedeutung der Varrusschlacht einordnest,
    \end{aufgabe}
    \begin{loesung}
        Wenn ich eine Lösung habe, dann kann ich auch Erwartungen haben
    \end{loesung}

    \begin{aufgabe}[points=30, subtitle=Beurteilung,symbol=\symBleistift]
        \dots und zu der Darstellung der Ereignisse durch Flores Stellung nimmst
    \end{aufgabe}

    Die nun folgende Aufgabe wird nur nicht angezeigt aber gezählt
    \begin{aufgabe}[points=20,print=false]
        Verwende die Regeln der deutschen Sprache und belege deine Aussagen anhand der Quelle
    \end{aufgabe}

    Die nun folgende Aufgabe wird versteckt und nicht genutzt also auch nicht gezählt
    \begin{aufgabe}[points=20,use=false,print=false]
        Verwende die Regeln der klingonischen Sprache und belege deine Aussagen anhand der Quelle
    \end{aufgabe}


    \xsimsetup{
        aufgabe/template=schule-randpunkte,
        aufgabe*/template=schule-randpunkte
    }

    \begin{aufgabe*}[symbol=\symSprechblase]
        Eine Aufgabe im default Layout
        \begin{bearbeitungshinweis}
            probier mal: 1+2
        \end{bearbeitungshinweis}
    \end{aufgabe*}

    \begin{aufgabe}[subtitle={Superschwere Aufgabe},points=5, bonus-points=10,symbol=\symHaken]
        Eine Zusatzaufgabe mit zwei Teilaufgaben aber ohne Lösung
        \begin{teilaufgaben}
            \teilaufgabe Erste Teilaufgabe.
            \teilaufgabe Zweite Teilaufgabe.
        \end{teilaufgaben}
    \end{aufgabe}

    \xsimsetup{
        aufgabe/template=schule-randpunkte,
    }

    \begin{aufgabe}[points=4]
        Kleine Aufgabe mit Punkten an der Seite.
    \end{aufgabe}


%     \todo[inline]{Warum geht aufgabe tcolorbox innerhalb von mcumgebung nicht? wie ist \textbackslash{}choice definiert?}
%     \todo[inline]{Wie handhaben wir die MC-Lösungs-Sache? aufgabeMC klappt gerade nicht -> https://github.com/cgnieder/xsim/issues/33}

    \xsimsetup{
        aufgabe/template=schule-tcolorbox
    }

    \begin{aufgabe}[points=3,subtitle=Eine Kreuzchenaufgabe]
        \begin{mcumgebung} %Lösung wird automatisch ergänzt
            \choice Erstens
            \choice Zweitens
            \choice Drittens
            \choice[\mcrichtig] Viertens
            \choice Fünftens
            \choice[\mcrichtig] Sechstens
            \choice Siebtens
            \choice[\mcrichtig] Achtens
        \end{mcumgebung}
        \begin{loesung}
                \begin{mcumgebung} %Lösung wird automatisch ergänzt
                \choice Erstens
                \choice Zweitens
                \choice Drittens
                \choice[\mcrichtig] Viertens
                \choice Fünftens
                \choice[\mcrichtig] Sechstens
                \choice Siebtens
                \choice[\mcrichtig] Achtens
            \end{mcumgebung}
        \end{loesung}
        \begin{bearbeitungshinweis}
            Mache die Kreuzchen an der richtigen Stelle
        \end{bearbeitungshinweis}
    \end{aufgabe}
    \bearbeitungshinweisZuAufgabe{10}

    \xsimsetup{
        aufgabe/template=schule-tcolorbox,
        aufgabe*/template=schule-tcolorbox
    }

    \begin{aufgabe}[symbol=\symBleistift]
        \begin{teilaufgaben}
            \teilaufgabe[2] Das ist eine feste Lücke: \luecke[style={\uwave{#1}}]{2cm}
            \teilaufgabe[3] Das ist eine \textluecke[style={\dashuline{#1}}]{Lücke für einen Text}.
            \teilaufgabe[20] Keine Aufgabe
        \end{teilaufgaben}
    \end{aufgabe}

    \begin{aufgabe*}[points=10,]
        Eine Zusatzaufgabe. v3

        \begin{teilaufgaben}
            \teilaufgabe Das ist eine feste Lücke: \luecke[style={\uwave{#1}}]{2cm}
            \teilaufgabe Das ist eine \textluecke[style=\dashuline{#1}]{Lücke für einen Text}.
            \teilaufgabe Keine Aufgabe
        \end{teilaufgaben}

        \begin{loesung*}
        \begin{teilaufgaben}
            \teilaufgabe Das ist eine feste Lücke: \luecke[style={\uwave{#1}}]{2cm}
            \teilaufgabe Das ist eine \textluecke[style={\dashuline{#1}}]{Lücke für einen Text}.
            \teilaufgabe Keine Aufgabe
        \end{teilaufgaben}

        \end{loesung*}
        \begin{erwartungen}
            \erwartung{sollte eine $7+4=11$ weitere Lösung angeben.}{}[2]
        \end{erwartungen}
        \begin{bearbeitungshinweis}
            Noch ein Hinweis.
        \end{bearbeitungshinweis}
    \end{aufgabe*}

    \bearbeitungshinweisZuAufgabe[zusatzaufgabe]{12}

    \begin{aufgabe}[points=5,subtitle=Aufgabe mit Quellcode und (Sub-)Titel]
        \setzeSymbol{\symHaken}
        Verbessere den folgenden Quellcode.
        \begin{lstlisting}[basicstyle={\ttfamily},numbers=left,gobble=10]
            for int 1 = 0; i++;{
                set #power out
            }
        \end{lstlisting}
    \end{aufgabe}
    \begin{loesung}
        Diese Aufgabe ist leichter als man so denkt:
        \begin{lstlisting}[basicstyle={\ttfamily},numbers=left,gobble=10]
            for int 1 = 0; i++;{
                set #power on
            }
        \end{lstlisting}
        Man muss eigentlich nur aus dem \enquote{out} ein \enquote{on} machen.
    \end{loesung}

    \begin{aufgabe*}[points=5, bonus-points=2,subtitle={Mein Titel}]
        hallo
        \begin{teilaufgaben}
            \teilaufgabe nix1
            \teilaufgabe nix4
        \end{teilaufgaben}

        \begin{lstlisting}
            mycode.org
        \end{lstlisting}

        \begin{loesung*}
            nix 3
            \begin{teilaufgaben}
                \teilaufgabe nix3
            \end{teilaufgaben}

            \begin{lstlisting}
                mycode.org
                nostyle
                what ever
            \end{lstlisting}
        \end{loesung*}
    \end{aufgabe*}

    \section*{Hinweisliste}
        \bearbeitungshinweisliste

    \section*{Punktübersicht}

        \punktuebersicht

    \subsection*{Punktübersicht Hochformat}
        \punktuebersicht[default]

    \section*{Notenverteilung}
        \notenverteilung
\end{document}