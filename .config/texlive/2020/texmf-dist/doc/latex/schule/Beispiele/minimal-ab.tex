\documentclass[a4paper]{scrartcl}
\usepackage[
    typ=ab,
    fach=Informatik,
    lerngruppe=9g,
    loesungen=seite,
    module={Symbole,Bewertung},
]{schule}

% Dieses Dokument gehört zu den Beispiel des LaTeX Paketes Schule und ist von den Autoren
% des Pakets erstellt worden.
%
% Das Dokument steht unter der Lizenz: Creative Commons by-nc-sa Version 4.0
% http://creativecommons.org/licenses/by-nc-sa/4.0/deed.de
%
% Nach dieser Lizenz darf das Dokument beliebig kopiert und bearbeitet werden,
% sofern das Folgeprodukt wiederum unter gleichen Lizenzbedingungen vertrieben
% und auf die ursprünglichen Urheber verwiesen wird.
% Eine kommerzielle Nutzung ist ausdrücklich ausgeschlossen.

\author{Test Person}
\date{\today}
\title{Ein Arbeitsblatt}

\usepackage{blindtext}

\begin{document}

\section*{\Titel}
\blindtext

\begin{exercise}
    A first example for an exercise.
\end{exercise}

\begin{solution}
    A first example for a solution.
\end{solution}

\section*{Aufgaben}
\begin{aufgabe}
    Eine Aufgabe.
\end{aufgabe}

\begin{aufgabe}
    \setzeSymbol{\symHaken}
    Eine Zusatzaufgabe mit zwei Teilaufgaben
    \begin{teilaufgaben}
        \teilaufgabe Erste Teilaufgabe.
        \teilaufgabe Zweite Teilaufgabe.
    \end{teilaufgaben}
\end{aufgabe}

\begin{aufgabe}[points=10,bonus-points=5]
    Hier steht eine Aufgabe. Die ist wirklich schwierig und hat auch noch Teilaufgaben.
    \begin{teilaufgaben}
        \teilaufgabe Das ist eine feste Lücke: \luecke[style={\uwave{#1}}]{2cm}
        \teilaufgabe Das ist eine \textluecke[style={\dashuline{#1}}]{Lücke für einen Text}.
        \teilaufgabe Keine Aufgabe
    \end{teilaufgaben}

    \begin{loesung}
        \begin{teilaufgaben}
            \teilaufgabe Das ist eine feste Lücke: \luecke[style={\uwave{#1}}]{2cm}
            \teilaufgabe Das ist eine \textluecke[style={\dashuline{#1}}]{Lücke für einen Text}.
            \teilaufgabe Keine Aufgabe
        \end{teilaufgaben}
    \end{loesung}
\end{aufgabe}

\begin{aufgabe*}[points=10,bonus-points=5]
    Eine Zusatzaufgabe.
    \begin{loesung*}
        Eine weitere Lösung.
    \end{loesung*}
    \begin{bearbeitungshinweis}
        Ein Hinweis für die Zusatzaufgabe
    \end{bearbeitungshinweis}
\end{aufgabe*}


\begin{aufgabe}
    \begin{mcumgebung}
        \choice Erstens
        \choice Zweitens
        \choice Drittens
        \choice[\mcrichtig] Viertens
        \choice! Fünftens
        \choice[\mcrichtig] Sechstens
        \choice Siebtens
        \choice[\mcrichtig] Achtens
    \end{mcumgebung}

    \begin{loesung}
        \begin{mcumgebung}
            \choice Erstens
            \choice Zweitens
            \choice Drittens
            \choice[\mcrichtig] Viertens
            \choice! Fünftens
            \choice[\mcrichtig] Sechstens
            \choice Siebtens
            \choice[\mcrichtig] Achtens
        \end{mcumgebung}
    \end{loesung}
\end{aufgabe}

\section*{Punktübersicht}
    \punktuebersicht

\section*{Notenverteilung}
    \notenverteilung
\end{document}