\documentclass[]{scrartcl}
\usepackage[
    typ=ab,
    fach=Geschichte,
    lerngruppe=9g,
    loesungen=folgend,
    module={Symbole,Texte},
    sprache={english,french},
]{schule}
\usepackage{blindtext}

% Dieses Dokument gehört zu den Beispiel des LaTeX Paketes Schule und ist von den Autoren
% des Pakets erstellt worden.
%
% Das Dokument steht unter der Lizenz: Creative Commons by-nc-sa Version 4.0
% http://creativecommons.org/licenses/by-nc-sa/4.0/deed.de
%
% Nach dieser Lizenz darf das Dokument beliebig kopiert und bearbeitet werden,
% sofern das Folgeprodukt wiederum unter gleichen Lizenzbedingungen vertrieben
% und auf die ursprünglichen Urheber verwiesen wird.
% Eine kommerzielle Nutzung ist ausdrücklich ausgeschlossen.

\author{Test Person}
\date{\today}
\title{Ein Arbeitsblatt}
\subtitle{Nicht schwer, sondern ein Beispiel}

\addbibresource[location=remote,type=file]{http://uni-w.de/1v} %Komplett.bib

\begin{document}

\maketitle

Klasse \Kurs in \Fach am \Datum von \Autor bearbeitet \enquote{\Titel}.

%\tableofcontents

\section{Materialsammlung}

\material{Über Lizenzkompetenz}
    Eingesehen am \today:

        \enquote{Wie bereits erwähnt, sollte die Lizenzkompetenz nicht nur, aber besonders im Informatikunterricht vermittelt werden. Informatiklehrkräfte sollen dabei die technischen Aspekte erläutern, sprich Datenübertragung und ihre Protokolle, Datenspeicherung, Computernetzwerke, Lizenzen und Rolle der Informatiksysteme, die in vielen Bereichen des Alltags Anwendung finden. Das bedeutet keineswegs, dass die gesellschaftliche Perspektive grundsätzlich anderen Fächern überlassen werden soll, denn auch der Themenbereich \enquote{Informatik und Gesellschaft} sollte im Lehrplan einen festen Platz haben.}\autocite[29]{Salamon2013}

\material{Churchill 1939}
    \begin{otherlanguage}{english}
        Saved at \today:
        \enquote{I cannot forecast to you the action of Russia. It is a riddle wrapped in a mystery inside an enigma: but perhaps there is a key. That key is Russian national interest.}
    \end{otherlanguage}

\material{de Gaule 1940}
\begin{otherlanguage}{french}
	\today.
	\enquote{Tant il est vrai que, face aux grands périls, le salut n'est que dans la grandeur.}
\end{otherlanguage}

\clearpage
\quelle[\section]{Eine wichtige Quelle}
\begin{zeilenNrZweispaltig}
	\blindtext
\end{zeilenNrZweispaltig}

\quelle{Eine wichtige Quelle}
	\blindtext

\vt[\section]{Auch wichtige Quelle} \blindtext
\section{Verweise}
Siehe \nameref{sec:mat1} oder siehe \nameref{sec:vt1} und \nameref{sec:quelle1}

\printbibliography

\end{document}