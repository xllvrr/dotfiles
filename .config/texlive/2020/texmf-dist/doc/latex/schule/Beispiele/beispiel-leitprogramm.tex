\documentclass[12pt,a4paper,parskip=half,chapterprefix,bibliography=totoc,footheight=57pt,numbers=noendperiod]{scrreprt}
\usepackage[
    typ=leit,
    fach=Informatik,
    farbig,
    module={Lizenzen},
    lizenz={cc-by-nc-sa-4},
]{schule}

% Dieses Dokument gehört zu den Beispiel des LaTeX Paketes Schule und ist von den Autoren
% des Pakets erstellt worden.
%
% Das Dokument steht unter der Lizenz: Creative Commons by-nc-sa Version 4.0
% http://creativecommons.org/licenses/by-nc-sa/4.0/deed.de
%
% Nach dieser Lizenz darf das Dokument beliebig kopiert und bearbeitet werden,
% sofern das Folgeprodukt wiederum unter gleichen Lizenzbedingungen vertrieben
% und auf die ursprünglichen Urheber verwiesen wird.
% Eine kommerzielle Nutzung ist ausdrücklich ausgeschlossen.

\title{Beispiel Leitprogramm}

\author{Fachgruppe Informatische Bildung in NRW der Gesellschaft für Informatik}
\usepackage{gitfile-info}
\usepackage{qrcode}
\ifoot*{Version vom \gfiGetDate{}}
\ofoot*{\lizenzSymbol ~\qrcode{http://ddi.uni-wuppertal.de/material/materialsammlung/}}

\renewcommand{\ttdefault}{pcr}

\usepackage[
    language=ngerman,
    backend=biber,
    sortlocale=de_DE,
    style=authoryear,
    bibencoding=UTF8,
    block=space,
    autocite=inline % \autocite[..][..]{..} erzeugt Literaturverweise mit runden Klammern
]{biblatex}
\addbibresource[location=remote,type=file]{http://uni-w.de/1v} %Komplett.bib

\usepackage{blindtext}

\begin{document}
    \begin{titlepage}
        \begin{center}
            \begin{Large}
                Leitprogramm als
            \end{Large}
        \end{center}
        \begin{center}
            \begin{huge}
                Beispiel
            \end{huge}
        \end{center}
        \begin{center}
            Stand \gfiGetDay. \monatWort{\gfiGetMonth} \gfiGetYear
        \end{center}
        \vfil
        \begin{center}
            \begin{tikzpicture}
                \draw(0,0) rectangle (6cm,8cm);
                \draw(0,0) -- (6cm,8cm);
                \draw(6cm,0) -- (0,8cm);
            \end{tikzpicture}
        \end{center}
    \end{titlepage}

    \tableofcontents

\chapter{Vorwort}
    Hier ist Vorworttext. Eine komplettes Leitprogramm gibt es in der Materialsammlung (siehe \autocite{Pieper2014}). Dort sind welche zu verschiedenen Themen vorhanden.

    \blindtext

\chapter{Weiteres Kapitel}
    Ein weiteres Kapitel mit Blindtext und einer Aufgabe: \blindtext

    \begin{aufgabe}
        Heute ganz wichtig aufpassen.
    \end{aufgabe}


    \section{Abschnitt}
        \blindtext

        \blindtext

        \begin{aufgabe}
            Mache etwas mit dieser Aufgabe und schreibe es in das Feld.
            \TextFeld{3cm}
        \end{aufgabe}

    \section{Weitere Abschnitt}
        Ein weiterer Abschnitt mit gleich zwei Aufgaben:
        \begin{aufgabe}
            Mache noch etwas. Nutze auch die Hinweise.
            \begin{bearbeitungshinweis}
                Wenn du diesen Hinweis gelesen hast, dann bist du deiner Aufgabe nachgekommen.
            \end{bearbeitungshinweis}
        \end{aufgabe}
        \begin{aufgabe}
            Entspanne dich und schau in die Lösung
        \end{aufgabe}
        \begin{loesung}
            Du bist bei dieser Lösung zur Entspannung genau richtig.
        \end{loesung}

        \begin{hinweisBox}
            An dieser Stelle steht ein Hinweis auf andere Dinge.
        \end{hinweisBox}


\chapter{Zwei Kapitel sind zu wenig}
    Deshalb kommt hier noch eine Aufgabe
    \begin{aufgabe} Mach folgendes
        \begin{enumerate}
            \item Lege dich hin
            \item Schlaf ein
            \item
        \end{enumerate}
        \TextFeld{3cm}
    \end{aufgabe}
    \begin{bearbeitungshinweis}
        Hinweis
    \end{bearbeitungshinweis}
    \begin{loesung}
        Zu dieser Aufgabe kommt auch eine Lösung noch dazu
    \end{loesung}

    Und eine Aufgabe mit Teilaufgaben
    \begin{aufgabe}
        \begin{teilaufgaben}
            \teilaufgabe erste Teilaufgabe mit einem etwas längeren Text, der auch eine Zeile wirklich überschreitet
            \TextFeld{3cm}
            \teilaufgabe zweite Teilaufgabe
            \TextFeld{3cm}
            \teilaufgabe Dritte
        \end{teilaufgaben}
        \begin{bearbeitungshinweis}
            \begin{teilaufgaben}
                \teilaufgabe Erster Hinweis
                \teilaufgabe Zweiter Hinweis
                \teilaufgabe Dritter Hinweis
            \end{teilaufgaben}
        \end{bearbeitungshinweis}
        \begin{loesung}
            \begin{teilaufgaben}
                \teilaufgabe Erste Lösung
                \teilaufgabeOhneLoesung
                \teilaufgabe Dritter Lösung (zweite gibt es nicht)
            \end{teilaufgaben}
        \end{loesung}
    \end{aufgabe}

\chapter{Hinweisliste}
    \bearbeitungshinweisliste

\chapter{Lösungen}
    \printsolutions

\printbibliography

\end{document}
