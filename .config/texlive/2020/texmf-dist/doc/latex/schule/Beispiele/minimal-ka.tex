\documentclass[a4paper]{scrartcl}
\usepackage[
    typ=kl,
    klausurtyp=klasse,
    fach=Deutsch,
    lerngruppe=1g,
    loesungen=seite,
    erwartungshorizontAnzeigen,
    erwartungshorizontStil=simpel,
]{schule}

% Dieses Dokument gehört zu den Beispiel des LaTeX Paketes Schule und ist von den Autoren
% des Pakets erstellt worden.
%
% Das Dokument steht unter der Lizenz: Creative Commons by-nc-sa Version 4.0
% http://creativecommons.org/licenses/by-nc-sa/4.0/deed.de
%
% Nach dieser Lizenz darf das Dokument beliebig kopiert und bearbeitet werden,
% sofern das Folgeprodukt wiederum unter gleichen Lizenzbedingungen vertrieben
% und auf die ursprünglichen Urheber verwiesen wird.
% Eine kommerzielle Nutzung ist ausdrücklich ausgeschlossen.

\author{Test Person}
\date{\today}
\title{Eine Klassenarbeit}

\usepackage{blindtext}

\begin{document}

\begin{aufgabe}[points={2},bonus-points={3}]
    \blindtext
    \begin{teilaufgaben}
        \teilaufgabe Lies den Text!
        \teilaufgabe Unterschlängele alle Nomen.
    \end{teilaufgaben}
    \begin{loesung}
        Dies hier ist ein \uwave{Blindtext} zum \uwave{Testen} von \uwave{Textausgaben}\dots
    \end{loesung}
    \begin{erwartungen}
        \erwartung{Du hast den Text gelesen.}{}
        \erwartung{Du hast alle alle Nomen unterschlängelt.}{}
    \end{erwartungen}
\end{aufgabe}

\begin{aufgabe*}
    Eine Zusatzaufgabe mit zwei Teilaufgaben
    \begin{teilaufgaben}
        \teilaufgabe Erste Teilaufgabe.
        \teilaufgabe Zweite Teilaufgabe.
    \end{teilaufgaben}
    \begin{loesung*}
        \begin{teilaufgaben}
            \teilaufgabe Die Lösung lautet 1.
            \teilaufgabe Die Lösung lautet 2.
        \end{teilaufgaben}
    \end{loesung*}
    \begin{erwartungen}
        \erwartung{Deine Lösung zu Teilaufgabe a) ist richtig.}{}
        \erwartung{Deine Lösung zu Teilaufgabe b) ist richtig.}{}
    \end{erwartungen}
\end{aufgabe*}

\vspace{1cm}
{\Huge Viel Erfolg!}

\end{document}