\documentclass[a4paper]{scrartcl}
\usepackage[
    typ=kl,
    klausurtyp=klausur,
    fach=Physik,
    lerngruppe=Q1/IV,
]{schule}

% Dieses Dokument gehört zu den Beispiel des LaTeX Paketes Schule und ist von den Autoren
% des Pakets erstellt worden.
%
% Das Dokument steht unter der Lizenz: Creative Commons by-nc-sa Version 4.0
% http://creativecommons.org/licenses/by-nc-sa/4.0/deed.de
%
% Nach dieser Lizenz darf das Dokument beliebig kopiert und bearbeitet werden,
% sofern das Folgeprodukt wiederum unter gleichen Lizenzbedingungen vertrieben
% und auf die ursprünglichen Urheber verwiesen wird.
% Eine kommerzielle Nutzung ist ausdrücklich ausgeschlossen.

\date{\today}
\title{Test von Teilpunkten}

\begin{document}
    \begin{aufgabe}[subtitle=Erste ohne Teilpunkte,points=34]
        Ohne Teilpunkte
    \end{aufgabe}

    \begin{aufgabe}[subtitle=Erste Aufgabe mit Teilpunkten]
        \begin{teilaufgaben}
            \teilaufgabe[4] Erste Teilaufgabe
            \teilaufgabe[6] Zweite Teilaufgabe
        \end{teilaufgaben}
    \end{aufgabe}

    \begin{aufgabe}[subtitle=Zweite mit Teilpunkten]
        Davor noch ein Text
        \begin{teilaufgaben}
            \teilaufgabe[10] Erste zweite Teilaufgabe

            \teilaufgabe[10] Zweite zweite Teilaufgabe
        \end{teilaufgaben}
    \end{aufgabe}

    \begin{aufgabe}[subtitle=Ohne Punkte]
        Ohne Teilpunkte und Punkte
    \end{aufgabe}

    \begin{aufgabe}[subtitle=Ohne Teilpunkte,points=23]
        Ohne Teilpunkte
    \end{aufgabe}

\vspace{1cm}

{\Huge Viel Erfolg!}

\vspace{1cm}

\punktuebersicht

\end{document}