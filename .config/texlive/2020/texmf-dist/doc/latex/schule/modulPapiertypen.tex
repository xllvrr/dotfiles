\section{Papiertypen}
\label{modul:papiertypen}
Das Modul \module{Papiertypen} stellt einige Makros bereit, die
es erlauben, Freiräume zum Bearbeiten von Aufgaben zu setzen. Hierzu
stehen verschiedene Muster zur Auswahl. Die entsprechenden Felder werden
dabei in der Breite jeweils auf \verbcode|\linewidth| skaliert, 
allerdings so, dass ein vollständiges Muster entsteht. Die zur Verfügung
stehende Breite wird also optimal genutzt.

\subsection{Befehle}
\begin{commands}
	\command{feldLin}[\oarg{Abstand}\marg{Anzahl}]
		setzt die angegebene Anzahl Linien mit dem angegebenen Abstand zueinander. Der Standardabstand beträgt $1cm$.
\begin{example}
  \feldLin[1cm]{4}
\end{example}

	\command{feldKar}[\oarg{Seitenlänge}\marg{Anzahl}]
		setzt die angegebene Anzahl von Karo-Kästchen mit einer
		gegebenen Seitenlänge. Der Standard für die Seitenlänge beträgt
		$0,5cm$.
\begin{example}
  \feldKar[0.5cm]{5}
\end{example}

	\command{feldMil}[\marg{Anzahl}]
		setzt die angegebene Anzahl von Kästchen im
		Millimeterpapiermuster untereinander. Die Farbe wird von der
		Paketoption \option{farbig} beeinflusst.
\begin{example}
  \feldMil{2}
\end{example}
\end{commands}