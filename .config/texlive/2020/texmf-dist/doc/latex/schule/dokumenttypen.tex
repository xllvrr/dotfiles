\part{Dokumenttypen}
	\label{sec:typen}
	Dokumenttypen dienen dazu, Vorgaben für spezielle Arten von
	Dokumenten zu machen und entsprechende Makros bereitzustellen.
	Üblicherweise wird eine bestimmte Dokumentklasse für die
	Verwendung empfohlen, grundsätzlich sind die Dokumenttypen aber
	unabhängig von der verwendeten Klasse.

	Die Typen werden über die Paketoption \option{typ} geladen.
	Ist der Typ unbekannt, wird ein Arbeitsblatt gesetzt und der
	die Bezeichnung des Dokuments wird auf den angegebenen Typ
	eingestellt.

	Wird kein Typ angegeben, ist der Kompatibilitätsmodus zum
	alten Schule-Paket aktiv. Es wird folglich kein Dokumenttyp
	geladen.
	Ebensowenig wird ein Dokumenttyp geladen, wenn der Typ auf
	\keyis{typ}{ohne} gesetzt wird, da sich das Paket dann im
	eingebetteten Modus befindet, also kein eigenständiges
	Dokument gesetzt werden soll.

	\section{Arbeitsblatt}
\label{typ:ab}
Dieser Dokumenttyp ist der Standard des Schulepakets. Wird ein unbekannter Dokumenttyp verwendet, wird stattdessen ein Arbeitsblatt gesetzt. Um gezielt das Arbeitsblatt zu verwenden ist als Typ \verbcode|ab| anzugeben.

Die empfohlene Dokumentklasse ist \cls{scrartcl}.

Die Vorgaben des Dokumenttyps definieren Kopf- und Fußzeilen in der üblichen Darstellung des Schule-Pakets. Darüber hinaus werden keine weiteren Vorgaben gemacht.

%\subsection{Paketoptionen}
%\begin{options}
%	\opt{X}
%		Y
%\end{options}

%\subsection{Befehle}
%\begin{commands}
%
%\command{X}
%	Y
%
%\end{commands}
%	\input{typBrief}
	\section{Klausur}
\label{typ:kl}
Dieser Dokumenttyp wird für Klausuren und Klassenarbeiten verwendet. Für die Verwendung der Klausur oder Klassenarbeit ist als Typ \verbcode|kl| anzugeben.

Die empfohlene Dokumentklasse ist \cls{scrartcl}.

Die Vorgaben des Dokumenttyps definieren Kopf- und Fußzeilen in der üblichen Darstellung des Schule-Pakets. Außerdem werden Namens- und Datumsfelder erzwungen.

\subsection{Paketoptionen}
\begin{options}
    \keychoice{klausurtyp}{klausur,klasse,kurs}\Default{klausur}
        legt fest, ob die Klausur als Klausur (Standard), Kursarbeit (\keyis{klausurtyp}{kurs}) oder Klassenarbeit (\keyis{klausurtyp}{klasse}) bezeichnet wird.
\end{options}

%\subsection{Befehle}
%\begin{commands}
%
%\command{X}
%	Y
%
%\end{commands}
%	\input{typKursbuch}
    \section{Leitprogramm}
\label{typ:Leitprogramm}
Als Leitprogramm wird eine Grundlage für den Unterricht bezeichnet, mit dem \SuS sich ein größeres Thema erarbeiten können. Ein Leitprogramm enthält dafür erklärende Texte sowie Aufgaben mit Hinweisen und Lösungen. Diese werden von Elemente werden von den Lernenden selbstständig gelesen und bearbeitet. Zum Abschluss eines Kapitels gehört in der Regel ein Kapiteltest. Dieses holen sich die \SuS bei der Lehrkraft ab um ihn zu bearbeiten und ihn anschließend direkt von der Lehrkraft kontrollieren zu lassen. Dieser Test wird dabei nur auf Grundlage des Erlernten und ohne direktes Hinzunehmen des Leitprogramms absolviert.

Der Dokumententyp Leitprogramm, als Typ ist \verbcode|leit| anzugeben, stellt die layouttechnischen Grundlagen bereit und sorgt für die Verknüpfungen zwischen den Aufgaben und den dazugehörenden Hinweisen und Lösungen. Der Dokumententyp lässt sich aber auch für ein Skript nutzen, dass aus verschiedenen Kapiteln besteht. Die empfohlene Dokumentklasse ist \cls{scrreprt}. Ein Beispiel ist unter \prettyref{example:beispiel-leitprogramm} aufgeführt.

\subsection{Paketoptionen}
Beim Leitprogramm werden standardmäßig von der Aufgabe Links zu möglichen vorhanden Lösungen oder Bearbeitungshinweisen gesetzt. Da dieses Schaltflächen auch angezeigt werden, wenn die Lösungen bzw. Hinweise nicht eingebunden wurden, kann die Anzeige über Paketoptionen ausgeschaltet werden.
\begin{options}
    \opt{hinweisLinkVerbergen} verbirgt Links bei der Aufgabe zu möglichen Bearbeitungshinweisen.
    \opt{loesungLinkVerbergen} verbirgt Links bei der Aufgabe zu möglichen Lösungen.
\end{options}

\subsection{Befehle}
\begin{commands}
    \command{TextFeld}[\marg{Höhe}] Erstellt ein Formularfeld mit der angegebenen Höhe und der aktuellen Spaltenbreite. Mit passenden Anzeigeprogrammen kann dann an dieser Stelle im PDF-Dokument Text eingegeben werden.
    \command{monatWort}[\marg{Monatszahl}] Übersetzt den als Zahl angegeben Monat in den deutschen Namen. Sollte die Zahl nicht erkannt werden, wird \enquote{unbekannter Monat} ausgegeben.
    \command{uebungBild} Erstellt ein Symbol für eine Übung, dass allen Aufgaben innerhalb eines Leitprogramms vorangestellt wird.
    \command{hinweisBild} Erstellt ein Symbol für ein Hinweis.
\end{commands}

\subsection{Umgebungen}
\begin{environments}
    \environment{hinweisBox} Erzeugt eine optisch hervorgehobene Box, die mit dem Symbol für einen Hinweis gekennzeichnet ist.
\end{environments}
%	\input{typLogbuch}
	\section{Lernzielkontrolle}
\label{typ:lzk}
Dieser Dokumenttyp wird für Lernzielkontrollen verwendet. Für die Verwendung der Lernzielkontrolle ist als Typ \verbcode|lzk| anzugeben.

Die empfohlene Dokumentklasse ist \cls{scrartcl}.

Die Vorgaben des Dokumenttyps definieren Kopf- und Fußzeilen in der üblichen Darstellung des Schule-Pakets. Außerdem werden Namens- und Datumsfelder erzwungen.

%\subsection{Paketoptionen}
%\begin{options}
%	\opt X
%		Y
%\end{options}

%\subsection{Befehle}
%\begin{commands}
%
%\command{X}
%	Y
%
%\end{commands}
	\section{Übungsblatt}
\label{typ:ueb}
Dieser Dokumenttyp wird für Übungsblätter verwendet. Der Hauptunterschied zum Typ \enquote{Arbeitsblatt} liegt darin, dass Aufgaben als \enquote{Übungen} bezeichnet werden. Für die Verwendung des Übungsblatts ist als Typ \verbcode|ueb| anzugeben.

Die empfohlene Dokumentklasse ist \cls{scrartcl}.

Die Vorgaben des Dokumenttyps definieren Kopf- und Fußzeilen in der üblichen Darstellung des Schule-Pakets. Darüber hinaus werden keine weiteren Vorgaben gemacht.

%\subsection{Paketoptionen}
%\begin{options}
%	\opt{X}
%		Y
%\end{options}

%\subsection{Befehle}
%\begin{commands}
%
%\command{X}
%	Y
%
%\end{commands}
    \section{Unterrichtsbesuch}
\label{typ:ub}
Dieser Dokumenttyp dient als Grobvorlage für Unterrichtsbesuche. Eine komplette Vorlage wird nicht angeboten, da die Studienseminare unterschiedliche Anforderungen stellen und es auch in den einzelnen Seminaren sehr häufig Änderungen an den layouttechnischen Aspekten gibt. Die Hauptanwendungen dieses Dokumenttyps sind daher Unterrichtsbesuche, bei denen es keine festen Vorgaben gibt, wie z.\,B. bei Revisionen oder der Materialsammlung für Informatik, die vollständig dieses \LaTeX-Paket nutzt. Für die Verwendung dieses Dokumententyps ist \verbcode|ub| anzugeben.

Als Dokumentklasse wird für diesen Typ \cls{scrartcl} empfohlen. Darin wird durch den Dokumenttyp Kopf- und Fußzeile gesetzt, sowie eine Titelseite erzeugt, die mit Angaben gefüllt wird, die für einen Unterrichtsbesuch typisch sind und entsprechend angegeben werden müssen.

%\subsection{Paketoptionen}
%\begin{options}
%	\opt{X}
%		Y
%\end{options}

\subsection{Befehle für Angaben zum Unterrichtsbesuch}
Mit den folgenden Befehlen werden Angaben gesetzt, die auf der Titelseite des Unterrichtsbesuch angezeigt werden.
\begin{commands}
    \command{besuchtitel}[\marg{Titel}]
        setzt den Eintrag, um was für eine Art es sich bei dem Unterrichtsbesuch handelt. Dieses kann z.\,B. sein: "`2. Unterichtsbesuch im Fach Informatik"'.
    \command{lehrer}[\marg{Lehrername}]
        setzt den Namen des Lehrers, der neben der Titelseite auch im Seitenkopf angezeigt wird.

    \command{schulform}[\marg{Schulform}]
        setzt den Eintrag für die Schulform wie z.\,B. Gesamtschule.

    \command{lerngruppe}[\oarg{Kurzform der Lerngruppe} \marg{Name der Lerngruppe} \marg{Anzahl weiblich} \marg{Anzahl männlich}]
        sorgt dafür, dass die Angaben zur Lerngruppe gesetzt werden. Der Name wird auf dem Titelblatt und im Seitenkopf angegeben, außer die optionale Möglichkeit der Kurzform wurde genutzt. In diesem Fall wird die Kurzform im Seitenkopf angegeben. Aus der Anzahl der weiblichen und männlichen Schülerinnen und Schüler wird automatisch die Gesamtzahl bestimmt, daher sind für diese Angaben nur Zahlen erlaubt.

    \command{zeit}[\marg{Startzeit} \marg{Endzeit} \marg{Stunde}]
        bietet die Möglichkeit, die Zeiten der Besuchsstunde anzugeben. Neben der Uhrzeit des Beginns und des Endes muss angegeben werden, um welche Stunde es sich an dem entsprechenden Tag handelt.

    \command{schule}[\marg{Name der Schule}]
        hierüber lässt sich der Name der Schule angeben, der auf der Titelseite angezeigt wird.

    \command{raum}[\marg{Raumbezeichnung}]
        bietet die Möglichkeit die Bezeichnung des Raumes anzugeben, in dem die Besuchsstunde stattfinden soll.

\end{commands}
    \section{Folie}
\label{typ:folie}
Bei der Nutzung des Dokumenttyps Folie wird eine Seite mit wenig Rand zur Verfügung gestellt, der die Fuß und Kopfzeile fehlt. So kann möglichst viel auf eine Folie gedruckt werden, wenn diese im Unterricht zum Einsatz kommen soll. Um den Dokumenttyp verwenden zu können muss \verbcode|folie| als Typ angegeben werden.

%\subsection{Paketoptionen}
%\begin{options}
% \opt{X}
%   Y
%\end{options}

% \subsection{Befehle}
% \begin{commands}
%     \command{x}[]
%         ...
% \end{commands}