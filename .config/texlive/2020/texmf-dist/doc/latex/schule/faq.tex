\part{Häufig gestellte Fragen}
	\section{Formatierung}
		\subsection{Kann ich ein anderes Papierformat als A4 verwenden?}
			Ja! Denn dies wird von der Dokumentenklasse und nicht vom
			Schule-Paket festgelegt, \zB\space ein A5-Blatt im Querformat:
\begin{sourcecode}
  \documentclass[a5paper,landscape]{scrartcl}
\end{sourcecode}			
		\subsection{Kann ich die Seitenränder festlegen?}
			Ja! Dazu kann einfach das übliche Paket \pkg{geometry} im
			Dokument genutzt werden, \zB:
\begin{sourcecode}
  \usepackage[
    left=3cm,
    right=2cm,
    top=2cm,
    bottom=2cm,
    footskip=1cm
  ]{geometry}
\end{sourcecode}
		\subsection{Ist es möglich, in den erstellten Materialien
		 Schreibschriften zu verwenden?}
			Ja! Das ist kein Problem, allerdings ist das in \LaTeX\space
			bereits so einfach, dass das Schule-Paket hier keine
			abweichenden Funktionen implementiert.
			
			Zu empfehlen ist hier das Paket \pkg{schulschriften}.
			Dieses bringt Fonts für die folgenden Schreibschriften mit:
			
			\begin{description}
				\item[wesu] Sütterlinschrift (1911)
				\item[wedn] Deutsche Normalschrift (1941)
				\item[wela] lateinische Ausgangsschrift (1953) 
				\item[wesa] Schulausgangsschrift (1968, ehem. DDR)
				\item[weva] Vereinfachte Ausgangsschrift (1972)
			\end{description}
			
			Diese können einfach verwendet werden, indem das jeweilige
			Paket, \zB\space\pkg{weva}, eingebunden wird. Danach kann
			die zugehörige Schrift verwendet werden
			
\begin{example}
  \weva Vereinfachte Ausgangsschrift
\end{example}