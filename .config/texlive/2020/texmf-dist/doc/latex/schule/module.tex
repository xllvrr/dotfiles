\part{Module}
    \label{sec:module}
    \section{Nutzung der Module}
        \subsection{Standardmodule}
        Standardmäßig wird davon ausgegangen, dass ein Dokument mit Schulkontext gesetzt werden soll (Arbeitsblatt, Klausur, etc). Dann lädt das Schule-Paket die Module \module{Metadaten}, \module{Format} und \module{Aufgaben}.

        Wird das Paket eingebettet verwendet, also mit der Paketoption \keyis{typ}{ohne} geladen, lädt das Schule-Paket nur die Module \module{Metadaten} und \module{Format}.

        Wird ein nicht definierter Typ angegeben, wird ein Arbeitsblatt gesetzt und der angegebene Typ wird als Bezeichner verwendet.

        \subsection{Laden weiterer Module}
        \begin{options}
            \keyval{module}{Modul1,Modul2,\dots}\Default Weitere Module können geladen werden, indem Sie der Paketoption \option{module} als kommaseparierte Liste übergeben werden.
        \end{options}

    \section{Aufgaben}
\label{modul:aufgaben}
Das Modul \module{Aufgaben} ist das umfangreichste Modul des Schule-Pakets. Es umfasst alles, was zum Setzen von verschiedenen Arbeitsblättern, Klausuren, Klassenarbeiten, Lernzielkontrollen\usw. notwendig ist.

Im Kern baut das Modul auf dem Paket \pkg{xsim} auf, sodass alle Funktionen dieses Pakets nutzbar sind.

Die vom Schulepaket gemachten Ergänzungen sind voll kompatibel zu \pkg{xsim}, so werden die Hinweise etwa in den Eigenschaften der Aufgaben gespeichert.

\subsection{Aufgaben}

\subsubsection{Befehle}
\begin{commands}
    \command{setzeSymbol}[\marg{Symbol}] kann nur innerhalb der Aufgabenumgebung genutzt werden und stellt der jeweiligen Aufgabe ein Symbol voran. Dies kann etwa genutzt werden, um die Arbeitsform oder bestimmte Aufgabentypen zu kennzeichnen.

    Eine Kombination mit dem Modul \module{Symbole} bietet sich an. So könnte etwa zur Kennzeichnung von Höraufgaben \verbcode|\setzeSymbol{\symOhr}| genutzt werden.

    Alternativ lässt sich das Symbol auch als Eigenschaft der Aufgabe direkt setzen. Dieses erfolgt z.\,B. durch \verbcode|\begin{aufgabe}[symbol=\symOhr]|.

    \command{punkteAufgabe} liefert die Punkte der aktuellen Aufgabe inkl. der Bezeichnung.

    \command{punkteTotal} liefert die Gesamtpunktezahl aller Aufgaben inkl. der Bezeichnung.

    \command{punktuebersicht}[\oarg{Darstellungsart}]\Default{kurz} setzt eine Übersichtstabelle über die in allen Aufgaben erreichbaren Punkte und Zusatzpunkte, sowie einer Leerzeile für die erreichten Punkte. Als optionalen Parameter kann zwischen verschiedenen Darstellungen gewählt werden. Alternativ zur Standardoptionen ist \verbcode{default}.
\end{commands}

\subsubsection{Umgebungen}
\begin{environments}
    \environment{aufgabe\sarg}[\oarg{Optionen}]\envidx{aufgabe}%
        setzt eine Aufgabe. Alle Aufgaben werden automatisch durchnummeriert. Wird der optionale Stern angegeben, wird die Aufgabe als Zusatzaufgabe gesetzt.

        Bei den optionales Argument können alle von \pkg{xsim} bereitgestellen Optionen angegeben werden. Dazu gehören unter anderem folgende:
        \begin{options}
            \keyval-{points}{Punkte} legt die Punkte der Aufgabe fest.
            \keyval-{bonus-points}{Zusatzaufgabe} legt die Punkte der Aufgabe fest.
            \keyval{subtitle}{Titel} setzt den Title der Aufgabe.
        \end{options}
\end{environments}

\subsubsection{Aufgabentemplates}
    Die Darstellung der Aufgaben erfolgt auf der Grundlage verschiedener Templates. Das Paket \pkg{schule} liefert dabei folgende Templates mit, die in darunter dargestellt sind.
    \begin{itemize}
        \item \verbcode|schule-binnen|
        \item \verbcode|schule-default|
        \item \verbcode|schule-keinenummer|
        \item \verbcode|schule-keinepunkte|
        \item \verbcode|schule-keintitel|
        \item \verbcode|schule-randpunkte|
        \item \verbcode|schule-tcolorbox|
    \end{itemize}

    \includepdfpage[page=1, scale=0.6, trim={3cm, 16cm, 1cm, 2.5cm}, clip] {beispiel-aufgabentemplates}

    \begin{commands}
        \command{setzeAufgabentemplate}[\marg{Templatename}] setzt das Template mit dem die folgenden Aufgaben dargestellt werden.
    \end{commands}

\subsection{Teilaufgaben}

\subsubsection{Befehle}
\begin{commands}
    \command{teilaufgabe}[\oarg{Punkte}] leitet innerhalb einer \env{teilaufgaben}-Umgebung eine Teilaufgabe ein. Teilaufgaben werden mit den Kleinbuchstaben von \textit{a} bis \textit{z} gekennzeichnet.

    Über den optionalen Parameter kann eine Punktzahl angegeben werden.

    \command{teilaufgabeOhneLoesung} Dient als Platzhalter bei Teilaufgaben, bei den keine Lösung angegeben wird. Die entsprechende Nummer wird bei den Lösungen nicht aufgeführt und die folgende Teilaufgaben bekommt den nächsten Buchstaben, so dass es übereinstimmt mit der Aufgabenstellung.
\end{commands}

\subsubsection{Umgebung}
\begin{environments}
    \environment{teilaufgaben} bietet die Möglichkeit, eine Aufgabe in verschiedene Teilaufgaben zu unterteilen.
    \begin{sourcecode}[gobble=8]
        \begin{aufgabe}
            Inhalt...
            \begin{teilaufgaben}
                \teilaufgabe Erstens.
                \teilaufgabe[5] Zweitens.
            \end{teilaufgaben}
        \end{aufgabe}
    \end{sourcecode}
    \hinweis{Teilaufgaben können auch in einer \env{loesung}- und \env{bearbeitungshinweis}-Umgebung verwendet werden!}
\end{environments}

\subsection{Lösungen}

\subsubsection{Paketoptionen}\label{subsubsec:paketoptionen}
\begin{options}
    \keychoice{loesungen}{folgend, keine, seite}\Default{keine}
        legt fest, ob die Lösungen direkt hinter die Aufgaben, als eigenständige Lösungsseite oder gar nicht gesetzt werden.

        \achtung{Die Option \keyis{loesungen}{seite} ist nur für eigenständige Dokumente, \zB\space mit der Dokumentenklasse \cls{scrartcl} gedacht. Sie greift tief in den Übersetzungsprozess ein und ist geeignet Fehler im Zusammenspiel mit anderen Paketen zu provozieren.}
\end{options}

\begin{commands}
    \command{printsolutions} Wenn keine der Standardoptionen genutzt wird, kann der Befehl zur Ausgabe der Lösungen aus dem \pkg{exsheets}-Paket genutzt werden.
\end{commands}

\subsubsection{Umgebungen}
\begin{environments}
    \environment{loesung\sarg}[] wird innerhalb oder direkt hinter einer \env{aufgabe} verwendet, um eine Lösung dazu anzugeben. Die Inhalte dieser Umgebung werden standardmäßig nicht gesetzt, sondern durch die entsprechende Konfiguration von \option{loesungen} an der entsprechenden Stelle gesetzt. Wichtig ist, dass bei Zusatzaufgaben auch bei der Lösung der Stern gesetzt werden muss.

    \begin{sourcecode}[gobble=8]
        \begin{aufgabe}
            Inhalt...
            \begin{teilaufgaben}
                \teilaufgabe Erstens.
                \teilaufgabe[5] Zweitens.
            \end{teilaufgaben}
            \begin{loesung}
                \begin{teilaufgaben}
                    \teilaufgabe Erste Lösung.
                    \teilaufgabe Zweite Lösung.
                \end{teilaufgaben}
            \end{loesung}
        \end{aufgabe}
    \end{sourcecode}
\end{environments}

\subsection{Lückentexte}

\subsubsection{Befehle}
\begin{commands}
    \command{luecke}[\oarg{Optionen für blank}\marg{Länge}]
        Setzt eine Lücke mit der angegebenen Länge. Der Befehl nutzt dazu den \cs{blank}-Befehl aus \pkg{xsim}. Mit dem optionalen Parameter können zusätzliche Optionen an diesen weitergereicht werden, \zB\space kann mit \texttt{style = \choices{line,wave,dline,dotted,dashed}} der Stil der Unterstreichung festgelegt werden.
    \command{textluecke}[\oarg{Optionen für blank}\marg{Text}]
        Setzt eine Lücke für den angegebenen Text, die Länge wird durch den angegebenen Text vorgegeben. Standardmäßig wird als Korrekturfaktor für das handschriftliche Ausfüllen $2$ genutzt.

        Der Befehl nutzt dazu den \cs{blank}-Befehl aus \pkg{xsim}. Mit dem optionalen Parameter können zusätzliche Optionen an diesen weitergereicht werden, \zB\space kann mit \verbcode|style=\dashuline{#1}| eine unterstrichelte Linie gesetzt werden. Mit \texttt{scale = 3} ließe sich der Korrekturfaktor auf $3$ anpassen. Wird als Option \texttt{nichts} angegeben, so wird die Lücke ohne Inhalt und Weite eingesetzt.

        Innerhalb von Lösungsumgebungen wird der Text in die Lücke eingesetzt.

        %%%%% Mit xsim nicht möglich
%     \command{aufgabeLueckentext}[\oarg{Punkte}\marg{Lückentext} \marg{Extras}\oarg{Symbol}\oarg{Optionen für die Aufgabenumgebung}]
%         Um Lückentexte mit Lösungen anzugeben, müsste der gesamte Text mit den Lücken zweimal im Dokument stehen: einmal in der Aufgaben- und einmal in der Lösungsumgebung. Um dies einfacher zu gestalten, wurde als Abkürzung dieser Befehl eingeführt, der intern eine entsprechende Aufgaben- und Lösungsumgebung erzeugt.
%
%         Der Parameter \enquote{Extras} dient dem Anhängen von Inhalten an den Aufgabentext und kann somit etwa für Erwartungen und Hinweise genutzt werden.

% \begin{sourcecode}
% \aufgabeLueckentext[4]{
%     Das ist ein total verrückter \textluecke{Lückentext}. Für alle
%     \textluecke{Menschen}, die \textluecke{Lücken} lieben.
% }{
%     \begin{erwartungen}
%         \erwartung{füllt die Lücken richtig aus.}{4}
%     \end{erwartungen}
% }[\symBleistift][name=Lückentext]
% \end{sourcecode}
\end{commands}

\subsection{Multiple-Choice}
Zwar ist es über das \module{Format}-Modul möglich, einzelne Kästchen zum Ankreuzen zu setzen. In der Regel sollten allerdings echte Multiple-Choice-Aufgaben vorgezogen werden, da diese besser formatiert werden können.% und sich auch direkt Lösungen angeben lassen.

\subsubsection{Befehle}
\begin{commands}
    \command{choice}[\oarg{richtig}!] Innerhalb einer \env{mcumgebung} können mit \cs{choice} die einzelnen Wahlmöglichkeiten angegeben werden.

        Falls im optionalen Parameter \cs{mcrichtig} steht, wird die Wahlmöglichkeit als richtig markiert und in Lösungsumgebungen entsprechend gesetzt.

        Ein optionales Ausrufezeichen hinter dem Befehl sorgt dafür, dass die Wahlmöglichkeit einzeln gesetzt und somit hervorgehoben wird.
    \command{mcrichtig} markiert innerhalb einer \env{mcumgebung} eine Wahlmöglichkeit als richtig.

%     %%%% Mit xsim nicht möglich
%     \command{aufgabeMC}[\oarg{Punkte}\marg{Auswahlmöglichkeiten} \oarg{Spaltenzahl}\marg{Extras}\oarg{Symbol}
%         \oarg{Optionen für die Aufgabenumgebung}]
%         Genau wie bei Lückentexten müsste die jeweilige \env{mcumgebung}  zweimal im Dokument stehen, um automatisch Lösungen zu generieren: einmal in der Aufgaben- und einmal in der Lösungsumgebung. Um dies einfacher zu gestalten, wurde als Abkürzung dieser Befehl eingeführt, der intern eine entsprechende Aufgaben- und Lösungsumgebung erzeugt.
%
%         Der Parameter \enquote{Extras} dient dem Anhängen von Inhalten an den Aufgabentext und kann somit etwa für Erwartungen und Hinweise genutzt werden.
% \begin{sourcecode}
%   \aufgabeMC[4]{
%     \choice Erstens
%     \choice Zweitens
%     \choice Drittens
%     \choice[\mcrichtig] Viertens
%     \choice! Fünftens
%     \choice[\mcrichtig] Sechstens
%     \choice Siebtens
%     \choice[\mcrichtig] Achtens
%   }[4]{
%     \begin{erwartungen}
%         \erwartung{kreuzt alles richtig an.}{4}
%     \end{erwartungen}
%   }[\symHaken]
% \end{sourcecode}
\end{commands}

\subsection{Umgebungen}
    \begin{environments}
        \environment{mcumgebung}[\darg{Spaltenzahl}] ermöglicht es Multiple-Choice-Aufgaben zu setzen.
    \end{environments}

\subsection{Bearbeitungshinweise}
Die Bearbeitungshinweise sind dazu gedacht, dass man den Lernenden Tipps zu den Aufgaben mitgibt. Dieses ist z.\,B. bei der Bearbeitung von Leitprogrammen (siehe \ref{typ:Leitprogramm}) der Fall. Dabei ist es angedacht, diese nicht direkt bei den Aufgaben stehen zu haben, sondern an einer anderen Stelle, damit sie nur bei Bedarf genutzt werden.

\subsubsection{Umgebungen}
\begin{environments}
    \environment{bearbeitungshinweis} erlaubt es, zu einzelnen Aufgaben Hinweise anzugeben. Der Hinweis kann dabei fast beliebigen \LaTeX-Code enthalten. Verbatim-Elemente, wie z.\,B. die Verwendung von Quellcode machen an Probleme. Es kann aber \cs{lstinputlisting} genutzt werden.
\end{environments}

\subsubsection{Befehle}
\begin{commands}
    \command{bearbeitungshinweisZuAufgabe}[\oarg{Aufgabentyp}\marg{AufgabenId}]\Default{aufgabe} Setzt die Bearbeitungshinweise für die angegebene Aufgabe. Die ID ist dabei fortlaufend über alle Aufgabentypen. Der optionale Parameter erlaubt es auch für andere Aufgabentypen wie der Zusatzaufgabe mit \verbcode|aufgabe*| den Hinweis direkt auszugeben. Wird als AufgabenId nichts angegeben, so wird die aktuelle Aufgabe genommen.
    \command{bearbeitungshinweisliste} Setzt die Bearbeitungshinweise zu allen Aufgaben als Liste.
\end{commands}
    \section{Bewertung}
\label{modul:bewertung}
Das Modul \module{Bewertung} ergänzt das Modul \module{Aufgaben} um die Möglichkeiten eines Erwartungshorizonts und der Berechnung der Notenverteilung. Die Punkteangaben beim Erwartungshorizont werden auch als Punkte für die Aufgaben herangezogen und müssen so nicht doppelt angegeben werden.

\achtung{Soll in einem Dokument ein Erwartungshorizont gesetzt werden, müssen \uline{alle} Aufgaben Erwartungen enthalten!}

\subsection{Paketoptionen}
\begin{options}
    \opt{erwartungshorizontAnzeigen} hängt den Erwartungshorizont im gewählten Stil automatisch an das Dokument an, setzt vorher die Seitennummerierung und die Dokumentbezeichnung in der Kopfzeile zurück.

    Unter dem Erwartungshorizont wird automatische eine Notenverteilung gesetzt.

    \achtung{Diese Option ist nur für eigenständige Dokumente, \zB\space mit der Dokumentenklasse \cls{scrartcl} gedacht. Sie greift tief in den Übersetzungsprozess ein und ist geeignet Fehler im Zusammenspiel mit anderen Paketen zu provozieren.}

    \keychoice{erwartungshorizontStil}{einzeltabellen,simpel,standard}
        \Default{standard} legt den Stil des Erwartungshorizonts fest.
        Bisher gibt es drei verschiedene Stile:
        \begin{description}
            \item[\keyis-{erwartungshorizontStil}{einzel}\space] setzt für jede Aufgabe eine eigene Überschrift und darunter eine Tabelle mit den einzelnen Erwartungen. Unter die Erwartungen aller Aufgaben wird mit \cs{punktuebersicht} eine Übersicht über die erreichten Punkte gesetzt.

                \includepdfpage[page=3, scale=0.3, trim={3cm, 12.5cm, 3cm, 1.5cm}, clip] {minimal-kl-et}

            \item[\keyis-{erwartungshorizontStil}{simpel}\space] setzt einen Bewertungsbogen ohne Punkte mit drei Smiley-Feldern zum Ankreuzen. Die Notenverteilung wird hier ebenfalls nicht gesetzt.

                \includepdfpage[page=3, scale=0.3, trim={3cm, 22cm, 3cm, 1.5cm}, clip]{minimal-ka}

            \item[\keyis-{erwartungshorizontStil}{standard}\space] setzt einen klassischen Erwartungshorizont in einer zusammenhängenden Tabelle. Die Umgebung ist \env{longtable}, die Tabelle bricht also bei längeren Erwartungshorizonten auf die nächste Seite um.

                \includepdfpage[page=3, scale=0.3, trim={3cm, 18cm, 3cm, 1.5cm}, clip]{minimal-kl}
        \end{description}

    \opt{kmkPunkte} schaltet alle benotungsrelevanten Funktionen vom normalen Notensystem (\textit{ungenügend} bis \textit{sehr gut}) auf KMK-Notenpunkte ($0$ bis $15$) um.

    \keyval{notenschema}{15=.95,\dots} gibt ein Notenschema für die Berechnung der Notenverteilung an. Es muss eine Liste mit der Zuordnung von Notenpunkten zu Prozentwerten übergeben werden. Die Prozentwerte geben dabei jeweils die untere Grenze für die jeweilige Note an.

    Das Standardnotenschema ist \texttt{15 = .95, 14 = .9, 13 = .85, 12 = .8, 11 = .75, 10 = .7, 9 = .65, 8 = .6, 7 = .55, 6 = .5, 5 = .45, 4 = .39, 3 = .33, 2 = .27, 1 = .2}
\end{options}

\subsubsection{Umgebungen}
\begin{environments}
    \environment{erwartungen} erlaubt es, zu einzelnen Aufgaben Erwartungen anzugeben. Die einzelnen Erwartungen werden dabei mit dem Makro \cs{erwartung} angegeben.
\end{environments}

\subsubsection{Befehle}
\begin{commands}
    \command{erwartung}[\marg{Erwartung}\marg{Punkte}\oarg{Zusatzpunkte}] definiert eine einzelne Erwartung innerhalb der Umgebung \env{erwartungen}. Der Parameter kann beliebigen \LaTeX-Code enthalten bis auf Verbatim-Elemente. Des weiteren werden die Punkte für diese Erwartung als Parameter erwartet. Als optionalen Parameter können Zusatzpunkte angegeben werden.
    \command{erwartungshorizont} setzt den Erwartungshorizont im gewählten Stil, falls die automatische Erzeugung über die Paketoption \option{erwartungshorizontAnzeigen} nicht genutzt wird.
    \command{notenverteilung} setzt die Notenverteilung, falls die automatische Erzeugung über den Erwartungshorizont nicht genutzt wird. Die Verteilung wird über die Gesamtpunkte aller Aufgaben unter Berücksichtigung des gewählten Notenschemas ermittelt.
\end{commands}

    \section{Format}
\label{modul:format}
Dieses Modul definiert einige grundlegende Paketoptionen für die Formatierung von Dokumenten und stellt passende Makros bereit. Außerdem bindet es das Paket \pkg{ulem} für verschiedene Textformatierungen ein.

\subsection{Formatierungen}
    Über verschiedene Paketoptionen kann das Aussehen der vom Schule-Paket erstellten Dokumente beeinflusst werden. Es sind zudem einige Makros vorhanden, die häufig verwendete Formatierungen und Sonderzeichen bereitstellen.

\subsubsection{Paketoptionen}
\begin{options}
    \opt{farbig} aktiviert die farbige Darstellung.
    \opt{sprache} fügt eine Liste von CSV Sprachen dem Babelpaket hinzu. \texttt{ngerman} ist immer geladen (als Hauptsprache)
\end{options}

\subsubsection{Befehle}
\begin{commands}
    \command{achtung}[\marg{Text}] Der Befehl \cs{achtung} stellt den angegebenen Text mit einem vorangestellte Warnsymbol und einem fettgedruckten \enquote{Achtung:} dar.
\begin{sidebyside}[gobble=4]
    \achtung{Dies ist ein Beispiel.}
\end{sidebyside}

    \command{chb}[\sarg] setzt eine ankreuzbares Kästchen, der optionale Stern markiert dieses.
\begin{sidebyside}[gobble=4]
    \chb \chb*
\end{sidebyside}

    \command{dashuline}[\marg{Text}] Der Befehl \cs{dashuline} stellt den angegebenen Text unterstrichelt dar.
\begin{sidebyside}[gobble=4]
    \dashuline{Dies ist ein Beispiel.}
\end{sidebyside}

    \command{dotuline}[\marg{Text}] Der Befehl \cs{dotuline} stellt den angegebenen Text unterpunktet dar.
\begin{sidebyside}[gobble=4]
    \dotuline{Dies ist ein Beispiel.}
\end{sidebyside}

    \command{hinweis}[\marg{Text}] Der Befehl \cs{hinweis} stellt den angegebenen Text mit einem vorangestellte Warnsymbol und einem fettgedruckten \enquote{Hinweis:} dar.
\begin{sidebyside}[gobble=4]
    \hinweis{Dies ist ein Beispiel.}
\end{sidebyside}

    \command{person}[\marg{Name}] Der Name einer Person wird mit dem Befehl \cs{person}\marg{Name} hervorgehoben.
\begin{sidebyside}[gobble=4]
    \person{Einstein}
\end{sidebyside}

    \command{so}[\marg{Text}] Der Befehl \cs{so} stellt den angegebenen Text durchgestrichen dar und ermöglicht es so in Wertetabellen bzw. Schreibtischtests einzelne Werte durchzustreichen.
\begin{sidebyside}[gobble=4]
    \so{Dies ist ein Beispiel.}
\end{sidebyside}

    \command{uline}[\marg{Text}] Der Befehl \cs{uline} stellt den angegebenen Text unterstrichen dar.
\begin{sidebyside}[gobble=4]
    \uline{Dies ist ein Beispiel.}
\end{sidebyside}

    \command{uuline}[\marg{Text}] Der Befehl \cs{uuline} stellt den angegebenen Text doppelt unterstrichen dar.
\begin{sidebyside}[gobble=4]
    \uuline{Dies ist ein Beispiel.}
\end{sidebyside}

    \command{uwave}[\marg{Text}] Der Befehl \cs{uwave} stellt den angegebenen Text unterschlängelt dar.
\begin{sidebyside}[gobble=4]
    \uwave{Dies ist ein Beispiel.}
\end{sidebyside}

    \command{xout}[\marg{Text}] Der Befehl \cs{xout} stellt den angegebenen Text durchgekreuzt dar.
\begin{sidebyside}[gobble=4]
    \xout{Dies ist ein Beispiel.}
\end{sidebyside}

\end{commands}

\subsection{Kopf- Fußzeilen}
    Das Modul stellt einige Standardformatierungen für Kopf- und Fußzeilen bereit, die in den Dokumenttypen verwendet werden.

\subsubsection{Paketoptionen}
\begin{options}
    \opt{datumAnzeigen} aktiviert die Darstellung des Datums in der Kopfzeile.
    \opt{namensfeldAnzeigen} aktiviert die Darstellung eines Namensfelds in der Kopfzeile.
\end{options}

\subsubsection{Befehle}
\begin{commands}
    \command{schule@kopfUmbruch} setzt einen Umbruch, wenn die Kopfzeile durch eine gesetzte Option mehrzeilig wird. Kann in Kopfzeilen verwendet werden, um sie gleichmäßig auszurichten.
    \command{schule@kopfInnen} setzt eine Kopfzeile mit der vollständigen Lerngruppenbezeichnung (Fach, Lerngruppe) und je nach Paketoption einem Namensfeld in der zweiten Zeile. Ein etwaiger Umbruch der anderen Kopfzeilenfelder wird berücksichtigt.
    \command{schule@kopfMitte} setzt eine Kopfzeile mit dem Titel des Dokuments. Ein etwaiger Umbruch der anderen Kopfzeilenfelder wird berücksichtigt.
    \command{schule@kopfAussen} setzt eine Kopfzeile mit dem gegebenen Parameter, üblicherweise dem Dokumenttypbezeichner und je nach Paketoption einem Datumsfeld in der zweiten Zeile. Ein etwaiger Umbruch der anderen Kopfzeilenfelder wird berücksichtigt.
\end{commands}

\subsection{Seitenzahlen}
    Die Darstellung von Seitenzahlen in Dokumenten kann ebenfalls beeinflusst werden.

\subsubsection{Paketoptionen}
\begin{options}
    \keychoice{seitenzahlen}{auto,autoGesamt,immer,immerGesamt,keine} \Default{autoGesamt}
        legt die Art der Darstellung von Seitenzahlen fest. Die verschiedenen Varianten sind davon abhängig, ob es sich um ein einseitiges oder mehrseitiges Dokument handelt:

        \begin{tabular}{lcc}
            \toprule
            \textbf{Variante}& \textbf{einseitig} & \textbf{mehrseitig} \\
            \midrule
            auto &  & $1$ \\
            autoGesamt &  & $1$ von $n$ \\
            immer & $1$ & $1$ \\
            immerGesamt & $1$ von $1$ &  $1$ von $n$ \\
            keine &  &  \\
            \bottomrule
        \end{tabular}
\end{options}

\subsubsection{Befehle}
\begin{commands}
     \command{Seitenzahlen} setzt die Seitenzahlen gemäß der über die Paketoption \option{seitenzahlen} gewählten Variante. Dieser Befehl kann in Kopf- oder Fußzeilen verwendet werden.
\end{commands}


\subsection{Strukturelemente}
    Verschiedene, häufig verwendete Strukturelemente gehören ebenfalls zum Umfang des Pakets. Darunter sind verschiedene Listen und Platzhalter.

\subsubsection{Umgebungen}
\begin{environments}%
    \environment{smalldescription} Die Listenumgebung \env{smalldescription} ist identisch zur \env*{description}-Standardumgebungen von \LaTeX, bis auf die Tatsache, dass zwischen den einzelnen Punkten der Abstand verkleinert wurde.
    \environment{smallenumerate} Die Listenumgebung \env{smallenumerate} ist identisch zur \env*{enumerate}-Standardumgebungen von \LaTeX, bis auf die Tatsache, dass zwischen den einzelnen Punkten der Abstand verkleinert wurde.
    \environment{smallitemize} Die Listenumgebung \env{smallitemize} ist identisch zur \env*{itemize}-Standardumgebungen von \LaTeX, bis auf die Tatsache, dass zwischen den einzelnen Punkten der Abstand verkleinert wurde.

    Der Unterschied wird besonders dann deutlich, wenn man die Umgebungen nebeneinander sieht:
\begin{example}[gobble=4]
    \begin{minipage}[t]{.4\textwidth}
        \texttt{itemize}-Umgebung:
        \begin{itemize}
            \item Punkt
            \item Punkt
            \item Punkt
        \end{itemize}
    \end{minipage}
    \begin{minipage}[t]{.4\textwidth}
        \texttt{smallitemize}-Umgebung:
        \begin{smallitemize}
            \item Punkt
            \item Punkt
            \item Punkt
        \end{smallitemize}
    \end{minipage}
\end{example}

\end{environments}


\subsection{Wörtliche Rede, Zitate und Anführungszeichen}
\subsubsection{Paketoptionen}
\begin{options}
    \keychoice{zitate}{guillemets,quotes,swiss} \Default{guillemets} Standardmäßig werden die deutschen \enquote{Möwchen}
    geladen. Über \verb|quotes| können doppelte \glqq Hochkommata\grqq\ (99-66) geladen werden. Die Darstellung
    von doppelten Hochkommata im ``Modus 66-99'' kann mittels \verb|swiss| erreicht werden.
\end{options}

\subsubsection{Befehle}
\begin{commands}
    \command{enquote}[\marg{Text}] Setzen von Passagen in typographische Anführungszeichen.
\begin{sidebyside}[gobble=4]
    \enquote{Beispiel}
\end{sidebyside}

    \command{diastring}[\marg{Zeichenkette}] Darstellung von Zeichenketten (strings) in Diagrammen usw.
\begin{sidebyside}[gobble=4]
    \diastring{Beispiel}
\end{sidebyside}
\end{commands}

    \textbf{Hinweis:} Teilweise kann es zu Fehlern kommen, wenn das Paket \pkg{csquotes} mit eigenen Optionen geladen wird.

    \section{Kuerzel}
\label{modul:kuerzel}
Das Modul \module{Kuerzel} stellt einige Makros bereit, die
Kurzschreibweisen für häufig verwendete Schreibweisen
bereitstellen. Sofern relevant, wird dabei die Schreibweise an
die gewählte Variante des Genderings, im Sinne einer
geschlechtergerechten Sprache angepasst.

\subsection{Paketoptionen}\begin{options}
	\keychoice{gendering}{binneni,fem,gap,mas,split,star}\Default{split}
	Standardmäßig wird die amtlich geforderte Schreibweise des Splittings (etwa Schülerinnen und Schüler) verwendet. 
	\begin{smallitemize}
		\item\hspace{1ex} Splitting: \keyis{gendering}{split}
	\end{smallitemize}
			
	Außerdem werden folgende Varianten unterstützt:
	\begin{smallitemize}
		\item\hspace{1ex} Gender-Gap: \keyis{gendering}{gap}
		\item\hspace{1ex} Gender-Star: \keyis{gendering}{star}
		\item\hspace{1ex} Binnen-I: \keyis{gendering}{binneni}
	\end{smallitemize}
			
	Für spezielle Fälle, kann auch die ausschließliche Nutzung einer Geschlechtsform erzwungen werden:
	\begin{smallitemize}
		\item \hspace{1ex}Generisches Femininum: \keyis{gendering}{fem}
		\item \hspace{1ex}Generisches Maskulinum: \keyis{gendering}{mas}
	\end{smallitemize}
	
\end{options}
\subsection{Befehle}\hfill

\begin{multicols}{2}
\begin{commands}
	\command{Lkr}
	Lehr\-kraft
	\command{Lkre}
	Lehr\-kräf\-te
	\command{Lpr}
	Lehr\-per\-son
	\command{Lprn}
	Lehr\-per\-so\-nen
	\command{EuE}
	El\-tern und Er\-zie\-hungs\-be\-recht\-ig\-te
	\command{EuEn}
	El\-tern und Er\-zie\-hungs\-be\-recht\-ig\-ten
	\command{EK}
	Er\-wei\-ter\-ungs\-kurs
	\command{EKe}
	Er\-wei\-ter\-ungs\-kurse
	\command{EKen}Er\-wei\-ter\-ungs\-kursen
	\command{GK}
	Grund\-kurs
	\command{GKe}
	Grund\-kurse
	\command{GKen}
	Grund\-kursen
	\command{LK}
	Leis\-tungs\-kurs
	\command{LKe}
	Leis\-tungs\-kurse
	\command{LKen}
	Leis\-tungs\-kursen
	\command{SuS}
{\tiny 			\begin{tabular}{ll}
		\toprule \textbf{Gendering} & \textbf{Ergebnis} \\ 
		\midrule binneni & SchülerInnen \\ 
		fem & Schülerinnen \\ 
		gap & Schüler\_innen \\ 
		mas & Schüler \\ 
		split & Schülerinnen und Schüler \\ 
		star & Schüler*innen \\ 
		\bottomrule 
	\end{tabular} }
	\command{SuSn}
	{\tiny \begin{tabular}{ll}
		\toprule \textbf{Gendering} & \textbf{Ergebnis} \\ 
		\midrule binneni & SchülerInnen \\ 
		fem & Schülerinnen \\ 
		gap & Schüler\_innen \\ 
		mas & Schülern \\ 
		split & Schülerinnen und Schülern \\ 
		star & Schüler*innen \\ 
		\bottomrule 
	\end{tabular} }
	
	\command{LuL}
	{\tiny \begin{tabular}{ll}
		\toprule \textbf{Gendering} & \textbf{Ergebnis} \\ 
		\midrule binneni & LehrerInnen \\ 
		fem & Lehrerinnen \\ 
		gap & Lehrer\_innen \\ 
		mas & Lehrer \\ 
		split & Lehrerinnen und Lehrer \\ 
		star & Lehrer*innen \\ 
		\bottomrule 
	\end{tabular} }
	
	\command{LuLn}
	{\tiny \begin{tabular}{ll}
		\toprule \textbf{Gendering} & \textbf{Ergebnis} \\ 
		\midrule binneni & LehrerInnen \\ 
		fem & Lehrerinnen \\ 
		gap & Lehrer\_innen \\ 
		mas & Lehrern \\ 
		split & Lehrerinnen und Lehrern \\ 
		star & Lehrer*innen \\ 
		\bottomrule 
	\end{tabular} }
	
	\command{KuK}
	{\tiny \begin{tabular}{ll}
		\toprule \textbf{Gendering} & \textbf{Ergebnis} \\ 
		\midrule binneni & KollegInnen \\ 
		fem & Kolleginnen \\ 
		gap & Kolleg\_innen \\ 
		mas & Kollegen \\ 
		split & Kolleginnen und Kollegen \\ 
		star & Kolleg*en*innen \\ 
		\bottomrule 
	\end{tabular} }
	
\end{commands}
\end{multicols}
    \section{Lizenzen}
\label{modul:lizenzen}
Dieses Modul definiert die Paketoptionen zum Festlegen der Lizenz des
Dokuments und bietet Makros zum Setzen des Lizenznamens und der
Lizenzsymbole an.

Außerdem werden automatisch passende XMP-Dateien in die PDF-Datei
eingebunden.

\subsection{Paketoptionen}
\begin{options}
	\keyval{lizenz}{Lizenzcode}\Default{cc-by-nc-sa-4}
		legt die Lizenz für das Dokument fest. Aktuell werden
		folgende Codes unterstützt: \begin{smallitemize}
			\item cc-by-4
			\item cc-by-sa-4
			\item cc-by-nc-sa-4
		\end{smallitemize}
\end{options}

\subsection{Befehle}
\begin{commands}
	\command{lizenzName}
		gibt den vollständigen Namen der Lizenz des Dokuments zurück.
	\command{lizenzNameKurz}
		gibt den gekürzten Namen der Lizenz des Dokuments zurück.
	\command{lizenzSymbol}
		setzt das Symbol der Lizenz des Dokuments.

\end{commands}
    \section{Metadaten}
\label{modul:metadaten}
Dieses Modul definiert die Paketoptionen zum Setzen bestimmter Metadaten 
und bietet Makros zum Zugriff darauf an.

Im Gegensatz zu älteren Versionen des Schule-Paketes werden für Metadaten
immer die Standardmakros von \LaTeX\space eingesetzt, soweit dies
möglich ist. Dies gilt etwa für Autor (\cs{author}\marg{Autor}), Datum 
(\cs{date}\marg{Datum}) und Titel (\cs{titel}\marg{Titel}) des 
Dokumentes. Es werden allerdings auch für diese Metadaten Makros zum
 einfachen Zugriff aus dem Dokument heraus definiert.

\subsection{Paketoptionen}
\begin{options}
	\keyval{fach}{Fach}
		legt das Fach für das Dokument fest, siehe \prettyref{sec:faecher}.
	\keyval{lerngruppe}{Lerngruppe}
		legt die Lerngruppe für das Dokument fest.
	\keyval{nummer}{Dokumentnummer}
		legt die Dokumentnummer fest.
\end{options}

\subsection{Befehle}
\begin{commands}
	\command{Autor}
		gibt den Autor des Dokuments zurück.
	\command{Datum}
		gibt das Datum des Dokuments zurück.
	\command{Fach}
		gibt das Fach des Dokuments zurück.
	\command{Lerngruppe}
		gibt die Lerngruppe des Dokuments zurück. Als Alternative ist auch \cs{Kurs} definiert.
	\command{Nummer}
		gibt die Nummer des Dokuments zurück.
	\command{Titel}
		gibt den Titel des Dokuments zurück.	
\end{commands}
    \section{Papiertypen}
\label{modul:papiertypen}
Das Modul \module{Papiertypen} stellt einige Makros bereit, die
es erlauben, Freiräume zum Bearbeiten von Aufgaben zu setzen. Hierzu
stehen verschiedene Muster zur Auswahl. Die entsprechenden Felder werden
dabei in der Breite jeweils auf \verbcode|\linewidth| skaliert, 
allerdings so, dass ein vollständiges Muster entsteht. Die zur Verfügung
stehende Breite wird also optimal genutzt.

\subsection{Befehle}
\begin{commands}
	\command{feldLin}[\oarg{Abstand}\marg{Anzahl}]
		setzt die angegebene Anzahl Linien mit dem angegebenen Abstand zueinander. Der Standardabstand beträgt $1cm$.
\begin{example}
  \feldLin[1cm]{4}
\end{example}

	\command{feldKar}[\oarg{Seitenlänge}\marg{Anzahl}]
		setzt die angegebene Anzahl von Karo-Kästchen mit einer
		gegebenen Seitenlänge. Der Standard für die Seitenlänge beträgt
		$0,5cm$.
\begin{example}
  \feldKar[0.5cm]{5}
\end{example}

	\command{feldMil}[\marg{Anzahl}]
		setzt die angegebene Anzahl von Kästchen im
		Millimeterpapiermuster untereinander. Die Farbe wird von der
		Paketoption \option{farbig} beeinflusst.
\begin{example}
  \feldMil{2}
\end{example}
\end{commands}
    \section{Symbole}
\label{modul:symbole}
Dieses Modul stellt einige für den Schulkontext relevante Unicode-Symbole aus den vom Paket \pkg{utfsym} unterstützten Blöcken als benannte Makros zur Verfügung.

\subsection{Befehle}
\begin{multicols}{4}

\paragraph{Körperteile}
\begin{commands}
	\command{symNase}
		{\Huge\usym{1F443}} (1F443)
	\command{symAuge}
		{\Huge\usym{1F441}} (1F441)
	\command{symAugen}
		{\Huge\usym{1F440}} (1F440)
	\command{symMund}
		{\Huge\usym{1F444}} (1F444)
	\command{symZunge}
		{\Huge\usym{1F445}} (1F445)
	\command{symOhr}
		{\Huge\usym{1F442}} (1F442)
	\command{symDaumenHoch}
		{\Huge\usym{1F44D}} (1F44D)
	\command{symDaumenRunter}
		{\Huge\usym{1F44E}} (1F44E)
	\command{symZeigefinger}
		{\Huge\usym{1F446}} (1F446)
	\command{symApplaus}
		{\Huge\usym{1F44F}} (1F44F)
	
\end{commands}

\paragraph{Kommunikation}
\begin{commands}
	\command{symSprechblase}
		{\Huge\usym{1F5E9}} (1F5E9)
	\command{symZweiSprechblasen}
		{\Huge\usym{1F5EA}} (1F5EA)
	\command{symDreiSprechblasen}
		{\Huge\usym{1F5EB}} (1F5EB)
	\command{symDenkblase}
		{\Huge\usym{1F5ED}} (1F5ED)
	
\end{commands}

\paragraph{Kunst}
\begin{commands}
	\command{symPalette}
		{\Huge\usym{1F3A8}} (1F3A8)
	
\end{commands}

\paragraph{Material}
\begin{commands}
	\command{symBleistift}
		{\Huge\usym{1F589}} (1F589)
	\command{symFueller}
		{\Huge\usym{1F58B}} (1F58B)
	\command{symKuli}
		{\Huge\usym{1F58A}} (1F58A)
	\command{symBuntstift}
		{\Huge\usym{1F58D}} (1F58D)
	\command{symLineal}
		{\Huge\usym{1F4CF}} (1F4CF)
	\command{symGeodreieck}
		{\Huge\usym{1F4D0}} (1F4D0)
	\command{symBueroklammer}
		{\Huge\usym{1F4CE}} (1F4CE)
	\command{symBueroklammern}
		{\Huge\usym{1F587}} (1F587)
	\command{symPin}
		{\Huge\usym{1F4CC}} (1F4CC)
	\command{symNadel}
		{\Huge\usym{1F4CD}} (1F4CD)
	\command{symPinsel}
		{\Huge\usym{1F58C}} (1F58C)
	\command{symBuch}
		{\Huge\usym{1F56E}} (1F56E)
	\command{symBild}
		{\Huge\usym{1F5BC}} (1F5BC)
	\command{symMikroskop}
		{\Huge\usym{1F52C}} (1F52C)
	\command{symHeft}
		{\Huge\usym{1F4D3}} (1F4D3)
	\command{symBuecher}
		{\Huge\usym{1F4DA}} (1F4DA)
	\command{symKlemmbrett}
		{\Huge\usym{1F4CB}} (1F4CB)
	\command{symCD}
		{\Huge\usym{1F4BF}} (1F4BF)
	\command{symZeitung}
		{\Huge\usym{1F4F0}} (1F4F0)
	\command{symThermometer}
		{\Huge\usym{1F321}} (1F321)
	\command{symSchere}
		{\Huge\usym{2700}} (2700)
	\command{symSchloss}
		{\Huge\usym{1F512}} (1F512)
	\command{symSchlossOffen}
		{\Huge\usym{1F513}} (1F513)
	\command{symSchluessel}
		{\Huge\usym{1F511}} (1F511)
	\command{symGlocke}
		{\Huge\usym{1F514}} (1F514)
	\command{symKeineGlocke}
		{\Huge\usym{1F515}} (1F515)
	\command{symLupe}
		{\Huge\usym{1F5FD}} (1F5FD)
	
\end{commands}

\paragraph{Musik}
\begin{commands}
	\command{symNote}
		{\Huge\usym{1F39C}} (1F39C)
	\command{symNoten}
		{\Huge\usym{1F3B6}} (1F3B6)
	
\end{commands}

\paragraph{Smileys}
\begin{commands}
	\command{symSmileyLachend}
		{\Huge\usym{1F642}} (1F642) 
	\command{symSmileyNeutral}
		{\Huge\usym{1F610}} (1F610) 
	\command{symSmileyTraurig}
		{\Huge\usym{1F641}} (1F641) 
	\command{symSmileyGrinsend}
		{\Huge\usym{1F600}} (1F600) 
	\command{symSmileySchlafend}
		{\Huge\usym{1F614}} (1F614) 
	\command{symSmileyZwinkernd}
		{\Huge\usym{1F609}} (1F609) 
	
	
\end{commands}

\paragraph{Sonstiges}
\begin{commands}
	\command{symKlee}
		{\Huge\usym{1F340}} (1F340)
	\command{symSonne}
		{\Huge\usym{1F323}} (1F323)
	\command{symMond}
		{\Huge\usym{1F319}} (1F319)
	\command{symStern}
		{\Huge\usym{1F31F}} (1F31F)
	\command{symUhr}
		{\Huge\usym{1F551}} (1F551)
	\command{symHaken}
		{\Huge\usym{1F5F8}} (1F5F8)
	
	
\end{commands}

\paragraph{Spielkarten}
\begin{commands}
	\command{symSpielkarte}
		{\Huge\usym{1F0A0}} (1F0A0)
	\command{symPik}
		{\Huge\usym{2660}} (2660)
	\command{symHerz}
		{\Huge\usym{2665}} (2665)
	\command{symKaro}
		{\Huge\usym{2666}} (2666)
	\command{symKreuz}
		{\Huge\usym{2663}} (2663)
	
	\command{symPikAss}
		{\Huge\usym{1F0A1}} (1F0A1) 
	\command{symPikZwei}
		{\Huge\usym{1F0A2}} (1F0A2) 
	\command{symPikDrei}
		{\Huge\usym{1F0A3}} (1F0A3) 
	\command{symPikVier}
		{\Huge\usym{1F0A4}} (1F0A4) 
	\command{symPikFuenf}
		{\Huge\usym{1F0A5}} (1F0A5) 
	\command{symPikSechs}
		{\Huge\usym{1F0A6}} (1F0A6)
	\command{symPikSieben}
		{\Huge\usym{1F0A7}} (1F0A7) 
	\command{symPikAcht}
		{\Huge\usym{1F0A8}} (1F0A8) 
	\command{symPikNeun}
		{\Huge\usym{1F0A9}} (1F0A9) 
	\command{symPikZehn}
		{\Huge\usym{1F0AA}} (1F0AA) 
	\command{symPikBube}
		{\Huge\usym{1F0AB}} (1F0AB) 
	\command{symPikDame}
		{\Huge\usym{1F0AD}} (1F0AD) 
	\command{symPikKoenig}
		{\Huge\usym{1F0AE}} (1F0AE) 
	
	\command{symHerzAss}
		{\Huge\usym{1F0B1}} (1F0B1) 
	\command{symHerzZwei}
		{\Huge\usym{1F0B2}} (1F0B2) 
	\command{symHerzDrei}
		{\Huge\usym{1F0B3}} (1F0B3) 
	\command{symHerzVier}
		{\Huge\usym{1F0B4}} (1F0B4) 
	\command{symHerzFuenf}
		{\Huge\usym{1F0B5}} (1F0B5) 
	\command{symHerzSechs}
		{\Huge\usym{1F0B6}} (1F0B6)
	\command{symHerzSieben}
		{\Huge\usym{1F0B7}} (1F0B7) 
	\command{symHerzAcht}
		{\Huge\usym{1F0B8}} (1F0B8) 
	\command{symHerzNeun}
		{\Huge\usym{1F0B9}} (1F0B9) 
	\command{symHerzZehn}
		{\Huge\usym{1F0BA}} (1F0BA) 
	\command{symHerzBube}
		{\Huge\usym{1F0BB}} (1F0BB) 
	\command{symHerzDame}
		{\Huge\usym{1F0BD}} (1F0BD) 
	\command{symHerzKoenig}
		{\Huge\usym{1F0BE}} (1F0BE) 
	
	\command{symKaroAss}
		{\Huge\usym{1F0C1}} (1F0C1) 
	\command{symKaroZwei}
		{\Huge\usym{1F0C2}} (1F0C2) 
	\command{symKaroDrei}
		{\Huge\usym{1F0C3}} (1F0C3) 
	\command{symKaroVier}
		{\Huge\usym{1F0C4}} (1F0C4) 
	\command{symKaroFuenf}
		{\Huge\usym{1F0C5}} (1F0C5) 
	\command{symKaroSechs}
		{\Huge\usym{1F0C6}} (1F0C6)
	\command{symKaroSieben}
		{\Huge\usym{1F0C7}} (1F0C7) 
	\command{symKaroAcht}
		{\Huge\usym{1F0C8}} (1F0C8) 
	\command{symKaroNeun}
		{\Huge\usym{1F0C9}} (1F0C9) 
	\command{symKaroZehn}
		{\Huge\usym{1F0CA}} (1F0CA) 
	\command{symKaroBube}
		{\Huge\usym{1F0CB}} (1F0CB) 
	\command{symKaroDame}
		{\Huge\usym{1F0CD}} (1F0CD) 
	\command{symKaroKoenig}
		{\Huge\usym{1F0CE}} (1F0CE) 
	
	\command{symKreuzAss}
		{\Huge\usym{1F0D1}} (1F0D1) 
	\command{symKreuzZwei}
		{\Huge\usym{1F0D2}} (1F0D2) 
	\command{symKreuzDrei}
		{\Huge\usym{1F0D3}} (1F0D3) 
	\command{symKreuzVier}
		{\Huge\usym{1F0D4}} (1F0D4) 
	\command{symKreuzFuenf}
		{\Huge\usym{1F0D5}} (1F0D5) 
	\command{symKreuzSechs}
		{\Huge\usym{1F0D6}} (1F0D6)
	\command{symKreuzSieben}
		{\Huge\usym{1F0D7}} (1F0D7) 
	\command{symKreuzAcht}
		{\Huge\usym{1F0D8}} (1F0D8) 
	\command{symKreuzNeun}
		{\Huge\usym{1F0D9}} (1F0D9) 
	\command{symKreuzZehn}
		{\Huge\usym{1F0DA}} (1F0DA) 
	\command{symKreuzBube}
		{\Huge\usym{1F0DB}} (1F0DB) 
	\command{symKreuzDame}
		{\Huge\usym{1F0DD}} (1F0DD) 
	\command{symKreuzKoenig}
		{\Huge\usym{1F0DE}} (1F0DE) 
	
\end{commands}

\paragraph{Sport}
\begin{commands}
	\command{symBaseball}
		{\Huge\usym{26BE}} (26BE)
	\command{symBasketball}
		{\Huge\usym{1F3C0}} (1F3C0)
	\command{symFussball}
		{\Huge\usym{26BD}} (26BD)
	\command{symVolleyball}
		{\Huge\usym{1F3D0}} (1F3D0)
	\command{symHockey}
		{\Huge\usym{1F3D1}} (1F3D1)
	
	\command{symLaufen}
		{\Huge\usym{1F3C3}} (1F3C3)
	\command{symReiten}
		{\Huge\usym{1F3C7}} (1F3C7)
	\command{symSchwimmen}
		{\Huge\usym{1F3CA}} (1F3CA)
	
	\command{symSki}
		{\Huge\usym{26F7}} (26F7)
	\command{symSnowboard}
		{\Huge\usym{1F3C2}} (1F3C2)
	
	\command{symSurfen}
		{\Huge\usym{1F3C4}} (1F3C4)
	
	\command{symTennis}
		{\Huge\usym{1F3BE}} (1F3BE)
	\command{symTischtennis}
		{\Huge\usym{1F3D3}} (1F3D3)
	
	\command{symPokal}
		{\Huge\usym{1F3C6}} (1F3C6)
	\command{symMedaille}
		{\Huge\usym{1F3C5}} (1F3C5)
	\command{symZielflagge}
		{\Huge\usym{1F3C1}} (1F3C1)
	
\end{commands}

\paragraph{Technik}
\begin{commands}
	\command{symHandy}
		{\Huge\usym{1F4F1}} (1F4F1)
	\command{symKeinHandy}
		{\Huge\usym{1F4F5}} (1F4F5)
	
\end{commands}

\paragraph{Theater}
\begin{commands}
	\command{symTheater}
		{\Huge\usym{1F0DD}} (1F0DD)
	
\end{commands}

\paragraph{Verkehrsmittel}
\begin{commands}
	\command{symAuto}
		{\Huge\usym{1F698}} (1F698)
	\command{symBus}
		{\Huge\usym{1F68C}} (1F68C)
	\command{symBahn}
		{\Huge\usym{1F682}} (1F682)
	\command{symStrassenbahn}
		{\Huge\usym{1F68B}} (1F68B)
	\command{symSchwebebahn}
		{\Huge\usym{1F69F}} (1F69F)
	\command{symSeilbahn}
		{\Huge\usym{1F6A1}} (1F6A1)
	\command{symSchiff}
		{\Huge\usym{1F6A2}} (1F6A2)
	\command{symBoot}
		{\Huge\usym{1F6A3}} (1F6A3)
	\command{symFahrrad}
		{\Huge\usym{1F6B2}} (1F6B2)
	\command{symFussgaenger}
		{\Huge\usym{1F6B8}} (1F6B8)
	\command{symRollstuhl}
		{\Huge\usym{267F}} (267F)
	
	
\end{commands}

\paragraph{Würfel}
\begin{commands}
	\command{symWuerfelEins}
		{\Huge\usym{2680}} (2680)
	\command{symWuerfelZwei}
		{\Huge\usym{2681}} (2681)
	\command{symWuerfelDrei}
		{\Huge\usym{2682}} (2682)
	\command{symWuerfelVier}
		{\Huge\usym{2683}} (2683)
	\command{symWuerfelFuenf}
		{\Huge\usym{2684}} (2684)
	\command{symWuerfelSechs}
		{\Huge\usym{2685}} (2685)	
\end{commands}
\end{multicols}
    \section{Texte}
\label{modul:texte}
Dieses Modul definiert einige Umgebungen, die für die Formatierung
längerer Texte hilfreich sind. 

\achtung{Da die Umgebungen mit Zeilennummern
nicht ohne große Klimmzüge in umrahmte Boxen gesetzt werden können,
sehen die Beispiele hier ein wenig anders aus.}

\subsection{Befehle}
\begin{commands}
	\command{resetZeilenNr}
		Standardmäßig werden die Zeilennummern über die
		Umgebungsgrenzen hinweg vergeben. Möchte man in jeder neuen
		Umgebung mit 1 beginnen, so muss man die Zeilennummer mit
		diesem Befehl zunächst zurücksetzen.
\end{commands}

\subsection{Umgebungen}
\begin{environments}

	\environment{mehrspaltig}[\oarg{Anzahl}]
		Setzt einen gegebenen Text mehrspaltig, wobei die Anzahl der
		Spalten angegeben werden kann. Die Standardanzahl ist 2.
\begin{example}
  \begin{mehrspaltig}
    \blindtext
  \end{mehrspaltig}
\end{example}
\resetZeilenNr
	\environment{zeilenNr}[\oarg{Modulo}]
		Setzt einen gegebenen Text mit Zeilennummern, wobei
		ein	Modulo für den Abstand der Zeilennummern angegeben werden
		kann.
		
		Der Standardmodulo beträgt 5.
\begin{example}[outside=true]
  \begin{zeilenNr}[1]
    \blindtext
  \end{zeilenNr}
\end{example}
\resetZeilenNr

	\environment{zeilenNrMehrspaltig}[\oarg{Modulo}\marg{Anzahl}]
		Setzt einen gegebenen Text mehrspaltig mit Zeilennummern, 
		wobei die Anzahl der Spalten angegeben werden muss und 
		zusätzlich ein Modulo für den Abstand der Zeilennummern
		angegeben werden kann.
		
		Die Zeilennummern stehen jeweils links neben der jeweiligen
		Spalte.	Der Standardmodulo beträgt 5.
\begin{example}[outside=true]
  \begin{zeilenNrMehrspaltig}[5]{3}
    \blindtext
  \end{zeilenNrMehrspaltig}
\end{example}
\resetZeilenNr

	\environment{zeilenNrZweispaltig}[\oarg{Modulo}]
		Setzt einen gegebenen Text zweispaltig mit Zeilennummern,
		wobei ein Modulo für den Abstand der Zeilennummern angegeben
		werden kann.
		
		Die Zeilennummern stehen links neben der linken Spalte und
		rechts neben der rechten Spalte. Der Standardmodulo beträgt 5.
		
\begin{example}[outside=true]
  \begin{zeilenNrZweispaltig}[3]
    \blindtext
  \end{zeilenNrZweispaltig}
\end{example}
\resetZeilenNr
\end{environments}