%======================================================================

\SbSSCT{Commande rput}{Macro rput}


\psset{fillcolor=yellow,fillstyle=solid,linecolor=blue}


\TFRGB{syntaxe : \BS{rput}*[point de référence]\AC{rotation}(coordonnées)\AC{contenu}}{syntax : \BS{rput}*[reference point]\AC{rotation}(coordinates)\AC{contents}}

\bigskip



\SbSbSSCTTC{Rôle de l'astérisque}{Rôle de l'astérisque \protect \footnote{La couleur de fond est en jaune et le point de référence en bleu}}{Role of the asterisk}{Role of the asterisk  \protect \footnote{Fillcolor=yellow and Reference point = blue disk}}

\begin{tabular}{|p{3cm}|p{3cm}|}\hline
\rput(1,0){objet}  &  \rput*(1,0){objet}  \\
\textbf{\BS{}rput}(1,0)\AC{objet} & \textbf{\BS{}rput*}(1,0)\AC{objet} \\ \hline
\end{tabular}
%-------------------------------------

\SbSbSSCT{Point de référence }{Reference point} % \protect \footnote{La couleur de fond est en jaune et le point de référence en bleu} 


\begin{tabular}{|c|l|p{3cm}|c|}

\hline \multicolumn{4}{|c|}{Horizontal}   								\\ \hline
 
\rule[-.35cm]{0mm}{.7cm} l	& \TFRGB{à gauche}{left}		
&  \rput*[l](1,0){objet}\qdisk(1,0){3pt} 			
&  \BS{rput*}[{\red l}](1,0)\AC{objet}\BS{qdisk}(1,0)\AC{3pt} \\ \hline

\rule[-.35cm]{0mm}{.7cm} r 	& \TFRGB{à droite}{right}		
&   \rput*[r](1,0){objet}\qdisk(1,0){3pt}				
& \BS{rput*}[{\red r}](1,0)\AC{objet}\BS{qdisk}(1,0)\AC{3pt}  \\  \hline
 \multicolumn{4}{|c|}{vertical}   								\\ \hline

\rule[-.5cm]{0mm}{.7cm} t	
& \TFRGB{en haut}{top}		
&  \rput*[t](1,0){objet}	\qdisk(1,0){3pt}		
& \BS{rput*}[{\red t}](1,0)\AC{objet}\BS{qdisk}(1,0)\AC{3pt}  \\ \hline


\rule[-.2cm]{0mm}{.7cm }b 	& \TFRGB{en bas}{bottom}		
&  \rput*[b ](1,0){objet}	 \qdisk(1,0){3pt}		&
\BS{rput*} [{\red b}](1,0)\AC{objet}\BS{qdisk}(1,0)\AC{3pt}  \\ \hline


\rule[-.25cm]{0mm}{.7cm} B 	& \TFRGB{sur la ligne d'écriture}{baseline}		& \rput*[B](1,0){objet}	 \qdisk(1,0){3pt} 			& \BS{rput*}[{\red B]}(1,0)\AC{objet}\BS{qdisk}(1,0)\AC{3pt} \\ \hline

 \multicolumn{4}{|c|}{horizontal \TFRGB{et}{and} vertical}   								\\ \hline

\rule[-.5cm]{0mm}{.7cm} rt	
& \TFRGB{à droite et en haut}{right and top}		
& \rput*[rt](1,0){objet}	\qdisk(1,0){3pt} 
& \BS{rput}*[{\red rt}](1,0)\AC{objet}\BS{qdisk}(1,0)\AC{3pt}  \\  \hline 
 \end{tabular} 

%------------------------------------------------------

\SbSbSSCT{Angle de rotation de l'objet}{Rotation angle of the object}


\begin{tabular}{|p{2cm}|p{2cm}|p{2cm}|p{2cm}|p{2cm}|p{2cm}|p{2cm}|}\hline
   \BS{rput*}[t]\AC{45} 
&  \BS{rput*}[t]\AC{90} 
&  \BS{rput*}[b]\AC{90}
&  \BS{rput*}[B]\AC{90} 
&  \BS{rput*}[l]\AC{90}
&  \BS{rput*}[r]\AC{90}  \\ 
\hline \rule[-1cm]{0mm}{2cm}  
\rput*[t]{45}(1,0){objet}\qdisk(1,0){3pt} & 
\rput*[t]{90}(1,0){objet}\qdisk(1,0){3pt} & 
\rput*[b]{90}(1,0){objet}\qdisk(1,0){3pt} & 
\rput*[B]{90}(1,0){objet}\qdisk(1,0){3pt} & 
\rput*[l]{90}(1,0){objet}\qdisk(1,0){3pt} & 
\rput*[r]{90}(1,0){objet}\qdisk(1,0){3pt}  \\ 
\hline 
\end{tabular} 
%
%----------------------------------------------------------

\SbSbSSCT{Angles de rotation  en points cardinaux}{Rotation angle in cardinal points}

\begin{tabular}{|p{2cm}|p{2cm}|p{2cm}|p{2cm}|p{2cm}|p{2cm}|p{2cm}|}
\hline  \TFRGB{haut}{top} 	&  \TFRGB{haut}{top}  	&  \TFRGB{haut}{top}  	& \TFRGB{haut}{top}   & \TFRGB{gauche}{left}   &  \TFRGB{droite}{right}  \\
\TFRGB{ et Est}{and east} 		& \TFRGB{ et Ouest}{and west} & \TFRGB{et Nord}{and north}  & \TFRGB{ et Sud}{and south} & \TFRGB{ et Est}{and east}  & \TFRGB{ et Est}{and east}  \\
\hline  
   \BS{}rput*[t]\AC{E} 
&  \BS{}rput*[t]\AC{W} 
& \BS{}rput*[t]\AC{N}  
&  \BS{}rput*[t]\AC{S}   
&  \BS{}rput*[l]\AC{W}  
&  \BS{}rput*[r]\AC{W}  \\ 
\hline \rule[-1cm]{0mm}{2cm}  \rput*[t]{E}(1,0){objet}\qdisk(1,0){3pt} & \rput*[t]{W}(1,0){objet}\qdisk(1,0){3pt} & \rput*[t]{N}(1,0){objet}\qdisk(1,0){3pt} & \rput*[t]{S}(1,0){objet}\qdisk(1,0){3pt} & \rput*[l]{W}(1,0){objet}\qdisk(1,0){3pt} & \rput*[r]{W}(1,0){objet}\qdisk(1,0){3pt}  \\ 
\hline 
\end{tabular} 



\newpage
%%---------------------------------------------------------------

\SbSSCT{Commande uput}{Macro uput}
\psset{fillcolor=yellow,fillstyle=solid,linecolor=blue}


\TFRGB{ syntaxe :\BSS{uput}*\AC{écartement}[point de référence]\AC{rotation}(coordonnées)\AC{contenu} }{syntax  : \BSS{uput}*\AC{spacing}[Reference point]\AC{rotation}(coordinates)\AC{content}}
\smallskip



\SbSbSSCTTC{Rôle de l'astérisque}{Rôle de l'astérisque \protect \footnote{La couleur de fond est en jaune et le point de référence en bleu}}{Role of the asterisk}{Role of the asterisk  \protect \footnote{Fillcolor=yellow and Reference point = blue disk}}

\begin{tabular}{|p{3cm}|p{3cm}|}\hline
 \uput[r](1,0){objet} &  \uput*[r](1,0){objet}  \\ \hline
\textbf{\BS{}uput}(1,0)\AC{objet} & \textbf{\BS{}uput*}(1,0)\AC{objet} \\ \hline
\end{tabular}
 
\SbSbSSCT{Point de référence  : angle}{Reference point : angle}

\begin{center}
\begin{tabular}{|c|p{3cm}|c|} \hline

 \rule[-1cm]{0mm}{2cm} à 45°	&  \uput*[45](1,0){objet}\qdisk(1,0){3pt} 			&  \BS{uput*}[{\red 45}](1,0)\AC{objet}\BS{qdisk}(1,0)\AC{3pt}\\ \hline

 \rule[-1cm]{0mm}{2cm} à 90° 			&   \uput*[90](1,0){objet}\qdisk(1,0){3pt}				& \BS{uput*}[{\red 90}](1,0)\AC{objet}\BS{qdisk}(1,0)\AC{3pt}\\  \hline

 \rule[-1cm]{0mm}{2cm} à 120°		&  \uput*[120](1,0){objet}	\qdisk(1,0){3pt}		& \BS{uput*}[{\red 120}](1,0)\AC{objet}\BS{qdisk}(1,0)\AC{3pt} \\ \hline
 \end{tabular} 
 \end{center}

\bigskip
%%-----------------------------------------------------------------

\SbSbSSCT{Point de référence  : points cardinaux }{Reference point : letter} 

\begin{tabular}{|p{2cm}|p{2cm}|p{2cm}|p{2cm}|p{2cm}|p{2cm}|p{2cm}|}
\hline 
  \BS{uput*}[{\red u}] 
&  \BS{uput*}[{\red r}]
&  \BS{uput*}[{\red d}]
& \BS{uput*}[{\red l}]   
& \BS{uput*}[{\red ul}]  
&  \BS{uput*}[{\red ur}]  \\ 
\hline  \rule[-1cm]{0mm}{2cm} 
  \uput*[u](1,0){objet}\qdisk(1,0){3pt} 
& \uput*[r](1,0){objet}\qdisk(1,0){3pt} 
& \uput*[d](1,0){objet}\qdisk(1,0){3pt} 
& \uput*[l](1,0){objet}\qdisk(1,0){3pt} 
& \uput*[ul](1,0){objet}\qdisk(1,0){3pt} 
& \uput*[ur](1,0){objet}\qdisk(1,0){3pt}  \\ \hline 
\end{tabular} 



\bigskip
\SbSbSSCT{Angle de rotation de l'objet}{Rotation angle of the object} 


\begin{tabular}{|p{2cm}|p{2cm}|p{2cm}|p{2cm}|p{2cm}|p{2cm}|p{2cm}|}
\hline  
  \BS{uput*}[{\red u}]\AC{{\red 45}} 
& \BS{uput*}[{\red u}]\AC{{\red 90}}
& \BS{uput*}[{\red d}]\AC{{\red 90}}
& \BS{uput*}[{\red l}]\AC{{\red 90}}
& \BS{uput*}[{\red r}]\AC{{\red 90}}  
& \BS{uput*}[{\red ur}]\AC{{\red 90}}  \\ \hline  
\rule[-1.5cm]{0mm}{3cm} 
\uput*[u]{45}(1,0){objet}\qdisk(1,0){3pt} 
& \uput*[u]{90}(1,0){objet}\qdisk(1,0){3pt} 
& \uput*[d]{90}(1,0){objet}\qdisk(1,0){3pt} 
& \uput*[l]{90}(1,0){objet}\qdisk(1,0){3pt} 
& \uput*[r]{90}(1,0){objet}\qdisk(1,0){3pt} 
& \uput*[ur]{90}(1,0){objet}\qdisk(1,0){3pt}  \\ \hline 
\end{tabular} 
%

 
\bigskip
\SbSbSSCT{\'Ecartement de l'objet par rapport au point de référence}{Spacing between object and reference point}

\dft: \RDD{labelsep}= 0.5 pt 
\bigskip

\TFRGB{Exemple}{E\emph{}xample} : 

\begin{tabular}{ll}
\BS{}psset\AC{\red labelsep=1cm } &  \% \TFRGB{nouveau écartement par défaut}{new default spacing} \\ 
\BS{}uput(1,0)\{ à 1cm \}& \%  \TFRGB{utilisation  nouveau écartement par défaut}{use of the new default spacing} \\ 
\BS{}uput {\red \AC{3cm}}(1,0)\AC{à 3cm} & \% \TFRGB{écartement spécifié à 3 cm}{spacing = 3cm}  \\ 
\BS{}uput{\red \AC{3cm}[-30]}(1,0)\AC{à 3cm et à -30°} & \% \TFRGB{écartement spécifié à 3  et à un angle de -30°}{spacing = 3cm angle= -30°} \\  
\BS{}qdisk(1,0)\AC{3pt}  & \%\TFRGB{ point de référence}{Reference point} \\ 
 
\end{tabular} 
\bigskip


{\blue

\psset{labelsep=1cm} 
\uput[0](1,0){ à 1cm }      \uput{3cm}[0](1,0){ à 3cm}  \uput{3cm}[-30](1,0){ à 3cm et à -30°}\qdisk(1,0){3pt}  }

\vspace{2cm}


\SbSSCT{Commande psrotate}{Macro psrotate}
\label{rot}
\psset{fillstyle=none                                                        }
\begin{tabular}{|c|}\hline  
\BSS{psrotate}(2,1)\AC{45}\AC{\BS{psline}(0, 1)(1, 2)(2, 2)(3, 4)}  \BSI{psrotate}{pstricks-add} 
\\ \hline  
\begin{psgraph}[axesstyle=none,xticksize=0 4cm,yticksize=0 4cm,subticks=0](0,0)(4,4){4cm}{4cm} 
\psline[linestyle=dotted](0, 1)(1, 2)(2, 2)(3, 4)
\psrotate(2,2){45}{\psline(0, 1)(1, 2)(2, 2)(3, 4)}
\qdisk(2,2){3pt}

 \end{psgraph}
\\ 
\hline 
\end{tabular} 
