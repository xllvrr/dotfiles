
%\subsection{$\backslash$multirput}
Utilisation du module \textbf{multido} \label{multido}

\psset{fillcolor=yellow,fillstyle=solid,linecolor=blue,unit=1cm}


%Syntaxe : 

%\begin{tabular}{ll}
% \BSS{multido} 				&   \\ 
%\AC{variables} 						& \% variable = valeur initial + incrément \\
%\AC{nombre de répétitions} 			& \% nombre entier\\
%\AC{élément à répéter} 				& \% on peut utiliser les variables \\
%\end{tabular}
%
%
%
%
%\bigskip
%
%Exemples : \BS{}multido\AC{\BS{}i=2+-3}\AC{10}\AC{\BS{}i, } donne : \multido{\i=2+-3}{10}{\i, }

%\multido{\i=2+1, \n=5.0-0.5}{10}{\psframe[fillsyle=none](\i,\n)}
\begin{tabular}{|c|}\hline
\begin{pspicture}(6,3.5)
%   \put(0,0){\vector(1,0){8}}
   \multido{\i=1+1, \n=3.+-0.5}{5}{\psframe[fillstyle=none](\i,\n) }
\end{pspicture}   \\  \hline 

\BSS{multido} \Rnode*{A}{\AC{\textbf{\BS{i}}=1+1,\BS{n}=3.+-0.5}} \Rnode*[fillcolor=green]{B}{\AC{5}} \AC{\BS{psframe} \Rnode*[fillcolor=cyan]{C}{(\BS{i},\BS{n})} }
 \\ \hline 
\\
\rnode{AA}{\TFRGB{variable = valeur initiale+incrément}{variable = initial value + increment}} \hspace{1cm} \rnode{BB}{5 \TFRGB{fois}{times}} \hspace{1cm} \rnode{CC}{utilisation}
 \\ \hline
\end{tabular} 
 \ncline[linecolor=blue]{A}{AA}  \ncline[linecolor=blue]{B}{BB}  \ncline[linecolor=blue]{C}{CC}
 
\bigskip

\begin{tabular}{|l|l|} \hline
 \multicolumn{2}{|c|}{ \TFRGB{Types de variables}{variables types}  } \\  \hline  
\TFRGB{initiale}{initial} & dimension \\ \hline
d ou D				&  \TFRGB{longueur}{lenght}  \\  \hline
i ou I				&  \TFRGB{nombre entier}{integer}\\ \hline
n ou N 				& \TFRGB{nombre réel (même nombre de décimales)}{real} \\ \hline
r ou R 				&  \TFRGB{Réel (4 chiffres maxima de part et d'autre)}{real}\\ \hline
\end{tabular}

