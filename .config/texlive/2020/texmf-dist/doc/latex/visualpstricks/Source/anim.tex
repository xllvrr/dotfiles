\label{anim}
\SbSSCT{Animation à partir de fichiers d'image }{Animation from picture files}

%\begin{verbatim}\animategraphics[<options>]{<frame rate>}{<file basename>}{<first>}{<last>} \end{verbatim}


\begin{tabular}{|c|c|} \hline 
\TFRGB{première image}{first frame} & \TFRGB{second et dernière image}{second and last frame}
\\ \hline
\includegraphics{XXX1.ps}
&  
\includegraphics{XXX2.ps}
\\ 
\hline \BS{includegraphics}\AC{XXX1.ps} &  \BS{includegraphics}\AC{XXX2.ps}\\ 
\hline 
\end{tabular} 



%
\begin{minipage}{7cm}
\animategraphics[controls,loop,autoplay]{4}{XXX}{1}{2}  
\end{minipage}\hfill
\begin{minipage}{7cm}
\begin{tabular}{|l@{:}l|}
\hline \BSS{animategraphics} &  \\ 
\hline [ controls, & \TFRGB{boutons de contrôle}{Inserts control buttons} \\ 
\hline loop &  \TFRGB{en boucle}{animation restarts automatically}\\ 
\hline autoplay ] & \TFRGB{auto demarrage}{Start animation automatically } \\ 
\hline \AC{4} &  \TFRGB{4 fois par seconde}{4 frame per second}\\ 
\hline \AC{XXX} & \TFRGB{base du nom fichier}{file base name} \\ 
\hline \AC{1} & \TFRGB{numero de debut}{number of the first frame} \\ 
\hline \AC{2} & \TFRGB{numeo de fin}{number of the last frame} \\ 
\hline 
\end{tabular} 

\end{minipage}

%--------------------------------------------------------------
\subsection{Animateinline}


\begin{minipage}{7cm}
\begin{center}
\begin{animateinline}[ controls,loop,autoplay]{5}%
\begin{pspicture}(6,6)
\psdiamond*[gangle=45](3,3)(2,.5)
\psdiamond*[gangle=135](3,3)(2,.5)
\end{pspicture}
\newframe%
\begin{pspicture}(6,6)
\psdiamond*[gangle=0](3,3)(2,.5)
\psdiamond*[gangle=90](3,3)(2,.5)
\end{pspicture}
\end{animateinline}% 
\end{center}
\end{minipage}\hfill
\begin{minipage}{7cm}
\BSS{begin\AC{animateinline}}[controls,loop,autoplay]\AC{5} \\

\emph{\% \TFRGB{première image}{first frame }}\\
\BS{begin}\AC{pspicture}(6,6) \\
\BS{psdiamond*}[gangle=45](3,3)(2,.5) \\
\BS{psdiamond*}[gangle=135](3,3)(2,.5) \\
\BS{end}\AC{pspicture} \\

\emph{\% \TFRGB{deuxième}{second frame }}\\
\BSS{newframe} \\
\BS{begin}\AC{pspicture}(6,6) \\
\BS{psdiamond*}[gangle=0](3,3)(2,.5) \\
\BS{psdiamond*}[gangle=90](3,3)(2,.5) \\
\BS{end}\AC{pspicture} \\
\\
\BSS{end\AC{animateinline}} \\
\end{minipage}
%
%%------------------------------------------------------------------------
%\newpage
\subsection{Multiframe}

\begin{minipage}{7cm}
\begin{center}
\begin{animateinline}[poster=first, controls, palindrome,autoplay]{12}%
\multiframe{29}{iAngle=80+10,Rdim=2.0+-0.2}{\begin{pspicture}(6,6)
 \psdiamond*[gangle=\iAngle](3,3)(\Rdim,.5)
 \rput(1,1){\iAngle}
 \rput(5,1){\Rdim}
\end{pspicture} }%
\end{animateinline}%
\end{center}
\end{minipage}\hfill
\begin{minipage}{7cm}

\BS{begin}\AC{animateinline}[poster=first,controls, palindrome]\AC{12} \\
\BSS{multiframe}\AC{29}\AC{{\red iAngle}=80+10, {\red Rdim}=2.0+-0.2}\AC{ \\
\BS{begin}\AC{pspicture}(6,6) \\
\BS{psdiamond*}[gangle={\red \BS{iAngle}}](3,3)({\red \BS{Rdim}},.5)\\
\BS{rput}(1,1)\AC{{\red \BS{iAngle}}} \\
\BS{rput}(5,1)\AC{{\red \BS{Rdim}}} \\
\BS{end}\AC{pspicture} } \\
\BS{end}\AC{animateinline}%


\end{minipage}
\bigskip

\TFRGB{L'initiale de la variable définit son type}{The first  letter of the variable name determines his type }

\begin{tabular}{|c|l|}
\hline  entier &  initiale : i ou I \\ 
\hline  réelles &  initiale : n, N, r ou R \\ 
\hline  longueurs & initiale : d ou D \\ 
\hline 
\end{tabular} 
%%---------------------------------------------------------------------------------
\newpage
\subsection{Timeline}

\begin{minipage}{6.5cm}
\begin{animateinline}[controls,autoplay,timeline=xxx.txt]{5}%
\begin{pspicture}(6,6)
\pscircle[fillcolor=yellow,fillstyle=solid](3,3){2.5}
\end{pspicture} 
\newframe%
\begin{pspicture}(6,6)
\pscircle[linecolor=red,fillcolor=green,fillstyle=solid](3,3){2.5}
\end{pspicture} 
\newframe
\multiframe{10}{iAngle=60+10}{\begin{pspicture}(6,6)
\psdiamond*[gangle=\iAngle](3,3)(2,.5)
\end{pspicture} }%
\end{animateinline}%

\end{minipage}\hfill
\begin{minipage}{8cm}

\BS{begin}\AC{animateinline}

[controls,autoplay,\RDD{timeline}=xxx.txt]\AC{5} \\

\emph{\% \TFRGB{1 image de fond}{}first background image (image \No 0)} \\
\BS{begin}\AC{pspicture}(6,6) \\
\BS{pscircle}[fillcolor=yellow,fillstyle=solid](3,3)\AC{2.5} \\
\BS{end}\AC{pspicture}  \\

\textbf{\BS{newframe}} \emph{\%  \TFRGB{2 page de fond}{second background image } (image \No 1)} \\
\BS{begin}\AC{pspicture}(6,6) \\ 
\BS{pscircle}[linecolor=red,fillcolor=green,fillstyle=solid](3,3)\AC{2.5} \\
\BS{end}\AC{pspicture} \\

\textbf{\BS{newframe}} \emph{\% \TFRGB{animation}{animation frames} (images \No 2 - 11) } \\
\BS{multiframe}\AC{10}\AC{iAngle=60+10}\AC{ \\
\BS{begin}\AC{pspicture}(6,6) \\
\BS{psdiamond*}[gangle=\BS{iAngle}](3,3)(2,.5) \\
\BS{end}\AC{pspicture}  } \\
\BS{end}\AC{animateinline} \\
\end{minipage}

\bigskip
\SbSbSSCT{Création du fichier pour timeline}{ Creation of the file for timeline}

\TFRGB{Pour créer le fichier xxx.txt , en insérant le code suivant avant}{to create the file xxx.txt, insert the following code before } \BS{begin\AC{document}}
\bigskip

\begin{minipage}{7cm}
\BSS{begin\AC{filecontents}}\AC{xxx.txt} \\
::\Rnode{A}{0x0,8} \\
::2 \\
::7 \\
::3 \\
::6 \\
::\Rnode{B}{c,1x3,5} \\ 
::4 \\
::11 \\
::5 \\
::7 \\
::9 \\
\BSS{end\AC{filecontents}}
\end{minipage}\hfill
\begin{minipage}{7cm}
\rnode{AA}{0x0 : image \No 0 \TFRGB{sert de fond tout le temps}{= background image for all frame}} \\

 \vspace{.5cm}
\Rnode{BB}{c : \TFRGB{efface les images précédentes}{clear the background image} } \\

 \vspace{.5cm}
\Rnode{CC}{1x3 : image \No 1  \TFRGB{sert de fond 3 fois}{= background image for 3  frames}} \\

 \vspace{.5cm}
 \TFRGB{Ordre de passage des images}{ Order of frames} : 8,2,7,3,6,5,4,11,5,7,9
\end{minipage}
 \ncline[linecolor=blue]{A}{AA}  \ncline[linecolor=blue]{B}{BB}  \ncline[linecolor=blue]{B}{CC}
 
\SbSbSSCT{option pour le fichier xxx.txt}{option for the file xxx.txt}
 
\begin{tabular}{|c|l|}  \hline  
* : : 3
&
\TFRGB{pause à l'image}{pause at frame } \No 3    
\\   \hline  
: 10 : 3
&  
\TFRGB{vitesse 10 par seconde à l'image}{10 frames per second at frame} \No  3
\\   \hline  
: : 3 : code 
&  
\TFRGB{code java possible à l'image}{java code at frame} \No  3
\\   \hline 
 \end{tabular}  
 
\newpage
\SbSSCT{Animation d'un graphe}{Graph animation}
\readdata{\dat}{mesdata.dat}

\begin{animateinline}[poster=last,controls]{5}%
\multiframe{70}{ifin=10+10}{
\begin{psgraph}[axesstyle=frame,xticksize=0 4cm,yticksize=0 9cm,subticks=0,Dx=100,Dy=.02](0,0)(750,.12){9cm}{4cm} 
\listplot[xEnd=\ifin,linecolor=blue,linewidth=5pt]{\dat}
\end{psgraph} }%
\end{animateinline}

\bigskip
\noindent

\BS{readdata}\AC{\BS{dat}}\AC{mesdata.dat} \\
\BS{begin}\AC{animateinline}[poster=last,controls]\AC{5} \\
\BS{multiframe}\AC{70}\AC{{\red ifin}=10+10}\AC{ \\
\BS{begin}\AC{psgraph}[axesstyle=frame,xticksize=0 4cm,yticksize=0 9cm,subticks=0,Dx=100,Dy=.02](0,0)(750,.12)\AC{9cm}\AC{4cm} \\
\BS{listplot}[xEnd={\red \BS{ifin}},linecolor=blue,linewidth=5pt]\AC{\BS{dat}} \\
\BS{end}\AC{psgraph} } \\
\BS{end}\AC{animateinline}








%
%\newpage
%\subsection{Command options}
%
%
%\begin{tabular}{|c|c|} \hline  
%poster[=first | none | last] &   \\  \hline  
%every=<num>					& Build animation from every <num>th frame only \\
% autopause 					& \TFRGB{animation en pause à la fermeture de la page}{Pause animation when the page is closed }  \\  \hline  
% autoplay 					&  Start animation after the page has opened.\\  \hline  
% autoresume 				&  Resume previously paused animation \\  \hline  
% loop 						&  The animation restarts immediately after reaching the end. \\  \hline  
% palindrome 				&  The animation continuously plays forwards and backwards. \\  \hline  
% step 						&  Step through the animation one frame at a time per mouse-click. \\  \hline  
% width=<width> 				&  \TFRGB{redimensionne la largeur }{Resize the width of the animation widget} \\  \hline  
% height=<height> 			&  \TFRGB{redimensionne la hauteur}{Resize the height of the animation widget} \\  \hline  
% depth=<depth> 				&  Resize the animation widget \\  \hline  
% scale=<factor> 			&  Scales the animation widget \\  \hline  
% bb=<llx> <lly> <urx> <ury> &   \\  \hline  
% viewport=<llx> <lly> <urx> <ury> &   \\  \hline  
% trim=<left> <bottom> <right> <top> &   \\  \hline  
% controls 					&  Inserts control buttons \\  \hline  
% buttonsize=<size> 			&  Changes the control button \\  \hline  
% buttonbg=<colour> 			&  strokes on transparent background \\  \hline  
% buttonfg=<colour> 			& specifies the stroke colour  \\  \hline  
% draft 						& animation is not embedded.Instead, a box with the exact
% dimensions  \\  \hline  
% final 						&   \\  \hline  
% useocg 					&  alternative animation method based on Optional Content Groups \\  \hline  
% measure 					&  Measures the frame rate during one cycle of the animation. \\  \hline  
%\end{tabular}


