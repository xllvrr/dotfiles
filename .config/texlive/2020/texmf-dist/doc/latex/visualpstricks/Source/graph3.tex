


\subsection{Macro psplot }

%\TFRGB{syntaxe}{syntax} :  \BSS{psplot} [Options] \AC{x min}\AC{x max}\AC{fonction}
% 
%
%
% \bigskip


\psset{yunit=1cm}
\begin{center}
\begin{tabular}{|l|} \hline

 \multicolumn{1}{|c|}{
 \begin{psgraph*}[,xticksize= -1.5 1.5 ,yticksize=12cm , subticks=0, dx=90,Dx=90 , dy=.5,Dy=.5](0,0)(0,-1.5)(720,1.5){12cm}{5cm } 
 \psplot[plotpoints=200,linecolor=red]{0}{720}{x sin}
 \end{psgraph*} 
%\pspicture(0,-1.5)(8,1.50)
%\psset{xunit=0.01cm}
%\psplot[plotpoints=200,linecolor=red]{0}{720}{x sin}
%\psline{->}(800,0)
%\endpspicture 
}\\ \hline
% \BS{}psset\AC{xunit=0.01cm}\\
\textbf{\BS{psplot}}[\rnode{A}{\RDD{plotpoints}=200},linecolor=red]\rnode{B}{\AC{0}\AC{720}}\rnode{C}{\AC{x sin}}  \RDI{plotpoints}{pst-plot}  \\ \hline
\\
\rnode{AA}{\TFRGB{ nombre de points utilisés}{number of point used}}  \hspace{.5cm}
\rnode{BB}{\TFRGB{unité de x en degré}{x unit : degree}}  \hspace{0.5cm}
\rnode{CC}{\TFRGB{fonction  en langage PostScript  \footnotemark[1]}{Function in postscript code }} 
 \\ \hline
{\blue \dft{} :  plotpoints = 50 }
\\ \hline 


\end{tabular}
\end{center}
\footnotetext[1]{\TFRGB{formule de calcul en langage PostScript (voir \pageref{postcript})}{formula in the PostScript language} }

 \ncline[linecolor=blue]{A}{AA}  \ncline[linecolor=blue]{B}{BB}  \ncline[linecolor=blue]{C}{CC}


 %-----------------------------------------------------
 \subsection{Macro  parametricplot}
 
%\TFRGB{syntaxe}{syntax} :  \BSS{parametricplot} [Options]\AC{t min}\AC{t max}\AC{x(t) y(t) }
%\BSI{parametricplot}{pst-plot} 
%  \bigskip

  
%\BS{}psset\AC{xunit=2cm}\\


%
%\begin{psgraph*}[,xticksize= -2 2 ,yticksize=-2 2, subticks=0, dx=.5,Dx=.5, dy=.5,Dy=.5](0,0)(-2,-2)(2,2){5cm}{5cm } 
%\parametricplot[linewidth=1.2pt,plotstyle=ccurve,linecolor=red]{0}{360}{t sin t 2 mul sin}
%\end{psgraph*}
 

\begin{center}
\begin{tabular}{|c|} \hline
\begin{psgraph*}[,xticksize= -1.5 1.5 ,yticksize=-1.5 1.5 , subticks=0, dx=.5,Dx=.5, dy=.5,Dy=.5](0,0)(-1.5,-1.5)(1.5,1.5){5cm}{5cm } 
\parametricplot[linewidth=1.2pt,plotstyle=ccurve,linecolor=red]{0}{360}{t sin t 2 mul sin}
\end{psgraph*}
%
%\begin{pspicture}(-5cm,-1.5)(5cm,1.5)
%
%\psset{xunit=2cm}
%\parametricplot[linewidth=1.2pt,plotstyle=ccurve,linecolor=red]{0}{360}{t sin t 2 mul sin}
%\psline{<->}(0,-1.2)(0,1.2)
%\psline{<->}(-1.2,0)(1.2,0)
%\end{pspicture}
\\ \hline
\textbf{\BS{}parametricplot}[linewidth=1.2pt,plotstyle=ccurve,linecolor=red]\\
\rnode{A}{\AC{0}\AC{360}}\Rnode*[fillcolor=yellow,fillstyle=solid]{B}{\AC{t sin t 2 mul sin} }\hspace{1cm} (functions : $sin(t)$ et  $sin(2t)$)\\
%\BS{psline}\AC{<->}(0,-1.2)(0,1.2)\\
%\BS{psline}\AC{<->}(-1.2,0)(1.2,0)\\ 
\hline
\\ \\
 \rnode{AA}{\TFRGB{L'unité  de t est le degré }{ t unit in degree}} \hspace{1cm} 
\rnode{BB}{\TFRGB{Les deux fonctions doivent être écrites en langage PostScript }{the 2 functions in postcript code } !}
\\ \hline
\end{tabular}
\end{center}
 \ncline[linecolor=blue]{A}{AA}  \ncline[linecolor=blue,armA=.5cm,angleA=90]{B}{BB}  
 
\newpage

%=====================================================
\SbSSCT{Graphe polaire}{Polar graph}

%\TFRGB{syntaxe}{syntax} :  
%\BS{}psplot [\RDD{polarplot}=true]\AC{angle début}\AC{angle fin}[commande postscript]\AC{équation} \RDI{polarplot}{pst-plot} 

%Exemple :



%\AC{}psplot[plotstyle=curve,polarplot=true,linecolor=red,algebraic=true]\AC{0}\AC{\BS{}psPiTwo}\AC{6*sin(2*x)}

%\begin{psgraph*}[axesstyle=none,xticksize= -5 5 ,yticksize=-5 5, subticks=0](0,0)(-5,-5)(5,5){5cm}{5cm } 
%\psplot[plotstyle=curve,polarplot=true,linecolor=red]{0}{360}{ x 2 mul sin 6 mul }
%\end{psgraph*}

\begin{center}
\begin{tabular}{|c|} \hline
\begin{psgraph*}[xticksize= -5 5 ,yticksize=-5 5, subticks=0](0,0)(-5,-5)(5,5){5cm}{5cm } 
\psplot[plotstyle=curve,polarplot=true,linecolor=red]{0}{360}{ x 2 mul sin 6 mul }
\end{psgraph*}
\\ \hline
\BS{}psplot[plotstyle=curve,{\red polarplot}=true,linecolor=red]\\
\AC{0}\AC{360} \AC{ x 2 mul sin 6 mul }  \\  \hline ($6*sin(2*x)$)\\ \hline
\end{tabular}
\end{center}

%\psplot[plotstyle=curve,polarplot=true,linecolor=red,algebraic=true]{0}{\psPiTwo}{6*sin(2*x)}


\subsection{Modules infix-RPN et pst-infixplot  \cite{pst-infix}}
% \begin{itemize}
% \item  Ces Modules permettent de s'affranchir des équations en langage PostScript
% \item Le module  infix-RPN utilise \BS{RPN} 
% \item Le module  pst-infixplot donne les commandes \BS{}psPlot et \BS{}parametricPlot 
%\end{itemize}
%
%
%\bigskip
%Exemples :
\label{infix-RPN}
\label{pst-infixplot}

%% \multicolumn{1}{|c|}{
% \begin{psgraph*}[,xticksize= -1.5 1.5 ,yticksize=12cm , subticks=0, dx=90,Dx=90 , dy=.5,Dy=.5](0,0)(0,-1.5)(720,1.5){12cm}{5cm } 
%\infixtoRPN{sin(x)} 
%\psplot[plotpoints=200,linecolor=red]{0}{720}{\RPN}
% \end{psgraph*} 
%%\pspicture(0,-1.5)(8,1.50)
%%\psset{xunit=0.01cm}
%%\psplot[plotpoints=200,linecolor=red]{0}{720}{x sin}
%%\psline{->}(800,0)
%%\endpspicture 
%%}


\begin{center}
\begin{tabular}{|l|} \hline
 \begin{psgraph*}[,xticksize= -1.5 1.5 ,yticksize=12cm , subticks=0, dx=90,Dx=90 , dy=.5,Dy=.5](0,0)(0,-1.5)(720,1.5){10cm}{3cm } 
\infixtoRPN{sin(x)} 
\psplot[plotpoints=200,linecolor=red]{0}{720}{\RPN}
 \end{psgraph*} 
%\pspicture(0,-1.5)(9cm,1.50)
%\psset{xunit=0.01cm}
%\infixtoRPN{sin(x)} 
%\psplot[plotpoints=200,linecolor=red]{0}{720}{\RPN}
%\psline{->}(800,0)
%\endpspicture 
\\ \hline
 \BSS{infixtoRPN}\AC{sin(x)} \BSI{infixtoRPN}{pst-plot}  \\
\BS{}psplot[plotpoints=200]\AC{0}\AC{720}\AC{{\red \BS{}RPN}}
\\ \hline
\end{tabular} 
\end{center}



\smallskip
\begin{center}
\begin{tabular}{|l|} \hline
 \begin{psgraph*}[,xticksize= -1.5 1.5 ,yticksize=12cm , subticks=0, dx=90,Dx=90 , dy=.5,Dy=.5](0,0)(0,-1.5)(720,1.5){10cm}{3cm } 
\psPlot[linecolor=red]{0}{720}{sin(x)}
 \end{psgraph*} 
 
%\pspicture(0,-1.1)(9cm,1.1)
%\psset{xunit=0.01cm}
%\psset{plotpoints=200}
%\psPlot[linecolor=red]{0}{720}{sin(x)}
%\psline{->}(8cm,0)
%\endpspicture 
\\ \hline
%\BS{}psset\AC{plotpoints=200} \\
 \BSS{psPlot}\AC{0}\AC{720}\AC{sin(x)}
\\ \hline
\end{tabular} 
\end{center}




\smallskip
\begin{center}
\begin{tabular}{|c|} \hline
\begin{psgraph*}[,xticksize= -1.5 1.5 ,yticksize=-1.5 1.5 , subticks=0, dx=.5,Dx=.5, dy=.5,Dy=.5](0,0)(-1.5,-1.5)(1.5,1.5){5cm}{5cm } 
\parametricPlot[linecolor=red,plotpoints=200]{0}{360}{sin(t)}{sin(2*t)}
\end{psgraph*}
%\psset{plotpoints=200}
%\pspicture(-2,-1.5)(2,1.5)
%\psset{xunit=1.7cm}
%\parametricPlot[linecolor=red,plotpoints=200]{0}{360}{sin(t)}{sin(2*t)}
%\psline{<->}(0,-1.2)(0,1.2)
%\psline{<->}(-1.2,0)(1.2,0)
%\endpspicture
\\ \hline
 \BSS{parametricPlot}[linecolor=red,plotpoints=200]\AC{0}\AC{360}\AC{sin(t)}\AC{sin(2*t)} \\ \hline
\end{tabular} 
\end{center}

%\newpage
\subsection{Option algebraic}
%\begin{itemize}
%\item  Cette option permet de s'affranchir des équations en langage PostScript
%\item L'unité de x ou de t est le\emph{ radian }
%\end{itemize}
%
%
%
%\bigskip
%%
%\BS{}psplot[algebraic,plotpoints=200]\AC{0}\AC{12.56}\AC{ sin(x)}
%\smallskip

\begin{center}
\begin{tabular}{|c|} \hline
 \begin{psgraph*}[,xticksize= -1.5 1.5 ,yticksize=13	 , subticks=0, dx=1,Dx=1, dy=.5,Dy=.5](0,0)(0,-1.5)(13,1.5){10cm}{3cm } 
\psplot[algebraic,plotpoints=200,linecolor=red]{0}{12.56}{ sin(x)}
 \end{psgraph*} 
%\pspicture(0,-1.1)(8,1.1)
%\psset{xunit=.5cm}
%\psplot[algebraic,plotpoints=200,linecolor=red]{0}{12.56}{ sin(x)}
%\psline{->}(15,0)
%\endpspicture
\\ \hline
\BS{}psplot[\RDD{algebraic},plotpoints=200]\AC{0}\AC{12.56}\AC{{\red sin(x)}} \RDI{algebraic}{pst-plot} \\ \hline
\TFRGB{L'unité de x  est le}{x unit in }\emph{ radian }
 \\ \hline
\end{tabular} 
\end{center}
\bigskip
%

\smallskip


\begin{center}
\begin{tabular}{|c|} \hline
\begin{psgraph*}[,xticksize= -1.5 1.5 ,yticksize=-1.5 1.5 , subticks=0, dx=.5,Dx=.5, dy=.5,Dy=.5](0,0)(-1.5,-1.5)(1.5,1.5){5cm}{5cm } 
\parametricplot[algebraic,plotpoints=200,linecolor=red]{0}{6.28}{sin(t)|sin(2*t)}
\end{psgraph*}
%\pspicture(-2,-1.5)(2,1.5)
%\psset{xunit=1.7cm}
%\parametricplot[algebraic,plotpoints=200,linecolor=red]{0}{6.28}{sin(t)|sin(2*t)}
%\psline{<->}(0,-1.2)(0,1.2)
%\psline{<->}(-1.2,0)(1.2,0)
%\endpspicture 
\\ \hline
\BSS{parametricplot}[{\red algebraic},plotpoints=200]\AC{0}\AC{6.28}\AC{{\red sin(t)|sin(2*t)}}
\\ \hline
\TFRGB{L'unité de  t est le}{t unit in }\emph{ radian }
\\ \hline
\end{tabular} 
\end{center}

\subsection{Options VarStep et VarStepEpsilon }
\label{var}

\begin{tabular}{|c|} \hline 
 \begin{psgraph*}[,xticksize= -1.5 1.5 ,yticksize=13	 , subticks=0, dx=1,Dx=1, dy=.5,Dy=.5](0,0)(0,-1.5)(13,1.5){10cm}{3cm } 
\psplot[algebraic,VarStep=true,linecolor=red,showpoints=true,VarStepEpsilon=1]{0}{12.56}{ sin(x)}
 \end{psgraph*} 
 \\ \hline   
\BS{psplot}[algebraic,\RDD{VarStep}=true,showpoints=true,\RDD{VarStepEpsilon}=1]{0}\AC{12.56}\AC{ sin(x)}  \RDI{VarStep}{pstricks-add}   \RDI{VarStepEpsilon}{pstricks-add} 
\\ \hline  
 \begin{psgraph*}[,xticksize= -1.5 1.5 ,yticksize=13	 , subticks=0, dx=1,Dx=1, dy=.5,Dy=.5](0,0)(0,-1.5)(13,1.5){10cm}{3cm } 
\psplot[algebraic,VarStep=true,linecolor=red,showpoints=true,VarStepEpsilon=.1]{0}{12.56}{ sin(x)}
 \end{psgraph*} 
 \\ \hline  
\BS{psplot}[algebraic,\RDD{VarStep}=true,showpoints=true,\RDD{VarStepEpsilon}=0.1]{0}\AC{12.56}\AC{ sin(x)}
\\ \hline  
 \begin{psgraph*}[,xticksize= -1.5 1.5 ,yticksize=13	 , subticks=0, dx=1,Dx=1, dy=.5,Dy=.5](0,0)(0,-1.5)(13,1.5){10cm}{3cm } 
\psplot[algebraic,VarStep=true,linecolor=red,showpoints=true,VarStepEpsilon=0.01]{0}{12.56}{ sin(x)}
 \end{psgraph*} 
 \\ \hline  
\BS{psplot}[algebraic,\RDD{VarStep}=true,showpoints=true,\RDD{VarStepEpsilon}=.001]{0}\AC{12.56}\AC{ sin(x)}
\\ \hline  

\end{tabular} 




