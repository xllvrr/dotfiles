\documentclass[ngerman,size=12pt,mode=present,paper=a4paper,
style=BerlinFU,mode=present]{pst-powerdot}

\usepackage[T1]{fontenc}
\usepackage[utf8]{inputenc}
\usepackage{babel}
\usepackage[scaled=0.88]{luximono}
\AtBeginDocument{\sffamily}

\usepackage{graphicx,pifont,marvosym,amsmath,textcomp,multicol,url,
pst-all,pst-jtree,pst-slpe,pstricks-add,dtklogos,
boxedminipage,tipa}

\usepackage{hyperref}
\hypersetup{pdfauthor={Herbert Voss},pdftitle={LaTeX WS 2008/09}}

\title{Wissenschaftliches Schreiben mit \LaTeX\\
WS 2008/09}
\titlegraphic{silberlaube}

\author{\tabular{@{}l}
  Herbert Voß \\
  Freie Universit\"at Berlin\\  
  Zentraleinrichtung Datenverarbeitung\\  
  E-Mail: \url{Herbert.Voss@FU-Berlin.de}\\  
  \url{http://userpages.fu-berlin.de/~latex/WS0809}
\endtabular}

\date{27.10. 2008}

\pdsetup{lf={\textcopyright~H.\,Voß},cf={\LaTeX\ WS 2008/09},trans=Dissolve}
%\usepackage{enumitem}

\newcommand\describelabel[1]{\texttimes #1}

\begin{document}
\maketitle

\begin{slide}[toc=,bm=]{Gliederung}
\tableofcontents[content=sections]
\end{slide}

\section{Einf\"uhrung}

\begin{slide}{\LaTeX -- der Name \ldots}
\onslide{1-}{%
\begin{block}{Ungeschickte Definition}
"`Ich danke meiner Sekret\"arin, die mich behutsam in Latex eingef\"uhrt hat."'{}

\raggedleft\tiny "`Aus einem Wirtschaftsskript der Universit\"at W\"urzburg."'{}
\end{block}
}

\psset{shadow=true,shadowcolor=red!30!black!30}
\smallskip
\onslide{2-}{%
\begin{block}{Bessere Definition}
\LaTeX{}, gesprochen "`Latech"'{}, Kurzform von "`Lesley Lamport's \TeX"'{}
\end{block}
}
\end{slide}


\begin{slide}{Der Name \TeX}
\begin{block}{}
\TeX{} ist abgeleitet aus $\tau\epsilon\chi$, phonetisch
	korrekt \textipa{[teX]} (im  deutschen Sprachraum auch \textipa{[te\c{c}]})
und von Donald Knuth ab 1977 in Stanford entwickelt worden. Befindet sich in einem quasi
"`eingefrorenen"' Zustand.
\end{block}
\pause
\begin{quote}
\glqq Mathematics books and journals do not
look as beautiful as they used to. It is not that their mathematical
content is unsatisfactory, rather that the old and well-developed
traditions of typesetting have become too expensive. Fortunately, it
now appears that mathematics itself can be used to solve this problem.\grqq\\
({\small\sc Donald E. Knuth: Mathematical Typography, 1978})
\end{quote}
\end{slide}


\begin{slide}{Der Name \TeX}
\begin{block}{Block}
\TeX{} ist abgeleitet aus $\tau\epsilon\chi$, phonetisch
	korrekt \textipa{[teX]} (im  deutschen Sprachraum auch \textipa{[te\c{c}]})
und von Donald Knuth ab 1977 in Stanford entwickelt worden. Befindet sich in einem quasi
"`eingefrorenen"' Zustand.
\end{block}

\begin{block}[\textwidth][example]{Example}
\TeX{} ist abgeleitet aus $\tau\epsilon\chi$, phonetisch
	korrekt \textipa{[teX]} (im  deutschen Sprachraum auch \textipa{[te\c{c}]})
und von Donald Knuth ab 1977 in Stanford entwickelt worden. Befindet sich in einem quasi
"`eingefrorenen"' Zustand.
\end{block}
\end{slide}



\begin{slide}{Historische Entwicklung von \LaTeX}
  \def\NtS{$\mathcal{N}$\kern-.1667em\lower.5ex\hbox{$\mathcal{T}$}\kern-.125em  $\mathcal{S}$} %$
  \begin{itemize}[type=1]%\itemsep=0pt\parsep=0pt\parskip=0pt
  	\item 1982: {\sc Leslie Lamport} beginnt die Entwicklung		\pause
	\item 1985: \LaTeX\/ Version 2.09 wird fertiggestellt		\pause
	\item 1994: Buch {\sl \LaTeX---A Document Preparation
		System User's Guide and Reference Manual} erscheint		\pause
	\item 1994: \LaTeXe\/ (\LaTeX3-Projektgruppe)		\pause
		\begin{itemize}
			\item NFSS(2) {\tiny (New Font Selection Scheme)}		\pause
			\item Konzept der Dokumentklassen {\tiny (vererbbare Optionen)}		\pause
			\item Unterst"utzung moderner Pakete (z.\,B. {\tt hyperref})		\pause
			\item Kompatibilit"atsmodus f"ur \LaTeX\/ 2.09 Dokumente		\pause
		\end{itemize}
  \end{itemize}
\end{slide}

\begin{slide}{Historische Entwicklung von \LaTeX}
  \def\NtS{$\mathcal{N}$\kern-.1667em\lower.5ex\hbox{$\mathcal{T}$}\kern-.125em  $\mathcal{S}$} %$
  \begin{itemize}[type=1]%\itemsep=0pt\parsep=0pt\parskip=0pt
   \item Zukunft: \NtS\/ {\tiny (New Typesetting System)}, $\Omega$, 
             $\varepsilon${\tiny$\chi$}\kern-2pt\TeX\/ [DTK 4/03]\\
		\XeTeX\ (Mac -- Unicode) oder $\mu$IMP (Wysiwyg -- \url{http://www.microimp.com}) 		\pause
    \item \texttt{pdf}\TeX, aktuell die "`eigentliche"' Maschine (Compiler)		\pause
    \item Lua\TeX\ ab Herbst 2008 als frei programmierbares Satzsystem. Integration beliebiger 
    Schnittstellen.		\pause
%\item \verb+\newcommand\XeTeX{X\kern-.1em\lower.5ex}
%+ \hspace*{1cm}\verb+\mbox{\reflectbox{E}}\kern-.15em\TeX}+
  \end{itemize}
\end{slide}


\end{document}


