%% Fichier contenant un article. La classe génère trois fichiers
%% d'articles par défaut. Une thèse compte généralement trois articles
%% et un mémoire, un. Si vous en avez besoin davantage,
%% enregistrez ce fichier sous un autre nom et incluez-le dans
%% votre gabarit avec la commande \include.
%%
%% Les articles sont structurés tel que vous le voyez ci-dessous :
%% un résumé non numéroté, une introduction, des sections et une
%% conclusion numérotées, et une bibliographie. Vous pouvez ajouter
%% ou supprimer des sections de développement selon vos besoins.
\chapter{Titre de l'article / Article title}
\thispagestyle{empty} % Première page non paginée / First page is unnumbered

\section*{\HECtdmResumeArticle}
\phantomsection\addcontentsline{toc}{section}{\HECtdmResumeArticle}

%% Rédigez votre résumé ici.

\section{Introduction}

%% Rédigez votre introduction d'article ici.

\section{Titre de la section de développement 1 / Section 1 title}

%% Rédigez votre section de développement ici.

\section{Titre de la section de développement 2 / Section 2 title}

%% Rédigez votre section de développement ici.

\section{Titre de la section de développement 3 / Section 3 title}

%% Rédigez votre section de développement ici.

\section{Conclusion}

%% Rédigez votre conclusion d'article ici.

\bibliographystyle{francais}
%% Inscrivez le nom de votre fichier .bib entre les accolades.
\bibliography{}
