\documentclass[a4paper,10pt]{article}

%%%%%%%%%%%%%%%%%%%%%%
%% Modèle Banlieues %%
%%%%%%%%%%%%%%%%%%%%%%

%% Au moins 1 argument obligatoire pour babel [french, english, ...]  
%% L'encodage par defaut pour inputenc est utf8, mais vous pouvez utiliser l'option [latin1]
%% vous pouvez ensuite utiliser \rsfbEncodage{} pour vérifier l'encodage latin1ou utf8
%% Exemples d'options (nb: si vous changez les options, compilez 2 fois de suite):
%\usepackage[latin1, french]{facture-belge-simple-sans-tva} % latin1, français
%\usepackage[french, english]{facture-belge-simple-sans-tva} % utf8, français, anglais
\usepackage[french]{modeles-factures-belges-associations} % utf8, français

% On évite le "Figure X" dans le caption
\usepackage{caption}
\captionsetup[figure]{name={}}

\begin{document}

%%%%%%%%%%%%%%%%%%%%%%%%%%%
%% Logo de l'association %%
%%%%%%%%%%%%%%%%%%%%%%%%%%%
%% Le logo doit être dans le répertoire courant (= du modèle)
%% Nom du logo de l'association et son échelle (1 = 100%, 0.5 = 50%, etc.)
\begin{figure}[!h]
\begin{flushleft}
\includegraphics[scale=0.2]{facture-banlieues-logo.png}
\end{flushleft}
\end{figure}


%% pour vérifier l'encodage UTF-8 ou latin1
%\rsfbEncodage{}

%%%%%%%%%%%%%%%%%%%%%
%% Titres tableaux %%
%%%%%%%%%%%%%%%%%%%%%
%% Titres du tableau d'adresses; utile si utilisé, cas le plus fréquent
\rsfbExpedition{Expedition}
\rsfbFacturation{Facturation}
\rsfbLivraison{Intervention}

%% Titres du tableau des produits:
%%%%%%%%%%%%%%%%%%%%%%%%%%%%%%%%%
%% La monnaie est fixée juste en dessous de ceci
%% et au dessus du point "Unité monétaire atomique"
\rsfbNatureTableauProduits{Nature}
\rsfbQuantiteTableauProduits{Qtté}
\rsfbPuTableauProduits{Prix unit.}

%% + quelques autres paramètres utiles du tableau des prodsuits:
%%%%%%%%%%%%%%%%%%%%%%%%%%%%%%%%%%%%%%%%%%%%%%%%%%%%%%%%%%%%%%%
% Par défaut 2 décimales, c'est plutôt correct, non?
\rsfbNbDecimalesTableauProduits{2}
% Et le mot "sur" ("on" en anglais)
\rsfbSurTableauProduits{sur}
% Unité monétaire dans les titres du tableau
\rsfbMonnaieTableauProduits{\euro}

%%%%%%%%%%%%%%%%%%%%%%%%%%%%%%
%% Unité monétaire atomique %%
%%%%%%%%%%%%%%%%%%%%%%%%%%%%%%
%% permet de choisir l'unité monétaire, sauf dans le header du tableau (package calctab)
%% \euro (defaut), \pounds (livre anglaise) \$ (dollar américain) \textyen (yen japonais)
%% pour être cohérent, les 2 unités \rsfbChoisirUniteMonetaire{\unite}
%% \ctcurrency{\unite} doivent ëtre les mêmes
% Decommenter ci-dessous pour voir les unités
% \${} \pounds{} \euro{} \textyen{}
\rsfbChoisirUniteMonetaire{\euro}

%% Peut être utile, où vous voulez, dans le texte:
%% Décommenter pour afficher l'unité monétaire atomique choisie:
%\rsfbchoisirUniteMonetaire{}


%%%%%%%%%%%%%%%%%%%%%%%%%%%%
%% Aération de la facture %%
%%%%%%%%%%%%%%%%%%%%%%%%%%%%

%% permet d'aérer la page verticalement; le paramètre est une mesure LaTeX
%% TeX comprend six unités de mesure :
%% + pt point = 0,35 mm
%% + mm millimètre
%% + ex correspond à la hauteur d'un x dans la fonte courante
%% + em correspond à la largeur d'un m dans la fonte courante
%% + cm centimètre
%% + in pouce = 2,54 cm
\rsfbAerationVerticale{0.3cm}
%\rsfbAerationVerticale{1.5cm}
%\rsfbAerationVerticale{1cm}
%\rsfbAerationVerticale{0.5cm}

%%%%%%%%%%%%%%%%%%%%%%%%%%%%%%%%%%
%% Numéro et date de la facture %%
%%%%%%%%%%%%%%%%%%%%%%%%%%%%%%%%%%
%\rsfbNoDate{<numéro=nombre>}{date=chaine}. 
% (Dé)commenter pour version avec ou sans mention n°
%% \rsfbNoDate{n° 1}{jj mois AAAA}, avec mention n°
%\rsfbNoDate{n\textsuperscript{o}X}{JJ mois AAAA}
%% \rsfbNoDate{1}{jj mois AAAA}, sans mention n°
\rsfbNoDate{XYZ1234}{JJ mois AAAA}

%% créent aussi les codes \rsfbmentionObligatoire{} et \rsfbdateFacture{} 
%% qui permettent de récupérer et d'utiliser un numéro de facture 
%% un peu personnalisé et la date de la facture.


%%%%%%%%%%%%%%%%%%%%%%%%%%%%%%%%%%%%%%%%%
%% Entrée du compte bancaire créancier %%
%%%%%%%%%%%%%%%%%%%%%%%%%%%%%%%%%%%%%%%%%
%% \rsfbCompteBancaireCreancier{n° compte}
%% récupérable via \rsfbcompteBancaireCreancier{}
%% !!! Attention a la différence: rsfbCompt ... et rsfbcompt ...
\rsfbCompteBancaireCreancier{BE01 2345 6789 0123}

%%%%%%%%%%%%%%%%%%%%%%%%%
%% Aération du texte   %%
\vspace{\rsfbespaceVertical}
%%%%%%%%%%%%%%%%%%%%%%%%%

%%%%%%%%%%%%%%%%%%%%%%%%%%%%%%%%%%%%%%%%%%%%%%%%%%%%%%%%%%%%%%%%%%%%%%%%%
%% Tableau des adresses (3): expédition, facturation, livraison  DEBUT %%
%%%%%%%%%%%%%%%%%%%%%%%%%%%%%%%%%%%%%%%%%%%%%%%%%%%%%%%%%%%%%%%%%%%%%%%%%

%%%%%%%%%%%%%%%%%%%%%%%%%%%%%%%%%%%%%%%%%%%%%%%%%%%%%%%%%%%%%%%%%%%%%%%%%%%%%%%%
%%  On ouvre l'entête du tableau des adresses; à n'utiliser qu'une seule fois.
\rsfbEnteteTableauAdresses{}
%%%%%%%%%%%%%%%%%%%%%%%%%%%%%%%%%%%%%%%%%%%%%%%%%%%%%%%%%%%%%%%%%%%%%%%%%%%%%%%%

%%%%%%%%%%%%%%%%%%%%%%%%%%%%%%%%%%%%%%%%%%%%%%%%%%%%%%%%%%%%%%%%%%%%%%%%%%%%%%%%%%
%% Entrée d'une ligne d'adresse, dans l'ordre expédition, facturation, livraison

%% 1) Voici un exemple complexe:
\rsfbLigneTableauAdresses{Prénom \textsc{Nom}}{ \textsc{Fbg}}{Voir facturation}
\rsfbLigneTableauAdresses{}{Fédération belge de gong}{}
\rsfbLigneTableauAdresses{N\up{o}, rue Delarue1}{DelarueStraat, no}{}
\rsfbLigneTableauAdresses{\textsc{1234 Ville1}}{\textsc{4321 Ville2}}{}
\rsfbLigneTableauAdresses{\href{mailto:user@domain.tld}{user@domain.tld}}{}{}
\rsfbLigneTableauAdresses{+32 684 037 078}{}{}
\rsfbLigneTableauAdresses{\rsfbcompteBancaireCreancier{}}{BE98 7654 3210 9876}{}

%% 2) Voici un exemple simple:
%\rsfbLigneTableauAdresses{\textsc{Prénom Nom}}{\textsc{Org 1}}{\textsc{Org 2}}
%\rsfbLigneTableauAdresses{}{Nom organisation 1}{Nom organisation 2}
%\rsfbLigneTableauAdresses{N\up{o}, rue Delarue1}{N\up{o}, rue Delarue2}{N\up{o}, rue Delarue3}
%\rsfbLigneTableauAdresses{\textsc{CCC1 Ville1}}{\textsc{CCC2 Ville2}}{\textsc{CCCC Ville3}}

%% 3) et enfin, si vous voulez, des lignes vides:
%\rsfbLigneTableauAdresses{}{}{}
%\rsfbLigneTableauAdresses{}{}{}
%\rsfbLigneTableauAdresses{}{}{}
%\rsfbLigneTableauAdresses{}{}{}

%%%%%%%%%%%%%%%%%%%%%%%%%%%%%%%%%%%%%%%%%%%%%%%%%%%%%%%%%%%%%%%%%%%%%%%%%%%%%%%
%%  on ferme le pied du tableau des adresses; à n'utiliser qu'une seule fois.
\rsfbPiedTableauAdresses{}
%%%%%%%%%%%%%%%%%%%%%%%%%%%%%%%%%%%%%%%%%%%%%%%%%%%%%%%%%%%%%%%%%%%%%%%%%%%%%%%

%%%%%%%%%%%%%%%%%%%%%%%%%%%%%%%%%%%%%%%%%%%%%%%%%%%%%%%%%%%%%%%%%%%%%%%
%% Tableau des adresses (3): expédition, facturation, livraison  FIN %%
%%%%%%%%%%%%%%%%%%%%%%%%%%%%%%%%%%%%%%%%%%%%%%%%%%%%%%%%%%%%%%%%%%%%%%%


%%%%%%%%%%%%%%%%%%%%%%%%%
%% Aération du texte   %%
\vspace{\rsfbespaceVertical}
%%%%%%%%%%%%%%%%%%%%%%%%%


%%%%%%%%%%%%%%%%%%%%%%%%%%%%%%%%%%%%%%%%%%%%%%%%%%%%%%%%%%%%%%%%%%%%%%%%%%%%%%%%
%% Tableau des produits, avec le package calctab, on choisit xcalctab, DEBUT  %%
%%%%%%%%%%%%%%%%%%%%%%%%%%%%%%%%%%%%%%%%%%%%%%%%%%%%%%%%%%%%%%%%%%%%%%%%%%%%%%%%
%% On ne va pas réinventer le monde, mais simplement réutiliser les commandes
%% simples et efficaces du package calctab, dans son environnement xcalctab 

%%%%%%%%%%%%%%%%%%%%%%%%%%%%%%%%%%%%%%%%%%%%%%%%%%%%%%%%%%%%%%%%%%%%%%%%%%%%%%%%
%%  Ensuite le tableau des produits avec nature, quantité, coût et éventuelles
%% remises

% On ouvre l'environnement xcalctab
\begin{xcalctab}
%% calctab est en police police sans empattements (sf, pour sans serif), 
%% on remet en normalfont de ce document
\normalfont
%% On ajoute des produits 
% \amount{nature}{quantité}{prix unitaire}
%% si amount comporte un id comme ci-dessous, on pourra lui appliquer une remise (-) (ou une taxe (+))
%% avec la commande \perc[identificateur]{Intitulé}{+/-pourcentage}
%% une id s'écrit ainsi: [id=identificateur] identificateur = 1 seul mot entier!
% \amount[id=identificateur]{nature}{quantité}{prix unitaire}
%% Simple, non?

% produits (avec ou sans id)
\amount[id=intervention]{Intervention du 20/09}{2}{37,50}
\amount[id=forfait]{Déplacement (forfait)}{1}{10}
\add[id=phtva,intervention,forfait]{Prix HTVA}

% TVA:
\perc[id=tva21,phtva]{TVA}{21}

% Grand total:
\add[phtva,tva21]{Total}

% On ferme l'environnement xcalctab
\end{xcalctab}

%%%%%%%%%%%%%%%%%%%%%%%%%%%%%%%%%%%%%%%%%%%%%%%%%%%%%%%%%
%% Tableau des produits, avec le package calctab, FIN  %%
%%%%%%%%%%%%%%%%%%%%%%%%%%%%%%%%%%%%%%%%%%%%%%%%%%%%%%%%%


%%%%%%%%%%%%%%%%%%%%%%%%%
%% Aération du texte   %%
\vspace{\rsfbespaceVertical}
%%%%%%%%%%%%%%%%%%%%%%%%%

%%%%%%%%%%%%%%%%%%%%%%%%%
%% Texte libre, DEBUT. %%
%%%%%%%%%%%%%%%%%%%%%%%%%

\paragraph*{Description du produit:}
Les montants ci-dessus sont calculés sur base de l’intervention du 20 septembre 2018 effectuée par notre collègue.


Merci de régler cette facture au compte \no \emph{\rsfbcompteBancaireCreancier{}} avec la mention obligatoire \emph{\rsfbmentionObligatoire{}}.

%%%%%%%%%%%%%%%%%%%%%%%
%% Texte libre, FIN. %%
%%%%%%%%%%%%%%%%%%%%%%%

%%%%%%%%%%%%%%%%%%%%%%%%%%%%%%%%%%%
%% Ajoute les crédits en footer. %%
%%%%%%%%%%%%%%%%%%%%%%%%%%%%%%%%%%%
\rsfbCredit{}



%%%%%%%%%%%%%%%%%%%%%%%%%%%%%%%%%%%%%%%%%
%% Logos des sponsors de l'association %%
%%%%%%%%%%%%%%%%%%%%%%%%%%%%%%%%%%%%%%%%%
%% Le logo doit être dans le répertoire courant (= du modèle)
%% \caption*{texte juste au dessus des logos}
%% Nom du logo de l'association et son échelle (1 = 100%, 0.5 = 50%, etc.)
\begin{figure}[b]
\begin{center}
\caption*{\small Avec le soutien des sponsors}
\includegraphics[scale=0.2]{facture-banlieues-sponsors.png}
\end{center}
\end{figure}

\end{document}
