\section{News and compatibility}


Some changes have been made to make the syntax more homogeneous and especially to distinguish the definition and search for coordinates from the rest, i.e. drawing, marking and labelling.
In the future, the definition macros being isolated, it will be easier to introduce a phase of coordinate calculations using \tkzimp{Lua}.

An important novelty is the recent replacement of the \tkzNamePack{fp} package by \tkzNamePack{xfp}.  This is to improve the calculations a little bit more and to make it easier to use.


Here are some of the changes. 
\vspace{1cm}
 \begin{itemize}\setlength{\itemsep}{10pt} 

\item Improved code and bug fixes;

\item With \tkzimp{tkz-euclide} loads all objects, so there's no need to place \tkzcname{usetkzobj\{all\}};\item The bounding box is now controlled in each macro (hopefully) to avoid the use of \tkzcname{tkzInit} followed by \tkzcname{tkzClip};\item Added macros for the bounding box: \tkzcname{tkzSaveBB} \tkzcname{tkzClipBB} and so on;\item  Logically most macros accept \TIKZ\ options. So I removed the "duplicate" options when possible thus the "label options" option is removed;

\item Random points are now in \tkzname{\tkznameofpack} and the macro \tkzcname{tkzGetRandPointOn} is replaced by \tkzcname{tkzDefRandPointOn}. For homogeneity reasons, the points must be retrieved with \tkzcname{tkzGetPoint};

\item The options \tkzname{end} and \tkzname{start} which allowed to give a label to a straight  line are removed. You now have to use the macro \tkzcname{tkzLabelLine};

\item Introduction of the libraries \NameLib{quotes} and \NameLib{angles}; it allows to give a label to a point, even if I am not in favour of this practice;

\item  The notion of vector disappears, to draw a vector just pass "->" as an option to \tkzcname{tkzDrawSegment};

\item Many macros still exist, but are obsolete and will disappear:
\begin{itemize}

\item |\tkzDrawMedians| trace and create midpoints on the sides of a triangle. The creation and drawing separation is not respected so it is preferable to first create the coordinates of these points with |\tkzSpcTriangle[median]| and then to choose the ones you are going to draw with |\tkzDrawSegments| or |\tkzDrawLines|;

\item |\tkzDrawMedians(A,B)(C)| is now spelled |\tkzDrawMedians(A,C,B)|. This defines the median from $C$;
  
\item Another example |\tkzDrawTriangle[equilateral]| was handy but it is better to get the third point with |\tkzDefTriangle[equilateral]| and then draw with |\tkzDrawPolygon|;
  
\item |\tkzDefRandPointOn| is replaced by |\tkzGetRandPointOn|;\item now |\tkzTangent| is replaced by |\tkzDefTangent|;

\item You can use |global path name| if you want find intersection  but it's very slow like in \TIKZ.

\end{itemize}


\item Appearance of the macro \tkzcname{usetkztool} which allows to load new "tools".
\end{itemize}

\endinput