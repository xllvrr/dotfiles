\section{Marks or symbols}

I distinguished between the points used in Euclidean geometry and the "marks or symbols" that can be found in statistics.

To position the symbol, we use the macro \tkzcname{tkzDefPoint} to correctly define a point, then the macro \tkzcname{tkzDrawMark} to draw the symbol.

It is common to have to draw a scatter plot, so I created a macro that allows you to define several points quickly.

A "mark" symbol can be scaled, which is sometimes useful, but, on the other hand, if you change the abscissa and ordinates differently then the "marks" are distorted.

Reminder: it was already possible to create a cloud of points with the macro \tkzcname{tkzDefPoints}, but this requires to give a reference (a name) to each point, which is sometimes tedious. The macro \tkzcname{tkzSetOfPoints} allows to define points \tkzname{tkzPt1}, \tkzname{tkzPt2}, etc.

This is frequently referred to as the "scatter plot". The difference from the macro \tkzcname{tkzDefPoints} is that the reference to the points is given by a prefix (default |tkzPt|) and the point number. 

The points are not drawn.

\subsection{\tkzcname{tkzDrawSetOfPoints}} 

\begin{NewMacroBox}{tkzDrawSetOfPoints}{\oarg{local options}}%
Allows you to place symbols on the points defined by \tkzcname{tkzDefSetOfPoints}.

\medskip
\begin{tabular}{lll}%
options             & default & definition   \\ 
\midrule
\TOline{prefix} {tkzPt} {point name prefix}
\end{tabular}
\end{NewMacroBox}

\subsubsection{Drawing of a scatter plot with \tkzcname{tkzDrawSetOfPoints}} 
\begin{tkzexample}[latex=6cm,small] 
\begin{tikzpicture}[scale=0.75]
\tkzInit[xmax=6,ymin=1000,ymax=5000,ystep=1000]
\tkzDrawX[label=$m$,below=10pt]
\tkzDrawY[label=$R(m)$,above=10pt] 
\tkzLabelX[font=\scriptsize]
\tkzLabelY[font=\scriptsize]
\tkzDefSetOfPoints[show]{1/2000,2/3000,4/2500,5/4200}
\tkzDrawSetOfPoints[mark=ball,mark size=3pt]  
\end{tikzpicture}
\end{tkzexample}

\subsection{\tkzcname{tkzJoinSetOfPoints}} 
\begin{NewMacroBox}{tkzJoinSetOfPoints}{\oarg{local options}}%
Allows the symbols to be joined by line segments. Of course, it is possible to use all the options of \TIKZ.

\medskip
\begin{tabular}{lll}%
options             & default & definition   \\ 
\midrule
\TOline{prefix} {tkzPt} {point name prefix}
\end{tabular}
\end{NewMacroBox} 

\subsubsection{Link the points of a scatter plot with \tkzcname{tkzJoinSetOfPoints}} 
\begin{tkzexample}[latex=7cm,small]
\begin{tikzpicture}[scale=1]
\tkzInit[xmax=5,ymin=1000,ymax=6000,ystep=1000]
\tkzDrawX[label=$m$,below=13pt]
\tkzDrawY[label=$R(m)$] 
\tkzLabelX[font=\scriptsize]
\tkzLabelY[font=\scriptsize]
\tkzDefSetOfPoints{%
   1/2000,2/3000,4/2500,5/4200}
\tkzJoinSetOfPoints[%
      thick, color=brown]
\tkzDrawSetOfPoints[%
      mark=ball, mark size=3pt]  
\end{tikzpicture} 
\end{tkzexample}

\subsubsection{Using the points of a scatter plot} 
\begin{tkzexample}[latex=7cm,small]
\begin{tikzpicture}[scale=.5]
\tkzInit[xmax=5,ymin=1000,
         ymax=6000,ystep=1000]
\tkzGrid[color=orange!30]
\tkzDrawX[label=$m$,below=13pt]
\tkzDrawY[label=$R(m)$] 
\tkzLabelX[font=\scriptsize]
\tkzLabelY[font=\scriptsize] 
\tkzDefSetOfPoints[prefix=P]{%
   1/2000,2/3000,3/2000,4/2500,5/4200} 
\tkzDrawPolySeg[%
     color=brown!50,
     line width=2pt](P1,P2,P3,P4,P5) 
\end{tikzpicture} 
\end{tkzexample}
 
% \subsection{Mark option \tkzname{mark} et \tkzname{size}}   
\subsection{\tkzcname{tkzSetUpMark}}
\begin{NewMacroBox}{tkzSetUpMark}{\oarg{local options}}%
\begin{tabular}{lll}%
options &  default & example                 \\ 
\midrule
\TOline{mark}{no default}{\tkzcname{tkzSetUpMark[mark=heart]}}
\end{tabular}
\end{NewMacroBox} 

\subsubsection{Two scatter plots} 
\begin{tkzexample}[latex=7cm,small]
\begin{tikzpicture}
\tkzInit[xmax=5.5,ymin=1000,%
         ymax=6000,ystep=2000]
\tkzGrid[color=orange!30]
\tkzDrawX[label=$m$,below=13pt]
\tkzDrawY[above left,label=$R(m)$] 
\tkzLabelX[below right,font=\scriptsize]
\tkzLabelY[font=\scriptsize]
\tkzDefSetOfPoints{1/2000,2/3000,3/2000,
    4/2500,5/4200} 
\tkzDefSetOfPoints[prefix=P]{1/3200,2/4100,
    3/3300,4/3300,5/5000}  
\tkzSetUpMark[mark=heart,color=black,
   fill=red!30,size=4pt]
\tkzJoinSetOfPoints[thick,color=blue,double]
\tkzDrawSetOfPoints
\tkzJoinSetOfPoints[prefix=P,thick,color=orange]
\tkzDrawSetOfPoints[prefix=P,mark=square*,
     mark size=4pt,
     mark options={color=blue,fill=blue!40}]  
\tkzText[draw,color = red,
        fill  = orange!20](3,5800)%
        {Revenue by month}
  \end{tikzpicture}  
\end{tkzexample}

\subsection{\tkzcname{tkzDrawMark}} 
\begin{NewMacroBox}{tkzDrawMark}{\oarg{local options}\parg{point}}%
Place a symbol. More efficient than the next to place a single symbol.

\medskip
\begin{tabular}{lll}%
options             & default & definition   \\ 
\midrule
\TOline{prefix} {tkzPt} {point name prefix}
\end{tabular}
\end{NewMacroBox} 

\subsubsection{Ball; use of \tkzcname{tkzDrawMarks}}
\begin{tkzexample}[latex=7cm,small]
  \begin{tikzpicture}
  \tkzInit[xmax=3,ymax=1]
  \tkzAxeXY 
  \tkzDrawMark[mark=ball](1,.5)
  \end{tikzpicture}
\end{tkzexample}

\subsection{\tkzcname{tkzDrawMarks}} 
\begin{NewMacroBox}{tkzDrawMarks}{\oarg{local options}\parg{list of points}}%
Allows you to place a series of marks.

\medskip
\begin{tabular}{lll}%
options             & default & definition   \\ 
\midrule
\TOline{prefix} {tkzPt} {point name prefix}
\end{tabular}
\end{NewMacroBox}   

\subsubsection{Mark and plot; use of \tkzcname{tkzDrawMarks}}
\begin{tkzexample}[latex=7cm,small]
  \begin{tikzpicture}
  \tkzInit[xmax=6,ymin=1000,
          ymax=5000,ystep=1000]
  \tkzAxeXY 
  \tkzDefSetOfPoints[prefix=P]{%
        1/2000, 
        2/3000,
        4/2500, 
        5/4200} 
  \tkzDrawSegments[color=brown!50]%
(P1,P2 P2,P3 P3,P4)  
  \tkzDrawMarks[mark=ball](P1,P2,P3,P4)
  \end{tikzpicture}
\end{tkzexample}
 
 
\endinput