\section{Galerie}
\subsection{Tableaux de signes}
L'exemple suivant provient de la documentation de l'excellent \tkzname{tablor.sty}.


\medskip


\begin{tkzexample}[vbox,small]
  \begin{tikzpicture}
    \tkzTabInit[lgt=3]
       {$x$                                  /1,
        Signe de\\ $-2+3$                    /1.5,
        Signe de\\ $-x+5$                    /1.5,
        Signe de\\ $(-2x+3)(-x+5)$           /1.5 }%
       {$-\infty$,$\dfrac{3}{2}$,$5$,$+\infty$}%
    \tkzTabLine { ,+,z,-,t,-, }
    \tkzTabLine { ,+,t,+,z,-, }
    \tkzTabLine { ,+,z,-,z,+, }
  \end{tikzpicture}
\end{tkzexample}
 

\subsection{Variations de fonctions}
 \subsubsection{Variation d'une fonction rationnelle}
 
 Cet exemple a été cité dans la documentation du package \tkzname{tabvar}

 Étude de la fonction $f~:~ x \longmapsto \frac{x^3+2}{2x}$ sur $]-\infty~;~+\infty[$
 
\begin{tkzexample}[vbox,small]
\begin{tikzpicture}
  \tkzTabInit[]
  {$x$  /1, $f'(x)$ /1,$f$  /3}
  {$-\infty$ , $0$ , $1$ , $+\infty$}
\tkzTabLine{,-,d,-,z,+,}
\tkzTabVar{+/$+\infty$ ,-D+/$-\infty$ / $+\infty$ ,-/$\frac{3}{2}$, +/$+\infty$}
\tkzTabVal{1}{2}{0.4}{$ -\sqrt[3]{2}$}{$0$}
\end{tikzpicture}
\end{tkzexample}  


\subsubsection{Variation d'une fonction irrationnelle}
 
Autre exemple  cité dans la documentation du package \tkzname{tabvar}

Étude de la fonction $f~:~ x \longmapsto \sqrt{\frac{x-1}{x+1}}$ sur $]-\infty~;~-1[\cup ]1~;~+\infty[$

\begin{tkzexample}[vbox,small] 
\begin{tikzpicture}
  \tkzTabInit[]
  {$x$  /1, $f'(x)$ /1,$f$  /3}
  {$-\infty$ , $-1$ , $1$ , $+\infty$}
\tkzTabLine{,+,d,h,d,+, }
\tkzTabSlope{ 3/ /+\infty}
\tkzTabVar{-/$1$ ,+DH/$+\infty$  ,-/$0$, +/$1$}
\end{tikzpicture}
\end{tkzexample}  

Un prolongement par continuité pourrait être : $f(x)=0$ sur $[-1~;~1]$ alors le tableau deviendrait

\begin{tkzexample}[vbox,small] 
\begin{tikzpicture}
  \tkzTabInit[]
  {$x$  /1, $f'(x)$ /1,$f$  /3}
  {$-\infty$ , $-1$ , $1$ , $+\infty$}
\tkzTabLine{,+,d,0,d,+, }
\tkzTabSlope{ 3/ /+\infty}
\tkzTabVar{-/$1$ ,+D-/$+\infty$/$0$  ,-/$0$, +/$1$}
\end{tikzpicture}
\end{tkzexample}  


\subsection{Fonctions trigonométriques}
\NameFonct{Fonctions trigonométriques}
 \subsubsection{Variation de la fonction tangente}
  \NameFonct{Fonction  tangente}
 Étude de la fonction $f~:~ x \longmapsto \tan{x}$ sur $[0~;\pi]$
 
 \begin{tkzexample}[vbox]
\begin{tikzpicture}
\tkzTabInit[espcl=6]{$x$ / 1,Signe de\\f'(x)/1, Variations de\\ $f$ / 3}%
     {$0$ ,$\frac{\pi}{2}$ , $\pi$}%
\tkzTabLine{ ,+,d,+, }
\tkzTabVar{-/$0$ ,  +D-/$+\infty$/$-\infty$  , +/$0$ }
\tkzTabVal{1}{2}{0.5}{$\frac{\pi}{4}$}{$1$}
\tkzTabVal{2}{3}{0.5}{$\frac{3\pi}{4}$}{$1$}
\end{tikzpicture}
\end{tkzexample}  

\subsubsection{Variation de la fonction cosinus}
 \NameFonct{Fonction  cosinus}
  Étude de la fonction $f~:~ x \longmapsto \cos{x}$ sur $[-\pi~;~+\pi]$
  
 \begin{tkzexample}[vbox]
\begin{tikzpicture}
\tkzTabInit[espcl=6]{$x$ / 1,Signe de\\f'(x)/1, Variations de\\ $f$ / 3}%
     {$0$ , $\pi$}%
\tkzTabLine{ , + , }
\tkzTabVar{+/$1$ ,   -/$-1$ }
\tkzTabVal{1}{2}{0.5}{$\frac{\pi}{2}$}{$0$}
\end{tikzpicture}
\end{tkzexample}  

\subsection{Fonctions paramétrées et trigonométriques}
\NameFonct{Fonctions paramétrées} \NameFonct{Fonctions trigonométriques}

Étude sur $\left[0~;~\frac{\pi}{2}\right]$
 \begin{equation*} 
\left\{%
\begin{array}{l} 
 x(t) = \cos(3t)\\ 
 y(t) = \sin(4t)
\end{array}\right. 
\end{equation*} 

\begin{tkzexample}[vbox,small]
  \begin{tikzpicture}
  \tkzTabInit[ lgt=3 , espcl=3]%
    {$t$                               /1,
    Signe de\\        $x'(t)$          /1.5,
    Variations de\\   $x$              /3,
     Variations de\\  $y$              /3,
    Signe de\\        $y'(t)$          /1.5}
              {$0$   , $\frac{\pi}{8}$             , $\frac{\pi}{3}$ ,
               $\frac{3\pi}{8}$ , $\frac{\pi}{2}$ }%
  \tkzTabLine {z , - ,-3\sin\left(\frac{3\pi}{8}\right) , - , z , +  ,%
   3\sin\left(\frac{\pi}{8}\right),+,3}
   \tkzTabVar  { +/$1$ , R/    , -/$-1$/ , R/     , +/$0$ }
    \tkzTabIma{1}{3}{2}{$\cos\left(\frac{3\pi}{8}\right)$} 
   \tkzTabIma{3}{5}{4}{$-\cos\left(\frac{\pi}{8}\right)$} 
     \tkzTabVar  { -/$0$ , +/$1$ , R/      , -/$-1$ , +/$0$ }
   \tkzTabIma{2}{4}{3}{$\frac{-\sqrt{3}}{2}$} 
   \tkzTabLine {4 , + , z , - , -2 , - ,  z ,+,4}
 \end{tikzpicture}
\end{tkzexample}


\subsection{Baccalauréat Asie ES 1998}
\index{Baccalauréat}
Une petite astuce, en principe \tkzname{z}  est le symbole à mettre dans la liste pour obtenir un zéro centré sur un trait en pointillés. Si on veut que le zéro soit sans le trait , il suffit de remplacer \tkzname{z} par \tkzname{0}. Celui-ci n'est pas un symbole reconnu, il est donc traiter comme une chaîne normale.

  Soit $f$ la fonction de variable réelle $x$, définie sur $\mathbf{R}$ par :
  \[
      f(x)=\E^x(\E^x+a)+b
  \]
  où $a$ et $b$ sont deux constantes réelles.

  Les renseignements connus sur $f$ sont donnés dans le tableau de variation ci-dessous.
   
  \medskip
  \begin{center}
    \begin{tikzpicture}
    \tkzTab[lgt=3,espcl=4]{$x$/1,Signe de $f'(x)$ /1,Variations de $f$ /2}%
    {$-\infty$,$0$,$+\infty$}%
    {,, z ,,}%
     {+/    $-3$  ,
      -/          ,
      +/          } 
    \end{tikzpicture}
  \end{center}

  \medskip
  \begin{enumerate}
      \item Calculer $f'(x)$ en fonction de $a$ ($f'$ désigne la fonction dérivée de $f$).
      \item \begin{enumerate}
             \item déterminer $a$ et $b$ en vous aidant des informations contenues dans le
              tableau ci-dessus.
             \item Calculer $f(0)$ et calculer la limite de $f$ en $+\infty$.
             \item Compléter, après l'avoir reproduit, le tableau de variations de $f$.
             \end{enumerate}
      \item Résoudre dans $\mathbf{R}$ l'équation 
      \[
          \E^x(\E^x-2)-3=0
      \]
      (on pourra pose $X=\E^x$).
      \item Résoudre dans $\mathbf{R}$ les inéquations : 
      \[
          \E^x(\E^x-2)-3\geq -4
      \]
      \[
          \E^x(\E^x-2)-3 \leq 0
      \]
       (On utilisera le tableau de variations donné ci-dessus et en particulier les
        informations obtenues en 2.b)
  \end{enumerate}

\begin{tkzexample}[code only,small]
  Soit $f$ la fonction de variable réelle $x$, définie sur $\mathbf{R}$ par :
  \[
      f(x)=\E^x(\E^x+a)+b
  \]
  où $a$ et $b$ sont deux constantes réelles.

  Les renseignements connus sur $f$ sont donnés dans le tableau de variation ci-dessous.
   
  \medskip
  \begin{center}
    \begin{tikzpicture}
    \tkzTab[lgt=3,espcl=4]{$x$/1,Signe de $f'(x)$ /1,Variations de $f$ /2}%
    {$-\infty$,$0$,$+\infty$}%
    {,, z ,,}%
     {+/    $-3$  ,
      -/          ,
      +/          } 
    \end{tikzpicture}
  \end{center}

  \medskip
  \begin{enumerate}
      \item Calculer $f'(x)$ en fonction de $a$ ($f'$ désigne la fonction dérivée de $f$).
      \item \begin{enumerate}
             \item déterminer $a$ et $b$ en vous aidant des informations contenues dans le
              tableau ci-dessus.
             \item Calculer $f(0)$ et calculer la limite de $f$ en $+\infty$.
             \item Compléter, après l'avoir reproduit, le tableau de variations de $f$.
             \end{enumerate}
      \item Résoudre dans $\mathbf{R}$ l'équation 
      \[
          \E^x(\E^x-2)-3=0
      \]
      (on pourra pose $X=\E^x$).
      \item Résoudre dans $\mathbf{R}$ les inéquations : 
      \[
          \E^x(\E^x-2)-3\geq -4
      \]
      \[
          \E^x(\E^x-2)-3 \leq 0
      \]
       (On utilisera le tableau de variations donné ci-dessus et en particulier les
        informations obtenues en 2.b)
  \end{enumerate}
\end{tkzexample}


\subsection{Baccalauréat}
\index{Baccalauréat}
	On considère la fonction $f$ définie sur $]-\infty~;~0[$ :

	\[
	  f(x)=ax+b+\ln(-2x)
	\]
	où $a$ et $b$ sont deux réels donnés.

	\begin{enumerate}
	\item Calculer $f'(x)$ en fonction de $a$ et $b$.
	\item Le tableau ci-dessous représente les variations d'une fonction particulière $f$.

	\medskip
	\begin{center}
	  \begin{tikzpicture}
	  \tkzTab[]%
	  {$x$/1.25,Signe de\\ $f'(x)$/1.5, Variations\\ de $f$/1.5}%
	  {$-\infty$,$\dfrac{-1}{2}$,$0$}%
	  {,+,$0$,-,}%
	  {-//,
	  +/$2$/,
	  -//}
	  \end{tikzpicture}
	\end{center}

	\medskip
	\begin{enumerate}
	\item En utilisant les données du tableau déterminer les valeurs $a$ et $b$ qui caractérisent
	 cette fonction.
	\item Pour cette fonction particulière $f$, déterminer 
	        $\displaystyle \lim_{x \xrightarrow[x<0]{} 0} f(x)$.
	\item Montrer que, dans l'intervalle $\Big[\dfrac{-1}{2}~;~0,01\Big]$, l'équation $f(x)=0$
	 admet une solution unique. En donner une valeur approchée à $10^{-3}$ près.
	\end{enumerate}
	\end{enumerate}
	
\begin{tkzexample}[small,code only]
	On considère la fonction $f$ définie sur $]-\infty~;~0[$ :

	\[
	  f(x)=ax+b+\ln(-2x)
	\]
	où $a$ et $b$ sont deux réels donnés.

	\begin{enumerate}
	\item Calculer $f'(x)$ en fonction de $a$ et $b$.
	\item Le tableau ci-dessous représente les variations d'une fonction particulière $f$.

	\medskip
	\begin{center}
	  \begin{tikzpicture}
	  \tkzTab[]%
	  {$x$/1.25,Signe de\\ $f'(x)$/1.5, Variations\\ de $f$/1.5}%
	  {$-\infty$,$\dfrac{-1}{2}$,$0$}%
	  {,+,$0$,-,}%
	  {-//,
	  +/$2$/,
	  -//}
	  \end{tikzpicture}
	\end{center}

	\medskip
	\begin{enumerate}
	\item En utilisant les données du tableau déterminer les valeurs $a$ et $b$ qui caractérisent
	 cette fonction.
	\item Pour cette fonction particulière $f$, déterminer 
	        $\displaystyle \lim_{x \xrightarrow[x<0]{} 0} f(x)$.
	\item Montrer que, dans l'intervalle $\Big[\dfrac{-1}{2}~;~0,01\Big]$, l'équation $f(x)=0$
	 admet une solution unique. En donner une valeur approchée à $10^{-3}$ près.
	\end{enumerate}
	\end{enumerate}
\end{tkzexample}



\subsection{Baccalauréat Guyane ES 1998 }
\index{Baccalauréat}
C'est cet exemple qui m'a obligé à penser aux   commandes du style $+V+$. Sans doute, voulait-on ne pas influencer les élèves avec la vision d'une double barre (trop souvent associée à la présence d'une asymptote).

\textbf{Le sujet :}

{\parindent=0pt
On considère une fonction $f$ de la variable $x$, dont on donne le tableau de variations :

\begin{center}
\begin{tikzpicture}
\tkzTab[lgt=3]%
{$x$/1.25,Signe de\\ $f'(x)$/1.5, Variations\\ de $f$/2.5}
{$-\infty$,$\dfrac{-1}{2}$,$1$,$+\infty$}
{,-,$0$,+, ,-,}
{+/ $1$ , -/$\dfrac{-1}{3}$ , +V+/ $+\infty$ /$+\infty$ , -/$1$}
\end{tikzpicture}
\end{center}

On appelle (C) la courbe représentative de $f$ dans un repère Le plan est muni d'un repère orthonormé $(O;\vec{\imath};\vec{\jmath})$ (unités graphiques 2 cm sur chaque axe)
 
\vspace{6pt}
\textbf{Première partie}

En interprétant le tableau donné ci-dessus :%
 
 \begin{enumerate}
 \item  Préciser l'ensemble de définition de $f$.
 \item  Placer dans le repère $(O;\vec{\imath};\vec{\jmath})$  :
 \begin{enumerate}
 \item  l'asymptote horizontale (D);
 \item  l'asymptote verticale (D');
 \item  le point $A$ où la tangente à (C) est horizontale.
 \end{enumerate}
 \end{enumerate}

\textbf{Seconde partie}

On donne maintenant l'expression de $f$ :
\[
f(x)=1 + \dfrac{4}{(x-1)}    + \dfrac{3}{(x-1)^2}
\]
\begin{enumerate}
 \item Résoudre les équations $f(x)=0$ et $f(x)=1$.
 \item  Au moyen de votre calculatrice, remplir le tableau suivant
  ( recopier ce tableau sur votre copie).
\end{enumerate}
\begin{tikzpicture}
   \tkzTabInit[deltacl=1,espcl=1]{ $x$/1 , $f(x)$ /1}%
              {-1,,{-0,75},,{0,5},,2,,3,,4}%
   \tkzTabLine{,,,,,,,,,,,,,,,,,,,,}%
   \makeatletter
   \foreach \x in {1,...,5}
    \setcounter{tkz@cnt@pred}{\x}\addtocounter{tkz@cnt@pred}{\x}
    \draw (N\thetkz@cnt@pred 0.center) to (N\thetkz@cnt@pred 2.center);
\end{tikzpicture}
}


\begin{tkzexample}[code only,small]
On considère une fonction $f$ de la variable $x$, dont on donne le tableau de variations :

\begin{center}
\begin{tikzpicture}
\tkzTab[]%
{$x$/1.25,Signe de\\ $f'(x)$/1.5, Variations\\ de $f$/2.5}
{$-\infty$,$\dfrac{-1}{2}$,$1$,$+\infty$}
{,-,$0$,+, ,-,}
{+/ $1$ , -/$\dfrac{-1}{3}$ , +V+/ $+\infty$ /$+\infty$ , -/$1$}
\end{tikzpicture}
\end{center}

 \vspace{6pt}
On appelle (C) la courbe représentative de $f$ dans un repère. Le plan est muni d'un repère%
 orthonormal  $(O;\vec{\imath};\vec{\jmath})$ (unités graphiques 2 cm sur chaque axe)%

\textbf{Première partie}

En interprétant le tableau donné ci-dessus :%
 
 \begin{enumerate}
 \item  Préciser l'ensemble de définition de $f$.
 \item  Placer dans le repère $(O;\vec{\imath};\vec{\jmath})$  :
 \begin{enumerate}
 \item  l'asymptote horizontale (D);
 \item  l'asymptote verticale (D');
 \item  le point $A$ où la tangente à (C) est horizontale.
 \end{enumerate}
 \end{enumerate}

\textbf{Seconde partie}

On donne maintenant l'expression de $f$ :
\[
f(x)=1 + \dfrac{4}{(x-1)}    + \dfrac{3}{(x-1)^2}
\]
\begin{enumerate}
 \item Résoudre les équations $f(x)=0$ et $f(x)=1$.
 \item  Au moyen de votre calculatrice, remplir le tableau suivant
  ( recopier ce tableau sur votre copie).
   \begin{tikzpicture}
   \tkzTabInit[deltacl=1,espcl=1]{ $x$/1,$f(x)$ /1}%
   {-1,,{-0,75},,{0,5},,2,,3,,4}%
   \tkzTabLine{,,,,,,,,,,,,,,,,,,,,}%
   \makeatletter
   \foreach \x in {1,...,5}
    \setcounter{tkz@cnt@pred}{\x}\addtocounter{tkz@cnt@pred}{\x}
    \draw (N\thetkz@cnt@pred 0.center) to (N\thetkz@cnt@pred 2.center);
    \end{tikzpicture}
\end{enumerate}

\vfill
\end{tkzexample}

\subsection{Exemple relatif à une question: problème de virgule}

Il suffit pour régler ce genre de problème d'utiliser le package numprint.
La macro |\np| permet d'afficher correctement $1,5$.
\begin{tkzexample}[code only, small]
\usepackage[french]{babel}
\usepackage[np]{numprint}
\end{tkzexample}

\begin{tkzexample}[vbox]
\begin{tikzpicture}[scale=1]
\tkzTabInit[lgt=6,espcl=7]
{$x$/0.75,
$x\np{-1.5}$/0.75}
{$-\infty$,$+\infty$}
\end{tikzpicture}
\end{tkzexample}

\subsection{Quelques tableaux classiques}

Tableau 1 (ln) :

\begin{tkzexample}[vbox,small]
  \begin{tikzpicture}
  \tkzTabInit[espcl=6]{$x$/1,$\ln'(x)$/1,$\ln(x)$/2}{$0$,$+\infty$}
  \tkzTabLine{d,+,}
  \tkzTabVar{D-/$-\infty$,+/$+\infty$,}
  \tkzTabVal{1}{2}{0.4}{1}{0}
  \tkzTabVal{1}{2}{0.67}{$\E$}{1}
  \end{tikzpicture}
\end{tkzexample}


\bigskip

Tableau 2 (racine) :

\begin{tkzexample}[vbox,small]
\begin{tikzpicture}
\tkzTabInit[lgt=3,espcl=6]{$x$/1,$f'(x)$/1,$f(x)=\sqrt{x}$/2}{$0$,$+\infty$}
\tkzTabLine{d,+,}
\tkzTabVar{-/0,+/$+\infty$,}
\end{tikzpicture}
\end{tkzexample}

\bigskip

Tableau 3 (inverse) :

\begin{tkzexample}[vbox,small]
\begin{tikzpicture}
\tkzTabInit[lgt=3,espcl=6]{$x$/1,$f'(x)=-\dfrac{1}{\ x^2}$/1,$f(x)=\dfrac{1}{x}$/2}{$-\infty$,$0$,$+\infty$}
\tkzTabLine{,-,d,-,}
\tkzTabVar{+/0,-D+/$-\infty$/$+\infty$,-/0}
\end{tikzpicture}
\end{tkzexample}

\bigskip

Tableau 4 (\(f(x)=\dfrac{x^2+4}{x}\)):
\begin{tkzexample}[vbox,small]		
\begin{tikzpicture}
\tkzTabInit[]{\(x$/1,$f'(x)\)/1,\(f(x)\)/2}{\(-\infty\),\(-2\),\(0\),2,\(+\infty\)}
\tkzTabLine{,+,z,-,d,-,z,+,}
\tkzTabVar{-/\(-\infty\),+/-4,-D+/$-\infty$/$+\infty$,-/4,+/\(+\infty\)}
\end{tikzpicture}
\end{tkzexample}

\bigskip


Tableau 5 (cos):
\begin{tkzexample}[vbox,small]
\begin{tikzpicture}
\tkzTabInit{$x$/1,$-\sin(x)$/1,$\cos(x)$/2}{$0$,$\pi$,$2\pi$}
\tkzTabLine{z,-,z,+,z}
\tkzTabVar{+/$1$,-/$-1$,+/$1$}
\tkzTabVal{1}{2}{0.5}{$\dfrac{\pi}{2}$}{0}
\tkzTabVal{2}{3}{0.5}{$\dfrac{3\pi}{2}$}{0}
\end{tikzpicture}
\end{tkzexample}

\subsection{Utilisation de la macro \tkzcname{par}}

Et bien c'est impossible. La meilleure solution est d'utiliser la macro |\parbox| sinon l'emploi de |\endgraf| définit par |\let\endgraf=\par| peut faire l'affaire.

\begin{tkzexample}[width=7cm,small]
  \begin{tikzpicture}
  \tkzTabInit {$x$ / 1 ,$f(x)$ /1}%
  {$0$,$1$}
  \tkzTabLine{,\parbox{3cm}{$u(x)$\\ $v(x)$},}
  \end{tikzpicture}
\end{tkzexample}

\subsection{Exemple utiisant l'option help}

\begin{tkzexample}[vbox,small]
  \begin{tikzpicture}
  \tkzTabInit[color, colorT = red!20, colorC = yellow!20,
   colorL = cyan!40,  colorV = lightgray!20, espcl=3,help]
        {$x$ /1, $f''$ /1,$f'$ /2,  $f$ /2}
        {$-\infty$ , $0$ ,$+\infty$}
  \tkzTabLine{, - , z , + ,}
  \tkzTabVar{+/$+\infty$ , -/$-2$ , +/$+\infty$}
  \tkzTabVal[draw]{1}{2}{.6}{$x_1$}{$0$}
  \tkzTabVal[draw]{2}{3}{.4}{$x_2$}{$0$}
  \end{tikzpicture}
\end{tkzexample}


\begin{tkzexample}[vbox,small]
  \begin{tikzpicture}
  \tikzset{arrow style/.style   = {black,
  >->           = latex’,thick ,
  shorten >   =  5pt,
  shorten <   =  5pt}}
  \tkzTabInit[color, colorT = red!20, colorC = yellow!20,
   colorL = cyan!40,  colorV = lightgray!20, espcl=3]
        {$x$ /1, $f''$ /1,$f'$ /2,  $f$ /2}
        {$-\infty$ , $0$ ,$+\infty$}
  \tkzTabLine{, - , z , + ,}
  \tkzTabVar{+/$+\infty$ , -/$-2$ , +/$+\infty$}
  \tkzTabVal[draw]{1}{2}{.6}{$x_1$}{$0$}
  \tkzTabVal[draw]{2}{3}{.4}{$x_2$}{$0$}
  \begin{scope}[>->,line width=1pt,>=stealth]
  \path (N13) -- (N23) node[midway,below=6pt](N){};
  \draw ([above=6pt]N14) to [bend left=45] ([left=1pt]N);
  \draw ([right=3pt]N) to [bend left=45] ([above=6pt]N24)  ;
  \draw ([above right=6pt]N24)to [bend right=40] ([below left=6pt]N33);
  \end{scope}
  \end{tikzpicture}
\end{tkzexample}

\subsection{Exemple modifiant la largeur d'une colonne}
Cette modification n'est pas prévue. Une astuce consiste à intoduire des colonnes vides.

\begin{tkzexample}[vbox,small]
  \begin{tikzpicture}
      \tikzset{t style/.style = {style = densely dashed}} 
      \tkzTabInit[lgt=4, espcl = 1.2, lw = 0.5pt, deltacl=0]
      { / 0.7 , $x(x^2-5x+4)$ / 1}{ ,$0$, $1$, , $4$,}
      \tkzTabLine{,-,t,+,t,,-,,t, +,}
      \draw[fill=black] (N21) circle(2pt);
      \draw[fill=black] (N31) circle(2pt);
      \draw[fill=black] (N51) circle(2pt);
      \draw[->=stealth, line width=1.5pt] (N11) -> (N61.east) node[right=2pt] {$x$};
  \end{tikzpicture}
\end{tkzexample}

\endinput