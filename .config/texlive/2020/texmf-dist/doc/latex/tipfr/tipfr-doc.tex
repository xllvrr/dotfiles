\documentclass[10pt,french,a4paper]{article}

\usepackage[utf8]{inputenc}
\usepackage[T1]{fontenc}
\usepackage{mathpazo}
\usepackage[euler-digits]{eulervm}
\usepackage[dvipsnames,table]{xcolor}
\usepackage{tipfr}
\usepackage[margin=2cm]{geometry}
\usepackage{tabularx,titlesec}
\usepackage{babel}
\usepackage[pdfborder={0 0 0},bookmarksnumbered,pdfpagelabels]{hyperref}
\usepackage{tcolorbox}
\tcbuselibrary{listingsutf8,documentation}
\tcbset{color key=blue,color value=ForestGreen,index colorize=true,doclang/keys=options,doclang/key=option,doclang/values=valeurs,doclang/value=valeur}
\usepackage{makeidx}
\makeindex

\makeatletter
\newcommand*{\Spec@nostar}{Cette option s'applique également aux touches spéciales}
\newcommand*{\Spec@star}{Cette option ne s'applique pas aux touches spéciales}
\newcommand*\Speciale{\@ifstar{\Spec@star}{\Spec@nostar}}
\makeatother

%%% FORMAT SECTIONS DE mathbook.cls de Stéphane PASQUET

% couleurs section
\definecolor{section@title@color}{cmyk}{1,0.2,0.3,0.1}
\definecolor{subsection@title@color}{cmyk}{0,0.6,0.9,0}
\definecolor{shadow@color}{cmyk}{.07,0,0,0.49}
% fontes section
\def\sectiontitle@font{\fontfamily{ugq}\selectfont}
\def\subsectiontitle@font{\fontfamily{ugq}\selectfont}
% Décalages numéro de sections / titres des sections
\newlength\decalnumsec
\newlength\decalnumsubsec
\setlength{\decalnumsec}{-0.5em}
\setlength{\decalnumsubsec}{-0.5em}
\newlength\decalxtitlesubsec
\setlength{\decalxtitlesubsec}{\parindent}
% Espace entre le numéro de section et le titre
\newlength\spacetitlesec
\newlength\spacetitlesubsec
\setlength{\spacetitlesec}{0.5em}
\setlength{\spacetitlesubsec}{0.2em}

%%%%%%%%%%%%% Titre de section

\renewcommand{\thesection}{\Roman{section}}
\titleformat{\section}[block]
{%
	\bfseries\Large
	\color{section@title@color}
	\sectiontitle@font
}
{
\raisebox{\decalnumsec}
{%
\begin{tikzpicture}
\node (numsec) {\sectiontitle@font\thesection};
\fill[rounded corners=2pt,fill=shadow@color] ($(numsec.north west)+(2pt,-2pt)$) -- ($(numsec.north east)+(1mm,0mm)+(2pt,-2pt)$) -- ($(numsec.south east)+(2pt,-2pt)$) -- ($(numsec.south west)+(-1mm,0)+(2pt,-2pt)$) -- cycle;
\fill[rounded corners=2pt,fill=section@title@color] (numsec.north west) -- ($(numsec.north east)+(1mm,0mm)$) -- (numsec.south east) -- ($(numsec.south west)+(-1mm,0)$) -- cycle;
\node[white] at (numsec) {\sectiontitle@font\thesection};
\end{tikzpicture}
}
}
{\spacetitlesec}
{}

%%%%%%%%%%%%% Titre de subsection

\renewcommand{\thesubsection}{\arabic{subsection}}
\titleformat{\subsection}[block]
{%
	\hspace*{\decalxtitlesubsec}
	\bfseries\large
	\color{subsection@title@color}
	\subsectiontitle@font
}
{
	\raisebox{\decalnumsubsec}
	{%
	\begin{tikzpicture}
	\node (numsec) {\subsectiontitle@font\thesubsection};
	\fill[rounded corners=2pt,fill=shadow@color] ($(numsec.north west)+(2pt,-2pt)$) -- ($(numsec.north east)+(1mm,0mm)+(2pt,-2pt)$) -- ($(numsec.south east)+(2pt,-2pt)$) -- ($(numsec.south west)+(-1mm,0)+(2pt,-2pt)$) -- cycle;
	\fill[rounded corners=2pt,fill=subsection@title@color] (numsec.north west) -- ($(numsec.north east)+(1mm,0mm)$) -- (numsec.south east) -- ($(numsec.south west)+(-1mm,0)$) -- cycle;
	\node[white] at (numsec) {\subsectiontitle@font\thesubsection};
	\end{tikzpicture}
	}
}
{%
	\spacetitlesubsec
}
{%
	%\itemclass{subsection@title@color}{\subsection@font}
}

\begin{document}
\thispagestyle{empty}

\begin{center}
    \begin{tcolorbox}[enhanced,lifted shadow={1mm}{-2mm}{3mm}{0.1mm}{black!50!white},width=0.65\linewidth]
    \Huge\bfseries\centering tipfr.sty
    \end{tcolorbox}\medskip

    \Calculatrice[documentation]

    \today, version 1.5\medskip

    Philippe \bsc{De Sousa} (\href{mailto:philou.desousa@gmail.com}{philou.desousa@gmail.com})
\end{center}

\begin{abstract}
    Travaillant en lycée, je suis souvent amené à travailler avec les élèves sur une calculatrice graphique. La technologie d'aujourd'hui nous permet de vidéo-projeter la calculatrice et manipuler en même temps que les élèves pour leur montrer les différentes fonctionnalités de toutes les touches.\par
    Mais une fois chez eux, comment se souvenir de ce qui a été fait en classe ?\par
    J'ai créé ce package en m'appuyant sur un modèle de calculatrice répandu dans mon lycée afin de constituer des fiches méthodes que les élèves pourront utiliser à la maison et conserver d'année en année.
\end{abstract}

\tableofcontents
\vspace*{2cm}

\section{Les touches}

\subsection{Dessiner une touche}

\begin{docCommand}{Touche}{\oarg{options}}
    Voici la commande qui fournit tous les dessins de touche. L'appel à la commande \cs{Touche} sans aucune option réalise la touche $\sin$ par défaut.
\end{docCommand}

\begin{dispExample*}{sidebyside}
\Touche
\end{dispExample*}

Pour obtenir d'autres types de touches, on utilise alors différentes options auxquelles on spécifie une valeur :

\begin{docKey}{style}{=\meta{text}}{valeur par défaut : \docValue{function}}
Crée un dessin de touche selon le style précisé. Les différentes valeurs sont :
\end{docKey}
\begin{description}
    \item[\docValue{function}]  dessine une touche noire pour les différentes fonctions de la calculatrice ;
    \item[\docValue{number}] dessine une touche blanche un peu plus grosse que les précédentes pour indiquer les chiffres ;
    \item[\docValue{operation}] dessine des touches grises pour les opérations ;
    \item[\docValue{graph}] dessine des touches grises plus fines que les précédentes pour les options liées aux dessins de graphiques de la calculatrice ;
    \item[\docValue{second}] dessine la touche spéciale \textit{2nde} qui permet d'accéder aux fonctions secondaires des touches de la calculatrice ;
    \item[\docValue{alpha}] dessine la touche spéciale \textit{alpha} qui permet d'accéder aux fonctions alphabétiques des touches de la calculatrice ;
    \item[\docValue{on}] dessine la touche spéciale \textit{on/off} ;
    \item[\docValue{enter}] dessine la touche spéciale \textit{entrer} qui permet d'évaluer un résultat ;
    \item[\docValue{arrows}] dessine la touche spéciale qui représente les quatres flèches de la calculatrice.
\end{description}

\begin{dispExample*}{sidebyside, center lower}
\Touche[style=function]
\Touche[style=number]
\Touche[style=operation]
\Touche[style=graph]
\Touche[style=second]

\Touche[style=alpha]
\Touche[style=enter]
\Touche[style=on]
\Touche[style=arrows]
\end{dispExample*}

Hormis pour les touches spéciales, on constate que la valeur principale par défaut est $\sin$. De plus, la touche graphique nous montre un problème d'alignement. Des options ont donc été créées pour apporter une solution :

\begin{docKey}{principal}{=\meta{text}}{valeur par défaut : \docValue{sin}}
Précise le \meta{text} à mettre à l'intérieur d'une touche. \Speciale* sauf la touche spéciale \docValue{enter}. Une fois encore on constate un problème de positionnement.
\end{docKey}

\begin{dispExample*}{sidebyside}
\Touche[style=function, principal={\large ,}]
\Touche[style=number, principal=2]
\Touche[style=operation, principal=$\times$]
\Touche[style=graph,principal=$f(x)$]
\Touche[style=enter, principal={=}]
\end{dispExample*}

La commande \cs{Circonflexe} a été créée pour dessiner un grand accent circonflexe : \Circonflexe

\begin{dispExample*}{sidebyside}
\Touche[style=function, principal=\Circonflexe]
\end{dispExample*}

\begin{docKey}{position}{=\meta{nombre}}{valeur par défaut : \docValue{0.7}}
Permet d'ajuster la position du texte principal à l'intérieur de la touche. \Speciale.
\end{docKey}

\begin{dispExample*}{sidebyside}
\Touche[style=on]
\Touche[style=on, position = 0.1]

\Touche[style=graph,principal=$f(x)$]
\Touche[style=graph,principal=$f(x)$, position = 0.9]

\Touche[style=enter, principal={\large =}]
\Touche[style=enter, principal={\large =}, position = 0.35]
\end{dispExample*}

\begin{docKey}{raise}{=\meta{dim}}{valeur par défaut : \docValue{0ex}}
Permet d'ajuster la hauteur de la touche par rapport à la ligne de base. Les valeurs négatives sont autorisées. \Speciale.
\end{docKey}

\noindent\begin{minipage}[t]{0.48\linewidth}
Hauteurs non modifiées\par
\begin{dispExample*}{center lower}
\Touche[style=graph,principal=$f(x)$, position = 0.9]
\Touche[style=number, principal=3]
\Touche[style=operation, principal=$\times$]
\Touche[style=function, principal={$x,t,\theta,n$}]
\Touche[style=enter, principal=entrer]
\end{dispExample*}
\end{minipage}\hfill
\begin{minipage}[t]{0.48\linewidth}
Hauteurs modifiées\par
\begin{dispExample*}{center lower}
\Touche[style=graph,principal=$f(x)$, position = 0.9]
\Touche[style=number, principal=3, raise=1.5ex]
\Touche[style=operation, principal=$\times$, raise=2ex]
\Touche[style=function, principal={$x,t,\theta,n$}, raise=2ex]
\Touche[style=enter, principal=entrer, raise=0.5ex]
\end{dispExample*}
\end{minipage}

\begin{docKey}{fontsize}{=\meta{dim}}{valeur par défaut : \docValue{8pt}}
    L'exemple précédent montre cette fois un débordement horizontal. L'option \docAuxKey{fontsize} permet de modifier ponctuellement la taille de la fonte utilisée dans la touche. \Speciale.
\end{docKey}
\begin{dispExample*}{sidebyside}
\Touche[style=function, principal={$x,t,\theta,n$}]
\Touche[style=function, principal={$x,t,\theta,n$},fontsize=6pt]
\end{dispExample*}
\bigskip

Les touches ne sont pas uniquement composées de leur fonction principale. Parfois, elles possèdent une fonction secondaire appelée à l'aide de la touche \Touche[style=second,raise=-1ex] et parfois même une fonction alphabétique appelée à l'aide de la touche \Touche[style=alpha,raise=-1ex].

\begin{docKey}{second}{=\meta{text}}{\sffamily fonction inactive par défaut}
    \'Ecrit en bleu une fonction secondaire au dessus de la touche. Lorsque l'option \docAuxKey{second} est spécifié sans l'option \docAuxKey{alpha} alors le \meta{text} est centré au dessus de la touche. \Speciale* sauf la touche spéciale \docValue{enter}.
\end{docKey}

\begin{dispExample*}{sidebyside}
\Touche[principal={suppr}, second={insérer}, position=0.65]
\end{dispExample*}

\begin{docKey}{alpha}{=\meta{text}}{\sffamily fonction inactive par défaut}
        \'Ecrit en vert une fonction alphabétique au dessus de la touche. Lorsque l'option \docAuxKey{alpha} est spécifié sans l'option \docAuxKey{second} alors rien n'est affiché. En effet, sur la calculatrice prise en modèle, il n'existe aucune touche possédant une fonction alphabétique sans avoir de fonction secondaire. \Speciale* sauf la touche spéciale \docValue{enter}.
\end{docKey}

\begin{dispExample*}{sidebyside}
\Touche[style=graph, principal=$f(x)$,second={gr.stats}, alpha=f1, position=0.95, fontsize=7pt]\quad
\Touche[style=number, principal={7}, second={$u_n$}, alpha={O}]\quad
\Touche[style=number, principal={7}, alpha={O}]
\Touche[style=enter,fontsize=7pt, principal=entrer, second=préc, alpha=résol]
\end{dispExample*}

L'option \docAuxKey{fontsize} est ici spécifiée afin de permettre au texte secondaire et au texte alphabétique de cohabiter et d'éviter une \textit{badbox}.\bigskip

La commande \cs{Racine} a été créée pour dessiner une racine carré : \Racine

\begin{dispExample*}{sidebyside}
\Touche[principal={$x^2$}, second={\Racine}, alpha={I}]
\end{dispExample*}

\subsection{Entourer une touche}

\begin{docKey}{circle}{\docValue*{=true|false}}{valeur par défaut : \docValue{false}}
    Permet d'entourer la touche à l'aide d'un cercle dont on peut alors préciser le rayon, l'épaisseur et la couleur. \Speciale\ sauf pour le style \docValue{arrows} qui bénéficie d'un traitement particulier.
\end{docKey}

\begin{docKey}{radius}{=\meta{dim}}{valeur par défaut : \docValue{20pt}}
    On spécifie ici le rayon du cercle qui ne sera pris en compte que si \docAuxKey{circle}=\docValue{true}.
\end{docKey}

\begin{docKey}{colour}{=\meta{colour}}{valeur par défaut : \docValue{red}}
    On spécifie ici la couleur du cercle qui ne sera prise en compte que si \docAuxKey{circle}=\docValue{true}. L'option \docAuxKey{color} est également possible.
\end{docKey}

\begin{docKey}{thickness}{=\meta{dim}}{valeur par défaut : \docValue{1pt}}
    On spécifie ici l'épaisseur du cercle qui ne sera prise en compte que si \docAuxKey{circle}=\docValue{true}.
\end{docKey}

\begin{dispExample*}{sidebyside}
\Touche[principal={$x,t,\theta,n$}, second={échanger}, fontsize=7pt, position=0.8, circle=true]
\Touche[style=enter,fontsize=7pt, principal=entrer, second=préc, alpha=résol, circle=true, thickness=0.5pt, colour=blue, radius=25pt]
\end{dispExample*}

\subsection{Touches flèches}

Les touches flèches ont un statut bien particulier qui leur permet d'avoir des commandes qui leur sont spécifiques.

\begin{docKey}{fixed}{\docValue*{=true|false}}{valeur par défaut : \docValue{true}}
    Les flèches sont dessinées à l'endroit même où la commande est appelée. Si on a spécifié \docAuxKey{fixed}=\docValue{false} alors les flèches vont pouvoir se déplacer sur la page.
\end{docKey}

\begin{docKey}{xoffset}{=\meta{dim}}{valeur par défaut : \docValue{0cm}}
    Permet de déplacer la touche horizontalement.
\end{docKey}

\begin{docKey}{yoffset}{=\meta{dim}}{valeur par défaut : \docValue{0cm}}
    Permet de déplacer la touche verticalement.
\end{docKey}

\begin{docKey}{scalearrows}{=\meta{nombre}}{valeur par défaut : \docValue{0.25}}
    Permet de modifier la taille de la touche.
\end{docKey}

\begin{dispExample*}{sidebyside}
Il faut donc appuyer sur les\Touche[style=arrows, raise=-0.15cm, scalearrows=0.25] pour voir les autres options du menu.
\end{dispExample*}

\begin{dispExample*}{sidebyside}
Les flèches\Touche[style=arrows, fixed=false, xoffset=4.5cm, scalearrows=0.5] ont disparu ! Ah non, les voilà :

Attention, des flèches non fixées \Touche[style=arrows, fixed=false, scalearrows=0.5] se superposent au texte !
\end{dispExample*}

Les options de décalage ont été utilisées pour positionner les flèches sur la calculatrice de la première page.

\begin{docKey}{arrowtot}{\docValue*{=true|false}}{valeur par défaut : \docValue{false}}
    Cette option permet d'entourer la touche flèches entièrement.
\end{docKey}

\begin{docKey}{arrowup}{\docValue*{=true|false}}{valeur par défaut : \docValue{false}}
    Cette option permet d'entourer la flèche du haut.
\end{docKey}

\begin{docKey}{arrowdown}{\docValue*{=true|false}}{valeur par défaut : \docValue{false}}
    Cette option permet d'entourer la flèche du bas.
\end{docKey}

\begin{docKey}{arrowleft}{\docValue*{=true|false}}{valeur par défaut : \docValue{false}}
    Cette option permet d'entourer la flèche de gauche.
\end{docKey}

\begin{docKey}{arrowright}{\docValue*{=true|false}}{valeur par défaut : \docValue{false}}
    Cette option permet d'entourer la flèche de droite.
\end{docKey}

Comme pour les autres touches, les options \docAuxKey{thickness} et \docAuxKey{colour} peuvent être employées. En revanche, le rayon du cercle est fixé.

\begin{dispExample*}{sidebyside}
\Touche[style=arrows, arrowtot=true, scalearrows=0.5]
\Touche[style=arrows, arrowleft=true, scalearrows=0.5]
\Touche[style=arrows, arrowup=true, scalearrows=0.5]
\end{dispExample*}

\subsection{Nommer une touche}\label{subsec:NomTouche}

\begin{docKey}{name}{=\meta{text}}{valeur par défaut : \docValue{NOM}}
    La touche sera référencée à l'aide d'un n{\oe}ud nommé \meta{text}.
\end{docKey}

\begin{dispExample}
Pour obtenir le nombre $\pi$ à la calculatrice, on utilise la séquence suivante :
\begin{center}
    \Touche[style=second]
    \Touche[principal={\Circonflexe},second={$\pi$},alpha={H},name=PI]
\end{center}
\begin{tikzpicture}[overlay, remember picture, >=latex']
    \draw[red, line width=1pt] ($(PI)+(-0.5,0.2)$) circle (7pt);
    \draw[blue, line width=0.5pt, <-, rounded corners=10pt]
        ($(PI)+(-0.6,-0.05)$) |- ($(PI)+(0.5,-0.7)$)
        node[right] {La lettre $\pi$ apparaît ici};
\end{tikzpicture}
\end{dispExample}

Les touches sont définies au sein d'un environnement \texttt{tikzpicture}. Afin de pouvoir s'y référer à l'intérieur d'un autre environnement de ce type, il faudra penser à utiliser les options \texttt{overlay} et \texttt{remember picture}. De plus, au minimum deux compilations seront nécessaires.



\section{Créer des menus}

En plus des différentes touches de la calculatrice, on pourra parler aux élèves des menus affichés par la calculatrice

\begin{docCommand}{Menu}{\oarg{options}\marg{nom}}
    Cette commande écrit \meta{nom} en majuscule dans une fonte à chasse fixe de type {\ttfamily machine à écrire} pour nommer un menu de calculatrice. Ce nom est enfermé dans une boîte à fond blanc exactement à sa taille.
\end{docCommand}

\begin{dispExample*}{sidebyside}
\Menu{Math} \Menu{num} \Menu{cpx} \Menu{prb}
\end{dispExample*}

La taille peut être modifiée à l'aide de l'option suivante

\begin{docKey}{size}{=\meta{dim}}{valeur par défaut : \docValue{15pt}}
    Si l'unité de mesure n'est pas spécifié dans \meta{dim}, alors le \texttt{pt} sera utilisé par défaut.
\end{docKey}

\begin{dispExample*}{sidebyside}
\Menu[size=1cm]{Math}
\Menu{num}
\Menu[size=8]{cpx}
\Menu[size=8pt]{prb}
\end{dispExample*}

\begin{docKey}{select}{\docValue*{=true|false}}{valeur par défaut : \docValue{false}}
    Permet d'écrire le nom du menu en blanc sur fond noir pour signifier qu'il est sélectionné.
\end{docKey}

\begin{dispExample*}{sidebyside}
\Menu{Math} \Menu{num}
\Menu[select=true]{cpx}
\Menu{prb}
\end{dispExample*}

\begin{docKey}{colourbox}{=\meta{colour}}{valeur par défaut : \docValue{white}}
    Détermine la couleur la boîte contenant le texte du menu lorsque celui-ci \textit{n'est pas} sélectionné. L'option \docAuxKey{colorbox} est autorisée.
\end{docKey}

\begin{dispExample*}{sidebyside}
\Menu{Math} \Menu{num}
\Menu[select=true]{cpx}
\Menu[colourbox=red]{prb}
\end{dispExample*}

\begin{docKey}{text}{=\meta{text}}{valeur par défaut : \docValue{\symbol{92}unskip}}
    Cette dernière option permet de spécifier si un texte doit être écrit à côté du nom du menu. Pratique pour les menus sous forme de listes verticales. Le \meta{text} est sensible à l'option \docAuxKey{size}.
\end{docKey}

\begin{dispExample*}{sidebyside}
\Menu[size=10pt, text={$\blacktriangleright$Frac}]{1 :}\par
\Menu[size=10pt, select=true, text={$\blacktriangleright$Dec}(]{2 :}\par
\Menu[size=10pt, text=\up{3}]{3 :}
\end{dispExample*}

Voilà par exemple les quatres menus disponibles avec la touche \Touche[principal={math},second={tests},alpha={A},position=0.8,raise=-1ex] :
\begin{center}\renewcommand\tabcolsep{-7pt}
    \begin{tabular}{*{4}{llll|}}
        \Menu[size=10,select=true]{Maths} & \Menu[size=10]{num} & \Menu[size=10]{cpx} & \Menu[size=10]{prb} &
                \Menu[size=10]{Maths} & \Menu[size=10,select=true]{num} & \Menu[size=10]{cpx} & \Menu[size=10]{prb} &
                \Menu[size=10]{Maths} & \Menu[size=10]{num} & \Menu[size=10,select=true]{cpx} & \Menu[size=10]{prb} &
                \Menu[size=10]{Maths} & \Menu[size=10]{num} & \Menu[size=10]{cpx} & \Menu[size=10,select=true]{prb} \\[-8pt]
%
        \multicolumn{4}{l|}{\Menu[select=true, size=9, text={$\blacktriangleright$Frac}]{1 :}} &
        \multicolumn{4}{l|}{\Menu[select=true, size=9, text=abs(]{1 :}} &
        \multicolumn{4}{l|}{\Menu[select=true, size=9, text=conj(]{1 :}} &
        \multicolumn{4}{l|}{\Menu[select=true, size=9, text=NbrAléat]{1 :}} \\[-8pt]
%
        \multicolumn{4}{l|}{\Menu[size=9, text={$\blacktriangleright$Dec}(]{2 :}} &
        \multicolumn{4}{l|}{\Menu[size=9, text=arrondi(]{2 :}} &
        \multicolumn{4}{l|}{\Menu[size=9, text=réel(]{2 :}} &
        \multicolumn{4}{l|}{\Menu[size=9, text=Arrangement]{2 :}} \\[-8pt]
%
        \multicolumn{4}{l|}{\Menu[size=9, text=\up{3}]{3 :}} &
        \multicolumn{4}{l|}{\Menu[size=9, text=ent(]{3 :}} &
        \multicolumn{4}{l|}{\Menu[size=9, text=imag(]{3 :}} &
        \multicolumn{4}{l|}{\Menu[size=9, text=Combinaison]{3 :}} \\[-8pt]
%
        \multicolumn{4}{l|}{\Menu[size=9, text={$^{\text{3}}\sqrt{\phantom x}$(}]{4 :}} &
        \multicolumn{4}{l|}{\Menu[size=9, text=partDéc(]{4 :}} &
        \multicolumn{4}{l|}{\Menu[size=9, text=argument(]{4 :}} &
        \multicolumn{4}{l|}{\Menu[size=9, text=!]{4 :}} \\[-8pt]
%
        \multicolumn{4}{l|}{\Menu[size=9, text={$^{\text{x}}\sqrt{\phantom x}$}]{5 :}} &
        \multicolumn{4}{l|}{\Menu[size=9, text=partEnt(]{5 :}}&
        \multicolumn{4}{l|}{\Menu[size=9, text=abs(]{5 :}} &
        \multicolumn{4}{l|}{\Menu[size=9, text=entAléat(]{5 :}} \\[-8pt]
%
        \multicolumn{4}{l|}{\Menu[size=9, text={$\times$fMin(}]{6 :}} &
        \multicolumn{4}{l|}{\Menu[size=9, text=min(]{6 :}} &
        \multicolumn{4}{l|}{\Menu[size=9, text={$\blacktriangleright$Rect}]{6 :}} &
        \multicolumn{4}{l|}{\Menu[size=9, text=normAléat(]{6 :}} \\[-8pt]
%
        \multicolumn{4}{l|}{\Menu[size=9, text={$\times$fMax(}]{7 $\downarrow$}} &
        \multicolumn{4}{l|}{\Menu[size=9, text=max(]{7 $\downarrow$}}&
        \multicolumn{4}{l|}{\Menu[size=9, text={$\blacktriangleright$Polaire}]{7 $\downarrow$}} &
        \multicolumn{4}{l|}{\Menu[size=9, text=BinAléat(]{7 $\downarrow$}}
    \end{tabular}
\end{center}


\section{Afficher un écran}
\subsection{Généralités}

\begin{docCommand}{Ecran}{\oarg{options}\marg{arguments}}
    Cette commande permet d'afficher un écran de calculatrice.
\end{docCommand}

\begin{dispExample*}{center lower}
\Ecran{}
\end{dispExample*}

On peut modifier l'aspect général à l'aide des options suivantes :

\begin{docKey}{screencolour}{=\meta{colour}}{valeur par défaut : \docValue{ForestGreen}\docValue*{!15}}
    Détermine la couleur de fond de l'écran. \docAuxKey{screencolor} est également possible.
\end{docKey}

\begin{docKey}{screenname}{=\meta{text}}{valeur par défaut : \docValue{ecran}}
    Donne un nom à l'écran afin de pouvoir s'y référencer plus tard avec des environnements \texttt{tikzpicture}. Les mêmes précautions que pour les touches doivent être prises (voir sous-section \ref{subsec:NomTouche} page \pageref{subsec:NomTouche}).
\end{docKey}

\begin{docKey}{width}{=\meta{number}}{valeur par défaut : \docValue{8}}
Permet de fixer la longueur de l'écran. L'unité de mesure est le \texttt{cm}.
\end{docKey}

\begin{docKey}{height}{=\meta{number}}{valeur par défaut : \docValue{5}}
Permet de fixer la largeur de l'écran. L'unité de mesure est le \texttt{cm}.
\end{docKey}

\begin{dispExample*}{sidebyside, center lower}
\Ecran[width=3, height=2, screenname=first]{}
\Ecran[width=3, height=2, screencolour=blue!50, screenname=second]{}
\tikz[remember picture, overlay]{\draw (first.center) circle (5pt);}
\tikz[remember picture, overlay]{\draw (second.north west) -- (second.south east);}
\end{dispExample*}

\subsection{\'Ecran de calculs}

\begin{docCommand}{Ecran}{\oarg{options}\marg{expression/résultat}}
    C'est l'écran par défaut. Il n'existe pas d'options supplémentaires que celles de la sous-section précédente. L'argument obligatoire est une liste de couples \meta{expression/résultat} séparés par une virgule.\par
    On peut ne rien écrire à la place de \meta{expression} ou \meta{résultat} mais, dans ce cas, il ne faut pas mettre d'espace non plus. Les deux peuvent être laissés vides mais alors rien ne se passe (pas de création de ligne vide).\par
    On pensera à utiliser des accolades si l'expression ou le résultat utilise les symboles \texttt{,} ou \texttt{/}.
\end{docCommand}\bigskip

\noindent Voilà un premier exemple un peu long qui utilise la commande \cs{Menu} :\smallskip

\begin{dispExample*}{center lower}
\Ecran[width=6,height=3]{
{\renewcommand\tabcolsep{-7pt}
\begin{tabular}{llll}
\Menu[size=10,select=true]{Maths} & \Menu[colourbox={ForestGreen!15}, size=10]{num} & \Menu[colourbox={ForestGreen!15}, size=10]{cpx} & \Menu[colourbox={ForestGreen!15}, size=10]{prb} \\[-8pt]
\multicolumn{4}{l}{\Menu[select=true, size=9, text={$\blacktriangleright$Frac}]{1 :}} \\[-8pt]
\multicolumn{4}{l}{\Menu[colourbox={ForestGreen!15}, size=9, text={$\blacktriangleright$Dec}(]{2 :}} \\[-8pt]
\multicolumn{4}{l}{\Menu[colourbox={ForestGreen!15}, size=9, text=\up{3}]{3 :}} \\[-8pt]
\multicolumn{4}{l}{\Menu[colourbox={ForestGreen!15}, size=9, text={$^{\text{3}}\sqrt{\phantom x}$(}]{4 :}} \\[-8pt]
\multicolumn{4}{l}{\Menu[colourbox={ForestGreen!15}, size=9, text={$^{\text{x}}\sqrt{\phantom x}$}]{5 :}} \\[-8pt]
\multicolumn{4}{l}{\Menu[colourbox={ForestGreen!15}, size=9, text={$\times$fMin(}]{6 :}} \\[-8pt]
\multicolumn{4}{l}{\Menu[colourbox={ForestGreen!15}, size=9, text={$\times$fMax(}]{7 $\downarrow$}}
\end{tabular}
}/
}
\end{dispExample*}

\noindent Les exemples suivants sont plus courts :\medskip

\begin{dispExample*}{sidebyside, center lower}
\Ecran[screencolour=blue!10, height=3, width=7]%
{{sin(5$\pi$/3)\up 2}/0.75,
3+2/5,
/,
1+2+3+4+5+6+7+8+9+10+11+12+13+14+15+16+17+%
$\blacktriangleright$/5050}
\end{dispExample*}

\begin{dispExample*}{sidebyside, center lower}
\Ecran[height=4, width=4]%
{
PROGRAM:DISTANCE/,
{:Input "XA=",A}/,
{:Input "YA=",B}/,
{:Input "XB=",C}/,
{:Input "YB=",D}/,
{:$\sqrt{\:}$((A-C)\up2+(B-D)\up2)$\to$L}/,
{:Disp "AB=",L}/
}
\end{dispExample*}

\begin{dispExample*}{sidebyside, center lower}
\Ecran[height=4,width=4]%
{
prgmDISTANCE/,
XA=4/,
YA=-1/,
XB=1/,
YB=3/,
AB=/,
/5,
/Done
}
\end{dispExample*}

\subsection{\'Ecran de graphiques}

\begin{docKey}{graphic}{\docValue*{=true|false}}{valeur par défaut : \docValue{false}}
    Cette option là permet de basculer d'un affichage dédié aux calculs à un affichage prévu pour les graphiques. On voit d'ailleurs apparaître des axes gradués.
\end{docKey}%

\begin{dispExample*}{sidebyside, center lower}
\Ecran[width=6, height=4, graphic=true]{}
\end{dispExample*}

\begin{docKey}{xgrad}{=\meta{number}}{valeur par défaut : \docValue{0.5}}
    Détermine l'unité de longueur en \texttt{cm} pour une unité sur l'axe des abscisses.
\end{docKey}

\begin{docKey}{ygrad}{=\meta{number}}{valeur par défaut : \docValue{0.5}}
    Détermine l'unité de longueur en \texttt{cm} pour une unité sur l'axe des ordonnées.
\end{docKey}

\begin{dispExample*}{sidebyside, center lower}
\Ecran[width=6, height=4, graphic=true, xgrad=0.25, ygrad=1]{}
\end{dispExample*}

On constate que, selon l'unité choisie, le nombre de graduations peut être insuffisant.

\begin{docKey}{nbgradx}{=\meta{number}}{valeur par défaut : \docValue{8}}
    Modifie le nombre de graduations sur un \textbf{demi}-axe des abscisses. Le nombre total de graduations est donc doublé.\par
    Cependant, la première graduation est tracée à l'origine donc n'est pas visible (car recouverte par l'axe des ordonnées). De même, il se peut qu'une graduation se retrouve exactement sur un bord de l'écran.
\end{docKey}

\begin{docKey}{nbgrady}{=\meta{number}}{valeur par défaut : \docValue{5}}
    Modifie le nombre de graduations sur un \textbf{demi}-axe des ordonnées. Le nombre total de graduations est donc doublé.\par
    Cependant, la première graduation est tracée à l'origine donc n'est pas visible (car recouverte par l'axe des abscisses). De même, il se peut qu'une graduation se retrouve exactement sur un bord de l'écran.
\end{docKey}

\begin{dispExample*}{sidebyside, center lower}
\Ecran[width=6, height=4, graphic=true, xgrad=0.25, nbgradx=13, ygrad=1, nbgrady=40]{}
\end{dispExample*}

\begin{docKey}{origin}{=\meta{coordonnées}}{valeur par défaut : \docValue{\meta{screenname}.center}}
    Modifie l'origine du repère. Celui-ci est utilisé pour tracer les axes gradués mais aussi les courbes. Pour information, le coin inférieur gauche de l'écran est le point \texttt{(0,0)}. Rappelons également que {\ttfamily \meta{screenname}=ecran} par défaut.
\end{docKey}

\begin{dispExample*}{sidebyside, center lower}
\Ecran[width=6, height=4, graphic=true, origin={(1,1)}, nbgradx=10, nbgrady=6]{}
\end{dispExample*}

\begin{docCommand}{Ecran}{\oarg{options}\marg{fonction/intervalle}}
    Avec cet argument obligatoire, on peut dessiner plusieurs courbes représentatives de fonctions, chacune définie sur un intervalle. Pour cela, la syntaxe suivante a été utilisé :
    \begin{center}
        \ttfamily
        \verb!\draw! plot[domain=\meta{intervalle}, samples=500] (\string\x, \meta{fonction});
    \end{center}

    On a donc {\ttfamily \meta{intervalle}=a:b} et {\ttfamily \meta{fonction}=f(\string\x)}. \meta{intervalle} peut être laissé vide (sans espace) et, dans ce cas, par défaut, {\ttfamily\meta{intervalle}=-6:6}.
\end{docCommand}

\begin{dispExample*}{sidebyside, center lower}
\Ecran[width=6, height=4, graphic=true]%
{%
\x/,
\x*\x/,
2/-3:3
}
\end{dispExample*}

On peut modifier l'aspect des courbes. Les options suivantes sont valables pour toutes les courbes tracées et ne peuvent être individualisées.

\begin{docKey}{plotcolour}{=\meta{colour}}{valeur par défaut : \docValue{blue}}
    Modifie la couleur des courbes. L'option \docAuxKey{plotcolor} est autorisée.
\end{docKey}

\begin{docKey}{plotwidth}{=\meta{dim}}{valeur par défaut : \docValue{1pt}}
    Modifie l'épaisseur des courbes.
\end{docKey}

\begin{dispExample*}{sidebyside, center lower}
\Ecran[plotwidth=0.5pt, plotcolour=red, graphic=true, ygrad=2, width=6.5]%
{
sin(\x r)/-3*pi:3*pi,
cos(\x r)/-2*pi:1.5*pi
}
\end{dispExample*}

\begin{dispExample*}{sidebyside, center lower}
\Ecran[graphic=true, height=2.75, width=6.5, nbgradx=15, origin={(0.5,0.5)}]%
{sqrt(\x)/0:15}
\end{dispExample*}

\begin{dispExample*}{sidebyside, center lower}
\Ecran[width=6.5, height=2.75, graphic=true, nbgradx=15, origin={($(ecran.west)+(0.5,0)$)}]%
{ln(\x)/0.01:15}
\end{dispExample*}


\section{La calculatrice}
\subsection{Version grand format}

\begin{docCommand}{Calculatrice}{\oarg{titre optionnel}}
    Cette commande permet d'afficher la calculatrice en entier. Chaque touche a été nommée individuellement pour pouvoir s'y référer.
\end{docCommand}

Le tableau ci-dessous montre le nom attribué à chacune des touches avec la commande \cs{Calculatrice}.\bigskip

\noindent\renewcommand\arraystretch{2}%
\begin{tabularx}{\linewidth}{>{\bfseries}c *{8}{>{\centering\arraybackslash} X}}
    \hline
        Touche & \Touche[style=graph,principal=$f(x)$,second={gr.stats},alpha=f1,position=0.95,fontsize=7pt,raise=-2ex] &
        \Touche[style=graph,principal={fenêtre},second=déf tab,alpha=f2,position=0.95,fontsize=7pt,raise=-2.2ex] &
        \Touche[style=graph,principal={zoom},second={format},alpha={f3},position=0.95,fontsize=7pt,raise=-2.2ex] &
        \Touche[style=graph,principal={trace},second={calculs},alpha={f4},position=0.95,fontsize=7pt,raise=-2.2ex] &
        \Touche[style=graph,principal={graphe},second={table},alpha={f5},position=0.95,fontsize=7pt,raise=-2.2ex] &
        \Touche[style=second] &
        \Touche[style=alpha,name=ALPHA] &
        \Touche[principal={suppr},second={insérer},position=0.65]\\
    \hline
    Nom & FX & FEN & ZOOM & TRC & GRA & SCD & ALPHA & SUP \\
%
    \hline\hline
%
        Touche & \Touche[principal={mode},second={quitter},position=0.8] &
        \Touche[principal={$x,t,\theta,n$},second={échanger},fontsize=7pt,position=0.8] &
        \Touche[principal={stats},second={listes}] &
        \Touche[principal={maths},second={tests},alpha={A},position=0.8] &
        \Touche[principal={\textcolor{purple}{apps}},second={angle},alpha={B},position=0.7] &
        \Touche[principal={prgm},second={dessin},alpha={C},position=0.7] &
        \Touche[principal={var},second={distrib}] &
        \Touche[principal={annul}]\\
    \hline
        Nom & MODE & XTN & STATS & MAT & APPS & PRGM & VAR & ANN \\
%
    \hline\hline
%
        Touche & \Touche[principal={$x^{-1}$},second={matrice},alpha={D},fontsize=7pt] &
        \Touche[principal={sin},second={arcsin},alpha={E}] &
        \Touche[principal={cos},second={arccos},alpha={F}] &
        \Touche[principal={tan},second={arctan},alpha={G}] &
        \Touche[principal={\Circonflexe},second={$\pi$},alpha={H}] &
        \Touche[principal={$x^2$},second={\Racine},alpha={I}] &
        \Touche[principal={\large ,},second={EE},alpha={J}] &
        \Touche[principal={log},second={$10^x$},alpha={N}] \\
    \hline
        Nom & INV & SIN & COS & TAN & POW & CARRE & VIRG & LOG\\
%
    \hline\hline
%
        Touche & \Touche[principal={ln},second={e$^x$},alpha={S},name=LN] &
        \Touche[principal={(},second={\{},alpha={K}] &
        \Touche[principal={)},second={\}},alpha={L}] &
        \Touche[principal={sto$\to$},second={rappel},alpha={X},name=STO] &
        \Touche[style=operation,principal={$\div$},second={e},alpha={M}] &
        \Touche[style=operation,principal={$\times$},second={[},alpha={R}] &
        \Touche[style=operation,principal={$-$},second={]},alpha={W},name=SUB] &
        \Touche[style=operation,principal={+},second={mém},alpha={"},name=ADD]\\
    \hline
        Nom & LN & PO & PF & STO & DIV & MUL & SUB & ADD \\
%
    \hline\hline
%
        Touche & \Touche[style=number,principal={1},second={L1},alpha={Y}] &
        \Touche[style=number,principal={2},second={L2},alpha={Z}] &
        \Touche[style=number,principal={3},second={L3},alpha={$\theta$}] &
        \Touche[style=number,principal={4},second={L4},alpha={T}] &
        \Touche[style=number,principal={5},second={L5},alpha={U}] &
        \Touche[style=number,principal={6},second={L6},alpha={V}] &
        \Touche[style=number,principal={7},second={$u_n$},alpha={O}] &
        \Touche[style=number,principal={8},second={$v_n$},alpha={P}] \\
    \hline
        Nom & T1 & T2 & T3 & T4 & T5 & T6 & T7 & T8 \\
%
    \hline\hline
%
        Touche & \Touche[style=number,principal={9},second={$w_n$}] &
        \Touche[style=number,principal={0},second={cat.},alpha={$\sqcup$}] &
        \Touche[style=number,principal={$\centerdot$},second={$i$},alpha={:}] &
        \Touche[style=number,principal={(--)},second={rép},alpha={?}] &
        \Touche[style=on] &
        \Touche[style=enter, fontsize=7pt, principal=entrer, second=préc, alpha=résol, name=ETR] &
        \Touche[style=arrows,scalearrows=0.5] &\\
    \hline
        Nom & T9 & T0 & DOT & MS & ON & ETR & FLE &\\
\end{tabularx}\label{tableau}

\begin{center}\label{calculatrice}
\Calculatrice[Structure]
\end{center}
\begin{tikzpicture}[overlay, remember picture]
        \draw[red, line width=1pt,rounded corners = 5pt] ($(STO) + (0.6,-0.5)$) |-
                                                    ($(PF) + (0.6,-0.5)$) |-
                                                    ($(POW) + (0.6,-0.5)$)|-
                                                    ($(STATS) + (0.6,-0.5)$)|-
                                                    ($(MODE) + (-0.6,0.5)$)|-
                                                    ($(MAT) + (-0.6,0.5)$)|- cycle;
        \path[red]  ($(MAT) + (-0.6,0.5)$) -- ($(STO) + (-0.6,-0.5)$) node[left,midway] {\textbf 1};
%
        \draw[blue, line width=1pt,rounded corners = 5pt] ($(T0) + (-0.6,-0.7)$) -|
                                                    ($(T9) + (0.6,0.5)$) -| cycle;
        \path[blue]  ($(T0) + (-0.6,-0.7)$) -- ($(MS) + (0.6,-0.7)$) node[below,midway] {\textbf 2};
%
        \draw[Orange, line width=1pt,rounded corners = 5pt] ($(ADD) + (-0.6,-0.5)$) -|
                                                    ($(DIV) + (0.6,0.5)$) -| cycle;
        \path[Orange]  ($(ADD) + (0.6,-0.5)$) -- ($(DIV) + (0.6,0.5)$) node[right,midway] {\textbf 3};
%
        \draw[pink!200, line width=1pt,rounded corners = 5pt] ($(FX) + (-0.6,-0.4)$) -|
                                                    ($(GRA) + (0.6,0.5)$) -| cycle;
        \draw[pink!200] ($ (GRA)+ (0.6,0)$) node[right] {\textbf 4};
%
        \draw[blue, line width=1pt,rounded corners = 5pt] ($(SCD) + (-0.6,-0.5)$) |-
                                                                                        ($(ALPHA) + (0.6,-0.5)$)|- ($(SCD) + (-0.6,0.1)$) -- cycle;
        \path ($ (SCD)+ (-0.6,0.1)$) --  ($(ALPHA)+ (-0.6,-0.5)$) node[midway,left] {\parbox{2cm}{\raggedleft\bfseries touches spéciales}};
%
        \draw[red, line width=1pt] (ETR) circle (22pt) node[right=20pt] {\color{black} \parbox{3cm}{\raggedright\bfseries pour évaluer les séquences tapées}};
%
        \draw[red, line width=1pt] (ON) circle (20pt) node[left=25pt] {\color{black} \parbox{3cm}{\raggedleft\bfseries pour allumer et éteindre la calculatrice}};
%
        \draw[red, line width=1pt] (FLE) circle (1cm) node[right=1.1cm] {\color{black} \parbox{3cm}{\raggedright\bfseries pour se déplacer dans les menus}};
    \end{tikzpicture}
\begin{enumerate}
    \item Touches de fonctions (le cadre a été obtenu en utilisant le code ci-dessous)
    \item Touches numériques pour écrire les nombres décimaux
    \item Touches pour écrire les opérations de base
    \item Touches pour les menus graphiques
\end{enumerate}\medskip

\begin{dispListing}
\draw[red, line width=1pt,rounded corners = 5pt]%
    ($(STO) + (0.6,-0.5)$) |- ($(PF) + (0.6,-0.5)$) |-
    ($(POW) + (0.6,-0.5)$) |- ($(STATS) + (0.6,-0.5)$) |-
    ($(MODE) + (-0.6,0.5)$) |- ($(MAT) + (-0.6,0.5)$) |- cycle;
\path[red]  ($(MAT) + (-0.6,0.5)$) -- ($(STO) + (-0.6,-0.5)$) node[left,midway] {\textbf 1};
\end{dispListing}

\subsection{Version petit format}

\begin{docCommand}{Calculatrice*}{\oarg{options}}
    Cette commande permet d'afficher une calculatrice en petit format à utiliser dans des fiches méthodes par exemple.
\end{docCommand}

\begin{dispExample*}{sidebyside}
\Calculatrice*
\end{dispExample*}

L'aspect de la calculatrice est modifiable :

\begin{docKey}{calcscale}{=\meta{number}}{valeur par défaut : \docValue{0.5}}
Permet de modifier la taille de la calculatrice. Plus la calculatrice est petite, moins les dessins de touches seront précis.
\end{docKey}

\begin{dispExample*}{sidebyside}
\Calculatrice*[calcscale=1]
\Calculatrice*[calcscale=0.25]
\end{dispExample*}

\begin{docKey}{calcrotate}{=\meta{number}}{valeur par défaut : \docValue{-30}}
Permet de changer l'angle d'affichage de la calculatrice.
\end{docKey}

\begin{dispExample*}{sidebyside}
\Calculatrice*[calcrotate=0] \textbf{Méthode}
\qquad
\Calculatrice*[calcrotate=90] \textbf{Méthode}
\par\bigskip
\Calculatrice*[calcrotate=-30]
\hspace{-1em}\textbf{Méthode}
\qquad
\rotatebox{90}{\textbf{Méthode}}
\Calculatrice*[calcrotate=0]
\end{dispExample*}

\begin{docKey}{calcraise}{=\meta{dim}}{valeur par défaut : \docValue{-2ex}}
Permet de modifier la hauteur de la calculatrice en fonction de la ligne de base.
\end{docKey}

\begin{dispExample*}{sidebyside}
\Calculatrice*[calcrotate=0, calcraise=0ex] \textbf{Méthode}
\qquad
\rotatebox{90}{\textbf{Méthode}}
\Calculatrice*[calcrotate=0, calcraise=-0.5ex]
\end{dispExample*}

\begin{dispListing}
\begin{center}
    \begin{tikzpicture}
        \fill[color=blue!15, rounded corners=5pt] (0,0) rectangle ++(0.75\linewidth,-3.2);
        \draw[line width = 2pt, color=blue, rounded corners=5pt, line cap=round] (0,0) |- ++(0.75\linewidth,-3.2) -- ++(0,0.2);
        \node (Calc) at (-0.1,0.1)
            {\rotatebox{45}{\textbf{Méthode}}\hspace{-2.5em}
            \Calculatrice*[calcrotate=-45, calcraise=-2.85ex]};
    \end{tikzpicture}

\vspace*{-3cm}
\hspace*{0.1\linewidth}
\parbox{0.7\linewidth}{
    Pour tracer une courbe à la calculatrice, on effectue les actions suivantes :
    \begin{itemize}
        \item appuyer sur \Touche[style=graph, principal=$f(x)$, second={gr.stats}, alpha=f1, fontsize=7pt, position=0.95, raise=-3.5ex] ;
        \item taper l'expression de la fonction ;
        \item appuyer sur \Touche[style=graph, principal={trace}, second={calculs}, alpha={f4}, position=0.95, fontsize=7pt, raise=-3.5ex].
    \end{itemize}
}
\end{center}
\end{dispListing}

\begin{center}
    \begin{tikzpicture}
        \fill[color=blue!15, rounded corners=5pt] (0,0) rectangle ++(0.75\linewidth,-3.2);
        \draw[line width = 2pt, color=blue, rounded corners=5pt, line cap=round] (0,0) |- ++(0.75\linewidth,-3.2) -- ++(0,0.2);
        \node (Calc) at (-0.1,0.1)
            {\rotatebox{45}{\textbf{Méthode}}\hspace{-2.5em}
            \Calculatrice*[calcrotate=-45, calcraise=-2.85ex]};
    \end{tikzpicture}
    
\vspace*{-3cm}
\hspace*{0.1\linewidth}
\parbox{0.7\linewidth}{
    Pour tracer une courbe à la calculatrice, on effectue les actions suivantes :
    \begin{itemize}
        \item appuyer sur \Touche[style=graph, principal=$f(x)$, second={gr.stats}, alpha=f1, fontsize=7pt, position=0.95, raise=-3.5ex] ;
        \item taper l'expression de la fonction ;
        \item appuyer sur \Touche[style=graph, principal={trace}, second={calculs}, alpha={f4}, position=0.95, fontsize=7pt, raise=-3.5ex].
    \end{itemize}
}
\end{center}

\clearpage

\printindex
\end{document} 