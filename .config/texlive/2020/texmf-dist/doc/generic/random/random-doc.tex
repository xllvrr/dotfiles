% plain TeX

\def\cmd#1{\leavevmode\hbox{\tt\string#1}}
\edef\{{\string{}
\edef\}{\string}}

\noindent{\bf RANDOM.TEX }\qquad\qquad  ver 0.2 \qquad\qquad  (Donald Arseneau)

\bigskip

\noindent Generating ``random'' numbers in \TeX. 

\medskip

  {\tt \string\setrannum \thinspace\{$\langle \it counter\rangle$\}\thinspace\{$\langle\it minimum\rangle$\}\thinspace\{$\langle\it maximum\rangle$\}}

  {\tt \string\setrandimen \thinspace\{$\langle \hbox{\it dimen-register}\rangle$\}\thinspace\{$\langle\it minimum\rangle$\}\thinspace\{$\langle\it maximum\rangle$\}}

\medskip
\noindent This software is released to the public domain.
\bigskip
\noindent Random integers are generated in the range 1 to 2147483646 by the
macro \cmd\nextrandom.  The result is returned in the counter \cmd\randomi.
Do not change \cmd\randomi\ except, perhaps, to initialize it at some
random (or specific) value.  If you do not initialize it, it will be initialized
using the time and date.  (This is a sparse initialization, giving
fewer than a million different starting values, but you should use
other sources of numbers if they are available---just remember that
most of the numbers available to \TeX\ are not at all random.)

The \cmd\nextrandom\ command is not very useful by itself, unless you
have exactly 2147483646 things to choose from.  Much more useful
is the \cmd\setrannum\ command which sets a given counter to a random
value within a specified range.  There are three parameters:
\medskip
{\tt \string\setrannum \thinspace\{$\langle \it counter\rangle$\}\thinspace\{$\langle\it minimum\rangle$\}\thinspace\{$\langle\it maximum\rangle$\}}
\medskip
\noindent For example, to simulate a die-roll:
\medskip
{\tt \string\setrannum\{\string\die\}\{1\}\{6\} \string\ifcase\string\die...}
\medskip

If you need random numbers that are not integers, you will have to
use dimen registers instead with \cmd\setrandimen.  For example, to set a random
page width between 3 and 6.5 inches:
\medskip
\cmd\setrandimen \cmd\hsize{\tt\{3in\}\{6.5in\}}
\medskip\noindent
The ``\cmd\pointless'' macro
will remove the ``{\tt pt}'' that \TeX\ gives so you can use the dimensions
as pure ``real'' numbers.  In that case, specify the range in {\tt pt} units.
For example,
\medskip
   {\tt\string\setrandimen\string\answer\{2.71828pt\}\{3.14159pt\}}
   \hfill\break\indent
   {\tt The answer is \string\pointless\string\answer.}
\medskip

The random number generator is the one by Lewis, Goodman, and Miller
(1969) and used as ``ran0'' in ``Numerical Recipies'' using Schrage's
method for avoiding overflows.  The multiplier is 16807 ($7^5$), the
added constant is 0, and the modulus is 2147483647 ($2^{31}-1$).

See CACM, Vol.~36, no. 7, (July 1993), p.~109.  The original authors 
Park and Miller have since concluded that a better multiplier is 48271,
rather than their original 16807.

The range of integers generated is $1 - 2147483646$.  A smaller range would
reduce the complexity of the macros a bit, but not much---most of the
code deals with initialization and type-conversion.  On the other hand,
the large range may be wasted due to the sparse seed initialization.


\bye
