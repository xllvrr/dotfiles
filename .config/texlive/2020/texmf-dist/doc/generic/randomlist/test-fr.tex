\documentclass{article}
\usepackage[T1]{fontenc}
\usepackage[utf8]{inputenc}
\usepackage[a4paper, margin=2.5cm, noheadfoot]{geometry}
\usepackage{amsmath}
\usepackage[seed=1]{randomlist}
\usepackage[french]{babel}

\pagestyle{empty}
\setlength{\parindent}{0pt}

\NewList{Eleves}
\NewList{Triplets}

\begin{document}
\ReadFileList{Eleves}{pupils.dat}
\ExtractFirstItem{Eleves}{NULL} % retire la ligne de titre
\ReadFileList{Triplets}{pythagoras.dat}
\ForEachFirstItem{Eleves}{Eleve}
{%
  \ReadFieldItem{\Eleve}{0}{Nom}
  \ReadFieldItem{\Eleve}{1}{Prenom}
  \ReadFieldItem{\Eleve}{2}{Note}
  \ExtractRandomItem{Triplets}{Triplet}
  \ReadFieldItem{\Triplet}{0}{Tripleta}
  \ReadFieldItem{\Triplet}{1}{Tripletb}
  \ReadFieldItem{\Triplet}{2}{Tripletc}
  \begin{center}
    \fbox{\huge\bfseries Devoir pour \Nom{} \Prenom}
  \end{center}
  \textbf{Exercise} \par
  \if A\Note
    La diagonale d'un rectangle fait \Tripletc\,cm et un des côtés de
    ce rectangle fait \Tripleta\,cm. Quelle est la longueur de l'autre
    côté du rectangle ?
  \else
    Calculer la longueur de la diagonale d'un rectangle de
    \Tripleta\,cm sur \Tripletb\,cm.
  \fi
  \newpage
  \begin{center}
    \fbox{\huge\bfseries Corrigé du devoir de \Nom{} \Prenom}
  \end{center}
  \textbf{Exercise} \par
  \if A\Note
    Il faut utiliser le théorème de Pythagore. On a :
    \[\text{diagonale}^2=\text{côté}_1^2+\text{côté}_2^2.\]
    Donc :
    \[\Tripletc^2=\Tripleta^2+\text{côté}_2^2\]
    et donc :
    \[\text{côté}_2=\sqrt{\Tripletc^2-\Tripleta^2} = \Tripletb.\]
  \else
    Avec le théorème de Pythagore, on a :
    \[\text{diagonale}^2=\text{côté}_1^2+\text{côté}_2^2.\]
    Here:
    \[\text{diagonale}^2=\Tripleta^2+\Tripletb^2\]
    and then
    \[\text{diagonale}=\sqrt{\Tripleta^2+\Tripletb^2} = \Tripletc.\]
  \fi
  \newpage
}
\end{document}
