% This is The Visual Doc for the Lecturer package.
% The EXHAUSTIVE documentation is lecturer-doc.pdf.
%
% You can recompile if you want, but you need LuaTeX 0.6
% + piTeX (available on CTAN, put it on top of plain TeX)
% + non-free fonts. So you'd better just copy and paste here
% and there.
% 
% Author: Paul Isambert.
% Date: July 2010.

\input pitex
\input lecturer


%
% That's piTeX, nothing to do with
% Lecturer, except the syntax (how strange).
%
\setparameter font:
  command = \mainfont
  name    = ChaparralPro-
  size    = 10pt
  roman   = Regular
  bold    = Bold
  italic  = Italic

\setparameter font:
  command = \titlefont
  name    = ChaparralPro-
  size    = 1.5cm
  roman   = Regular
  bold    = Bold
  italic  = Italic

\setparameter font:
  command = \codefont
  name    = LTYPE
  size    = 8pt
  type    = ttf
  roman   = {}
  bold    = B
  italic  = O

\setparameter font:
  command = \centaur
  type    = ttf
  name    = Centaur
  roman   = {}

\setcatcodes{_=12}
%
% End of piTeX stuff.
%










\setparameter job:
  title          = "{\it Lecturer:} a Visual Doc"
  pdftitle       = "Lecturer's Visual Doc (alternate title set with pdftitle in the job parameter)"
  author         = "Paul Isambert"
  background     = lightgrey
  menutext       = "Value to the job's submenutext attribute"
  fullscreen     = true
  autofullscreen = true

\newcolor{lightgrey}{grey}{.9}
\newcolor{attribute}{rgb}{.9 0 0}
\newcolor{value}{rgb}{0 .7 0}
\newcolor{instruction}{grey}{.5}
\def\Parameter#1{\bold{#1}}
\def\Attribute#1{\usecolor{attribute}{#1}}
\def\Value#1{\usecolor{value}{#1}}
\def\Instruction#1{\usecolor{instruction}{\ital{#1}}}
\def\com#1{\usecolor{attribute}{\string#1}\antigobblespace}

\setarea{controlbar}
  width      = 3cm
  top        = 1cm
  left       = .3cm
  font       = \mainfont
  foreground = value
  background = lightgrey

\newsymbol\menu[.3cm]{%
  pen .05, color attribute,
  circle ur 1, circle rd .9, circle dl .8, circle lu .7,
  circle ur .6, circle rd .5, circle dl .4, circle lu .3,
  circle ur .2, circle rd .1,
  }


\setparameter slide:
  height     = 16cm
  width      = 19cm
  left       = 3.5cm
  right      = 1cm
  top        = 1cm
  bottom     = 1.5cm
  hpos       = ff
  vpos       = top
  font       = \mainfont
  everyslide = \everyslide

\def\everyslide{%
  \position{controlbar}[0cm,11.5cm]{%
    \firstslide{\codefont\com\firstslide}
    \prevslide{\codefont\com\prevslide}
    \prevstep{\codefont\com\prevstep}
    \nextstep[flash]{\codefont\com\nexstep[flash]}
    \nextslide[push]{\codefont\com\nextslide[push]}
    \lastslide[none]{\codefont\com\lastslide[none]}
    \kern.2cm
    \showbookmarks\menu
    }
  }










%
%
% TITLE SLIDE
%
%

\setarea{title}
  hshift  = 4cm
  hshift* = 1cm
  vshift  = 3cm
  baselineskip = 2cm
  visible = false
  font    = \titlefont
  hpos    = rr

\setarea{showtitle showauthor showdate}
  visible    = step
  height     = 2cm
  frame      = "width=-.3cm, corner=bevel, color=lightgrey"

\setarea{showtitle}
  top     = -2cm
  right   = 2.5cm
  hshift  = 4cm
  hshift* = 1cm
  vshift  = 2cm

\setarea{showauthor}
  hshift  = 6cm
  hshift* = 3cm
  left    = -4cm
  hsize   = 3.7cm
  vshift  = 4cm
  hpos    = rf
  
\setarea{showdate}
  top     = 2cm
  hshift  = 7cm
  hshift* = 4cm
  vshift  = 6cm

\setarea{showall}
  visible    = step 
  height     = \pdfpageheight
  hshift     = 3cm
  left       = 8cm
  hpos       = rf
  background = lightgrey

\newcolor{tblack}{grey}[.9]{0}
\setarea{warning}
  hshift* vshift* = 1cm
  right           = 3cm
  frame           = "width=1cm, corner=round"
  foreground      = white
  background      = tblack
  hpos            = ff
  vpos            = center
  visible         = step
  parskip         = \baselineskip
  




\slide[The Job]

\createbookmark{1.1}{%
  This bookmark has been created just
  to show the menutext attribute above;
  it is also probably the longest bookmark
  ever created. It was done with the createbookmark
  command, of course.
  }

\position{title}{\Title\par\Author\par\Date}

\step[off=all]
  \position{showtitle}
  {That's the title, set in the \Attribute{title} attribute
   of the \Parameter{job} parameter and stored in \com\Title.
   (Since it includes an italic command, an alternative one
   has been declared in \Attribute{pdftitle}, so it appears
   correctly. Look in the document's properties.)}

\step[off=all]
\position{showauthor}
  {That's \Attribute{author}, again in
  the \Parameter{job} parameter, and then in \com\Author;
  \Attribute{pdfauthor} can be used to define
  an alternative version.}

\step[off=all]
\position{showdate}
  {That's the \Attribute{date} attribute, still in \Parameter{job},
   and stored as \com\Date.
   In this document, it hasn't been specified, so it's
   automatically created with the format shown here.}

\step[all]
\position{showall}[,12cm]
  {That's your PDF reader's background, whose color
  you set with \Attribute{background} in the \Parameter{job} parameter
  (how surprising).}

\step
\position{showall}[-10cm,9cm]
  {And these are navigation commands, which take an argument
   and create a clickable hbox with it. Below is
   a symbol drawn with \com\newsymbol, which displays
   the bookmark pop-up menu when clicked because it's
   the argument to \com\showbookmarks\ (note the value
   of \Attribute{submenutext} from the \Parameter{job} parameter).
   This material is positioned here on every slide thanks
   to the \Attribute{everyslide} attribute of the \Parameter{slide} parameter,
   using \com\position.}

\step
\position{warning}{%
  {\titlefont warning}
  
  \kern\baselineskip
  This visual documentation isn't exhaustive, and doesn't
  show all there is to know in \ital{Lecturer}. It's for
  instance a bit light on navigation, because navigation
  commands aren't the more visual things on earth. The
  reference remains \tcode{lecturer-doc.pdf}.
  
  It says nothing either on what a good presentation is,
  which I don't know, except you probably spend more time
  drawing beautiful proportions between elements than
  devising special effects. Unlike this doc. There are
  more succesful example presentations distributed with \ital{Lecturer}.

  Anyway, since we're at it, this super-area has a transparent
  background defined with \com\newcolor, its \Attribute{hshift*}
  and \Attribute{vshift*} are \Value{1cm}, (thus setting \Attribute{hshift}
  and \Attribute{vshift} automatically to the same values), \Attribute{right} is
  \Value{1cm} (hence \Attribute{left} too), \Attribute{parskip}
  is \string\baselineskip, its \Attribute{hpos}
  is \Value{ff} and \Attribute{vpos} is \Value{center}.
  The \Attribute{frame} has a width of \Value{1cm},
  with \Value{round} \Attribute{corner}s. Finally the value to 
  \Attribute{visible} is \Value{step}, that's why the area shows up
  only right now, with its content.
  
  Barring personal macros for verbatim text and font selection,
  all you see here has been done with \ital{Lecturer}. Even the arrows on
  \goto{dimensions}{the slide about dimensions} are areas without
  width, with symbols as heads. (In the previous sentence, the
  phrase \goto{dimensions}{\ital{the slide about dimensions}} sends to the said slide
  when clicked, because it is the argument of a \com\goto command
  referring to the value of the \Attribute{anchor} attribute of
  this slide.)
  
  The fonts used in this documentation are Chaparral, Lucida Typewriter,
  and Centaur for the example text on \goto{dimensions}{the slide about dimensions},
  which Centaur is out of sync with the text displayed but I couldn't resist.
  }
  

\endslide




\let\Doexample\doexample
\let\doexample\doexamplenogroup
\let\exampleskip\relax

 
\def\Mystep[#1]{%
  \mystep[#1]
  {\oldcodefont
    \string\mystep
    \reverse\iffemptystring{#1}
      {{\setcatcodes{\\\~=12}[\usecolor{attribute}{\scantextokens{#1}}]}}
      }%
  \verb
  }




\slide[Self-describing Step, font=\codefont, vpos = center]

\Example
\setstep\mystep
  vskip      = \baselineskip
  everyvstep = $\bullet$~
  font       = \rm

\newimage{universe}[,1cm]{universe.jpg}
\Example/
\let\oldcodefont\codefont
\let\codefont\mainfont
\setcatcodes{|=0,<=1,>=2}

\kern2\baselineskip

\Mystep[] "This step is normal." \Mystep[] "This one too. And horizontal. |gotoA<hyperjump><|oldcodefont\gotoA{hyperjump}{Here.}>"

\Mystep[A] "Let's jump!"
\Mystep[font=\it] "Phew."

\Mystep[visible=true, left=1cm] "I'm already on the screen."

\Mystep[lovely, color=rgb .5 0 .5, on=A, off=B] "Lovely!"

\Mystep[B] "Get off with your color!
<|oldcodefont|showorhide<toggle=lovely><\showorhide{|usecolor<attribute><toggle=lovely>}{Color or not?}>>"

\Mystep[] "<|oldcodefont\createbookmark[bold,1 0 0]{1.1}{Bookmark created on slide \slidetitle}>"
\createbookmark[bold,1 0 0]{1.1}{Bookmark created on slide \slidetitle}

\Mystep[] "And the universe appeared: <|oldcodefont\useimage{|useimage<universe>}>."

\Mystep[] "The universe ain't what it used to be. |gotoB<hyperjump><|oldcodefont\gotoB{hyperjump}{There.}>"

\endslide


\let\codefont\oldcodefont












\def\valuebox#1#2#3#4{\hbox{\showorhide{on = #3, off = #4}{#1}\kern1em\bold{#2}}}
\def\Valuebox#1#2#3#4{\hbox{\bold{#1}\kern1em\showorhide{on = #3, off = #4}{#2}}}


\slide[Areas in Steps, vpos=top, top=.5cm, bottom=8cm, scale=true]

\step[scaled,visible=true]
\position{controlbar}[,]{%
  \hbox{\Attribute{scale}}
  \Valuebox{true}{false}{unscaled}{scaled}
  }

\Example
\showgrid[11cm,8cm]{.5cm}
\showgrid[11cm,8cm]{1cm}[red]

\setarea{one two three}
  visible      = step
  width height = 8cm
  right bottom = 1cm
  topskip      = 0cm
  foreground   = white

\setarea{one two}
  hshift* = 0cm

\setarea{three}
  hshift* vshift = 8cm
  background     = blue
  hpos           = rf

\setarea{one}
  background = red
  vpos       = bottom

\setarea{two}
  background = "cmyk  1 0 1 0"
  vshift     = 8cm 

\step\position{one}{ONE}
\step\position{two}{TWO}
\step\position{three}{THREE}
\Example/

\setarea{unscaled}
  hshift       = 3.5cm
  width height = 8cm
  vshift       = 0cm
  top          = .5cm
  baselineskip = 12pt
  topskip      = 12pt
  vpos         = top
  visible      = step 

\step[unscaled, on=]
\position{unscaled}{{\typesetexample}}
\position{controlbar}[,]{%
  \hbox{\Attribute{scale}}
  \valuebox{true}{false}{scaled}{unscaled}
  }
\step
\position{two}{\tcode{\string\position$\{$two$\}\{$this$\}$}}
\step
\position{two}{\tcode{\string\position$\{$two$\}\{$and this$\}$}}
\step
\position{two}[1cm,5cm]{\tcode{\string\position$\{$two$\}$[1cm,5cm]$\{$and that$\}$}}
\endslide






\hidegrids




\setparameter slide:
  top   = 3cm
  left  = 7cm
  right = 4cm
  vpos  = top
  vsize  = "19\baselineskip"

% An empty value for "on" makes the step not 
% a step at all, but it can be switched on and off.
\setparameter step: on = ""\par

\setarea{text bordermiter borderbevel borderround}
  hshift  = "\pdfhorigin-1cm"
  hshift* = 3cm
  vshift  = 2cm
  vshift* = 4cm
  background = "grey .8"
  visible = step

\setarea{bordermiter borderbevel borderround}
  frame = "width = 1cm, color = grey .9"

\setarea{bordermiter} frame_corner = miter\par
\setarea{borderbevel} frame_corner = bevel\par
\setarea{borderround} frame_corner = round\par

\setarea{authorff authorfr authorrf authorrr}
  vshift* = 1cm
  height  = 2cm
  hshift* = 5cm
  width   = 4cm
  font    = \centaur
  visible = step
  parindent = 0pt

\setarea{authorff} hpos = ff\par
\setarea{authorrf} hpos = rf\par
\setarea{authorrr} hpos = rr\par

\newsymbol\ltriangle[.1cm, padding=0pt]{% The color must be specified, otherwise
  color black, 1 .5, + 0 -1, fill       % the aera's foerground will be used.
  }
\newsymbol\rtriangle[.1cm, padding=0pt]{%
  color black, move 1 0, + -1 .5, + 0 -1, fill
  }
\newsymbol\ttriangle[.1cm, padding=0pt]{%
  color black, move -.5 0, +  1 0, + -.5 1, fill
  }  
\newsymbol\btriangle[.1cm, padding=0pt]{%
  color black, .5 1, + -1 0, fill
  }

\hfuzz=1cm
\def\Hline{%
  \ifnext*{\gobbleoneand{\hLine{\baselineskip}}}{\hLine{-.15cm}}%
  }
\def\hLine#1#2 #3,#4,#5,#6{%
  \setarea{hline#2}
    hshift  = "\pdfhorigin+#3"
    width   = "#4"
    vshift  = "#5"
    height  = 0pt
    topskip = 0pt
    frame   = "width = .15pt, color = black"
    hpos    = ff
    font    = \mainfont
    parindent = 0pt
    foreground = attribute
    visible = step
  \par\position{hline#2}{\quitvmode\ltriangle\hfill\rtriangle}%
  \position{hline#2}[0pt,#1]{\hfill#6\hfill\kern0pt}%
  }

\def\Vline{%
  \ifnext*{\gobbleoneand{\vLine{rf}}}{\vLine{fr}}%
  }
\def\vLine#1#2 #3,#4,#5,#6{%
  \setarea{vline#2}
    vshift  = "#3"
    height  = "#4"
    hshift  = "\pdfhorigin+#5"
    width   = 0pt
    hsize   = 3cm
    topskip = .1cm
    frame   = "width = .15pt, color = black"
    vpos    = center
    font    = \mainfont
    hpos    = #1
    baselineskip = 0pt
    parindent = 0pt
    foreground = attribute
    visible = step
  \par
  \position{vline#2}{\ttriangle\vfill}%
  \position{vline#2}[\iffstring{#1}{rf}{-3cm-}.3em,.1cm]{#6}%
  \position{vline#2}{\vfill\btriangle}%
  }





\slide[Main Dimensions, anchor=dimensions]


\Hline*X1 0cm, \hsize, 11.3cm,                       {hsize}
\Vline X1 3cm, 19\baselineskip, \hsize+.3cm,         {vsize}
\Hline X2 0cm,  1cm,  2.7cm,                         {parindent} 
\Vline*X2 3cm, \baselineskip, -.3cm,                 {topskip}
\Vline*X3 3cm+5\baselineskip, \baselineskip, -.3cm,  {baselineskip}
\Vline X4 3cm+11\baselineskip, \baselineskip, 5.5cm, {parskip}

\step[slidedim, visible=true]
\Hline S1 -4cm, 16cm, \pdfpageheight-.5cm,           {width}
\Vline*S1 0cm, \pdfpageheight, \hsize+3.5cm,         {height}
\Hline S2 -4cm, 4cm, 10cm,                           {left}
\Vline S2 0cm, \pdfvorigin, 7cm,                     {top}
\Hline S3 8cm, 4cm,  5cm,                            {right}
\Vline*S3 11cm, 5cm,  1cm,                           {bottom}

\position{controlbar}[0cm,2.5cm]{%
  \showorhide{ on = areadim bordermiter, off = slidedim}{\Instruction{Show area}}}


\step[maincontrol, visible=true]
\position{controlbar}[1.2cm,3cm]{%
  \hbox{\Attribute{vpos}}
  \showorhide{on = top topcontrol, off = maincontrol center bottom centercontrol bottomcontrol}{top}
  \showorhide{on = center centercontrol, off = maincontrol top bottom topcontrol bottomcontrol}{center}
  \showorhide{on = bottom bottomcontrol, off = maincontrol top center topcontrol centercontrol}{bottom}
  \showorhide{on = top bottom center maincontrol, off = topcontrol centercontrol bottomcontrol}{\Instruction{reset}}
  }

\step[topcontrol]
\position{controlbar}[1.2cm,3cm]{%
  \hbox{\Attribute{vpos}}
  \showorhide{on = top topcontrol, off = maincontrol center bottom centercontrol bottomcontrol}{\bold{top}}
  \showorhide{on = center centercontrol, off = maincontrol top bottom topcontrol bottomcontrol}{center}
  \showorhide{on = bottom bottomcontrol, off = maincontrol top center topcontrol centercontrol}{bottom}
  \showorhide{on = top bottom center maincontrol, off = topcontrol centercontrol bottomcontrol}{\Instruction{reset}}
  }

\step[centercontrol]
\position{controlbar}[1.2cm,3cm]{%
  \hbox{\Attribute{vpos}}
  \showorhide{on = top topcontrol, off = maincontrol center bottom centercontrol bottomcontrol}{top}
  \showorhide{on = center centercontrol, off = maincontrol top bottom topcontrol bottomcontrol}{\bold{center}}
  \showorhide{on = bottom bottomcontrol, off = maincontrol top center topcontrol centercontrol}{bottom}
  \showorhide{on = top bottom center maincontrol, off = topcontrol centercontrol bottomcontrol}{\Instruction{reset}}
  }

\step[bottomcontrol]
\position{controlbar}[1.2cm,3cm]{%
  \hbox{\Attribute{vpos}}
  \showorhide{on = top topcontrol, off = maincontrol center bottom centercontrol bottomcontrol}{top}
  \showorhide{on = center centercontrol, off = maincontrol top bottom topcontrol bottomcontrol}{center}
  \showorhide{on = bottom bottomcontrol, off = maincontrol top center topcontrol centercontrol}{\bold{bottom}}
  \showorhide{on = top bottom center maincontrol, off = topcontrol centercontrol bottomcontrol}{\Instruction{reset}}
  }

\step[areadim]
\position{text}{}
\position{controlbar}[0cm,2.5cm]{%
  \showorhide{ on = slidedim, off = areadim bordermiter borderbevel borderround}{\Instruction{Show slide}}}
\Hline*A0 -4cm, 3cm, 8cm,                           {hshift}
\Vline*A0 0cm, 2cm, 2cm,                            {vshift}
\Hline*A0a 9cm, 3cm, 10cm,                          {\kern4em hshift*}
\Vline*A0a 12cm, 4cm, 2cm,                          {vshift*}
\Hline*A1 -1cm, \hsize+2cm, \pdfpageheight-1.5cm,   {width}
\Vline A1 2cm, 10cm, \hsize+2.5cm,                  {height}
\Hline A2 -1cm, 1cm, 10cm,                          {left}
\Vline A2 2cm, 1cm, 7cm,                            {top}
\Hline A3 8cm, 1cm,  5cm,                           {\kern1em right}
\Vline*A3 11cm, 1cm,  1cm,                          {bottom}
\Hline*A4 -2cm, 1cm, 12cm,                          {frame_width}

\step[bordermiter]
\position{controlbar}[0cm,6cm]{%
  \hbox{\Attribute{frame_corner}}
  \showorhide{on = bordermiter, off = borderbevel borderround}{\bold{miter}}
  \showorhide{on = borderbevel, off = bordermiter borderround}{bevel}
  \showorhide{on = borderround, off = bordermiter borderbevel}{round}
  }
\position{bordermiter}{}


\step[borderbevel]
\position{controlbar}[0cm,6cm]{%
  \hbox{\Attribute{frame_corner}}
  \showorhide{on = bordermiter, off = borderbevel borderround}{miter}
  \showorhide{on = borderbevel, off = bordermiter borderround}{\bold{bevel}}
  \showorhide{on = borderround, off = bordermiter borderbevel}{round}
  }
\position{borderbevel}{}

\step[borderround]
\position{controlbar}[0cm,6cm]{%
  \hbox{\Attribute{frame_corner}}
  \showorhide{on = bordermiter, off = borderbevel borderround}{miter}
  \showorhide{on = borderbevel, off = bordermiter borderround}{bevel}
  \showorhide{on = borderround, off = bordermiter borderbevel}{\bold{round}}
  }
\position{borderround}{}


\step[ff, visible=true]
\position{controlbar}[0cm,3cm]{%
  \hbox{\Attribute{hpos}}
  \showorhide{on = ff, off = fr rf rr}{\bold{ff}}
  \showorhide{on = fr, off = ff rf rr}{fr}
  \showorhide{on = rf, off = ff fr rr}{rf}
  \showorhide{on = rr, off = ff fr rf}{rr}
  }
\position{authorff}{Savinien Cyrano de Bergerac, Etats et Empires de la Lune, c.1650}
\step[fr]
\position{controlbar}[0cm,3cm]{%
  \hbox{\Attribute{hpos}}
  \showorhide{on = ff, off = fr rf rr}{ff}
  \showorhide{on = fr, off = ff rf rr}{\bold{fr}}
  \showorhide{on = rf, off = ff fr rr}{rf}
  \showorhide{on = rr, off = ff fr rf}{rr}
  }
\position{authorfr}{Savinien Cyrano de Bergerac, Etats et Empires de la Lune, c.1650}
\step[rf]
\position{controlbar}[0cm,3cm]{%
  \hbox{\Attribute{hpos}}
  \showorhide{on = ff, off = fr rf rr}{ff}
  \showorhide{on = fr, off = ff rf rr}{fr}
  \showorhide{on = rf, off = ff fr rr}{\bold{rf}}
  \showorhide{on = rr, off = ff fr rf}{rr}
  }
\position{authorrf}{Savinien Cyrano de Bergerac, Etats et Empires de la Lune, c.1650}
\step[rr]
\position{controlbar}[0cm,3cm]{%
  \hbox{\Attribute{hpos}}
  \showorhide{on = ff, off = fr rf rr}{ff}
  \showorhide{on = fr, off = ff rf rr}{fr}
  \showorhide{on = rf, off = ff fr rr}{rf}
  \showorhide{on = rr, off = ff fr rf}{\bold{rr}}
  }
\position{authorrr}{Savinien Cyrano de Bergerac, Etats et Empires de la Lune, c.1650}

\parfillskip=0pt
\step[top, visible=true,font=\centaur]
\quitvmode\kern1cm La lune �tait en son plein, le ciel �tait d�couvert, et neuf heures
du soir �taient sonn�es lorsque nous revenions d'une maison proche de Paris,
quatre de mes amis et moi. Les diverses pens�es que nous donna la vue
de cette boule de safran nous d�fray�rent sur le chemin. Les yeux noy�s
dans ce grand astre, tant�t l'un le prenait pour une lucarne du ciel
par o� l'on en\-%
%
\step[center, visible=true,font=\centaur]
trevoyait la gloire des bienheureux ; tant�t l'autre protestait
que c'�tait la platine o� Diane dresse les rabats d'Apollon ;
tant�t un autre s'�criait que ce pourrait bien �tre le soleil lui-m�me,
qui s'�tant au soir d�pouill� de ses rayons regardait par un trou
ce qu'on faisait au monde quand il n'y �tait plus.\hfil\kern0pt

\kern\baselineskip
\step[bottom, visible=true,font=\centaur]
�~Et moi, dis-je, qui souhaite m�ler mes enthousiasmes aux v�tres,
je crois sans m'amuser aux imaginations pointues dont vous chatouillez
le temps pour le faire marcher plus vite, que la lune est un monde comme celui-ci,
� qui le n�tre sert de lune.~�
La compagnie me r�gala d'un grand �clat de rire.
�~Ainsi peut-�tre, leur dis-je, se moque-t-on maintenant dans la lune,
de quelque autre, qui soutient que ce globe-ci est un monde.~�
Mais j'eus beau leur all�guer 
%que Pythagore, Epicure,
%D�mocrite et, de notre �ge, Copernic et Kepler, avaient

\endslide







\setarea{code result}
  topskip = 1cm
  everyposition = "\vskip4pt"

\setarea{code}
  hshift = 3cm
  width  = 11cm

\setarea{result}
  hshift* = 0cm
  width   = 6cm


\def\processexample{%
  \position{code}\typesetexample
  \doexample
  \position{result}{\quitvmode\lastsymbol}
  }

\slide[Symbols]
\def\lastsymbol{\square}
\Example
\newsymbol\square[1cm, padding=.15cm]{%
  pen .3,
  1 0, + 0 1, + -1 0, close, stroke,
  color red, move 1.5 0, + 1 0, + 0 1, + -1 0, fill,
  color green, move 3 0, + 1 0, + 0 1, + -1 0, paint,
  }
\Example/

\position{result}{\kern.5cm}

\def\lastsymbol{\triangle}
\Example
\newsymbol\triangle[1cm]{%
  color rgb .8 0 0,  1 .5, + -1 .5, fill,
  }
\Example/

\position{result}{\kern.1cm}

\def\lastsymbol{\spirale}
\Example
\newsymbol\spirale[.1mm]{%
  circle ur 100, circle rd 95, color red, stroke,
  circle dl  90, circle lu 85, color black, stroke,
  circle ur  80, circle rd 75, color red, stroke,
  circle dl  70, circle lu 65, color black
% stroke automatically added if there remains a path.
  }
\Example/

\position{result}{\kern.3cm}

\def\lastsymbol{\diamond}
\Example
\newsymbol\diamond[.5mm]{%
  color red, move 5 0, + 5 7, + 5 -7, + -5 -7, fill
  }
\Example/

\position{result}{\kern.5cm}

\def\lastsymbol{\trystero}
\Example
\newsymbol\trystero[.5cm]{%
  pen .1, move 2 2,
  circle ld 1, circle dr 1, circle ru 1, circle ul 1, stroke,
  move -.75 2, + 4.5 0, stroke,
  + 1.2 1.2, + -0 -2.4, close,
  move + 0 2, + .4 .4, + 0 -2.4, + -.4 .4,
  }
\Example/
\endslide















\setarea{
  fullshade1 fullshade2 fullshade3 fullshade4 fullshade5 fullshade6 fullshade7 fullshade8
  fullshade9 fullshade10 fullshade11 fullshade12 fullshade13 fullshade14 fullshade15 fullshade16
  mask}
  hshift  = 3cm
  visible = step

\setarea{mask}
  background = white
 

\setarea{
  shade1 shade2 shade3 shade4 shade5 shade6 shade7 shade8
  shade9 shade10 shade11 shade12 shade13 shade14 shade15 shade16
  border}
  hshift  = 4cm
  hshift* = 2cm
  vshift  = 1cm
  vshift* = 11cm
  visible = step
  
\setarea{
  shade1b shade2b shade3b shade4b shade5b shade6b shade7b shade8b
  shade9b shade10b shade11b shade12b shade13b shade14b shade15b shade16b
  borderb}
  hshift  = 14cm
  hshift* = 2cm
  vshift  = 6cm
  vshift* = 2cm
  visible = step

\setarea{border borderb}
  frame_width = .1pt
  frame_color = white


\newshade{shade1}
  fixed  = true
  width  = "\pdfpagewidth-3cm"
  height = \pdfpageheight

\setarea{shade1 shade1b fullshade1 fullshade2 fullshade9 fullshade10} background = shade1\par

\newshade{shade2}
  fixed = false
  width  = "\pdfpagewidth-3cm"
  height = \pdfpageheight

\setarea{shade2 shade2b} background = shade2\par

\newshade{shade3}
  fixed  = true
  width  = "\pdfpagewidth-3cm"
  height = \pdfpageheight
  angle  = -45

\setarea{shade3 shade3b fullshade3 fullshade4 fullshade11 fullshade12} background = shade3\par

\newshade{shade4}
  fixed  = false
  width  = "\pdfpagewidth-3cm"
  height = \pdfpageheight
  angle  = -45

\setarea{shade4 shade4b} background = shade4\par

\newshade{shade5}
  fixed  = true
  width  = "\pdfpagewidth-3cm"
  height = \pdfpageheight
  speed  = 2

\setarea{shade5 shade5b fullshade5 fullshade6 fullshade13 fullshade14} background = shade5\par

\newshade{shade6}
  fixed = false
  width  = "\pdfpagewidth-3cm"
  height = \pdfpageheight
  speed  = 2
  
\setarea{shade6 shade6b} background = shade6\par

\newshade{shade7}
  fixed  = true
  width  = "\pdfpagewidth-3cm"
  height = \pdfpageheight
  angle  = -45
  speed  = 2
  
\setarea{shade7 shade7b fullshade7 fullshade8 fullshade15 fullshade16} background = shade7\par

\newshade{shade8}
  fixed = false
  width  = "\pdfpagewidth-3cm"
  height = \pdfpageheight
  angle = -45
  speed  = 2
  
\setarea{shade8 shade8b} background = shade8\par

\newshade{shade9}
  fixed  = true

\setarea{shade9 shade9b} background = shade9\par

\newshade{shade10}
  fixed = false

\setarea{shade10 shade10b} background = shade10\par

\newshade{shade11}
  fixed  = true
  angle  = -45

\setarea{shade11 shade11b} background = shade11\par

\newshade{shade12}
  fixed = false
  angle = -45

\setarea{shade12 shade12b} background = shade12\par

\newshade{shade13}
  fixed  = true
  speed  = 2

\setarea{shade13 shade13b} background = shade13\par

\newshade{shade14}
  fixed = false
  speed  = 2
  
\setarea{shade14 shade14b} background = shade14\par

\newshade{shade15}
  fixed  = true
  angle  = -45
  speed  = 2
  
\setarea{shade15 shade15b} background = shade15\par

\newshade{shade16}
  fixed = false
  angle = -45
  speed  = 2
  
\setarea{shade16 shade16b} background = shade16\par



\setarea{comment}
  hshift  = 4cm
  width   = 4cm
  vshift  = 10cm
  vshift* = 2cm
  font    = \mainfont
  foreground = attribute

\def\comment{\position{comment}[0pt,0pt]}







\slide[Shades]

\position{border}{} \position{borderb}{}

\step[mask, visible=true]
\position{mask}{}
\position{controlbar}[0cm,4cm]{\showorhide{toggle = mask nomask}{\Instruction{Show slide's shade}}}

\step[nomask]
\position{controlbar}[0cm,4cm]{\showorhide{toggle = mask nomask}{\Instruction{Hide slide's shade}}}





\step[shade1, visible=true]
\position{shade1}{} \position{shade1b}{} \position{fullshade1}{}
\position{controlbar}[0cm,0cm]{%
  \hbox{\Attribute{fixed}}
  \Valuebox{true}{false}{shade2}{shade1}
  \hbox{\Attribute{angle}}
  \valuebox{-45}{90}{shade3}{shade1}
  \hbox{\Attribute{speed}}
  \Valuebox{1}{2}{shade5}{shade1}
  \hbox{\Attribute{height \& width}}
  \Valuebox{true}{false}{shade9}{shade1}
  }
\comment{The areas are open windows on the underlying shade,
which goes from black at the top of the slide to white at
the bottom.}

\step[shade2]
\position{shade2}{} \position{shade2b}{} \position{fullshade2}{}
\position{controlbar}[0cm,0cm]{%
  \hbox{\Attribute{fixed}}
  \valuebox{true}{false}{shade1}{shade2}
  \hbox{\Attribute{angle}}
  \valuebox{-45}{90}{shade4}{shade2}
  \hbox{\Attribute{speed}}
  \Valuebox{1}{2}{shade6}{shade2}
  \hbox{\Attribute{height \& width}}
  \Valuebox{true}{false}{shade10}{shade2}
  }
\comment{The shade goes from black at the top of each
areas to white somewhere at the bottom of its height;
the areas show the same shade at different position; the
area above is almost totally black because it displays the
shade's upper part, and the shade takes its time to reach
its bottom.}

\step[shade3]
\position{shade3}{} \position{shade3b}{} \position{fullshade3}{}
\position{controlbar}[0cm,0cm]{%
  \hbox{\Attribute{fixed}}
  \Valuebox{true}{false}{shade4}{shade3}
  \hbox{\Attribute{angle}}
  \Valuebox{-45}{90}{shade1}{shade3}
  \hbox{\Attribute{speed}}
  \Valuebox{1}{2}{shade7}{shade3}
  \hbox{\Attribute{height \& width}}
  \Valuebox{true}{false}{shade11}{shade3}
  }
\comment{The shade goes from black at the top right
corner to white at the bottom left one; the areas display
the parts they cover.}

\step[shade4]
\position{shade4}{} \position{shade4b}{} \position{fullshade4}{}
\position{controlbar}[0cm,0cm]{%
  \hbox{\Attribute{fixed}}
  \valuebox{true}{false}{shade3}{shade4}
  \hbox{\Attribute{angle}}
  \Valuebox{-45}{90}{shade2}{shade4}
  \hbox{\Attribute{speed}}
  \Valuebox{1}{2}{shade8}{shade4}
  \hbox{\Attribute{height \& width}}
  \Valuebox{true}{false}{shade12}{shade4}
  }
\comment{The shade makes an angle of 45 degrees from
its top right corner and goes to white at its bottom left one;
it is positioned at the top right corner of the areas, which
are thus almost entirely black.}

\step[shade5]
\position{shade5}{} \position{shade5b}{} \position{fullshade5}{}
\position{controlbar}[0cm,0cm]{%
  \hbox{\Attribute{fixed}}
  \Valuebox{true}{false}{shade6}{shade5}
  \hbox{\Attribute{angle}}
  \valuebox{-45}{90}{shade7}{shade5}
  \hbox{\Attribute{speed}}
  \valuebox{1}{2}{shade1}{shade5}
  \hbox{\Attribute{height \& width}}
  \Valuebox{true}{false}{shade13}{shade5}
  }
\comment{The shade goes twice faster. It reaches white at its middle;
since it is fixed, the area on the right has nothing much to show.}

\step[shade6]
\position{shade6}{} \position{shade6b}{} \position{fullshade6}{}
\position{controlbar}[0cm,0cm]{%
  \hbox{\Attribute{fixed}}
  \valuebox{true}{false}{shade5}{shade6}
  \hbox{\Attribute{angle}}
  \valuebox{-45}{90}{shade8}{shade6}
  \hbox{\Attribute{speed}}
  \valuebox{1}{2}{shade2}{shade6}
  \hbox{\Attribute{height \& width}}
  \Valuebox{true}{false}{shade14}{shade6}
  }
\comment{The shade goes twice faster from its top to its
bottom, reaching white at half its height; the areas display
its upper part corner only.}

\step[shade7]
\position{shade7}{} \position{shade7b}{} \position{fullshade7}{}
\position{controlbar}[0cm,0cm]{%
  \hbox{\Attribute{fixed}}
  \Valuebox{true}{false}{shade8}{shade7}
  \hbox{\Attribute{angle}}
  \Valuebox{-45}{90}{shade5}{shade7}
  \hbox{\Attribute{speed}}
  \valuebox{1}{2}{shade3}{shade7}
  \hbox{\Attribute{height \& width}}
  \Valuebox{true}{false}{shade15}{shade7}
  }
\comment{The shade runs at 45 degrees from its upper right
corner and reaches white on its diagonal, because it goes
twice faster than normal; this diagonal visibly crosses the
areas, which seem really connected.}


\step[shade8]
\position{shade8}{} \position{shade8b}{} \position{fullshade8}{}
\position{controlbar}[0cm,0cm]{%
  \hbox{\Attribute{fixed}}
  \valuebox{true}{false}{shade7}{shade8}
  \hbox{\Attribute{angle}}
  \Valuebox{-45}{90}{shade6}{shade8}
  \hbox{\Attribute{speed}}
  \valuebox{1}{2}{shade4}{shade8}
  \hbox{\Attribute{height \& width}}
  \Valuebox{true}{false}{shade16}{shade8}
  }
\comment{The shade makes a vanishing triangle
from its upper right corner, which is the part
both areas display.}











\step[shade9]
\position{shade9}{} \position{shade9b}{} \position{fullshade9}{}
\position{controlbar}[0cm,0cm]{%
  \hbox{\Attribute{fixed}}
  \Valuebox{true}{false}{shade10}{shade9}
  \hbox{\Attribute{angle}}
  \valuebox{-45}{90}{shade11}{shade9}
  \hbox{\Attribute{speed}}
  \Valuebox{1}{2}{shade13}{shade9}
  \hbox{\Attribute{height \& width}}
  \valuebox{true}{false}{shade1}{shade9}
  }
\comment{The area is fixed at the top of the slide,
but since it has no dimensions it takes those of
the areas in which it is painted; hence the area
on the right shows its own bottom.}


\step[shade10]
\position{shade10}{} \position{shade10b}{} \position{fullshade10}{}
\position{controlbar}[0cm,0cm]{%
  \hbox{\Attribute{fixed}}
  \valuebox{true}{false}{shade9}{shade10}
  \hbox{\Attribute{angle}}
  \valuebox{-45}{90}{shade12}{shade10}
  \hbox{\Attribute{speed}}
  \Valuebox{1}{2}{shade14}{shade10}
  \hbox{\Attribute{height \& width}}
  \valuebox{true}{false}{shade2}{shade10}
  }
\comment{The shade adapts to the areas' shapes, and goes
  from black at their tops to white at their bottom.}


\step[shade11]
\position{shade11}{} \position{shade11b}{} \position{fullshade11}{}
\position{controlbar}[0cm,0cm]{%
  \hbox{\Attribute{fixed}}
  \Valuebox{true}{false}{shade12}{shade11}
  \hbox{\Attribute{angle}}
  \Valuebox{-45}{90}{shade9}{shade11}
  \hbox{\Attribute{speed}}
  \Valuebox{1}{2}{shade15}{shade11}
  \hbox{\Attribute{height \& width}}
  \valuebox{true}{false}{shade3}{shade11}
  }
\comment{The shade is a replica of the areas it paints in the
top right corner of the shade, hence the varying angles;
the result is quite unpredictable.}

\step[shade12]
\position{shade12}{} \position{shade12b}{} \position{fullshade12}{}
\position{controlbar}[0cm,0cm]{%
  \hbox{\Attribute{fixed}}
  \valuebox{true}{false}{shade11}{shade12}
  \hbox{\Attribute{angle}}
  \Valuebox{-45}{90}{shade10}{shade12}
  \hbox{\Attribute{speed}}
  \Valuebox{1}{2}{shade16}{shade12}
  \hbox{\Attribute{height \& width}}
  \valuebox{true}{false}{shade4}{shade12}
  }
\comment{The shade goes from each area's top right corner
to its bottom left one; in an ideal world where all rectangles
are square, it is slanted by and angle of 45 degrees.}

\step[shade13]
\position{shade13}{} \position{shade13b}{} \position{fullshade13}{}
\position{controlbar}[0cm,0cm]{%
  \hbox{\Attribute{fixed}}
  \Valuebox{true}{false}{shade14}{shade13}
  \hbox{\Attribute{angle}}
  \valuebox{-45}{90}{shade15}{shade13}
  \hbox{\Attribute{speed}}
  \valuebox{1}{2}{shade9}{shade13}
  \hbox{\Attribute{height \& width}}
  \valuebox{true}{false}{shade5}{shade13}
  }
\comment{The shade runs a distance the height of the areas
it paints, but is still fixed at the top of the slide; because
of its speed, the area on the right is totally blanck.}

\step[shade14]
\position{shade14}{} \position{shade14b}{} \position{fullshade14}{}
\position{controlbar}[0cm,0cm]{%
  \hbox{\Attribute{fixed}}
  \valuebox{true}{false}{shade13}{shade14}
  \hbox{\Attribute{angle}}
  \valuebox{-45}{90}{shade16}{shade14}
  \hbox{\Attribute{speed}}
  \valuebox{1}{2}{shade10}{shade14}
  \hbox{\Attribute{height \& width}}
  \valuebox{true}{false}{shade6}{shade14}
  }
\comment{The shade fills the areas down to their
middle, because it goes twice faster than normal. The areas
look like scaling of each other.}

\step[shade15]
\position{shade15}{} \position{shade15b}{} \position{fullshade15}{}
\position{controlbar}[0cm,0cm]{%
  \hbox{\Attribute{fixed}}
  \Valuebox{true}{false}{shade16}{shade15}
  \hbox{\Attribute{angle}}
  \Valuebox{-45}{90}{shade13}{shade15}
  \hbox{\Attribute{speed}}
  \valuebox{1}{2}{shade11}{shade15}
  \hbox{\Attribute{height \& width}}
  \valuebox{true}{false}{shade7}{shade15}
  }
\comment{The area on the right can't catch
itself, it goes too fast; speed, angle and length
vary according to the area painted, but not the
position; thus the result is totally unpredictable.}

\step[shade16]
\position{shade16}{} \position{shade16b}{} \position{fullshade16}{}
\position{controlbar}[0cm,0cm]{%
  \hbox{\Attribute{fixed}}
  \valuebox{true}{false}{shade15}{shade16}
  \hbox{\Attribute{angle}}
  \Valuebox{-45}{90}{shade14}{shade16}
  \hbox{\Attribute{speed}}
  \valuebox{1}{2}{shade12}{shade16}
  \hbox{\Attribute{height \& width}}
  \valuebox{true}{false}{shade8}{shade16}
  }
\comment{The areas show their diagonals;
the variation in angles is quite blatant, because
the shade adapts to their dissimilar shapes and
increase the differences with speed.}



\endslide







\setarea{split blinds box wipe dissolve glitter fly push cover uncover fade}
  hshift     = 3cm
  foreground = white
  visible    = step
  hpos       = rr
  vpos       = center
  

\setarea{split box dissolve fly cover fade}
  background = attribute

\setarea{blinds wipe glitter push uncover}
  background = value
  
\setparameter step:
  on = here

\setparameter slide:
  top    = 0cm
  bottom = 0cm
  left   = 3cm
  right  = 0cm
  hpos   = rr
  vpos   = center


\slide[Default transitions]

\step[visible=true, off=next]
Let's \ital{split}!

\step[next, transition=split]
\position{split}{Let's \ital{blinds}}

\step[transition=blinds]
\position{blinds}{Let's \ital{box}}

\step[transition=box]
\position{box}{Let's \ital{wipe}}

\step[transition=wipe]
\position{wipe}{Let's \ital{dissolve}}

\step[transition=dissolve]
\position{dissolve}{Let's \ital{glitter}}

\step[transition=glitter]
\position{glitter}{Let's \ital{fly}}

\step[transition=fly]
\position{fly}{Let's \ital{push}}

\step[transition=push]
\position{push}{Let's \ital{cover}}

\step[transition=cover]
\position{cover}{Let's \ital{uncover}}

\step[transition=uncover]
\position{uncover}{Let's \ital{fade}}
\endslide






\slide[right=0cm, font=\titlefont, transition=fade, background=lightgrey, everyslide=, bookmark=false]
THE END
\endslide



\bye