% This is a demonstration file distributed with the
% Lecturer package (see lecturer-doc.pdf).
%
% You can recompile the file with a basic TeX implementation,
% using pdfTeX or LuaTeX with the plain format.
% 
% The reusable part ends somewhere around line 250.
%
% Author: Paul Isambert.
% Date: July 2010. 

\input lecturer

% Uncomment these to show the grid on which the presentation is built.
%\showgrid{1mm}
%\showgrid[-.5cm,-.5cm]{1cm}[black]
%\showgrid{1cm}[red][.4pt]


\font\maintitlefont  = cmdunh10 at 23pt
\font\slidetitlefont = cmss12   at 20pt
\font\subtitlefont   = cmss8
\font\titlefont      = cmss10   at 20pt

\newcolor{darkred}{rgb}{.6 0 0}
\setparameter job:
  mode           = presentation
  author         = "Pete Agoras"
  title          = "The square root of 2 ain't rational"
  date           = "Some centuries B.C\rlap." % \rlap creates protrusion
  fullscreen     = true
  autofullscreen = true
  font           = \tenrm


%
% Slides for the main matter.
%
\setparameter slide:
  everyslide    = \everyslide
  top           = 3cm
  topskip       = 1cm
  left          = 3.1cm
  right         = 5cm
  baselineskip  = .5cm
  areas*        = "title by author date"
  bookmarkstyle = italic
  font          = \tenrm
  vpos          = top

\def\everyslide{%
  \position{top}\slidetitle
  \position{left}{\Author\par\Date}
  \position{menu}{\quitvmode\showbookmarks\Bookmarks}
  \position{menu}[0pt,0pt]{\the\numexpr\slidenumber-1\relax}
  }

%
% Left and top areas.
% 
\setarea{left}
  hshift     = 0pt
  width      = 3cm
  right      = .1cm
  hpos       = rf
  bottom     = .5cm
  vpos       = bottom
  foreground = darkred
  font       = \subtitlefont

\setarea{top}
  left   = 3.1cm
  vshift = 0pt
  height = 2cm
  vpos   = bottom
  font   = \titlefont

\setarea{vline hline}
  frame  = "width = .15pt, color = black"

\setarea{vline}
  hshift = 3cm
  width  = 0pt

\setarea{hline}
  vshift = 2cm
  height = 0pt

%
% The circle with the slide number.
%
\setarea{menu}
  width height = 1cm
  vshift* hshift* topskip = .5cm 
  vpos = bottom
  hpos = rr

\newsymbol\Bookmarks[.35cm,padding=.1pt]
  {pen .05,
   move 1 -.6,
   circle lu 1,
   circle ur 1,
   circle rd 1,
   circle dl 1,
   }

%
% The maths on the right.
%
\setarea{math}
  width      = 2cm
  hshift*    = 1.1cm
  vshift*    = 5.35cm
  vshift     = 5.25cm
  topskip    = .8cm
  foreground = white
  background = black
  frame      = width=.1cm,corner=round % No space => no need for quotes or braces. No readability either.
  hpos       = rr


%
% The title slide.
%
\setslide{titleslide}
  areas      = "title by author date"
  bookmark   = false
  hpos       = rr
  everyslide = {}

\setarea{title}
  hshift*    = 1cm
  hpos       = rr
  vshift     = 2cm
  height     = 1.3cm
  topskip    = 1cm
  background = darkred
  foreground = white
  frame      = "width = .5em, corner = round, color = black"
  font       = \maintitlefont

\setarea{by}
  hshift*    = 5.8cm
  hpos       = rr
  vshift     = 4.5cm
  height     = .8cm
  topskip    = 10pt
  top        = 3pt
  background = black
  foreground = white
  frame      = "width = .5em, corner = round"
  font       = \it
  

\setarea{author}
  hshift*    = 5cm
  hpos       = rr
  vshift     = 6.3cm
  height     = 25pt
  topskip    = 15pt
  top        = 4pt
  background = black
  foreground = white
  frame      = "width = .5em, corner = round, color = darkred"
  font       = \slidetitlefont
  
\setarea{date}
  hshift*    = 5.5cm
  vshift*    = 1.7cm
  height     = .8cm
  bottom     = .3cm
  topskip    = 0pt
  vpos       = bottom
  background = white
  foreground = black
  frame      = "width = .5em, corner = bevel, color = darkred"
  hpos       = rr


%
% For the appendices.
% We remove the math area instead
% of redefining \everyslide.
%
\setslide{appslide}
  top   = 2.5cm
  areas*= "math title by author date"

\abovedisplayshortskip = 0cm
\belowdisplayshortskip = 0cm
\abovedisplayskip = 0pt
\belowdisplayskip = 0pt

%
% Steps.
%
\setparameter step:
  vskip = .5cm

% With this macro, the item symbol is always on the screen
% (since the step is "visible").
\def\Step{%
  \step[visible=true]\quitvmode\llap{\stepsym\kern.2cm}%
  \step}

%
% The maths.
%
\setstep\emptystep
  everyvstep everyhstep = {}
  vskip hskip = 0pt

\def\domath#1 #2 #3{%
  \emptystep[on=#1,off=#2]
  \domathstep{#3}%
  }
\def\Domath#1 #2{%
  \emptystep[off=#1,visible=true]
  \domathstep{#2}%
  }
\def\domathstep#1{%
  \position{math}[0pt,0pt]{$#1$}%
  }
%  
% Item symbols.
%
\newsymbol\stepsym[2mm,padding=0pt]
  {color darkred,
  + 1 .5, + -1 .5, fill,}
\newsymbol\backsym[2mm,padding=0pt]
  {color darkred, move 1 0, + -1 .5, + 1 .5, fill,}
%
% Navigation to the appendices.
%
\def\Back#1{%
  \presentationonly{\usecolor{darkred}{\gotoB{#1}{\subtitlefont\backsym\kern.3em Back}}}%
  }
\def\To#1{%
  \usecolor{darkred}{\gotoA{#1}{\subtitlefont\stepsym\kern.3em See why}}%
  }
%
% And some structure.
%
\def\section#1{%
  \def\sectiontitle{#1}%
  \createbookmark[nosubmenutext,open]{.5}{#1}%
  }

% \endinput
%%%%%%%%%%%%%%%%%%%%%%%%%%%%%%%%%%%%%%%%%%%%%%%%%%%%%%%%%%%%%%%
%                                                             %
% UNCOMMENT THE PREVIOUS LINE TO USE THIS FILE AS A TEMPLATE, %
% OR REMOVE EVERYTHING BELOW.                                 %
%                                                             %
%%%%%%%%%%%%%%%%%%%%%%%%%%%%%%%%%%%%%%%%%%%%%%%%%%%%%%%%%%%%%%%





\titleslide

\position{title}\Title
\position{by}{A Casual Talk By}
\position{author}\Author
\position{date}\Date

\endtitleslide





\section{Demonstration}



\slide[A simple assumption,font=]

\step[on=A]\position{math}[0cm,0cm]{}

\domath A B  {{a\over b}=\sqrt2}
\domath B C  {({a\over b})^2=2}
\domath C D  {{a^2\over b^2}=2}
\domath D {} {a^2=2b^2}

\Step Some say the square root of 2 isn't rational.
\step Suppose it were.

\Step[A] Then we could write it as this, where $a$ and $b$ are
integers without a common factor.
\To{app1}

\Step[B] But then we can also write this.
\step[C] And this.
\step[D] And finally this.
\step Which means that $a^2$ is even.

\endslide



\slide[Its consequences]

\step[on=A]\position{math}[0cm,0cm]{}

\Domath A   {a^2=2b^2}
\domath A B {(2k)^2=2b^2}
\domath B C {4k^2=2b^2}
\domath C D {2k^2=b^2}

\Step[visible=true] So what?
\step So $a$ is even. Because only even numbers produce even squares.
\To{app2}

\Step[A] Being even means being expressible in the form $2k$, where $k$ is
any integer.
\step[B] And $(2k)^2$ square gives $4k^2$.

\Step[C] Let's simplify.
\step Thus $b^2$ is even.
\step And $b$ is too.

\endslide



\slide[The problem]

\step[on=A]\position{math}[0cm,0cm]{}

\domath A {} {{a\over b}\neq\sqrt2}

\Step[visible=true] And but so we said $a$ and $b$ have no common factor.

\Step If both are even they do have a common factor: 2.
\step Which is absurd.

\Step Thus, our basic assumption is false.
\step[A] There are no such $a$ and $b$.

\Step The square root of 2 is irrational.
\step Too bad.

\endslide





\section{Appendices}



\appslide[All fractions are reducible]

\Step[visible=true]
Suppose $c\over d$ is a rational number. If c and d have no common
factor, then $a=b$ and $b=d$. If they have a common factor,
divide both by their greatest common divisor. The result
is $a\over b$, with no common factor.
\kern1em\Back{app1}

\endappslide



\appslide[An even square has an even root]

\Step[visible=true]
An even number, by definition, is expressible
in the form $2k$, where $k$ is any integer.
On the other hand, an odd number is expressible by
%
$$2k+1$$
%
Thus the square of an odd number is
%
$$(2k+1)^2$$
i.e. $$4k^2+4k+1$$
i.e. $$2\times2(k^2+k)+1$$
%
which is of the form $2k+1$
with $2(k^2+k)$ as $k$.
Hence, an odd number produces an odd square,
and thus if a square is even its root is even too.
\kern1em\Back{app2}

\endappslide


\bye