% This is a demonstration file distributed with the
% Lecturer package (see lecturer-doc.pdf).
%
% You can recompile the file with a basic TeX implementation,
% using pdfTeX or LuaTeX with the plain format.
% 
% You can retrieve code between each slide to reuse it.
%
% Author: Paul Isambert.
% Date: July 2010.

\input lecturer


\font\mainfont=cmss10

\setparameter job:
  author    = "F. Runobulax"
  pdfauthor = "Paul Isambert"
  title     = "The Poodle Lectures"
  fullscreen = true

\setparameter slide:
  background = black
  left       = 3cm
  top        = 2cm
  font       = \mainfont
  everyslide = \everyslide

\def\everyslide{%
  \position{title}{\slidetitle}
  \position{author}{\Author}
  \step This is Step One.\par
    \substep High-school diploma.\par
    \substep Stuffed with a gym sock.\par
  \step This is Step Two.\par
    \substep College degree.\par
    \substep Stuffed with absolutely nothing at all.\par
  \step One more time for the world.
  }

\setparameter step:
  vskip      = "2\baselineskip"
  visible    = true
  everyvstep = "\quitvmode\llap{\stepsymbol\kern1em}"

\setstep\substep
  vskip      = 0pt
  everyvstep = "\quitvmode\llap{\substepsymbol\kern1em}"
  left       = 1cm






\newcolor{color1}{cmyk}{0 .27 0 .45}
\newcolor{color2}{cmyk}{0 .27 .0 .25}
\setarea{background1}
  background = color1
  frame      = "width=-11.5cm, corner=round"

\setarea{background2}
  vshift* hshift* = 1cm
  background      = color1
  frame           = "width=-10.5cm, color=color2, corner=round"

\setarea{author title}
 
\newcolor{symbol}{cmyk}{0 .06 .63 .02}
\newsymbol\stepsymbol[.5em]{%
  color symbol,
  + 1.2 .5, + -1.2 .5, fill
  }
\newsymbol\substepsymbol[.5em]{%
  color symbol,
  move 0 .5, circle ur .4, circle rd .4,
  circle dl .4, circle lu .4, fill
  }
  
\slide[Duke of Prunes, areas*= author title]

\endslide









\newcolor{color1}{cmyk}{0 .51 .90 .22}
\newcolor{color2}{cmyk}{0 .51 .90 .02}

\setparameter slide:
  background = white

\setarea{background1}
  hshift* vshift* = 1cm
  background      = white
  frame           = "width=.2cm, color=color1, corner=bevel"

\setarea{background2}
  hshift vshift hshift* vshift* = 1.2cm
  background                    = white
  frame                         = "width=-.15cm, color=color2, corner=miter"

\setarea{author title}
  background = white
  height     = .5cm
  font       = \it
  vpos       = bottom
  left       = .1cm  

\setarea{author}
  hshift* = 2cm
  width   = 2.3cm
  vshift* = 1.2cm
  
\setarea{title}
  hshift  = 2cm
  width   = 3.7cm
  vshift  = .5cm

\newcolor{symbol}{cmyk}{0 .31 1.0 .02}
\newsymbol\stepsymbol[.5em]{%
  color symbol,
  + 1 0, + 0 1, + -1 0, fill
  }
\newsymbol\substepsymbol[.5em]{%
  color symbol,
  move 0 .5, circle ur .4, circle rd .4,
  circle dl .4, circle lu .4, fill
  }

\slide[Son of Orange County]
\endslide













\setparameter slide:
  background = black
  foreground = white

\setarea{background1 background2 title}
  hshift hshift* = 0pt
  height         = 1ex
  background     = ""
  foreground     = white
  frame          = "color=grey .6, corner=miter, width=.1pt"

\setarea{background1}
  vshift = 1cm

\setarea{background2}
  vshift = 1cm+2ex

\setarea{title}
  vshift = 0cm
  height = 1cm+4ex
  bottom = 1ex
  vpos   = bottom
  font   = \bf
  
\newsymbol\stepsymbol[.3em]{%
  pen 0.2,
  move 0 .3, circle ur .4, circle rd .4,
  circle dl .4, circle lu .4, fill,
  move .8 .3, + 0 1.8,
  }

\newsymbol\substepsymbol[.3em]{%
  pen 0.2,
  move 0 .3, circle ur .4, circle rd .4,
  circle dl .4, circle lu .4, fill,
  move .8 .3, + 0 1.8, + 1.5 .5, stroke,
  move 1.5 .5,
  move + 0 .3, circle ur .4, circle rd .4,
  circle dl .4, circle lu .4, fill,
  move 1.5 .5,
  move + .8 .3, + 0 1.8,
  }

\slide[The Black Page, areas*=author]
\endslide













\newcolor{color1}{cmyk}{0 .1 .26 .07}
\newcolor{color2}{cmyk}{0 .1 .46 .47}

\setparameter slide:
  background = color1
  foreground = color2

\setarea{background1}
  vshift      = 10cm
  height      = 1.5cm
  hpos        = ff
  vpos        = bottom
  frame_width = 0pt 
  left right  = -1cm

\setarea{background2}
  hshift hshift* = 2cm
  vshift     = 2.5cm
  vshift*    = 2cm
  background = ""
  frame      = "width=.05cm, color=color2"

\setarea{title}
  vshift     = .5cm
  height     = .7cm
  topskip    = .2cm
  vpos       = center
  hshift     = 2cm
  width      = 4cm
  hpos       = rr
  foreground = color2
  frame      = "width=1pt, color=color2"
  font       = \tt

\newsymbol\fresco[.5cm, top = .07cm, bottom = 0.7cm]{%
  pen 0.07, color color2, 
  move -.1 0, +.1 0, + 0 1, + .9 0, + 0 -.3, + -.6 0, + 0 -.7, +1.1 0
  }
\newsymbol\stepsymbol[.5em]{%
  + 1 0, + -.5 1, fill
  }
\newsymbol\substepsymbol[.5em]{%
  pen 0.1, move 0 .5,
  circle ur .5, circle rd .5,
  circle dl .5, circle lu .5, stroke,
  move .3 .5,
  circle ur .2, circle rd .2,
  circle dl .2, circle lu .2
  }

\slide[Regyptian Strut, areas*=author]
\position{background1}{\quitvmode\leaders\fresco\hfill\kern0pt}
\endslide















\newshade{ocean}
  model = rgb
  from  = ".95 .9 1"
  to    = ".35 .3 1"
  
\setparameter slide:
  background = ocean
  foreground = white

\newsymbol\stepsymbol[.5em]{%
  pen .1,
  color red,
  + 1 .7, + -1 .7, fill,
  move .8 .7,
  circle ur .6, + .3 0, circle rd .9,
  circle dl .3, 
  + -1 0, circle lu .6, fill,
  color black,
  move 1.8 .8, + .2 0,
  }
\newcolor{symbol}{cmyk}{0 .26 .43 0}
\newsymbol\substepsymbol[.5em]{%
  pen .1, color symbol,
  .8 0, + -.1 .2, + .5 .3,
  circle ul .8, circle ld .8,
  + .5 -.3, close, stroke,
  move .3 .2, + -.4 .7, stroke,
  move .4 .2, + 0 .9, stroke,
  move .5 .2, + .4 .7
  }

\setarea{title}
  hshift*     = 1cm
  hshift      = 7cm
  font        = \it
  hpos        = rf
  foreground  = "cmyk 1 1 0 0"
  frame_width = 0pt

\slide[The Ocean is the Ultimate Solution, areas=title]
\endslide

\bye