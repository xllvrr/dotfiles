
\chyph % use csplain

\newcount\firstpage   \firstpage=1
\newcount\numpages    \numpages=14

\def\addcmd #1#2{\expandafter\def\expandafter\tmpa\expandafter{#1}%
   \def\tmpb{#1}%
   \ifx\tmpa\tmpb % the #1 is not a macro
      \expandafter\let\csname \string#1-original\endcsname =#1%
      \expandafter\def\expandafter #1\expandafter{%
           \csname\string#1-original\endcsname #2}%
   \else          % the #1 is a macro
      \expandafter\def\expandafter #1\expandafter{#1#2}%
   \fi}


\input ofs [ffonts]                  % Charter je ve skupin� free fonts

\setfonts [Charter/10pt]             % v�choz� rodina
\setmath[//]                         % inicializace matematiky
\fontdef\tt [CMTypewriter/mag1.1]    % strojopis, korekce st�edn� v��ky
\fontdef\verbtt [CMTypewriter-rm/8]  % strojopis pro display uk�zky
\fontdef\small [!/9]                 % zmen�en� pro abstrakt a z�hlav�
\addcmd \small {\baselineskip11pt \rm \def\mathversion{normal}\setmath[//]}
\fontdef\fontsekce [!/12]            % pro nadpisy sekc�
\addcmd \fontsekce {\bf \let\it=\bi \def\mathversion{bold}\setmath[//]} 
\fontdef\fonttitul [!-bf/14.4]       % titul

\hsize=12,2cm
\vsize=19,3cm
\hoffset=63pt
\voffset=43pt
\parindent=14pt

\lineskiplimit=-10pt
\raggedbottom

\exhyphenpenalty=10000
\widowpenalty=10000
\clubpenalty=10000

\showboxdepth=10 \showboxbreadth=50

\def\starthead{\global\headline=
     {\small \ifodd\pageno \hfil \thetitul \headspace \the\pageno 
     \else \the\pageno \headspace \theauthor \hfil \fi}}
\headline={\hfil\starthead}
\def\headspace{\hskip2.5em\relax}
\def\makeheadline{\vbox to0pt{\vskip-25pt
  \line{\vbox to8.5pt{}\the\headline}\vss}\nointerlineskip}
\footline={\setfonts[/7]\setmath[//]\baselineskip=9pt
     \vbox{\hbox{\copytext}\hbox{\copykonvoj}}\hfil
     \global\footline={}}
\def\lastpage{\advance\firstpage by\numpages \advance\firstpage by-1
     \the\firstpage}

\def\copytext{Euro\TeX{} 2003}
\def\copykonvoj{}

\newcount\subnum

\def\kap #1\par{\advance\subnum by1
   \removelastskip %\goodbreak
   \vskip17pt plus2pt minus1pt\noindent{\fontsekce
   \ifnum\subnum>1 \the\subnum\enspace\enspace\fi #1}%
   \par\nobreak\vskip11pt plus2pt minus1pt 
   \everypar{\setbox0=\lastbox \everypar={}}}
\def\reference {\subnum=-1 \kap Reference\par \small}
\def\bib{\par\advance\subnum by1 \leftskip=\parindent 
   \noindent\llap{\the\subnum.\enspace}\ignorespaces}

\let\orifootnote=\footnote
\newcount\footnotenum
\def\footnote#1{\global\advance\footnotenum by1
   \orifootnote{$\setmath[//]^\the\footnotenum$}{\small 
        \hangindent=\parindent #1\par}}
\def\footnoterule{\kern-3pt \hrule width 2cm \kern2.6pt }

\def\titul #1 \par{\def\thetitul{#1}\pageno=\firstpage
   \centerline{\fonttitul #1}\vskip20pt\relax}
\def\autor #1 \par{\def\theauthor{#1}
   \centerline{#1}\vskip10pt\relax}
\def\institut #1 \par{\centerline{\small #1}}
\def\email #1 \par{\centerline{\small Email: \tt #1}\vskip20pt\relax}
\def\abstrakt {\bgroup \small
   \leftskip=3em \rightskip=3em
   \noindent{\bf Abstract:}\enspace}
\def\endabstrakt {\par\egroup\bigskip}
\def\url#1{{\tt#1}}

\def\begitems{\medskip\bgroup\catcode`\*=13 \narrower}
\def\enditems{\par\egroup\medskip}
{\catcode`\*=13 \gdef*{\par\noindent\llap{$\bullet$\kern.6em }\ignorespaces}
\gdef\numerate{%                         \numerate napsat t�sn� za \begitems
  \def*{\par\advance\itemnum by1\noindent
    \llap{\bf\the\itemnum.\kern.6em }\ignorespaces}}}


\font\mflogo=logo10                       % METAFONT logo
\font\boldsy=cmbsy10 at12pt               % Tu�n� \CS, \AMS.
\font\bboldsy=cmbsy10 scaled\magstep2     % Tu�n� \CS v titulu

\def\CS{$\cal C\kern-.1667em\lower.5ex\hbox{$\cal S$}\kern-.075em $}
\def\bCS{{\boldsy C\kern-.1667em\lower.5ex\hbox{S}\kern-.04em}}
\def\CSTeX{\CS\TeX}
\def\CSTUG{\CS TUG}
\def\CSfont{\CS\kern.1em font}
\def\csplain{\CS\kern.1em plain} \let\CSplain=\csplain
\def\mf{{\mflogo META}\-{\mflogo FONT}}
\def\mp{{\mflogo META}\-{\mflogo POST}}
\def\LaTeX{L\kern-.2em\raise.45ex\hbox{\setfonts[/mag.7] A}\kern-.05em\TeX}
\def\twoe{$2_{\textstyle\varepsilon}$}
\def\LaTeXe{\LaTeX\thinspace\twoe}
\def\cslatex{\CS\kern.05em\LaTeX} \let\CSLaTeX=\cslatex
\def\twoe{$2_{\textstyle\varepsilon}$}
\def\LaTeXe{\LaTeX\thinspace\twoe}
\def\AMS{$\cal A\kern-.166em\lower.5ex\hbox{$\cal M$}\kern-.075em S$}
\def\bAMS{{\boldsy A\kern-.166em\lower.5ex\hbox{M}\kern-.075em S}}
\def\ps{Post\-Script}
\def\SLT{S\kern-0.025em\lower.5ex\hbox{L}\kern-.2emT}
\def\DOS{\hbox{DOS}}

%% Definice verbatim prost�ed� %%          !verb.
\catcode`\"=13
\def"{\hbox\bgroup\let"=\egroup\setverb\tt}
\def\setverb{\def\do##1{\catcode`##1=12}\dospecials\obeyspaces}
\def\begtt{\medskip\bgroup
   \nobreak\setverb \parskip=0pt %\parindent=0pt
   \catcode`\"=12\catcode`\~=13 \obeylines
   \baselineskip=10pt
   \let\tt=\verbtt \startverb}
{\catcode`\|=0 \catcode`\\=12
  |gdef|startverb#1\endtt{%
        |tt#1|nobreak|egroup|penalty0|medskip|scannexttoken}}
{\obeyspaces\gdef {\ }}
\long\def\scannexttoken#1{\ifx#1\par\else\noindent#1\fi}

\def\citeref #1 #2 {\expandafter\def\csname cit:#1\endcsname{#2}}
\citeref wwwstorm    1
\citeref pismolijna  2
\citeref stormtex    3
\citeref slido       4
\citeref ofsftp      5
\citeref ofsdoc      6
\citeref stormcl     7
\citeref berry       8
\def\cite#1{\expandafter\ifx \csname cit:#1\endcsname\relax
   \message{Warning: cite{#1} is not defined}[??]%
   \else [\csname cit:#1\endcsname]\fi}

%-------------------------------------------------------------------------

This version of the paper is only draft! It waits to Petr Sojka's corrections.
\bigskip

\titul  The font management with the OFS
        %%%%%%%%%%%%%%%%%%%%%%%%%%%%%%%%

\autor Petr Ol\v s\'ak, Petr Sojka

\institut Faculty of Electro Engineering CVUT, Prague,
          Faculty of Informatics MU, Brno

\email petr@olsak.net, sojka@fi.muni.cz

\abstrakt 

The widely used New Font Selection System (NFSS) has its pros and cons.
To have uniform platform for large font set management and selection for
both plain and \LaTeX\ users, and to extend the possibilities
of the NFSS, an Ol\v{s}\'ak's Font System (OFS) has been designed
and implemented.

The user interface of this macro is the same for plain and
\LaTeX, with two independent implementations. First implementation
is based on plain.tex macros only; the second one for \LaTeX\ users
is implemented as an additional layer over NFSS macros, and is  
slightly less elaborate (support for math font families is missing).

The main features of the system:

\begitems
* The OFS defines declarative language to define the mapping from 
  the original full names of fonts to the tfm names (in plain) or 
  to NFSS short names of the font families (in \LaTeX). 
  The user do not need to remember the short names of tfm files 
  and/or the short names of NFSS font families. This allows for 
  user-defined logical font markup: one can use the real font 
  (foundry) names from font (catalog).
* The user of the OFS can use a font separated into two \TeX{} metrics 
  (basic and extended tfm). The font is from the user perspective seen 
  as one font only. The mapping works even for fonts with more than 256 
  characters (if one does not want to kern between all characters).
* One can choose the \TeX{} internal encoding of fonts for your
  language at the beginning of your document. This feature is commonly 
  used for Czech and Slovak languages: there are \TeX{} fonts which 
  encode the alphabets of these languages by Cork (T1 encoding) or by
  ISO-8859-2 (IL2 encoding), or even others.
* There are at least four independent parameters of fonts 
  (family, size, encoding, variant). Possibility to switch one 
  font parameter independently on the others is important under 
  plain.tex where NFSS is not implemented.
* The OFS for plain includes the support of the math fonts manipulation
  when the PostScript fonts and/or fonts at the different sizes are used.
* One can use the "\fontusage" command which lists the short usage
  about OFS macros on the screen and to the log file.
* The "\showfonts" macro lists all available font families 
  on the screen and to the log file.
* One can print the short catalogue of the all installed fonts by
  using "\input ofscatal.tex".
\enditems

The macro is available at ftp://math.feld.cvut.cz/pub/olsak/ofs .
The declaration and support OFS files for all of the fonts offered 
by the Storm Type Foundry (www.stormtype.com), and e.g. MathTimes
fonts have been created.  The macro is in use by CSTUG users since 2001.
The English documentation (in addition to the current 
Czech one) will be released in January 2003. 

The full paper will discuss in detail the features of OFS listed 
above, will give examples of usage and pros and cons over current 
macro packages for font switching and handling in \TeX{}.

\endabstrakt


\kap Historical notes, plain\TeX{} and \LaTeX{} version of OFS

In 2000, Petr Ol\v s\'ak got whole set of fonts from Storm Type Foundry
(made by Franti\v sek \v Storm) in order to make the \TeX{} support for
these fonts. The number of fonts was very large (many hundreds) and the
names used in Typo-catalog was a bit different from pfb file names and
they was {\it very} different from new tfm names used in \TeX{} support 
because these names follow the Karl Berry's recommendation~\cite{berry}.

There was a presumption that the user will want to choose individual
fonts and to make his/her own font collections for document
design. The simple solution like "\usepackage{times}" is not possible
because the user have to choose the sans serif variants, serif
variants etc. from a big offer by Storm Type Foundry. There are big
number of such combinations---we are not able to support a style file
for each such combination. Moreover, there are a great font collection
for titles and posters in \v Storm's package. The conclusion is: the user
needs to manage with individual fonts with the same names as in
Typo-catalog provided by Storm Type Foundry. The usage of tfm names or
short names for font families provided by NFSS are not acceptable at
user level.  The support for managing of large sets of fonts
implemented at macro level is needed. This was an inspiration of OFS.

The OFS was written by Petr Ol\v s\'ak primary for plain\TeX{} because he
is a plain\TeX{} user. He never used \LaTeX{} nor NFSS.  Thus the
first version of OFS was usable only in plain\TeX{} environment
without NFSS. This version substitutes some features solved in NFSS
already: an independency of encoding/size/family/variant font
setting at user level, declaration of encoding-de\-pendent commands,
more comfortable declaration of math fonts (but plain\TeX{} approach of
math fonts declarations is kept because this is more understandable for
plain\TeX{} users than NFSS approach).

The \LaTeX{} users asked the similar environment as OFS for
plain\TeX{} after OFS for plain was released. This was a reason why
Petr Ol\v s\'ak made the same user environment of OFS for \LaTeX{} (but
never used it:-).
This work was done by totally different macro code because the NFSS
features was used in this situation. The OFS for \LaTeX{} is only the
new macro level over NFSS. Roughly speaking, the OFS for 
\LaTeX{} implements only the dictionary from long names of fonts used
in Font-catalog to the short family names used in NFSS and defines
the same ``user level environment'' as in OFS for plain\TeX{}.

Because the OFS for plain\TeX{} is more powerful and includes more
interesting subjects in its implementation, we will describe only the
plain\TeX{} version of OFS in this paper.

The OFS macro is released with \TeX{}-like license. It means, that the
names of the files and its content cannot be changed if this macro is
re-distributed. Only Petr Ol\v s\'ak can do changes.
There are no more restrictions for distributors and there are no
restrictions for users.

The macro is released in~\cite{ftpofs} and the same version
is included in \TeX{} support of fonts from Storm Type 
Foundry~\cite{stormtex}. The \TeX{} font metrics and fd files
are included in this support. The real fonts (implemented in pfb files) 
are not free. If you take up your interest
in these fonts you have to buy them. See~\cite{wwwstorm} for more details.
One family called Lido is offered as free for non-commercial use
by Storm Type Foundry. There is the special \TeX{} support only for
this family~\cite{slido} including the OFS.

The first release of OFS macro (September 2000) had only Czech
documentation. The English version of the documentation was released
in February 2003 with help of Mat\v ej Cepl. The concept of OFS was not
changed since its first release and Petr Ol\v s\'ak does not plan any changes
to the future. He is ready only to remove bugs, if any occur.

The documentation can be found at the same place as the OFS macro.
It is a bit more technical than this paper.


\kap Basics of user environment
%------------------------------

The user environment is the same in both: OFS for plain\TeX{} and
\LaTeX{}. Of course, you have to include the OFS macro in plain\TeX{}
by little bit different way:

\begtt
\input ofs [fonts, collections]
\endtt
%
than in \LaTeX{}: 

\begtt
\usepackage [fonts, collections] {ofs}
\endtt
%
All other user-level commands have the same syntax and meaning.

You can use the command "\fontusage". This command prints to the
terminal and to the log file the basic usage of OFS:

\begtt
$ tex ofs \\fontusage
This is TeX, Version 3.14159 (Web2C 7.3beta5)
(/usr/local/share/texmf/tex/csplain/ofs.tex
OFS (Olsak's Font System) based on plain initialized. <Oct. 2002>
(/usr/local/share/texmf/tex/csplain/ofsdef.tex))
\fontusage: ============== Olsak's Font System, usage: =================
\input ofs [sjannon, sdynamo, a35] ... for example
\showfonts  ... shows all loaded font families (by previous \input)
\setfonts [Family/] ... local switch to the new family, after this, the
  \rm, \bf, \it, bi will switch to the variants. The current size is used.
\setfonts [/size] ... local switch to the new size of fonts, the family is
  not changed. The "size" has the following possible formats:
    at<dimen>       ... the same as \font\something=file at<dimen>
    <dimen>         ... the same as at<dimen>
    <number>        ... the same as at<number>pt
    scaled<number>  ... the same as \font\something=file scaled<number>
    mag<decimal-number> fonts will be magnified by given coefficient
                        depend on current size of the fonts.
\setfonts [Family/size] ... switch to the new family at given size
\setfonts [Family-vr/]  ... switch to the specified font, the current size
  is used. The "vr" is acronym for variant (bf for example).
\setfonts [Family-vr/size] ... switch to the specified font.
\fontdef\name [Family/size] ... same as \gdef\name{\setfonts[Family/size]}
  The "Family" or "size" parameter may be empty.
\fontdef\name [Family-vr/size] ... \name is fixed-font switch iff:
  "size" is no empty and no mag<dec-number>.
  Fixed-font switch "\name" is implemented as \global\font\name=file.
\setmath [size/size/size] ... set math it/rm as current it/rm + use PS Symbol
\nofontmessages, \logfontmessages, \displayfontmessages, \detailfontmessages
  ... the levels of log.
*
\endtt

The "\showfonts" prints the used font collections. You can include the
OFS macro without additional font collections in square brackets (it means
use only "\input ofs"). The basic font collection is initialised in
such situation:

\begtt
*\showfonts
OFS (l.0): The list of known font families:
defaults:
   [CMRoman/]            \rm, \bf, \it, \bi, \sl
   [CMSans/]             \rm, \bf, \it,  - 
   [CMTypewriter/]       \rm,  - , \it,  - , \sl
   [Times/]              \rm, \bf, \it, \bi
   [Helvetica/]          \rm, \bf, \it, \bi, \nrm, \nbf, \nit, \nbi
   [Courier/]            \rm, \bf, \it, \bi
\endtt
%
The family names are printed in square brackets here and followed by varinat
switches usable for such font family.

The basic fonts from Computer Moder family (by Donald Knuth) are
collected into three families here: "CMRoman", "CMSans" and 
"CMTypewriter". This is sel explanatory. Note that the common
varinat "BoldItalic" ("\bi") is missing in "CMSans" and
"CMTypewriter". On the other hand the 
``special'' variant "\sl" (slanted) is available in 
"CMRoman" and "CMTypewriter" families.

The declaration of another families are stored in ``declaration
files'' (the extension "tex" is used in OFS for plain\TeX{} and "sty"
in OFS for \LaTeX{}. It is recommended to manage the file
"allfonts.tex" in all \TeX{} distribution. This file includes the
"\input"s to the declaration files of all fonts installed in such
distribution. The "allfonts.tex" file may look like:

\begtt
$ cat `kpsewhich allfonts.tex`
%%% All OFS families declared on this TeX
%%%%%%%%%%%%%%%%%%%%%%%%%%%%%%%%%%%%%%%%%
%%%                            Petr Olsak
\input a35       % PostScript 35
\input ffonts    % Another free fonts
\input btfonts   % Bitstream fonts
\input skatalog  % Stromtype foundry, 89 families
\endtt
%
When we write "$ tex allfonts \\showfonts \\end | less", we get the
listing of all font families available on \TeX{} distribution. This
listing includes cca three hundred font families on the computer of
one of the author of this article. He need not remember the tfm file
names of thousands of fonts. The part of the listing follows:

\begtt
OFS (l.0): The list of known font families:
defaults:
   [CMRoman/]            \rm, \bf, \it, \bi, \sl
   [CMSans/]             \rm, \bf, \it,  - 
   [CMTypewriter/]       \rm,  - , \it,  - , \sl
   [Times/]              \rm, \bf, \it, \bi
   [Helvetica/]          \rm, \bf, \it, \bi, \nrm, \nbf, \nit, \nbi
   [Courier/]            \rm, \bf, \it, \bi
a35.tex:
   [AvantGarde/]         \rm, \bf, \it, \bi
   [Bookman/]            \rm, \bf, \it, \bi
   [NewCentury/]         \rm, \bf, \it, \bi
   [Palatino/]           \rm, \bf, \it, \bi
   [ZapfChancery/]       \rm,  - , \it,  - 
   [ZapfDingbats/]       \rm,  - ,  - ,  - 
   [Symbol/]             \rm,  - , \it,  - 
ffonts.tex:
   [Charter/]            \rm, \bf, \it, \bi
...
sjannon.tex:
   [JannonAntikva/]      \rm, \bf, \it, \bi, \mr, \mi
   [JannonText/]         \rm, \bf, \it, \bi, \mr, \mi
   [JannonCaps/]         \rm, \bf, \it, \bi
...
sdynamo.tex:
...
   [DynaGroteskLE/]      \rm, \bf, \it, \bi
   [DynaGroteskD/]       \rm, \bf, \it, \bi
   [DynaGroteskR/]       \rm, \bf, \it, \bi
...
stitul.tex:
   [Alcoholica/]         \rm, \bf, \it,  - 
   [Monarchia/]          \rm, \bf,  - ,  - 
   [MonarchiaText/]      \rm, \bf,  - ,  - 
   [Clichee/]            \rm, \bf, \it, \bi
   [Regula/]             \rm,  - , \it,  - 
   [Splendid/]           \rm, \bf, \it,  - , \script, \sans
   [Cobra/]              \rm, \bf,  - ,  - 
   [ExcelScript/]        \rm,  - ,  - ,  - , \ext
   [ExcelScriptText/]    \rm,  - ,  - ,  - , \ext
   [Zeppelin/]           \rm, \bf,  - ,  - , \lr, \coll
   [Negro/]              \rm,  - ,  - ,  - 
   [Farao/]              \rm, \bf,  - ,  - , \kr, \coll
...
slido.tex:
   [Lido/]               \rm, \bf, \it, \bi, \crm, \cbf
\endtt

If you need recall what the special variant switches means
("\mr" and "\mi" in JannonText family, for instance), you can try 
to switch to this family and look to the log file 
(or to the terminal if "\displayfontmessages" is set):

\begtt
*\displayfontmessages \setfonts [JannonText/]
OFS (l.0): Font family JannonText at10pt (enc=8z) activated:
OFS (l.0):  \rm () \bf (Bold) \it (Italic) \bi (BoldItalic)
OFS (l.0):  \mr (Medium) \mi (MediumItalic)
*
\endtt

Aha, "\mr" means MediumItalic variant of JannonText family.
The "\it" means sometimes "Italic" variant and sometimes "Oblique"
variant. You can try:

\begtt
*\setfonts[Helvetica/]
OFS (l.0): Font family Helvetica at10pt (enc=8z) activated:
OFS (l.0):  \rm () \bf (Bold) \it (Oblique) \bi (BoldOblique)
OFS (l.0):  \nrm (Narrow) \nbf (NarrowBold) \nit (NarrowItalic) 
            \nbi (NarrowBoldItalic)
*
\endtt

The declaration file name is mentioned in the listing of all families
(see above). For example the file "a35.tex" declares the font families
from standard PostScript ``35 fonts collection'' from Adobe (no all
ten families are declared here because three families of them are 
declared in basic OFS font collection).

You can write "\input a35" instead of "\input ofs [a35]". The second
alternative is recommended if more font collections are used together,
for example: "\input ofs [a35, sdynamo, slido]". You can use

\begtt
\input ofs [allfonts]
\endtt
%
(or simply "\input allfonts") in your document
if all font collections are needed or if you need not use the
names of individual font collections.

OFS provides the printing of simple font catalog. For example, you can
write on your command line:

\begtt
$ pdfcsplain ofscatal [sjannon, sdynamo, stitul] ; acroread ofscatal.pdf
\endtt


\kap The setfonts command

You can select the font family and/or font size by "\setfonts"
command. Two parameters are separated by slash in square brackets. 
The first parameter is the font family name and the second one is the
font size. If one of the parameters is missing then this aspect stay
unchanged. After the font family is selected, you can use the variant
switches. Most common switches are "\rm", "\bf", "\it", "\bi", but
another switches can be available for some font family. See the
listing produced by "\showfonts" for more detail.

The "\setfonts" command keeps the variant from previous family
unchanged after setting of the new family if the current variant is
available in new font family. If not then the "\rm" variant is
initialized. All families have to support at least the "\rm" variant.

Examples:

\begtt
\setfonts [JannonText/10.5]  % the normal font
\setfonts [/14]\bf           % used for titles
\setfonts [/8]\rm            % for footnotes
\setfonts [DynagroteskR/]\it % for citations
\setfonts [CMTypewriter/]    % monospaced font
       % if you use \setfonts [CMTypewriter/] when
       % it is active then Typewriter italics is used.
\endtt

The main advantage is that you can use the same names of font families
as in Typo-catalog and you need not remember the cryptical names of tfm
files or abbreviations of family names in NFSS. 

If the family name is not present in the internal OFS dictionary
(perhaps misspelling) then "\setfonts" command prints warning plus
all available families to log and to the terminal (the same as the
"\showfonts" command).

The font size can be specified as decimal number without unit 
(the unit "pt" is appended automatically) or you can type number with
arbitrary \TeX{} unit ("mm" for example). Moreover you can use the
keyword "scaled" before number with the same meaning as in "\font"
primitive. OFS introduces the new keyword "mag" followed by decimal
number (decimal point is required). This number denotes the fraction
for the current font size. For example:

\begtt
\def\smaller{\setfonts [/mag0.8]}
This text is {\smaller typeset by small \smaller and more small
\smaller and more more small characters} and the normal size is used
here.
\endtt
%
yields to:

\medskip
\def\smaller{\setfonts [/mag0.8]}
This text is {\smaller typeset by small \smaller and more small
\smaller and more more small characters} and the normal size is used
here.
\medskip

We can use this feature in \LaTeX{} logo, for example:

\begtt
\def\LaTeX{L\kern-.2em\raise.45ex\hbox{\setfonts[/mag.7] A}\kern-.05em\TeX}
\endtt

This solution works in titles (the raised A is bod in this situation),
in normal text, footnotes, italics etc and in all font families.
This feature is not implemented in NFSS thus the \LaTeX{} logo has its
A implemented as superscript math font in \LaTeX{} kernel. I mean that
this is not a best idea.

The another usage of mag keyword is to make a corrections of not
perfectly the same visual ex height of used font. This problem arises
if you combine the CMTypewriter family with some common PostScript
font families for instance. The CMTypewriter font seems to be smaller
if you use exactly the same design size. This is no problem:
you can define "\tt" as "\setfonts [CMTypewriter/mag1.1]" and the ex
height is balanced. This definition works in all sizes and font
variants.

The NFSS keeps another aspect of fonts: the font weight. The main reason of
this feature is to boldify all text (in titles, for example) including
its italics part. I decided that this feature is not needed in OFS
because plain user can define titles in the following way:

\begtt
\def\chapterfont {\setfonts[/14]\bf \let\it=\bi}
\endtt
%
Moreover you can simply define the macros which keeps more than five
aspects (NFSS keeps exactly five aspects). You can find the examples
in \TeX{} support of \v Storm's font where the big Dynagrotesk family implements
the special font selector which keeps the level of ``condensation of
the font'' in addition to the weight, variant (normal/italics), size
and encoding. 

You can declare you own font-switches in OFS by "\fontdef" command.
The code "\fontdef\fontswitch [Family/size]" declares the
"\fontswitch" similar as "\gdef\fontswitch{\setfonts[Family/size]}".
Example:

\begtt
\input ofs [ffonts]                 % font Charter is in free fonts
\setfonts [Charter/10pt]            % default family
\fontdef\tt [CMTypewriter/mag1.1]   % monospaced font
\fontdef\verbtt [CMTypewriter/8]    % monospaced font in verbatim
\fontdef\headfont [Charter/9]       % the font+size for headers
\endtt

It would be more convenient to write the word ``Charter'' only once
(this word seems to be redundant in declaration of "\headfont"). But the
macro code "\fontdef \headfont [/9]" cannot work everywhere because the
"\headfont" can be used for headers in output routine and this header have
to be in Charter every time, no in CMTypewriter, if CMTypewriter family
is the current when output routine is called out. 

You can solve this problem by using the ``"!"''
character for the family parameter in "\fontdef". This character is
substituted by the current family in the time of "\fontdef" declaration:

\begtt
\fontdef\headfont [!/9] \addcmd\small {\rm} % the font for headers
\endtt
%
Moreover the "\addcmd" is used here. This command adds the additional
code to the macro "\headfont" thus, the example above has the same effect
as:

\begtt
\gdef\headfont {\setfonts[Charter/9]\rm} % the font for headers
\endtt
%
but the word Charter is not used here. Now, you can change the default
family only on one place and the new default family will be used in
headers too.

The whole font declaration for your plain\TeX{} document can be 
localized in one place in your macro code and it can be look 
like the following code:

\begtt
\input ofs [ffonts]                    % font Charter is in free fonts
\setfonts [Charter/10pt]               % default family
\fontdef \tt [CMTypewriter/mag1.1]     % monospaced font
\fontdef \verbtt [CMTypewriter-rm/8]   % monospaced font in verbatim
\fontdef \headfont [!/9]     \addcmd \headfont {\rm}
\fontdef \abstractfont [!/9] \addcmd \abstractfont {\baselineskip11pt\rm}
\fontdef \sectionfont [!/12] \addcmd \sectionfont {\bf \let\it=\bi}
\fontdef \titlefont [!-bf/14.4]        % for the title of the document
\endtt

If the ``"-vs"'' (where "\vs" is the variant switch) is used in
parameter of "\fontdef" or "\setfonts" then the whole family is not
initialized but only the single font is used as the current font. 
The "\fontdef\name" is the equivalent of "\global\font\name" in such
situation. This convention is used in the example above for title font
because no variant switches is used in text of
title.


\kap Font encoding
%-----------------

OFS for plain\TeX{}% 
   \footnote{OFS for \LaTeX{} does not solve font encoding,
             use NFSS tools for font encoding manipulation}
initializes the CSfont encoding by default.
This does no matter for English users because CSfonts have absolutely
the same encoding(s) as Computer Modern in slots 0 to 127.
The users of T1 encoded fonts have to set the default font encoding by
the code: "\def\fotenc{8t}" before OFS is loaded. The macro "\fotenc"
includes a part of used metric file names. If this macro is not
defined then OFS define it as "8z"---it means CSfont encoding.

If "\fotenc" has the value "8z" then the families "CMRoman",
"CMTypewriter" and "CMSans" will work with CSfonts and the font
metrics with "8z" encoding is used for another font families. If
"\fotenc" is set to "8t" then DC fonts are used in "CMRoman",
"CMTypewriter" and "CMSans" families. The EC fonts can be used instead
DC fonts, but you have to write the declaration file for OFS and for
EC fonts. This file does not exist now, sorry.

Theoretically, you can switch between encoding inside the
document but this is not common used practice:

\begtt
\input ofs
\setfonts [Times/]  text 1 % used metric: ptmr8z, CSfont encoding
\def\fotenc{8t}
\setfonts [/]       text 2 % used metric: ptmr8t, T1 encoding
\endtt
%
If you are using another font encoding with metric names "*8x" (for
example) then you can do "\def\fotenc{8x}". No problem.

You can find the files "ofs-8z.tex" and "ofs-8t.tex" in OFS
package. The accents declaration and another encoding-dependent macros
are included here. By default, no of these files is read, it means
that the accent macros keep its original meaning from plain\TeX{}.
You can use "\input" of one of these files or "\input" both files
(there are no conflicts in these files). Assume that both files are
read. Then accent macros (such as "\v", "\'" etc.) followed by
a character expands to the new character code of CSfonts 
(if "\fotenc" is "8z") or to the new code of T1 encoding
(if "\fotenc" is "8t"). Another encoding dependent macros (such as
"\promile") work in the chosen encoding by "\fotenc".

We can look into the parts of "ofs-8z.tex" and "ofs-8t.tex" files:

\begtt
%%% Default accents in CM
\accentdef  \` *    8z  {\accent 18 }                   % grave
\accentdef  \' *    8z  {\accent 19 }                   % acute
\accentdef  \v *    8z  {\accent 20 }                   % caron
\accentdef  \u *    8z  {\accent 21 }                   % breve
...
%%% Standard characters in plain (redefined here)
\def\aa{\r a} 
\def\AA{\r A}
\characterdef \i    8z  16 
\characterdef \j    8z  17
\characterdef \SS   8z  {SS}
\characterdef \AE   8z  29
...
%%% Extra characters from CS fonts
\characterdef \promile      8z  141
\characterdef \varhyphen    8z  156
\characterdef \flqq         8z  158
\characterdef \frqq         8z  159
\characterdef \clqq         8z  254
\characterdef \crqq         8z  255
...
%%% Accented letters from CS fonts
\accentdef  \` A      8z  152
\accentdef  \' A      8z  193
\accentdef  \" A      8z  196
\accentdef  \` a      8z  184
\accentdef  \' a      8z  225
...
%%% Default accents in Cork
\accentdef  \` *    8t  {\accent 0 }               
\accentdef  \' *    8t  {\accent 1 }
\accentdef  \^ *    8t  {\accent 2 }
\accentdef  \~ *    8t  {\accent 3 }
\accentdef  \" *    8t  {\accent 4 }
...
%%% Standard characters in plain (redefined here)
\def\aa{\r a} 
\def\AA{\r A}
\characterdef \i    8t  25
\characterdef \j    8t  26
\characterdef \SS   8t  223
\characterdef \AE   8t  198
...
\characterdef \promile              8t {\%\char 24 }        
\characterdef \textpertenthousand   8t {\%\char 24\char 24 } 
...
%%% Accented letters from T1 encoding
\accentdef  \. i    8t `\i
\accentdef  \u A    8t 128
\accentdef  \k A    8t 129
\accentdef  \' C    8t 130
\accentdef  \v C    8t 131
...
\endtt

This example is self explanatory and illustrates the language for
accents and encoding-dependent macro declarations. You can find more
information in documentation~\cite{ofsdoc}.

The \LaTeX{} user can be familiar with the commands with similar 
meaning: "\DeclareTextSymbol", "\DeclareTextComposite", etc.
 
OFS calculates with the possibility of existence of a special metric
with extra characters (such us "\euro"). Each basic metric can be
connected with such ``extra metric'' to the couple. These couples are
used in \TeX{} support of Storm Type Foundry because these fonts have
more than 256 characters.  There is prepared an extra metric "*6s"
for each basic metric "*8z" or "*8t" in \TeX{} support of \v Storm's
fonts.  The "*6s"-encoding specific macros are declared in
"stormenc.tex" by "\accentdef" and "\characterdef" commands.  If the
current metric has declared an extra metric and user calls the
macro declared by "\characterdef" or "\accentdef" and this macro is
not declared for current encoding but only for the encoding of the
extra metric then the extra metric is temporary used without any user
intervene. Example:

\begtt
\characterdef \euro           6s  136
Now the \euro{} works in all fonts with extra metric *6s
\endtt



\kap Declaration files
%---------------------

Look into "a35.tex" now for an example of language of 
the declaration files:

\begtt
%%% Times, Helvetica, Courier is in OFS defaults

\ofsdeclarefamily [AvantGarde] {% -------------------- AvantGarde
   \loadtextfam (Book)         pagk\fotenc;%    \rm
                (Demi)         pagd\fotenc;%    \bf
                (BookOblique)  pagko\fotenc;%   \it
                (DemiOblique)  pagdo\fotenc;;%  \bi
   \def\TeX{T\kern-.08em\lower.3333ex\hbox{E}\kern-0.09emX}%
}
\ofsdeclarefamily [NewCentury] {% -------------------- NewCenturySchlbk
   \loadtextfam (Roman)        pncr\fotenc;%    \rm
                               pncb\fotenc;%    \bf
                               pncri\fotenc;%   \it
                               pncbi\fotenc;;%  \bi
   \def\TeX{T\kern-.1667em\lower.5ex\hbox{E}\kern-.125emX}%
}
...
\endtt
%
The mapping between font family names and metric files is defined
here.%
\footnote{OFS for \LaTeX{} uses other declaration files:
          the mapping between family names and NFSS short names
          are defined here.
}
Each family declares four metrics for four common variants "\rm",
"\bf", "\it" and "\bi". Empty parameter means that the variant 
is missing. The metric names are saved by "\fotenc" macro in order 
to possibility of changing the used encoding. The extra metric can be
written before the last semicolon (not used in this example). 
The commands from parameter of the "\ofsdeclarefamily" are processed
when "\setfonts" is used. You can see the alternative definition of
\TeX{} logo for each family in "a35.tex" file on order to optimize the
visual aspect of this logo.

The optional parameters of "\loadtextfam" macro is written in brackets
and these parameters declares mapping of short variant switches and
the full variant names printed to the log and terminal. For example,
BookOblique is a "\it" variant in AvantGarde font family. If the
common name is used (Bold/Italics/BoldItalics) then this parameter can
be missing (see NewCentury family, for instance). 

How the special variants are declared? Look into the file
"sjannon.tex", for example:

\begtt
\ofsdeclarefamily [JannonAntikva] {% -------- Jannon Antikva
   \loadtextfam sjnr\fotenc;%
                sjnb\fotenc;%
                sjnri\fotenc;%
                sjnbi\fotenc;6s;%
   \newvariant2 \mr (Medium)        sjnm\fotenc;6s;%
   \newvariant3 \mi (MediumItalic)  sjnmi\fotenc;6s;%
}
\ofsdeclarefamily [JannonText] {% ------------ Jannon Text
   \loadtextfam sjnrg\fotenc;%
                sjnbg\fotenc;%
                sjnrig\fotenc;%
                sjnbig\fotenc;6s;%
   \newvariant2 \mr (Medium)        sjnmg\fotenc;6s;%
   \newvariant3 \mi (MediumItalic)  sjnmig\fotenc;6s;%
}
...
\endtt
%
Note that there is declared the extra metric with~"6s" encoding here.
See the documentation~\cite{ofsdoc} for more information.

How the "CMRoman" etc. families are declared? These families have two
special subjects. First: the metrics file names do not include the "8z" or
"8t" acronym for encoding. Second: the different metrics are used
for different font sizes. Both problems are solved by the OFS command
"\registertfm" (see the "ofsdef.tex" file): 

\begtt
\registertfm cmr8z     -      csr10  % metric for all sizes
\registertfm cmr8z  0pt-6pt   csr5
\registertfm cmr8z  6pt-7pt   csr6
\registertfm cmr8z  7pt-8pt   csr7
\registertfm cmr8z  8pt-9pt   csr8
\registertfm cmr8z  9pt-10pt  csr9
\registertfm cmr8z  10pt-12pt csr10
\registertfm cmr8z  12pt-17pt csr12
\registertfm cmr8z  17pt-*    csr17
...
\registertfm cmr8t     -   dcr10     % metris for all sizes
...
\ofsdeclarefamily [CMRoman] {% ------------ Computer Modern Roman
   \loadtextfam cmr\fotenc;%
                cmbx\fotenc;%
                cmti\fotenc;%
                cmbxti\fotenc;;%
   \newvariant8 \sl (Slanted) cmsl\fotenc;;%
}
\endtt


\kap Math fonts
%--------------

The math fonts are collected in math families (three fonts 
per one family)
by "\textfont", "\scriptfont" and "\scriptscriptfont" primitives.  The
math families with number 0, 1, 2, 3 have special meaning in math
typesetting. The declaration of new math family by \TeX{} primitives
is not too comfortable. Plain\TeX{} user can declare the new math
family by OFS macro "\loadmathfam". This command will be described
below.

Plain\TeX{} user have to initialize the math fonts in OFS by
"\setmath" command. The math fonts are in the same state as declared
in plain\TeX{} macro until the "\setmath" command is used. It means
that the Computer Modern at 10/7/5 pt size are used.
The "\setmath" command have three parameters in square brackets 
separated by slashes. These parameters means the
text/script/scriptscript size of the math fonts. The empty parameter
means that the mag1.0/mag.7/mag.5 (relatively to the current size of
textual font) is substituted:

\begtt
\setmath [//] is the same as \setmath [mag1.0/mag.7/mag.5]
\endtt

The "\setmath" command calculates the needed sizes from given
parameters and starts the "\mathfonts" macro followed by macro
"\mathchars". The plain\TeX{} user can define these macros in 
his own opinion but OFS gives the reasonable default values of these
macros. These macros run the variant part of the code depend on the
values of the "\fomenc" and "\mathversion" macros.

If "\def\fomenc{PS}" is used (it is default value in OFS) then "\setmath"
initializes math fonts by the following way: math italic is loaded
from text italics of the current text font family, family~0 is loaded
from "\rm" variant of the current family. The math symbols are loaded
(if it is possible) from common PostScript font Symbol. The rest
(what is not included in Symbol font) is realized from Computer Modern
fonts. The math encoding is redefined (by "\mathchardef" etc. primitives) 
for many symbols in order we keep the accessibility of all math 
characters declared in plain\TeX{}. For example the lower letter greek
characters are loaded from the slanted variant of PostScript Symbol font.

If you use "\def\fomenc{CM}" then "\setmath" loads the math fonts from
Computer Modern family and does not change the math encoding.
In this case, the "\setmath[//]" command only sets the actual sizes of these fonts
depend on current text font size. 

If you buy the MathTimes family, you can "\input" the "ofsmtdef.tex"
file (this is a part of OFS package) and you can use "\def\fomenc{MT}".
Then the "\setmath" command loads the math italics from current "\it"
font and family~0 from current "\rm" font. Moreover, it loads the
MathTimes fonts for all math symbols. Unlike "\def\fotens{PS}" mode
there is no ``rest'': no symbols needs to be realized from Computer
Modern family.

You can control the math families collection loaded by "\setmath" by
the value of "\mathversion" macro. OFS declares two
math families collections: "\def\mathversion{normal}"
and "bold". You can declare more collections if you need it.
The "bold" collection is the same as "normal", but bold variants of
italics, family~0 and math symbols (if accessible) are loaded.

The example from "ofsdef.tex" file illustrates the language of
declarations of the math fonts:

\begtt
\def\defaultmathfonts{\csname load\fomenc\mathversion math\endcsname}
\def\defaultmathchars{\csname set\fomenc mathchars\endcsname}
\def\mathfonts{\defaultmathfonts}
\def\mathchars{\defaultmathchars 
   \let\mathchars=\relax % to protect the twice math-setting
}
\def\loadPSnormalmath{%
  \loadmathfam 0[tenrm/]%              Actual Roman font
  \loadmathfam 1[tenit/]%              Actual Italic font
  \loadmathfam 2[/cmsy]%               Standard symbols from CM
  \noindexsize\loadmathfam 3[tenex/]%  Standard extra symbols from CM
  \chardef\itfam=1 \chardef\bifam=5 
  \loadmathfam \bffam [tenbf/]%        Actual Bold font
  \loadmathfam \bifam [tenbi/]%        Actual Bold Italic  
  \newmathfam\symbfam
  \loadmathfam \symbfam [/psyr]%       PostScript Symbol
  \newmathfam\symbofam
  \loadmathfam \symbofam [/psyro]%     PostScript Symbol Oblique
}
\def\loadPSboldmath{%
  \loadmathfam 0[tenbf/]%              Actual Bold font
  \loadmathfam 1[tenbi/]%              Actual Bold-Italic font
...
}
\def\loadCMnormalmath{%
  \loadmathfam 0[/cmr8z]%                    Roman font
  \loadmathfam 1[/cmmi]%                     Math Italic font
  \loadmathfam 2[/cmsy]%                     Standard symbols from CM
  \noindexsize\loadmathfam 3[/cmex10]%       Standard extra symbols from CM
...
}
\def\loadCMboldmath{%
  \loadmathfam 0[/cmbx8z]%                     Roman font
  \loadmathfam 1[/cmmib10]%                    Math Italic font
...
}
\def\setPSmathchars{%
  \fontmessage{\ofsmessageheader Math codes are set for PS encoding}%
  \mathcode`\,="602C
  \mathcode`\.="002E
  \delcode`\<="\hex\symbfam E130A
  \delcode`\>="\hex\symbfam F130B
  \delcode`\|="\hex\symbfam 7C30C
  \edef\langle{\delimiter"4\hex\symbfam E130A }%
  \edef\rangle{\delimiter"5\hex\symbfam F130B }%
  \mathchardef\alpha    "0\hex\symbofam 61
  \mathchardef\beta     "0\hex\symbofam 62
  \mathchardef\gamma    "0\hex\symbofam 67
...
}
\endtt

If you need to add next math families (math alphabets in NFSS
terminology) then you can use the cod similar as the following:

\begtt
\def\mathfonts{\defaultmathfonts
  \newmathfam\bbfam 
  \loadmathfam \bbfam [/bbold12]%  Dvojit� vertik�ln� kresba
  \def\bb{\fam\bbfam}%
}
\def\mathchars{\defaultmathchars 
  \mathchardef\balpha    "0\hex\bbfam 0B
  \mathchardef\bbeta     "0\hex\bbfam 0C
  ...
} 
\endtt

We re-write the example from section~3 in order to support the math
typesetting: 

\begtt
\input ofs [ffonts]                  % font Charter is in free fonts
\setfonts [Charter/10pt]             % default family
\setmath[//]                         % math initialisation
\fontdef\tt [CMTypewriter/mag1.1]    % monospaced font
\fontdef\verbtt [CMTypewriter-rm/8]  % monospaced font in verbatim
\fontdef\headfont [!/9]              % running heads
\addcmd \headfont {\baselineskip11pt\rm\def\mathversion{normal}\setmath[//]}
\fontdef\sectionfont [!/12]          % section font
\addcmd \sectionfont {\bf \let\it=\bi \def\mathversion{bold}\setmath[//]} 
\fontdef\titlefont [!-bf/14.4]       % for the title of the document
\endtt

The math fonts works in all sizes in our virtual example. The math
formulae are in bold variant and in right size in section headers.

\reference

\bib \url{www.stormtype.com}

\bib \url{www.pismolijna.cz}, 

\bib \url{www.cstug.cz/stormtype}.

\bib \url{www.cstug.cz/stormtype/slido.html}.

\bib \url{ftp://math.feld.cvut.cz/pub/olsak/ofs}.

\bib Petr Ol��k. {\it OFS: Ol��k�v fontov� syst�m}. 2001.
     The English documentation is in files "ofsdoc-e.tex", "ofsdoc-e.pdf"

\bib Petr Ol��k. {\it Jak \TeX{} k~font�m ze St�e�ovic p�i�el}. Bulletin
     of CSTUG, number 4/2001, pp~153--180.

\bib Karl Berry. {\it Fontname}, March 1999. The documentation is included
     in {\tt web2c} \TeX, file {\tt fontname.texi}, {\tt fontname.pdf}


\bye 

