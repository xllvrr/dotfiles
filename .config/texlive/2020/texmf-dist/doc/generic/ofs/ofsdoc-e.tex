% A documentation for OFS package (pseudo English, sorry)
%%%%%%%%%%%%%%%%%%%%%%%%%%%%%%%%%%%%%%%%%%%%%%%%%%%%%%%%%
% July 2001                                    Petr Olsak

\magnification\magstephalf % use plain or csplain

\headline={\llap{\tt May 2004}\hfil See the ofsdoc.tex for Czech version 
           of this documentation.\global\headline={\hfil}}

\hoffset=4pc
\voffset=1pc
\hsize=31.6pc
\vsize=49pc
\raggedbottom
\parindent=15pt
\emergencystretch=2em

\font\smallbf=csb10 at7pt
\font\smalltt=cstt8
\font\smallit=csti8
\font\smallrm=csr8
\font\titulfont=\fontname\tenbf\space scaled\magstep4
\font\bigbf=\fontname\tenbf\space scaled\magstep1

\def\Red{}
\def\Black{}
\def\Blue{}
\def\Green{}
\def\beglink#1{}
\def\endlink{}
\def\aimlink#1{}

\def\uv#1{``#1''}

\ifx\pdfoutput\undefined\else   %%%% pdfTeX is used %%%%%%%%%%
\ifnum\pdfoutput>0    

\ifx \pdfstartlink\undefined %%% PDFTeX version <= 13
   \let\pdfstartlink=\pdfannotlink
\fi

\def\beglink#1{%          % Za��tek textu odkazu, #1 je kl�� odkazu
   \Green \pdfstartlink height9pt depth3pt 
     attr{/Border[0 0 0]} goto name{#1}\relax}
\def\endlink{\pdfendlink\Black}

\def\aimlink#1{%          % M�sto c�le odkazu, #1 je kl�� odkazu
   \expandafter\ifx \csname aim:#1\endcsname \relax
      \expandafter\gdef \csname aim:#1\endcsname {}%
      \vbox to0pt{\vss\hbox{\pdfdest name{#1} fith}\kern15pt}%
   \fi
}
\def\pdfsetcmykcolor#1{\special{PDF:#1 k}}
\def\Red{\leavevmode\pdfsetcmykcolor{0.1 0.9 0.9 0}}
\def\Black{\leavevmode\pdfsetcmykcolor{0 0 0 1}}
\def\Green{\leavevmode\pdfsetcmykcolor{0.9 0.1 0.9 .3}}
\def\Blue{\leavevmode\pdfsetcmykcolor{0.9 0.9 0.1 0}}

\pdfcompresslevel=9
\pdfinfo{/Author (Petr Olsak)
         /CreationDate (July 2001) 
         /ModDate (May 2004)
         /Creator (TeX)
         /Producer (pdfTeX)
         /Title (OFS: Olsak's Font System)
         /Subject (Documentation)
         /Keywords (TeX, fonts)
}

\fi\fi %%%%%%%%%%%%%%%%%%%%%%%%%%%% End of pdfTeX macros %%%%%


\def\PLAIN #1:{{\smallbf \Red PLAINTEX\Black\if|#1|\else\space#1\fi}:}
\def\LATEX #1:{{\smallbf \Red LATEX\Black \if|#1|\else\space#1\fi}:}
\def\OBA   #1:{{\smallbf \Red PLAINTEX+LATEX\Black \if|#1|\else\space#1\fi}:}

\newcount\secnum \newcount\subsecnum

\def\sec #1\par{\ifnum\secnum>0 \goodbreak\fi\removelastskip
   \vskip2\baselineskip
   \subsecnum=0 \advance\secnum by1
   \noindent{\bigbf\llap{\the\secnum.\quad}#1}\par\nobreak\medskip}
\def\subsec #1\par{\removelastskip\bigskip
   \advance\subsecnum by1
   \noindent{\bf \llap{\aimlink{\the\secnum.\the\subsecnum}%
      \the\secnum.\the\subsecnum.\quad}#1}\par\nobreak\medskip}
\def\title #1\par{\vglue2\baselineskip 
   \centerline{\titulfont #1}\vskip2\baselineskip}

\def\LaTeX{La\TeX}

\catcode`<=13
\def<#1>{\hbox{$\langle$\it#1\/$\rangle$}}
\def\,{\thinspace}

{\obeyspaces \gdef\activespace{\obeyspaces\let =\ }}
\def\setverb{\def\do##1{\catcode`##1=12}\dospecials}
\def\begtt{\medskip\bgroup \setverb \catcode`@=0 \activespace
   \parindent=0pt \catcode`\"=12
   \def\par##1{\endgraf\ifx##1\par\leavevmode\fi ##1}
   \obeylines \startverb}
{\catcode`\|=0 \catcode`\\=12
|gdef|startverb#1\endtt{|tt#1|egroup|medskip|testpar}}
\long\def\testpar#1{\ifx\par#1\else\noindent\fi#1}

\catcode`"=13
\def"{\hbox\bgroup\setverb\activespace\tt\readverb}
\def\readverb #1"{#1\egroup}

\def\begitems{\medskip\bgroup \catcode`*=13 }
{\catcode`*=13 \gdef*{\item{$\bullet$}}}
\def\enditems{\medskip\egroup}

\newwrite\indout
\immediate\openout\indout=\jobname.ind
\def\inl[#1]{\strut
   \write\indout{\noexpand\indexentry{#1}{\the\pageno}}%
   \vadjust{\vbox to0pt{\vss
   \hbox to\hsize{\llap{{\aimlink{#1}\smalltt\Blue\char`\\ #1%
      \ifx\extratext\empty\else\space\extratext\fi\Black}%
      \kern2pt\quad}\strut\hfil}%
   \hrule height0pt}}\def\extratext{}}
\def\inll[#1 #2]{\def\extratext{#2}\inl[#1]}
\def\indexentry #1#2{\expandafter \ifx \csname ind:#1\endcsname \relax
      \expandafter \def \csname ind:#1\endcsname {#2}%
   \else 
      \expandafter \edef \csname ind:#1\endcsname {%
          \csname ind:#1\endcsname, #2}%
   \fi
}
\def\pg[#1]{\csname ind:#1\endcsname}

\chardef\back=`\\

%%%%%%%%%%%%%%%%%%%%%%%%%%%%%%%%%%%%%%%%%%%%%%%%%%%%%%%%%%%%%%%%%%%


\title   OFS: The Ol\v s\'ak's Font System
%%%%%%%%%%%%%%%%%%%%%%%%%%%%%%%%%%%%%%%%%%

The OFS is a \TeX{} macro for managing large sets of fonts.  You can
select the appropriate fonts comfortably by the names from font
catalog used by a font foundry. It means you don't have to remember short
names of tfm files and/or short names of NFSS font families. The user
interface of this macro is the same in \LaTeX{} and in plain but there
are two independent implementations of this macro: first and more
elaborate: based only on plain macros; second: based on NFSS macros
for \LaTeX{} users.

If a text in this documentation is applicable only for the plain\TeX{}
version of the OFS then the text is introduced by the word \PLAIN:. 
If a text is applicable only for OFS version based on NFSS then it is
introduced by the word \LATEX:.

After the OFS macro is loaded, the following message is printed:

\begtt
@PLAIN: OFS (Olsak's Font System) based on plain initialized. @char`<ver.>
@LATEX: OFS (Olsak's Font System) based on NFSS initialized. @char`<ver.>
\endtt
Main features of the OFS:

\begitems
* The user interface is the same for plain  and for \LaTeX.
* You can use the "\fontusage" command which displays a short description
  of OFS macros in terminal and in the log file.
* You can use the real font names from font catalog.
* You can use the font divided into two \TeX{} metrics (basic and
  extended tfm) and they seem from the user's point of view to be only one
  font.  This is useful for fonts with more than 256 characters if you
  don't want to use Omega.
* You can choose the \TeX{} internal encoding of fonts for your
  language in the beginning of your document. This feature is commonly used
  for Czech and Slovak languages: there are \TeX{} fonts which encode
  the alphabets of these languages by Cork (T1 encoding) or by
  ISO-8859-2 (IL2 encoding). T1 font encoding is common in LaTeX
  world, whereas IL2 encoding is widely used in the Czech and Slovak
  \TeX{} community.
* The OFS defines the language of declaration files. These files
  define mapping of full names of fonts to \PLAIN: tfm
  names or \LATEX: NFSS short names of the font families.
* \PLAIN: You can use individual variants of fonts in an indepent manner
  in similar way as in NFSS (family, size, encoding, variant).
* \PLAIN: You can declare different fonts for different sizes
  in the declaration files (usable for Computer Modern family first
  of all).
* \PLAIN: The OFS includes support of the math fonts manipulation
  when the PostScript fonts and/or fonts at different sizes are
  used.
* Interactive macro "ofstex.tex" enables printing the paragraph samples
  in the chosen font families, printing the font table, font registers,
  samples of the mathematic and the lists of characters, including
  their \TeX\ sequences. All to do is to write "tex ofstest [allfonts]"
  or "csplain ofstest [allfonts]" on the command line and to follow
  the orders on the terminal.
\enditems

\sec The font families
%%%%%%%%%%%%%%%%%%%%%%

The most current font families have the four commonly used variants:
normal: ("\rm"), bold ("\bf"), italic ("\it")
and bold italic ("\bi")\inll[rm \back bf \back it \back bi].
These variants are called \uv{standard variants} in OFS.
After the font family is activated by "\setfonts" command (see
bellow), you can use the commands "\rm", "\bf", "\it" a "\bi"
as a variant switches for the current font family. The first three
commands are well known in plain and the fourth command switches into
BoldItalic variant and it is introduced in the OFS.

Additional \uv{nonstandard variants} can be declared in some font families
and vice versa some \uv{standard variants} can be missing in
other font families. Only the "\rm" variant has to be present in all
families.

The original names of fonts can be a little different from the names mentioned
above in some font families but the font switches names "\rm", "\bf",
"\it", "\bi" can be unchanged. For example the family Helvetica has
the variant \uv{Oblique}, but we are still using the "\it" switch
for this variant.

If you want to use the font switches from another font families
at the same time, these switches can be declared by the "\fontdef"
command (see bellow).

The font families are declared in the declaration files. These files
have the similar meaning in the plain\TeX{} as {\tt fd} files in NFSS. The
recommended suffix for the files are \PLAIN:~{\tt tex} (they map the
family names to \TeX{} metrics) or \LATEX: {\tt sty} (they map the
family names to the NFSS short family names). These files can include
more than one font family declaration. The names of these files are
chosen in order to you can recognize which families are declared
here. Examples:

\begtt
@PLAIN:         @LATEX:
sjannon.tex,  sjannon.sty  ... the big Jannon family by stormtype.com
a35.tex,      a35.sty      ... the basic 35 fonts by Adobe
\endtt

The user chooses from these files only such files needed by his/her
document and writes the names of these files into the header of
the document. For more simplicity, \uv{global} files
include the \PLAIN: "\input" or \LATEX: the "\RequirePackage"
command for the single declaration files. Examples:

\begtt
skatalog.tex, skatalog.sty ... all fonts by stormtype.com
allfonts.tex, allfonts.sty ... all fonts at your TeX installation
\endtt

\sec The user interface
%%%%%%%%%%%%%%%%%%%%%%%

\subsec The header
%%%%%%%%%%%%%%%%%%

For example, let us suppose that we will use fonts from Jannon and
DynaGrotesk family by Storm Type Foundry. Their
declaration files are included in the "storm" directory. You can get
the \TeX{} metrics plus the OFS macro package from
"www.cstug.cz/stormtype". You can buy the real fonts from Storm Type
Foundry. See "www.stormtype.com" for more details.
We need to use
\PLAIN: "sjannon.tex", "sdynamo.tex";
or \LATEX: "sjannon.sty", "sdynamo.sty".
declaration files. The first letter "s"
in their file name means that the fonts are made
by Storm Type Foundry. We can use the following line in the header of
our document:

\begtt
@PLAIN: \input ofs [sjannon, sdynamo] % space before "[" is necessary
@LATEX: \usepackage [sjannon,sdynamo]{ofs}
\endtt

Now we can use the families declared in "sjannon" and "sdynamo"
declaration files.
The "\showfonts"\inl[showfonts] command writes a list of available
families on terminal and to a log file. In the above case, the
"\showfonts" lists the following text:

\begtt
OFS (l.1): The list of known font families:
defaults:
   [CMRoman/]            \rm, \bf, \it, \bi, \sl
   [CMSans/]             \rm, \bf, \it,  -
   [CMTypewriter/]       \rm,  - , \it,  - , \sl
   [Times/]              \rm, \bf, \it, \bi
   [Helvetica/]          \rm, \bf, \it, \bi, \nrm, \nbf, \nit, \nbi
   [Courier/]            \rm, \bf, \it, \bi
sjannon.tex:
   [JannonAntikva/]      \rm, \bf, \it, \bi, \mr, \mi
   [JannonText/]         \rm, \bf, \it, \bi, \mr, \mi
   [JannonCaps/]         \rm, \bf, \it, \bi
sdynamo.tex:
   [DynaGroteskDXE/]     \rm, \bf, \it, \bi
   [DynaGroteskRXE/]     \rm, \bf, \it, \bi
   [DynaGroteskLXE/]     \rm, \bf, \it, \bi
   ... <next 15 families of DynaGrotesk> ...
\endtt

The first 6 families are declared in OFS internally (you need not write
any declaration file to use them). The next families are declared in
specified declaration files.

Beside each family are listed option switches. The first
four switches set \uv{standard variants}. If a standard
variant is not available then a dash sign is listed instead of
the switch. The fifth and any other switches correspond to the
\uv{nonstandard variants}, if these exists. For example,
the JannonAntikva and JannonText families have the extra variants
medium and medium-italic ("\mr" and "\mi" switches are used here).

If you use the "\fontusage"\inl[fontusage] command, then short
description of OFS use is printed to the terminal and into the log file.

We mentioned the header in the form:

\begtt
@PLAIN: \input ofs [<file>, <file>, ...]
@LATEX: \usepackage [<file>,<file>,...]{ofs}
\endtt
%
Instead of the header of this type you can also include definition
files directly. In such case, the "ofs.tex"
or "ofs.sty" do not have to be explicitly mentioned:

\begtt
@PLAIN: \input <file> \input <file> ...
@LATEX: \usepackage {<file>} \usepackage {<file>} ...
\endtt
%
Example:

\begtt
@PLAIN: \input sjannon \input sdynamo
@LATEX: \usepackage {sjannon} \usepackage {sdynamo}
\endtt
%
We don't recommend to mix the both variants of the header (in \LaTeX{}
specially).

\subsec The {\tt\back setfonts} command
%%%%%%%%%%%%%%%%%%%%%%%%%%%%%%%%%%%%%%%

Let us suppose\inl[setfonts]
that the "sjannon" and "sdynamo"
declaration files were given in the header for the next examples.
For example,
the command:

\begtt
\setfonts [JannonText/12pt]
\endtt
%
sets the switches "\rm", "\bf", "\it", "\bi", "\mr" and "\mi".
These switches set variants of the JannonText family to
12pt font size.

The current variant used before "\setfonts" command is saved, but the
family/size parameters of the current font are changed by this
command. For example, if the BoldItalic variant for the TimesRoman was a
current then the BoldItalic of JannonText family of size 12pt is
current font set by "\setfonts [JannonText/12pt]".  If the new family
has the current variant switch undeclared then the "\rm" variant is
used.

The "\setfonts" makes all changes locally, so that \TeX{} returns to
the previous font and font family after the group is ended.

The parameters of the "\setfonts[<FamName>/<size>]" command can be empty:
the "\setfonts[<FamName>/]" switches to the new font family but keeps
the current font size and the "\setfonts[/<size>]" changes the font
size for all corresponded variant switches but keeps the current
family unchanged. The command "\setfonts[/]" is syntactically correct
but without any effect.

The CMRoman/10pt is default <FamName>/<size> after OFS for plain\TeX{}
is initialized.

The parameter <FamName>, unless not empty has to be the same as the
name of a family listed by the "\showfonts". This parameter is case
sensitive. If the parameter does not match any font family from given
declaration files, the "\setfonts" command acts the same as the
"\showfonts" command: all available families are listed. Thus, you can
use the "\setfonts [?/]" with the same effect as the
"\showfonts".

\LATEX: The <FamName> parameter can be not only the \uv{long-named}
font family from "\showfonts" list, but the short name from NFSS can
be used too. For example, the "\setfonts[Times/]" and the
"\setfonts[ptm/]" has the same effect in \LaTeX.

\OBA:
The parameter <size> can be written in more ways:

\begitems
* <number>               \dots{} example: "12", "17.4",
* <number><unit>         \dots{} example: "12pt", "17.4pt",  "10dd",
* "at"<number><unit>     \dots{} example: "at12pt", "at17.4pt", "at10dd",
* "scaled"<integer>      \dots{} example: "scaled1200",  "scaled\magstep3",
* "mag"<decimal number>  \dots{} example: "mag1.2",  "mag.7",  "mag2.0".
\enditems

The first three possibilities have the same meaning. The keyword "at" is
optional and if you omit the keyword and the unit, the "at<number>pt"
is used. We can use the "true.." unit if we need not the relative unit
associated by current "\magnification" factor: for example:
"17truept". If the "at" keyword is present then the unit can't be
omitted---it means: "at12" is not correct, but
"at12pt" or simply "12" are correct values.

The "scaled" keyword has the same meaning as in \TeX{} primitive
"\font". For example, if the design size of the font is
"10pt" (this is a common value) then "scaled1200" is the same as
"at12pt".

The last possibility (keyword "mag") is a new feature in OFS.
The new font size is calculated from the current font size by
multiplying by <decimal number>.
For example, by the command "\setfonts[/12pt]" followed by "\setfonts[/mag2.]"
the current font size is changed to 24\,pt.
The decimal point is required in the <decimal number>.
Another example:

\begtt
\def\small{\setfonts[/mag.7]}
The text {\small is smaller \small and smaller \small and more smaller}
and the normal size is used here.
\endtt

Note: the <size> parameter in "\setfonts" command changes only the
font size but not the "\baselineskip". The user has to do the
"\baselineskip" setting by another way.
\LATEX: The "\setfonts" sets the "\baselineskip" to the current value
stored in NFSS. It means that if you want to change the
"\baselineskip" register for your goal, you have to do it
{\it after} "\setfonts" command in \LaTeX{}.

You can set only one variant of a font family (not the whole family) by
the "\setfonts" command. This is done if the ``"-<variant>"'' is appended
to the <FamName> parameter. In such case, the switches of the current
family are not changed and only the new font is set. The
<variant> means the name of variant switch here. Examples:

\begtt
\setfonts [JannonText-it/12] ..... sets the italics of the JannonText
                                   at 12 pt as the current font.
                                   The \rm, \bf, etc. are unchanged
\setfonts [JannonText-rm/] ....... sets the normal variant of the
                                   JannonText at the current size.
\setfonts [CMTypeWriter-sl/] ..... sets the cmsltt font as the current
                                   font at the current size.
\endtt

You can omit the family name even if the "-<variant>" is present.
The actual family is substituted in such case. Examples:

\begtt
\setfonts [JannonText/12]
\setfonts [-bf/17]     ... variant Bold of JannonText, size 17pt.
                           Selectors \rm, \bf, \it and \bi keeps its
                           original meaning: they switches between
                           variants of JannonText in 12pt size. For
                           example, the next command \setfonts[Times/]
                           sets the Times family in 12pt size.
BUT:
\setfonts [/17]\bf    ...  variant Bold of current family, size 17pt.
                           Selectors \rm, \bf, \it and \bi switches
                           in 17pt size now.
\endtt

\LATEX: The text above about the keeping of original
meaning of variant switches is not true in \LaTeX{} because it may
break the NFSS philosophy. Thus, the commands
"\setfonts [-bf/17]" and "\setfonts [/17]\bf" has the same meaning in
\LaTeX.


\subsec The commands {\tt\back fontdef} and {\tt\back addcmd}
%%%%%%%%%%%%%%%%%%%%%%%%%%%%%%%%%%%%%%%%%%%%%%%%%%%%%%%%%%%%%

The "\fontdef"\inl[fontdef] command declares a new font switch.

\begtt
\fontdef \<fontswitch> [<FamName>/<size>]
\endtt
%
This declaration is roughly similar to

\begtt
\gdef \<fontswitch> {\setfonts [<FamName>/<size>]}
\endtt

\PLAIN: If the ``"-<variant>"'' is appended in <FamName> and the
parameter <size> is not empty and it is not specified by "mag" keyword
then the new declared control sequence "\<fontswitch>" is not a macro
but it is implemented by "\global\font\<fontswitch>".
It is called \uv{fixed font} in an~OFS terminology.
The user can implement his/her native "\<fontswitch>" by "\fontdef"
without a knowledge about "tfm" filename.

\LATEX: The declared "\<fontswitch>" is always implemented as a macro
including the "\setfonts" command. The reason is that the user access
to native "\<fontswitch>" is not simply possible in NFSS.

\OBA:
You can type the exclamation mark \uv{!} instead of <FamName> and the
family current in the moment the "\fontdef" command was used is substituted.
On the other hand the empty <FamName> means that the family current in the moment
the "\<fontswitch>" command was used is substituted. You can use the
exclamation mark \uv{!} instead of <size> parameter with the same meaning.
Examples:

\begtt
\setfonts [JannonAntikva/]
\fontdef \small  [/7]             % \small = \setfonts [/7pt]
\fontdef \sffam  [DynaGroteskR/]  % \sffam = \setfonts [DynagroteskR/]
\fontdef \bigf  [Times/17]        % \bigf = \setfonts [Times/17pt]
\fontdef \ttfam  [Courier/]       % \ttfam = \setfonts [Courier/]
\fontdef \mylogo [Times-rm/mag.8] % \mylogo = \setfonts [Times-rm/mag.8]
                                  % the fontsize will be always
                                  % 0.8 times of the current size.
\fontdef \timbf  [Times-bf/12]    % \timbf = fixed-font, the same as:
                                  % \global\font\timbf=ptmb8z at12pt
\fontdef \jansmall [!/7]          % \jansmall=\setfonts[JannonAntikva/7]
\fontdef \janbi [!-bi/17]         % \janbi = fixed-font, the same as:
                                  % \global\font\janbi=sjnbi8z at17pt
\fontdef \tt [Courier-rm/!]       % \tt = fixed-font, the same as
                                  % \global\font\tt=pcrr8u at10pt
\endtt

The declaration of the "\<fontswitch>" by "\fontdef" command is global
but the "\<fontswitch>" itself has a local effect in the place where it is
used.

Since OFS version Oct.~2002, the "\addcmd"\inl[addcmd] command is
supported, which makes possible to concentrate whole font
management in one place. The format of "\addcmd" is:

\begtt
\addcmd \<fontswitch> {<commands>}
\endtt
%
The meaning is same as

\begtt
\def\<fontswitch> {<the original meaning of fontswitch><commands>}
\endtt
%
You can include new <commands> into the original content of the macro
"\<fontswitch>". The control sequence "\<fontswitch>" has to be
defined as a macro without parameters or as unexpandable control
sequence (by "\font", "\chardef" etc.) before "\addcmd" is used.
The "\addcmd" redefines "\<fontswitch>" as a macro without parameters in
all cases. You can apply "\addcmd" on the same "\<fontswitch>"
more than once.

Examples:

\begtt
\setfonts [JannonText/]
\fontdef \footnotefont  [!/7]
\addcmd  \footnotefont  {\rm \baselineskip=9pt \relax}
\fontdef \sectionfont  [!/12]
\addcmd  \sectionfont  {\bf \let\it=\bi}
\endtt

\subsec Test of a family name existence
%%%%%%%%%%%%%%%%%%%%%%%%%%%%%%%%%%%%%%%

\PLAIN: 
  You are able to test the font family declaration in your own marcos,
  it means whether the font is loaded from the file of declarations.
  The sequence "\knownfam <FamilyName>? \iftrue"\inl[knownfam] 
  is used for such test.
  This sequence expands to "\iftrue", if the font family is declared
  otherwise expands to "\iffalse". The parameter <FamilyName> has to be
  brought in without the variant specification.

\OBA: By reason of the backward compatibility with the older version of OFS
  the "\ifknownfam [<FamilyName>]"\inl[ifknownfam] does the same as 
  "\knownchar" macro.
  Since the version Feb.~2004 of the OFS for plain it is recommended
  to use "\knownfam", because of correct alignment of the primitives
  "\if*", "\else", "\fi". La\TeX\ user can define "\knownfam" quite easily.

\subsec \LATEX: OFS and NFSS
%%%%%%%%%%%%%%%%%%%%%%%%%%%%

This section is intended only for \LaTeX{} users.
The command\inl[OFSfamily]

\begtt
\OFSfamily [<FamName>]
\endtt
%
converts the long family name to internal short NFSS family name.
For example

\begtt
\OFSfamily [Times]
\endtt
%
expands to "ptm". The macro works only on expand processor level thus
we don't get the error message if the <FamName> is not known. The
"\OFSfamily" expands to the text ``"unknown"'' in such case.
If you are using the "\OFSfamily" in your macro files and the
NFSS try to substitute the "unknown" family then you can be sure that
some misspelling occurs in <FamName> parameter or the required family
is not known.

The example:

\begtt
\usepackage [sjannon, sdynamo] {ofs}
\edef\rmdefault {\OFSfamily [JannonAntikva]}
\edef\sfdefault {\OFSfamily [DynaGroteskR]}
\edef\ttdefault {\OFSfamily [Courirer]}
\endtt

The meaning of the macros "\rmdefault", "\sfdefault", "\ttdefault" is
described in the NFSS documentation.

OFS defines the command\inl[OFSfamilydefault]

\begtt
\OFSfamilydefault [<FamName>]
\endtt
%
which sets the basic family of the whole document. This family is used
in the plain text and in the chapter headers etc. too (if the used class
file is made by common \LaTeX{} conventions). The command
"\OFSfamilydefault" internally does:

\begtt
\edef\familydefault {\OFSfamily [<FamName>]}
\endtt
%
and moreover it also cares for the case of the unknown <FamName>.
If the <FamName> is not known then the list of the supported families is
printed.

\subsec The font encoding
%%%%%%%%%%%%%%%%%%%%%%%%%

\LATEX: The font encoding switching is the subject to the NFSS and OFS
defines nothing more (i.e., packages "fontenc" and "inputenc"
work as expected).

\PLAIN (to the end of this section):
OFS sets the font encoding into CSfonts encoding by default.
If you need to use fonts encoded in another encoding (T1 by Cork,
for example) then you write "\def\fotenc{8t}"\inl[fotenc]
before the OFS is included and OFS will operate the fonts with this encoding.
The name ``"8t"'' is an example of T1 encoding.

You can even switch the encoding inside the document:

\begtt
\def\fotenc{8z} \setfonts[/] ... fonts in CSfont encoding
\def\fotenc{8t} \setfonts[/] ... fonts in encoding by Cork
\endtt

Moreover there are tools in OFS for correct macros expansion,
that are dependent upon the font encoding (for example "\v", "\'", "\ae").

By default, OFS set "\loadingenc=0"\inl[loadingenc],
which means, that font encoding change nor the command "\setfonts" does not
change of the macros of the type "\v", "\ae". These macros hold their
original \hbox{plain} meaning. Such a feature will welcome plain users,
who dislikes too inteligent packages.

But, if the document starts with "\loadingenc=1", for example:

\begtt
\input ofs [a35,sjannon]  \loadingenc=1
\endtt
%
then \TeX{} checks during every run of the "\setfont" command,
whether all necessary files named "ofs-<encoding>.tex" are loaded.
These files contain macro definitions dependent on the preset font encoding.
If these files are not loaded, \TeX{} loads them during the run of "\setfonts".
More informations on this topic can be found in the sections~\beglink{3.3}3.3\endlink{}
to~\beglink{3.5}3.5\endlink.

The OFS package contains three basic files with macros definitions dependent
on the encoding: "ofs-8z.tex", "ofs-8t.tex" and "ofs-8c.tex". If you use another
font encoding, you can create a new similar file. Commands "\accentdef"
and "\characterdef" used in these files are described in details in the
\beglink{3.4}section~3.4\endlink.

\subsec The fonts in mathematics
%%%%%%%%%%%%%%%%%%%%%%%%%%%%%%%%

\LATEX: OFS for \LaTeX{} does nothing with the issue of math fonts.
You need to use some \LaTeX{} package or build on the capabilities of NFSS.

\PLAIN (to the end of this section):
The command "\setfonts" and the control sequences declared by "\fontdef"
switch fonts in text mode only. Until you use the
"\setmath" command, all text between dollars is in Computer Modern in
10\,pt/7\,pt/5\,pt sizes (text/index/index of index).

The "\setmath"\inl[setmath] command has the following syntax:

\begtt
\setmath [<text size>/<index size>/<indexindex size>]
\endtt
%
The parameters set sizes of mathematical fonts and they
have same syntax as in the "\setfonts" command.
The keyword "mag" means the magnification to the size of
the current textual font. The empty parameter means the following
substitution:

\begitems
* text size: "mag1.0"
* index size: "mag0.7"
* indexindex size: "mag0.5"
\enditems

It means that "\setmath[//]" has the same effect as
"\setmath[mag1./mag.7/mag.5]". The
"\setsimplemath"\inl[setsimplemath]
command is defined in OFS as the equivalent of the
"\setmath[//]" command.

The "\setmath" sets the mathematical fonts depending on the values of the
macros "\fomenc" and "\mathversion". The "\fomenc"\inl[fomenc] means the
mathematical encoding and the "\mathversion" means
the version of the math-families set.

Mathematical encoding is set by the value of the macro
"\fomenc".\inl[fomenc].
There exist following possibilities:

\begitems
* If the default value "\def\fomenc{PS}" (PostScript fonts) is used 
  then "\setmath" sets the
  italic from the current font family as a mathematical italic ("\fam1").
  It uses "\rm" from the current font family for
  digits and a some other symbols. The variant selectors
  "\rm", "\it", "\bf" and "\bi" work in math mode too.
%
  Moreover, when "\def\fomenc{PS}" is used then the math symbols from "\fam2" and Greek
  symbols (originally located in "\fam1") are used from PostScript
  Symbol font. This font is much better visual compatible with most
  Postscript fonts than the Computer Modern symbols. Other symbols (e.\,g.,~big operators
  and big braces) stay in Computer Modern font because unfortunately
  there is no common alternative for these glyphs.

* If "\def\fomenc{CM}" is used then "\setmath" command keeps the Computer Modern
  fonts in math formulae. It sets only the desired sizes given in the
  parameters.

* It is possible to use "\def\fomenc{AMS}" after loading the file "amsfn.tex".
  Mathematic symbols are the same as while using "CM", but in addition
  you can use all mathematics symbols from~AMS\TeX{}.

* After loading the file "txfn.tex", you can use two new encodings:
  "\def\fomenc{TX}" or "\def\fomenc{PX}".
  In both cases the free TXfonts are used for mathematics, they are compatible 
  with the font families Times and Helvetica. If you choose the value "TX" then 
  pure TXfonts are used for all mathematic symbols, whereas 
  the value "PX" means that TXfonts
  are going to be combined with italics and with the basic fontface of the actual
  font family (similar to "PS" encoding). OFS supports all control sequences
  of the mathematic symbols, as is mentioned in the TXfonts reference. There are
  all symbols from the AMS\TeX{} and many more there. There are
  hunderts sybols from TXfonts.

* After loading the file "mtfn.tex", it is possible to use "\def\fomenc{MT}".
  Mathematics will contain italics and basic fontface of the actual font family
  as well. And it will be combined with the characters of the commercial version
  of the mathematics fonts MathTimes.
\enditems

OFS supports two versions of math formulae as default: normal and
bold version. The user can define another versions---see
\beglink{3.6}section 3.6\endlink.
The current version is set by the contents
of the "\mathversion"\inl[mathversion]
macro. You can say "\def\mathversion{normal}" or "\def\mathversion{bold}"
before "\setmath" command. The "normal" version is used as
default. Examples:

\begtt
\setmath [//] $formula$      % formula in "normal" version
$\def\mathversion{bold}\setmath[//] formula$ % formula in "bold" version
\endtt

\sec The inside of OFS for insiders
%%%%%%%%%%%%%%%%%%%%%%%%%%%%%%%%%%%

\LATEX: All auxiliary macros of OFS are defined in "ofs.sty" file with
the name "\ofs@macroname" in order to avoid the confusion with other
style files. The macros used in declaration files have the name
"\OFSmacroname". Moreover OFS defines user-level macros "\fontdef", "\showfonts",
"\fontusage", "\rm", "\bf", "\it" and "\bi".

\PLAIN: All auxiliary macros of OFS are defined in "ofs.tex" file and
they are listed in index at the end of this document. The convention
with the~"@" character is not used because I personally hate this character in
macro names.

\subsec Debugging
%%%%%%%%%%%%%%%%%

\LATEX: Use the standard NFSS package "tracefnt" for tracing purposes.

\medskip
\PLAIN: There are four commands for tracing OFS:
\begitems
* "\nofontmessages"\inl[nofontmessages] sets no tracing info.
* "\logfontmessages"\inl[logfontmessages] sets the tracing info to log file only.
* "\displayfontmessages"\inl[displayfontmessages] sets the tracing
  info to log file and to terminal.
* "\detailfontmessages"\inl[detailfontmessages] sets the detailed
  tracing info to log and to terminal (all "\font" primitives are traced).
\enditems
The "\logfontmessages" is initialized by default.

The warnings about unaccessible characters or encodings are always displayed
on the terminal and they are written into the log file. If you want log output only,
you can write "\let\displaymessage=\wlog"\inl[displaymessage],
because OFS uses this sequence for displaying messages on the terminal.

\subsec The robust and fragile commands
%%%%%%%%%%%%%%%%%%%%%%%%%%%%%%%%%%%%%%%%

\LATEX: \LaTeX{}2e has its conventions to define robust commands.
The command "\setfonts" and the "\<fontswitches>" declared by
"\fontdef" are robust, of course. They can be used in texts for table
of contents, indexes etc.

\PLAIN (to the end of this section):
Plain\TeX{} does not solve the problem of fragile commands and its
users have their own solutions without any standardization. One
solution is used in OFS.

What is a fragile command? We sometimes need to send some part of
text to auxiliary file (for table of contents, index, etc.). We
are doing it by "\write" primitive and in second run of \TeX{} this
file is "\input"-ted. The problem is that the "\write" primitive prints
the text to the file after all macro expansions and it may cause
problems. For example, if the font switch is implemented as a
complicated macro and it is used in "\write" parameter then the macro
is stored in the file after its expansion.
The error can occurs in most cases during the "\input" of such file.
We say that the \uv{fragile}
command was used in "\write" parameter and that this command
\uv{was got spilt} in auxiliary file.

If the (potentially) fragile command defined in OFS is being
sent to the file then the
following message is printed on the terminal and to the log file during the
"\input" of the file:

\begtt
ERROR !! The fragile command in the toc/ind/aux or similar file.
You can solve this problem by the following steps:
1. Remove the auxiliary file with this command.
2. Include the following macro code into your document header:
     \let\orishipout=\shipout
     \def\shipout#1#2{\setbox0=#1{#2}\bgroup
          \let\expandaction=\noexpand \orishipout\box0 \egroup}
3. Run TeX on your document again and again...
See the OFS documentation for more info.
! The fragile command in auxiliary file
\endtt

You can follow this hint to remove the problem.

The detail explanation of this behavior follows. The "\setfonts" and
other (potentially) fragile macros are implemented in OFS by the
following way:

\begtt
\def\macro {%
   \ifx\expandaction\noexpand
      \noexpand\macro
   \else
      \csname fragilecommand!\endcsname
      <the macro code>
   \fi
}
\endtt

If "\expandaction"\inl[expandaction] does not have the meaning
"\noexpand" then the "\else" part of the macro
code is performed. The
"\csname fragilecommand!\endcsname"\inl[fragilecommand!]
expands to "\fragilecomand!" and this text occurs in auxiliary file.
If this file is included in next \TeX{} run then the
"\fragilecomand"\inl[fragilecommand]
runs and this command prints the
message with the hint mentioned above.

If the user follows the hint then the "\expandaction"
has the meaning of "\noexpand" when the "\shipout" is active
(it means during the no-immediate "\write" parameter expansion).
The "\macro" expands to "\macro" and this text is stored in
auxiliary file.

The "\shipout" is not re-defined in OFS by default---only a suggestion
is printed if fragile command in the "\write" parameter occurs.  The
reason is that a plain\TeX{}
user may have his/her own redefinitions of "\output" routine or
"\shipout" primitive thus OFS for plain does not do any redefinitions at
this level itself. Unlike \LaTeX{} users the plain user needs exactly
to know how his macros work.

The code above illustrate the definition of an abstract macro
"\macro". The similar code is used for the following actual macros in
OFS: "\setfonts" (\beglink{2.2}section~2.2\endlink),
"\setmath" (\beglink{2.7}section~2.7\endlink),
"\<fontswitches>" declared by "\fontdef"
macro (\beglink{2.3}section~2.3\endlink),
"\setextrafont", "\printcharacter", "\printaccent"
(\beglink{3.4}section~3.4\endlink),
"\accentabove" and "\accentbelow" (\beglink{3.6}section~3.6\endlink).
If you use the hint above then these macros become \uv{robust}
in \LaTeX{} sense of this word.

\subsec Declaration files
%%%%%%%%%%%%%%%%%%%%%%%%%

\LATEX: The declaration files for OFS are common \LaTeX{} style files
which includes mapping from long family names to NFSS family names
of one or more font families. These NFSS families are declared in
common~"fd" files. Use NFSS documentation to create "fd" files.
You can use the following commands in the OFS declaration files:

\begitems
* "\OFSprocessoptions"\inl[OFSprocessoptions]:
  This macro is undefined by default but it has the meaning "\relax"
  during "ofs.sty" file is scanned and its options are included.
  You can test by "\ifx" the value of this control sequence in order
  to skip the "\RequirePackage{ofs}".
* "\OFSextraencoding {<extra encoding>}"\inl[OFSextraencoding]:
  The macro stores the <extra encoding> into memory and does
  "\input {<extra encoding>ini.def}".
  We assume that the corresponding definitions for <extra encoding>
  are included in this file. See "se1ini.def". This file includes the
  declarations for extra encoding SE1 for fonts by Storm Type Foundry.
  If the "<extra encoding>ini.def" file was included already it is not
  included again. Attention: use uppercase letters for <extra encoding>
  parameter but use lowercase letters in filename.
* "\OFSputfamlist {<text>}"\inl[OFSputfamlist]:
  The macro adds the <text> into the list of family names.
  This list is printed by "\showfonts" command or if unknown family is
  used in "\setfonts".
* "\OFSdeclarefamily [<FamName>] {<NFSS-name>}"\inl[OFSdeclarefamily]
  This macro does an actual mapping from <FamName> (long family name)
  to the <NFSS-name> (short NFSS family name). Moreover, it stores the
  line about this family name to the list of family names which is
  printed by "\showfonts" command.
* "\OFSnormalvariants"\inl[OFSnormalvariants]:
  This macro stores the list of standard switches
  "\rm,"~"\bf," "\it," "\bi" into the list of family names which is
  printed by "\showfonts" command.
\enditems

\PLAIN (to the end of this section):
The declaration files have the extension "tex" and we assume that there is
a locking code in them so that the file will not be read twice.
If the "ofs.tex" is not included already then it have to be included at the
begin of declaration file.

You have to define the mapping from long family names to "tfm" names in
the declaration files. You can use the following commands:

\begitems
* "\protectreading <filename><space>"\inl[protectreading] --- 
  the flag about the <file> reading is saved in the memory.
  If the command is run with the same parameter once more, 
  the "\endinput" is executed. It means
  that next declarations are protected against the multiple reading.
* "\ofsputfamlist {<text>}"\inl[ofsputfamlist]:
  The macro adds the <text> into the list of family names.
  This list is printed by "\showfonts" command or if unknown family is
  used in "\setfonts".
* "\ofsdeclarefamily [<FamName>] {<commands>}"\inl[ofsdeclarefamily]:
  This macro declares the new font family with the name <FamName>.
  The <FamName> is stored into the list of family names which is
  printed by "\showfonts" command. If this family is used by
  "\setfonts" then the <commands> are performed. We assume that the
  <commands> include "\loadtextfam" command and (zero or more)
  "\newvariant" commands.
* "\loadtextfam"\inl[loadtextfam]:
  This macro loads four fonts with given metrics. See below for the syntax and
  the detail explanation of this command.
* "\newvariant<digit> \<switch> (<Variant>) <space> <metric>;<extra-enc>;"\inl[newvariant]:
  \null\space This macro sets the \uv{nonstandard} variant for given font family.
  See below for the detail explanation of this command.
* "\modifyenc <encoding>:<identifier>;"\inl[modifyenc] --- the exceptions
  are added with respect to the basic encoding, see~\beglink{3.5}section~3.5\endlink.
* "\fosize"\inl[fosize]:
  The information about the actual font size of the last selected font
  family is stored in this macro. The value of this macro can be in
  one of the two forms: "at<dimen>" or "scaled<number>". This depends
  on the form of <size> parameter given in "\setfonts" command.
* "\fotenc"\inl[fotenc]:
  The actual encoding is stored in this macro. The common values are:
  "8z" for encoding by CS-fonts (by ISO-8859-2) or "8t" for encoding
  by Cork. The part of "tfm" name (where encoding is specified)
  is recommended for values of "\fotenc" macro.
  If "\fotenc" is undefined at the time of OFS is initializing, the
  OFS makes "\def\fotenc{8z}" else the "\fotenc" is unchanged.
* "\extranec"\inl[extraenc] --- macro, that stores the information
  about the extra encoding. This information is copied from the parameter
  <extra-enc>, that is located in the "\laodatextfam" command.
* "\defaultextraenc"\inl[defaultextraenc] --- if you redefine this macro, the
  extra encoding of the basic families and the families from~"a35.tex" can be
  changed. The default value of this macro is "8c".
* "\setfontshook"\inl[setfontshook]:
  This macro is called from "\setfonts" macro before <commands> from
  "\ofsdeclarefamily" are performed.
* "\registertfm <symbolic name> <from>-<to> <real metric>"\inl[registertfm]:
  You can declare different "tfm" names for different font sizes.
  See \beglink{3.7}section~3.7\endlink{} for more details.
* "\registerenc <FamilyName>: <encoding> <space>"\inl[registerenc]
  --- enables the limitation of usage the font families only for
  certain encodings. See~\beglink{3.9}section~3.9\endlink.
\enditems

The "\loadtextfam" command used in declaration files
has the following syntax:

\begtt
\loadtextfam (<Variant-rm>) <space> <metric-rm>;%
             (<Variant-bf>) <space> <metric-bf>;%
             (<Variant-it>) <space> <metric-it>;%
             (<Variant-bi>) <space> <metric-bi>;<extra-enc>;%
\endtt
%
The percent characters at the ends of lines here mean that no spaces are
allowed after semicolons.
You can save the long name of the used variant by "(<Variant-..>)" parameter.
This name us used only when the OFS is traced by "\logfontmessages" or
others commands. The parameters ``"(<Variant-..>) <space>"'' are optional.
If this parameter is omitted then default value is stored:
"rm: ()", "bf: (Bold)", "it: (Italic)", "bi: (BoldItalic)".

The parameters <metric-..> are the names of the "tfm" files for the
appropriate variants. The "\loadtextfam"\inl[loadtextfam]
does roughly the following work:

\begtt
\font\tenrm=<metric-rm> \fosize
\font\tenbf=<metric-bf> \fosize
\font\tenit=<metric-it> \fosize
\font\tenbi=<metric-bi> \fosize
\endtt
%
We assume that all <metric-..> parameters are written with the
"\fotenc" macro in order to make the switching to others
encodings possible.

The <extra-enc> parameter is the name of the extra encoding. If this
parameter is non-empty then the "\loadtextfam" command redefines
temporally the "\fotenc" macro: "\def\fotenc{<extra-enc>}" and it
expands all parameters <metric-rm>, <metric-bf>, <metric-it> and
<metric-bi> again. The results of these expansions are stored into
memory. They are the \uv{extra metrics} connected to the
\uv{basic metrics}.
If the <extra-enc> parameter is empty then there are no
\uv{extra metrics} connected to \uv{basic metrics}.
One can use a macro which can need the access
to the extra metric concerned to the basic metric of the current font.
The macro for "\euro" character is the good example of these needs.

Some <metric-..> (except of <metric-rm>) can be omitted. When the
<metric-XX> is empty then the "\loadtextfam" command
does roughly the following work:

\begtt
\def\tenXX{\message{WARNING: the needed font variant is missings}}
\endtt
%
It means that if the user needs the omitted variant then the message
is printed to the log file and to the terminal and no font change is done.

The "\setfonts"\inl[setfonts] command does not change the meaning of
the macros "\rm", "\bf", "\it" and "\bi". It only changes the font
switches "\tenrm", "\tenbf", "\tenit",\inll[tenrm \back tenbf \back tenit]
and "\tenbi" respectively\inl[tenbi]. The first three font switches are known in
plain and the last one is introduced in OFS.
The macros "\rm", "\bf", "\it" and "\bi" store the information about
last selected variant into control sequence "\currentvariant"\inl[currentvariant].
This information has the form of the letter "M" (for "\rm"), "F" (for "\bf"),
"T" (for "\it") and "I" (for "\bi"). It is stored by
"\let\currentvariant=<letter>"
because this code is not expanded. Thus we need not to implement a
special \uv{robust} code to the macros for variant switches.

Note that the "\loadtextfam" command sets the font switches
"\tenrm", "\tenbf", "\tenit" and "\tenbi" to the fonts of arbitrary
size given by the contents of the "\fosize" macro. The word ``"ten"''
in names of font switches is used only for the historical reasons and it does
not mean that the font is loaded at 10\,pt size.

You can object that the repetitive calls of "\setfonts" runs the
font loading on the four fonts in given font family again and again.
This can be time consuming operation. But you are not right. \TeX{}
stores the font information from font loading in its internal memory
and if the "\font" primitive is applied again to the same font then
\TeX{} uses the information stored before and it needs not to load the
font again.

If "\loadingenc>0"\inl[loadingenc] the command "\loadtextfam" reads
the file "ofs-\fotenc.tex" before fonts loading.
If the parameter <extra-enc> is non-empty, it loads moreover the file
"ofs-<extra-enc>.tex". These files are read only once. Empty lines 
and ends of lines are ignored during reading. Reading is performed inside the
group. The character categories are locally set in accordance to
plain\TeX{} (and~"\catcode`@=11"). The "\globaldefs=1" is set. It means,
that all macros and values from the file "ofs-<encoding>.tex" are
defined globally.
That does not matter, because newly loaded encoding files do not conflict
with the previous ones (see the commands "\characterdef" and "\accentdef"
in~\beglink{3.4}section~3.4\endlink). It does not matter at all, if there
are loaded more files, than is necessary at a given instant. It is not wise
to read the encoding file repeatedly, when the command "{...\setfonts...}"
is executed. This is the reason of the global definition.
If the user dislikes the global predefined macros (he/she wants to add the
article into the proceedings, where such a predefined macros can collide with
other articles), he lets the "\loadingenc=0". In this case, the
declaration files have to be loaded manually by "\input" command at
the beginning of the article.

You can declare a nonstandard variant in <commands> of the
"\ofsdeclarefamily" by the "\newvariant"\inl[newvariant] command.
The "\newvariant" command does roughly the following:

\begtt
\font\ten<switch>=<metric> \fosize
\def \<switch> {\let\currentvariant=<digit> \ten<switch>}
\endtt
%
Moreover the "\newvariant" stores the \uv{extended metric}
connected to the \uv{basic metric} if the <extra-enc> is not empty.

If the OFS needs to return to the last \uv{nonstandard variant} then
it does it by the value of the "\currentvariant". If the new family
has the \uv{nonstandard variant} with the same <digit> as a previous
family then this variant is used and OFS does not switch to the "\rm"
variant. You can declare the variants of various families but the
similar \uv{type} with the same <digit>.
There are only ten digits thus we can distinguish only ten different
\uv{types} of \uv{nonstandard variants}.

The macro "\setfonts" can change the meaning of the macros
"\loadtextfam" and "\newvariant" if the
"-<variant>" is specified in <FamName> parameter of "\setfonts".
It is sufficient to load only one font in such case but not the whole
family. If the ``"-<variant>"'' is \uv{standard} then "\newvariant" is
redefined so that it do nothing and "\loadtextfam" is redefined in
order to load only one font of the variant specified.
If the "-<variant>" is is \uv{nonstandard} then "\loadtextfam" do
nothing and "\newvariant" loads the font only if it loads the variant
specified.

We describe the operations of "\setfonts [<FamName>/<size>]"\inl[setfonts]
command here in detail. This command calculates and defines the
"\fosize" macro by the <size> parameter. If the <FamName> is not empty
and the "-<variant>" is not given then "\setfonts" performs
"\def\currentfamily{<FamName>}"\inl[currentfamily]. On the other hand,
if the <FamName> is empty then the "\currentfamily" is used for
restoring the family name. If the "-<variant>" is given then
"\setfonts" redefines the "\loadtextfam" and "\newvariant" macros at
the temporary time. This behavior is described in previous paragraph.
Then the "\setfonts" runs "\setfontshook" and <commands>
specified as a parameter of "\ofsdeclarefamily" of the
appropriate <FamName>. It also runs macro "\runmodifylist"\inl[runmodifylist],
that at certain circumstances sets the exceptions from the chosen encodings
(see~\beglink{3.5}section~3.5\endlink). Finally the "\setfonts" runs
"\ignorespaces" at the end of its run in order to ignoring the possibly
forgotten space after ``"]"''.

\subsec The font encoding and the character declaration
%%%%%%%%%%%%%%%%%%%%%%%%%%%%%%%%%%%%%%%%%%%%%%%%%%%%%%%

\PLAIN: You can use the macro "\setextrafont"\inl[setextrafont] to
switch to the extra metric of the current font. If the extra metric
connected to the metric of current font is stored in \TeX{} memory (by
"\loadtexfam" or "\newvariant" command) then "\setextrafont" do roughly
the following work:

\begtt
\font\extrafont=<extra metric connected to the current metric> \extrafont@inl[extrafont]
\endtt

\LATEX: You can use the macro "\setextrafont" to switch to the extra
encoding declared by "\OFSextraencoding" command. If this encoding is
declared then "\setextrafont" do roughly
the following work:

\begtt
\fontencoding{<extra encoding>}\selectfont
\endtt

\OBA:
If you need to print the character from extra metric/encoding from
slot of <number> position then you can use the macro
"\extchar <number>"\inl[extchar].

\PLAIN (to the end of this section):
You can use the commands "\characterdef" and "\accentdef" to declare
the macros which depend on font encoding. See the "ofs-8t.tex" and
"ofs-8z.tex" for a good illustration.

The "\characterdef"\inl[characterdef] has the following syntax:

\begtt
\characterdef \<sequence>  <encoding> <space> <number>
% example:
\characterdef \promile  8z  141
% or
\characterdef \<sequence>  <encoding> <space> {<commands>}
% example:
\characterdef \promile  8t  {\%\char24 }
\characterdef \promile  *   {\vrule height1ex width1ex\relax}
                            % in another encodings
\endtt

If the current encoding is the same as <encoding> then "\<sequence>"
will expand to the token of category~12 with the code <number>
or it expands to the <commands>.
All work is done at expand processor level when "\<sequence>" is used.
You can declare the same "\<sequence>"
for more encodings, see the "\promile" declarations in previous examples:

\begtt
\def\fotenc{8z} \promile % expands to the token with the code 141
\def\fotenc{8t} \promile % expands to the commands \%\char24
\endtt

Moreover you can simply declare the access to the extra encoding:

\begtt
\characterdef \euro  8z  134
\characterdef \euro  6s  37

\def\fotenc{8z} \euro % expands to the token of the code 134
\def\fotenc{8t} \euro % expands to: {\setextrafont <token with 37 code>}
\endtt

The second example is working only if the extra metric connected to
the current metric exists (see "\loadtextfam" and "\newvariant" commands)
and the extra metric has the "6s" encoding. If this is not valid then
the "\euro" prints the warning about inaccessibility of the "\euro"
character to the terminal and to the log file.

Now, we explain the behavior of the "\characterdef"ed macros
in more details. The "\characterdef" command defines
the "\<sequence>" as "\printcharacter{<sequence>}",\inl[printcharacter]
it means that "\promile" expands to "\printcharacter{promile}" and "\euro" to
"\printcharacter{euro}" in our examples.
If you use the "\characterdef" twice to the same "\<sequence>" then it
does not matter because the definition is still the same.
Moreover, "\characterdef" defines the special macro
"\<sequence>:-<encoding>" in order to this macro expands to the
token of given <number> code or to the given <commands>.
The more work is done by the "\printcharacter" macro.
This macro checks if the "\<sequence>:-\fotenc" is defined.
If it is true then "\printcharacter" expands to the contents of this
macro. Else "\printcharacter" checks if the extra metric is connected
to the current font. If it is true then "\printcharacter" checks if the
"\<sequence>:-<extra-enc>" is defined where <extra-enc> is the
encoding of the extra metric. If it is true then
"\printcharacter" expands to
{\hbadness=1110\par}

\begtt
{\setextrafont \<sequence>:-<extra-enc>}
\endtt
%
If all attempts fail then the "\printcharacter" try to print the default character
independent on encoding. It means, the "\printcharacter" checks if the
"\<sequence>:-*" is defined and if true, it expands to this macro.

If this is false then the "\printcharacterwarn{<sekvence>}"
is run.\inl[printcharacterwarn]
The implicit value of this macro prints out a warning, that the
character <sequence> is not available. It is printed on the
terminal and into the log file. 
No character is printed to "dvi" output.

If we want to omit the warning printing, we can redefine the
"\printcharacterwarn" for example by following way:

\begtt
\def\printcharacterwarn #1{?(#1)?}
\endtt

The "\characterdef" does not redefine the defined control
sequences since the version OFS Mar. 2004. It defines only sequences,
that have the meaning "\undefined" or "\relax". Otherwise
(and also if the sequence is not defined by previous "\characterdef")
it prints the warning, that the definition is ignored.
The reason is that the encoding files declares by "\characterdef" command
enormous amount of new control sequences. But the programmer
have not to know all of them. It is possible, that he/she uses the same name
for his/her own macro. In this case, the "\characterdef" keeps the macro
defined by the programmer and lets the appropriate character unaccessible.
You have to write "\let\<sequence>=\relax" befor of the "\characterdef" command
if it is really necessary to redefine some control sequences
(this procedure is needed for macros dependent on the encoding and
defined in plain\TeX).

The macro "\safelet"\inl[safelet] has been added into OFS for the same reasons.
It acts similar to "\let", but resists to redefine the predefined control
sequences. The warning is printed by "\safeletwarn"\inl[safeletwarn]
macro instead of redefinition.

You can use "\accentdef"\inl[accentdef] command to declaration of the
accent macros "\'", "\v", etc. depend on encoding.
This command has the following syntax:

\begtt
\accentdef \<sequence> <char> <optional space> <encoding> <space> <number>
% example:
\accentdef \v E  8z  204           % Ecaron
\accentdef \v e  8z  233           % ecaron
% or
\accentdef \<sequence> <char> <optional space> <encoding> <space> {<commands>}
% example:
\accentdef \v *  8z  {\accent20 }  % default caron in 8z
\accentdef \v *  *   {\blackbox }  % default caron
\endtt

If the current encoding is the same as <encoding> then "\<sequence>"
followed by <char> expands to the token of category~12 with the code
<number> or to the <commands>. This work is done at expand processor
level. If the declared <char> is "*" then the "\<sequence>"
expands to the token of given <number> code or to the
given <commands> in the case of the
actual <char> does not match with all declared <chars>.

The possibility of the use of the extra metric is the same in
"\accentdef"-ed macros as in the "\characterdef"-ed macros.

Now, we explain the functionality of the "\accentdef"-ed macros
in more details. The "\accentdef" command defines the "\<sequence>"
as a macro with one non separated parameter~"#1" which expands to the
"\printaccent{<sequence>}{#1}"\inl[printaccent]. For example, "\v E"
expands to "\printaccent{v}{E}". Moreover, "\accentdef" defines the
macro "\<sequence>:<char>:-<encoding>" in order to this macro expands to the
token of given <number> code or to the given <commands>.
The more work is done by the "\printaccent" macro.
This command checks if the "\<sequence>:<char>:-\fotenc" is
defined. If it is true then "\printaccent" expands to the contents of
this macro. Else the "\printaccent" checks if the extra metric is connected
to the current font. If it is true and if this extra metric has
<extra-enc> encoding then "\printaccent" checks if the
"\<sequence>:<char>:-<extra-enc>" is defined. If it is true then
"\printaccent" expands to "{\setextrafont \<sequence>:<char>:-<extra-enc>}".
Else "\printaccent" checks if the macros "\<sequence>:*:-\fotenc"
and "\<sequence>:*:-<extra-enc>" are defined in this order.
If the first one is defined then "\printaccent" expands to this macro
and appends the <char>. If only the second one is defined then
"\printaccent" expands to:

\begtt
{\setextrafont \<sequence>:*:-<extra-enc> <normalfont> <char>}
\endtt
%
where <normalfont> is the font switch to the current font at the start
of "\printaccent" macro. If all attempts fail so far then
"\printaccent" try to use the macros "\<sequence>:<char>:-*"
or "\<sequence>:*:-* <char>" in this order. If all the previous commands
fail, the "\printaccentwarn{<sequence>}{<character>}"\inl[printaccentwarn] is run.
The default value of this macro prints out on the terminal and into the log
file the warning about the unaccessibility of the accented character
and no character is printed on "dvi" output.

Note that the character from extra metric inside the word breaks the
kerning around this character and breaks the possibility of
hyphenation of this word. It is extremely recommended that a basic
metric encodes all alphabet used in current language in order to
minimize switching to extra metric. For example, the "8t" and
"8z" encodings are good choice as basic metric for Czech and Slovak
languages.

If we want to take out predeclared character\inl[characterdel]
(see so called <exceptions> in the next section), we can use the
commands"\characterdel" and "\accentdel".\inl[accentdel]
These commands have to have the same parameters like "\characterdef" and
"\accentdef" respectively and they take out the command definition
"\<sequence>:-<encoding>" and "\<sequence>:<char>:-<encoding>"
respectively.

%The last part of this section is applied to the example of the inteligent
%"\Uppercase" definition:
%
%\begtt
%\def\Uppercase #1{\bgroup
%   \let\expandaction=\noexpand % \euro stays unchanged
%
%   \u{g} unchanged as well
%   \def\ss{SS}\def\l{\L}\def\o{\O}\def\ae{\AE}\def\oe{\OE}%
%   \edef\tmp{#1}\expandafter
%   \egroup \expandafter\uppercase\expandafter{\tmp}%
%}                              % \uppercase changes \u{g} to \u{G}.
%\endtt

\subsec \PLAIN: Macro files dependent on the encoding and encoding exceptions
%%%%%%%%%%%%%%%%%%%%%%%%%%%%%%%%%%%%%%%%%%%%%%%%%%%%%%%%%%%%%%%%%%%%%%%%%%%%%

Commands "\characterdef" and "\accentdef" described in the previous
section redefines macros dependent on the encoding ("\v", "\ae", etc.).
In this section, we are going to describe the conception of placement
these macros. 

Macros declarations by means of "\characterdef" and "\accentdef"
have to be written into the files called "ofs-<encoding>.tex"
(so called~{\it encoding files\/}). The command "\loadtextfam" (called
from the "\setfonts" macro) reads these declarations from these files while
"\loadingenc=1"\inl[loadingenc] is set.

Every encoding contains its own basic set of characters and accented types.
This set is registered in the encoding file by the
"\characterdef" and "\accentdef" commands. Particular font families can contain
some additional characters or some characters can be missing in reference
to that basic set. These exceptions are declared by means of the command
"\modifydef"\inl[modifydef]:

\begtt
\modifydef <encoding>:<identifier>; {<exceptions>}
\endtt
%
The <exceptions> contain the commands "\characterdef", "\accentdef",
"\characterdel" and "\accentdel". The "\*del" commands have to contain
the the same value of the character in the argument as in basic encoding set.
If any character has to be redefined, the commands "\*del" and "\*def" corresponding
to this character must be written one after another.

The command "\modifyenc"\inl[modifyenc] used in the pearameter of the
"\ofsdeclarefamily" macro is a ``link to <exceptions>''. You can mark
by "\modifyenc" command that the family 
contains <exceptions> with respect to the basic encoding set.
The command has following notation:

\begtt
\modifyenc <encoding>:<identifier>;%
\endtt
%
You can list more of one command for every font family
(these commands can contain different <encoding> as well).
Nothing is done, if this command links to <exceptions>, that were not
yet declared.

An example of the exceptions declaration "8z:csfonts" can be
found in the file "ofs-8z.tex" and links to them are used in families "CM*"
in the file "ofsdef.tex".

Commands "\modifyenc" are run at each "\setfonts". As the matter of fact
these commands only stores their parameters into so called ``list of links''
(to the macro "\newmodifylist"\inl[newmodifylist]). At each start of the "\setfonts",
the new list of links is created. The "\modifyenc" command stores its
parameter into this list
only if <encoding> is equal to "\fotenc" or "\extraenc". The exception handling provides
the command "\runmodifylist" that is run on the end of the "\setfonts" command.
Its activity is an object of the next paragraphs.

The macro "\runmodifylist"\inl[runmodifylist] compares the ``list of links''
of the previous family ("\modifylist"\inl[modifylist]) with the ``list
of links'' of the newly
set family ("\newmodifylist"). The "\runmodifylist" finishes its activity,
if both lists are the same or "\modifylist" has the meaning "\relax".
Otherwise, the setting of exceptions is run: At first, meanings of
"\characterdef"$\leftrightarrow$"\characterdel" and
"\accentdef"$\leftrightarrow$"\accentdel" are exchanged and "\modifylist" is run.
In other words, the macros dependent on the encoding are returned
to the initial state (without exceptions). During this activity, 
the deleting of the character is ignored,
if the character was declared immediatelly before 
(see the rule about character redefinition above). 
Next, the command "\runmodifylist" returns
the "\characterdef" and "\accentdef" into the initial state and run "\newmodifylist".
All the redefinitions, that takes place during this activity, are just local.
The mechanism of two lists ensures, that for example:

\begtt
\setfonts [Family1/] ... \setfonts [Family2/] ...
\endtt
%
the exceptions of the actual encoding will be correctly set even
for the Family2, eventhough the Family1 has another set of exceptions
than Family2.

The control sequence "\modifylist" has the meaning of the empty
macro after the OFS startup. The macro programmer can set
"\let\modifylist=\relax" to override every set of exceptions.
Note, that declaration commands "\modifydef" store <exceptions> into the memory
and execute them in order to define all declared sequences corresponding to
"\printcharacter" and "\printaccent" respectively. It means, that every
control sequence from all exceptions is defined (the message {\tt undefined
control sequence} is not displayed). Moreover OFS has perfect view whether the
control sequence is available or not in the actual family. The macro programmer
can then redefine macros "\print*warn".

The command "\modifydef" slightly changes commands
"\accentdef", "\accentdel", etc. for a temporary time and then
it executes the
<exceptions>. The control sequence "\skipfirststep"\inl[skipfirststep] forbids the
execution of the macros in the <exceptions> during the activity of "\modifydef".
It acts just like "\relax", but during the execution of the <exceptions> by means
of "\modifydef" the whole part of <exceptions> behind this sequence is omited.

Identifiers "<encoding>:lccodes" and "<encoding>:ienc" are reserved for usage
in the macros out of OFS. OFS does not consider the setting of the characters
"\lccode", "\uccode". A macro package taking care of these values can properly
define commands "\lccodes"\inl[lccodes] and "\lccodesloop" and runs
"\csname <encoding>:lccodes\endcsname". These above mentioned commands are
not defined in OFS at all.\inl[lccodesloop]
The declarations "<encoding>:lccodes" are loceted in the files "ofs-8t.tex" and
"ofs-8z.tex", eventhough they are not used in OFS. It bears upon the text
fonts encoding. An example of "<encoding>:lccodes" usage is placed in the
macro called "lang.tex". Macro "inec.tex" uses "<encoding>:ienc". More
informations can be found in the appropriate documentation.
{\hbadness=1617 \par}

The declarations of the most commonly used exceptions are written directly
into the encoding files. The declarations of the less usual exceptions
(related only to some font families) can be written behind "\endinput"
of the declaration files. 
You can use a command "\modifyread"\inl[modifyread]
in <commands> of "\ofsdeclarefamily":

\begtt
\modifyread <filename>;%
\endtt
%
This command reads the file from the first apperance
of the sequence "\modifytext"\inl[modifytext], but only if
the "\loadingenc" is positive. It is suitable to place the "\modifytext"
sequence behind "\endinput", so the "\modifyread" command
reads only that part of the file, which has not been read before.
Assigning is global during the reading, the empty lines and
ends of lines are omited. The file is not loaded repeatedly.

This command gives you chance to concentrate the font families
declarations and encoding exceptions declarations into the single file.
\TeX\ reads the exceptions in the time it needs them, so the \TeX\ memory
is spared. An example is placed in the file "slido.tex".

OFS offers testing macro as well, whether the control sequence corresponds
to the character, that is available in the font or not:\inl[knownchar]

\begtt
\knownchar <character or accent+character>? \iftrue character is available
           \else   character is unavailable or undefined \fi
% example:
\def\tryeuro{\knownchar \euro? \iftrue \euro \else Euro\fi}
\endtt


\subsec The auxiliary macros for accents and characters
%%%%%%%%%%%%%%%%%%%%%%%%%%%%%%%%%%%%%%%%%%%%%%%%%%%%%%%

You can declare the default accents in OFS\inl[accentabove]
not only by the "\accent"
primitive but by the macros "\accentabove" and "\accentbelow" too.
The syntax follows:\inl[accentbelow]

\begtt
\accentabove {<accent char>}{<vertical skip>}{<base char>}
\accentbelow {<accent char>}{<vertical skip>}{<base char>}
\endtt

The "\accentabove" command put the <accent char> above the
<base char> with the <vertical skip> between them. The "\accentbelow"
command does the same work, but put the <accent char> below the
<base char>. In both cases characters are placed on the joint vertical axis
and if the font is slanted then this axis is slanted too. The width of
the resulting character is derived from the width of the <base char>
only. The macros are implemented by the "\vbox", "\vtop" and "\halign"
primitives with the calculation of the (possibly) slanted axis.

The accented characters for accent above are commonly designed in the
height 1\,ex for most fonts. It means, that placing such character by
"\accentabove" command needs the "-1ex" compensation:

\begtt
\accentabove {<accent char>}{-1ex}{<base char>}
\endtt

In such case, the "\accentabove" has the same behavior as the "\accent"
primitive. The difference is that you can compose more than one accent
by "\accentabove" and "\accentbelow" macros. You can try:

\begtt
\it \accentabove {.}{.1ex}{\accentabove {,}{.1ex}{\v A}}
\endtt

\PLAIN:
Unfortunately, the declaration macro "\accentdef" is not able to
declare a macros which construct more than two accents. Moreover, the
first accent has to be to join to the <base char> as one compact
character in the font.  If you needs more accents then you can use the
macros "\accentabove" and "\accentbelow" directly in the document.

\PLAIN:
Since the version OFS Feb. 2004, the macro "\ofshexbox"\inl[ofshexbox]
is available. It acts very similar to the plain one "\mathhexbox", 
furthermore it can set the font in accordance to actual version. 
You can declare the family of four metrics by means of the
command "\ofshexboxdef"\inl[ofshexboxdef]:

\begtt
\ofshexboxdef <family>{<metrics-rm>}{<metrics-bf>}{<metrics-it>}{<metrics-bi>}
% example:
\ofshexboxdef 2 {cmsy}{cmbsy10}{cmsy}{cmbsy10} % an examle
\endtt
%
The command "\ofshexbox <family><hexa-code>" prints out the requested
character. Its font is one of four declared ones and its size is defined by
the command "\fosize". The font choice depends on the actual version. If the
version is different than "\bf", "\it", "\bi", the <metrics-rm> is used.

OFS declares only <family> = 2 by default, because plain uses
only "\mathhexbox2..". The purpose of this macro was to define characters
"\S", "\dag", "\ddag", "\P" for CMfonts/CSfonts. The way of the definition
has to be independent on actual setting of mathematical fonts, but dependent
on actual size and version. See the file "ofs-8z.tex".

We can easily define the "\euro" symbol by means of "\ofshexbox" for
every font encoding, where it is unavailable:

\begtt
\ofshexboxdef {TS1}{tcrm1000}{tcbx1000}{tcti1000}{tcbi1000}
\characterdef \euro * {\ofshexbox{TS1}BF}
\endtt

\subsec \PLAIN: The fonts in mathematics (the second apperance)
%%%%%%%%%%%%%%%%%%%%%%%%%%%%%%%%%%%%%%%%%%%%%%%%%%%%%%%%%%%%%%%

The user interface to the math fonts was described in
\beglink{2.7}section 2.7\endlink. Now, it is the time to describe the principles of
math fonts in detail.

The "\setmath"\inl[setmath] command calculates the text, index and
indexindex sizes from its parameters.
The results are stored into
macros "\textfosize"\inl[textfosize],
"\scriptfosize" and "\scriptscriptfosize"\inl[scriptfosize].
The values are in the form
"at<dimen>" or "scaled<number>" depending on
the format of the parameters.\inl[scriptscriptfosize]
Then the "\setmath" runs the macro
"\mathfonts".\inl[mathfonts]
You can define the math fonts loading here but some
conventions are recommended, see below.
If the macro "\setmath" is run for the first time or the value "\fomenc"
has been changed, then the  "\setmath"
runs the macro "\mathchars".\inl[mathchars]
You can define the math encoding by the
"\matchcode", "\mathchardef" etc. primitives here but some conventions
are recommended, see below. The OFS serves the default values of the
"\mathfonts" and "\mathchars" macros, see below.

You can load a whole math family (text, index and indexindex size of
one font) in "\mathfonts" macro by the
"\loadmathfam"\inl[loadmathfam] command. This command has the
following syntax possibilities:

\begtt
%%                                  % font is declared by::
\loadmathfam <family>[/<metrics>]   % metrics
\loadmathfam <family>[-<version>/]  % actual family version
\loadmathfam <family>[<switch>/]    % textual font switch
\loadmathfam <family>[X<switch>/]   % extending switch metrics
\endtt
%
Examples:

\begtt
\loadmathfam 0[tenrm/]   % metrics is in accordance to the switch \tenrm
\loadmathfam 5[-bi/]     % actual text family metrics, version bi
\newmathfam \symbfam
\loadmathfam \symbfam [/psyr]     % psyr metrics
\newmathfam \extitfam
\loadmathfam \extitfam [Xtenit/]  % extending metrics for tenit switch
\endtt
%
This example shows, that the textual font with the "\tenrm" font
switch is used for math family~0. In the family~5, fonts are initiated
just like in the case of "\setfonts [-bi/]". A new family "\symbfam"
is declared as well. Fonts of the metric "psyr" are initiated into it.

There exist a slight difference between usage of "\loadmathfam 5[tenbi/]" and
"\loadmathfam 5[-bi/]" command. In the first case, OFS finds out a metrics
of the "\tenbi" switch and the same metrics is then used for
all font sizes. The only modification is done by the key word "at<dimen>".
The other case means, that different sizes can use different metrics, but only if
such a font ability is declared (see \beglink{3.8}section~3.8\endlink).

The new family was in the example declared by means of the
"\newmathfam"\inl[newmathfam]. It is an alternative commnd to~"\newfam".
The reason for such a solution is evident. The plain macro "\newfam" is
defined as "\outer" one. It means that it is not possible to use it inside
any definition. Moreover the macro "\newmathfam" is local, so it spares some
place for user families, than plain one "\newfam".
New mathematical families are in the basic mathematical encodings
defided exactly by "\chardef". The command
"\lastfam=<number>"\inl[lastfam] sets the maximal
used value. Such a structure guarantees, that user can use "\newmathfam" later on
and the new family numbers are alocated with numbers greater than "\lastfam".

Lets check out the the principle of the "\loadmathfam" macro activity. This macro
finds out a metrics, that corresponds to a given parameter. Next, the primitive "\font"
is executed three times:

\begtt
\font \<name>-Mt = <metric> \textfosize
\font \<name>-Ms = <metric> \scriptfosize
\font \<name>-Mss = <metric> \scriptscriptfosize
\textfont <math fam> = \<name>-Mt
\scriptfont <math fam> = \<name>-Ms
\scriptscriptfont <math fam> = \<name>-Mss
\endtt
%
Nevertheless <name> is the text of the parameter "\loadmathfam", that
declares metrics (switch, version or metrics).

The <name> is generated as name of the <fontswitch> or the name of the
<metric> if only the <metric> is given as parameter of the "\loadmathfam".

The fonts of math family 3 are loaded without of the size changes
of index and indexindex fonts in plain\TeX. If you need this
feature then you can use the prefix "\noindexsize"\inl[noindexsize]
before "\loadmathfam". The macro "\loadmathfam" loads all three fonts
at the same "\textfosize" size. Example:

\begtt
\noindexsize\loadmathfam 3[tenex/]% Standard extra symbols from CM
\endtt

OFS defines four different macros for math font loading. Look at
"ofsdef.tex" file for these definitions. Which of these four macros is
used depends on the contents of macros "\fomenc" and "\mathversion"
We assume two possibilities of "\fomenc": "CM" or "PS" and two
possibilities of "\mathversion": "normal" and "bold".
OFS defines two macros with mathcodes. Which macro is used depends on
the contents of the macro "\fomenc". The list of these macros follows:

\begitems
* "\loadCMnormalmath"\inl[loadCMnormalmath]
  --- loads CM fonts in \uv{normal} version.
* "\loadCMboldmath"\inl[loadCMboldmath]
  --- loads CM fonts in \uv{bold} version.
* "\loadPSnormalmath"\inl[loadPSnormalmath]
  --- loads PostScript fonts in \uv{normal} version.
* "\loadPSboldmath"\inl[loadPSboldmath]
  --- loads PostScript fonts in \uv{bold} version.
* "\setCMmathchars"\inl[setCMmathchars]
  --- keeps the mathcodes from plain.
* "\setPSmathchars"\inl[setPSmathchars]
  --- sets the mathcodes in order to some characters
  are used from Symbol font.
\enditems

OFS sets the following defaults in the "ofsdef.tex" file:

\begtt
\ifx \fomenc\undefined \def\fomenc{PS}\fi
\def\mathversion{normal}
\def\defaultmathfonts{\csname load\fomenc\mathversion math\endcsname}@inl[defaultmathfonts]
\def\defaultmathchars{\csname set\fomenc mathchars\endcsname}@inl[defaultmathchars]
\def\mathfonts{\defaultmathfonts}
\def\mathchars{\defaultmathchars}
\endtt

It is possible to add the another math families to the list of loaded
math families in "\loadmath" and "\mathchars" macros. You can do it, for example, by the
following code:

The example how to add the Euler fraktur from AMS follows:

\begtt
\input amsfn     % here are declared metrics eufm and eufb
\addcmd\mathfonts{\def\tmpa{bold}%
   \ifx\mathversion\tmpa \def\tmpa{b}\else\def\tmpa{b}\fi
   \newmathfam\frakfam \loadmathfam\frakfam [/euf\tmpa]}
\def\frak#1{{\fam\frakfam#1}}
\endtt
%
Another examples of mathematical families declaration are located
in files "amsfn.tex", "txfn.tex" and "mtfn.tex". These files define
the implicit groups of the mathematical fonts for "AMS", "TX", "PX", "MT"
encodings. Finally they also contain comments (including examples), how to
load additional font families by means of the command "\addcmd".

I would like to load all OFS font declarations in the~ini\TeX\ (see the Ok\TeX\ project),
but I want to spare the \TeX\ memory as much as possible too. So I suggest not to load
the large definitions containing declarations of mathematical fonts encodings by means
of "\mathchardef", etc. immediately, but during the first use in the document.
So the OFS version Apr. 2004 contains redefined macro "\setPSmathchars":

\begtt
\def\setPSmathchars{\mathencread ofs-ps;}
\endtt
%
The command "\mathencread <file>;"\inl[mathencread] loads the encoding
commands included in the file <file.tex>. The file "ofs-ps.tex" contains encoding
commands for "PS" encoding, "ofs-ams.tex" contains encoding
commands for "AMS" encoding, etc.

The files "ofs-ps.tex", "ofs-tx.tex" etc. contain encoding commands
``packed'' into groups and defined by the "\mathencdef"\inl[mathencdef]
command. By default, this command acts this way: ``define, run and forget''. Such a model
spares the memory, but a disadvantage of its procedure is, that during repeated
changes, encoding files are read over and over again. Mostly it does not matter at all,
because the mathematical encoding is the same for the whole document.
If this model is for somebody not suitable, the solution can be found in the macros
"\mathencread" and "\mathencdef". They can be redefined by the way, that
the files are read only once and during new execution of "\mathchars", 
encoding commands are read out of remembered macros.

The command "\mathencread <file>;" works in group. The catcodes
are set in accordance to plain\TeX\ ("\catcode`@=11" excluded) and the
"<file>.tex" is read. The "\globaldefs=1" are during the reading, the empty lines
and ends of lines are ignored. The commands "\mathencdef" run and forget defined
macros after group enclosure. It means, that these macros run the "\mathchardef"
commands in the sense of local settings.

Let us explain an ensurance of restoring the delault values during
switching between mathematical encodings. The command "\setmath"
runs the macro "\mathcharsback"\inl[mathcharsback] in a time instant
before "\mathchars". It serves the restoring of the mathematical
encoding to the default state in accordance to plain\TeX{}.
This macro is set to "\relax" by default, because the mathematical
encoding is set in accordance to plain\TeX{}.
If the command "\set<fomenc>mathchars" changes the values preset in
the plain\TeX{}, it should also define the "\mathcharsback" macro.
A procedure of setting the values into the initial
state have to be stored in it. 
A command "\mathencread ofs-cm;" can be used in the macro
"\mathcharsback", because the file "ofs-cm.tex" contains declarations
of the mathematical encoding in accordance to plain\TeX{}.

If we insist on declaring our own mathematical encodings (different
from preprepared "PS", "CM", "AMS" etc.), the next principle has to be
fullfiled: in all versions of mathematical fonts (normal/bold/etc.) should be
set the numbers of the mathematical families by the same way
and ended by the same "\lastfam". Such a principle is necessary, because
after the version switching, the macro "\mathchars" is not run again.
It is not recomended to change especially the family numbers, that are linked
with the macro "\set<encoding>mathchars".

Let us now descibe another macros, that helps us with the mathematical
encoding declaration.
Macro "\hex"\inl[hex] converts a number to a singledigit hexadeximal
number. The usage of such a macro can be found in the files "ofs-ps.tex",
"ofs-tx.tex", etc.

You can define the control sequences which are working in both: text
mode and math mode. You can use the
"\safemathchardef"\inl[safemathchardef] macro instead
"\mathchardef" primitive in "\mathchars" macro for this purpose:

\begtt
\safemathchardef \<sequence> <number>
\endtt

If the "\<sequence>" is not defined then "\safemathchardef" does the
same work as "\mathchardef" primitive. If the "\<sequence>" is defined
(we assume that this definition is used in text mode) then
"\safemathchardef" saves the meaning of "\<sequence>" in
"\T<sequence>", it performs "\mathchardef \M<sequence> = <number>"
and it defines "\<sequence>" by the following way:

\begtt
\def \<sequence> {\ifmmode \expandafter \M<sequence>
                 \else \expandafter \T<sequence> \fi}
\endtt
%
Now, the "\<sequence>" works in both: text and math mode.
If the "\safemathchardef" is applied on the same "\<sequence>"
repetitively, then the second and another use of "\safemathchardef"
does nothing.

We assume that all declarations of characters in text mode are
performed before the first use of "\setmath" and that the
"\safemathchardef" macros are used in "\mathchars" macro.

The "\safemathaccentdef"\inl[safemathaccentdef] performs similar as
"\safemathchardef". Only the "\mathaccentdef"\inl[mathaccentdef]
macro is used instead
"\mathchardef" primitive. This macro does roughly the following:

\begtt
\def \<sequence> {\mathaccent<number> }
\endtt

If you do not want to declare the whole new math family for only one
character or a few characters (the number of math families is
restricted to~16 for one formula in \TeX) then you can use the
"\pickmathfont"\inl[pickmathfont] macro. This macro has the following
syntax:

\begtt
\pickmathfont {<metric>}{<text>}
% example:
\mathbin {\pickmathfont {psybo}{\char"C4}}
\endtt
%
The "\pickmathfont" command uses the "\font" primitive with the given
<metric> and it prints the <text> in this font. The result is an atom
of type Ord in the math list. The appropriate size is set
(text/index/indexindex) because the "\mathchoice" primitive is used by
"\pickmathfont" macro and the "\font" is loaded three times for each size.

\subsec The different metrics for different sizes of font
%%%%%%%%%%%%%%%%%%%%%%%%%%%%%%%%%%%%%%%%%%%%%%%%%%%%%%%%%

\LATEX: This problem is solved in NFSS, see the syntax of the "fd"
files in NFSS documentation. OFS for \LaTeX{} does no more things.

\PLAIN (to the end of this section):
There exist special font families (Computer Modern, for instance)
where the different metrics are used for the different font sizes.
This feature is implemented in OFS too. The <metric> parameter of the
commands "\loadtextfam", "\loadmathfam" and "\pickmathfont" can be
only the <symbolic name> and not the name of "tfm" file exactly.
In such case, the <real metric> is calculated from the font size and
from the data stored by a set of "\registertfm"\inl[registertfm]
commands. The
"\registertfm" has a following syntax:

\begtt
\registertfm <symbolic name> <space> <from>-<to> <space> <real metric> <space>
\endtt

The <from> and <to> parameters have to include the unit and they
declares the interval of sizes with the following property: if the
desired size of font is in interval <from>-<to> then the <real metric>
is used instead the <symbolic name>. You need to use more
"\registertfm" commands to the same <symbolic name> in order to use different
<real metric> for different <from>-<to> intervals.
The <from>-<to> interval is closed interval (including the boundary
values) but the next "\registertfm" has a precedence over previous
one. Then you can construct the non closed intervals by the choose of
the right order of "\registertfm" commands. See the "ofsdef.tex" file
for an example.

If the both parameters <from> and <to> are empty then the
<real metric> is used when the "scaled" keyword is given in desired
font size or if desired font size does not lie in any declared
interval. You can use the "*" symbol in <to> parameter---it
means the infinity.

The command "\registertfm <name> <space> - <space> - <space>"
erases the previous registrations for a given <name>. Moreover, it marks
<name> as an unavailable font. OFS then acts, as if the corresponding
variant has not been declared at all. It gives us a possibility to mark
nonexisting variants of a declared family, but only for certain encodings
(see "cmssbxti" in the file "ofsdef.tex").

The parameters <symbolic name> and <real metric> are expanded during
the command "\registertfm" is working. Thus the "\fotenc"
macro should not be used in these parameters. But you can use an "\edef"
construction if necessary.

The macro "\registerECfont"\inl[registerECfont] is an abbreviation
of the repeated execution of the command "\registertfm" to all EC fonts
sizes between "0500" up to "3583".
The macro "\registerECTTfont"\inl[registerECTTfont] is an abbreviation
for sizes "0800" do "3583", that are used for the typewriter fonts.
Definitions and usage of these macros are located in the file "ofsdef.tex".
Macros can be used for another fonts as well. But they have to have similarly
scaled sizes like EC fonts (LH fonts with Russian alphabet or the fonts
derived from CM~super).

\subsec The limitations of usage the font family for certain encodings
%%%%%%%%%%%%%%%%%%%%%%%%%%%%%%%%%%%%%%%%%%%%%%%%%%%%%%%%%%%%%%%%%%%%%%%%

\LATEX: The usage of a family in a given encoding is dependent on the
existence of "fd" file. It is all controlled by NFSS.

\PLAIN (to the end of this section):
The OFS version Feb.~2004 introduces a command
"\registerenc"\inl[registerenc]. This command limits the usage of a declared font
family only for a certain encoding. If the user writes "\setfonts" and
requires the family in an unregistered encoding, the OFS prints out a
warning on the terminal and does not switch to the requested family.
It avoids an attempt to loading a nonexisting font metrics. The
command "\registerenc" has two parameters.

\begtt
\registerenc <FamilyName>: <encoding><space>
\registerenc Times: 8t   % example
\registerenc Times: 8z   % Times is registered for two encodings
\endtt
%
If the family is not registered for any encoding, OFS then suggests, that
it is available in every encoding (dingbats fonts for example).

The family usage is limited for a certain encoding by
"\registerenc" in declaration files. The user can add another
encoding by the command "\registerenc". The command
"\registerenc <FamilyName>: * " means, that the family is available
for arbitrary encodings.

The parameter <FamilyName> can remain empty. In such case, 
the "\registerenc" uses the family declared in the last 
command "\ofsdeclarefamily".

It is possible to find out, whether the family is registered
for actual value of the "\fotenc":\inl[registeredfam]

\vbox \bgroup
\begtt
\registeredfam <Family Name>? \iftrue
         Family has the \fotenc registered or any encoding is enabled
         or is not declared at all.
  \else  Family has registered the encoding,
           but \fotenc is not among them
  \fi
\endtt \egroup

If "\setfonts[<FamilyName>/]" is run for an undeclared family or for a family
with unavailable encoding then OFS prints out a warning on the terminal and
returns the "\setfontsOK"\inl[setfontsOK] with the value "\undefined".
The "\setfontsOK" has asigned the value "\relax", whenever the font is successfully
set.


\sec The license
%%%%%%%%%%%%%%%%

The OFS package may be used by everybody without any license fees.
Everybody can re-distribute this package, if no changes in files
"readme.ofs", "ofs.tex", "ofsdef.tex", "ofs.sty", "ofs-8z.tex",
"ofs-8t.tex", "a35.tex", "a35.sty", "ofsdoc.tex", "ofsdoc-e.tex",
"ofsmtdef.tex" are done and
all these files are included in the re-distribution. Only
the author has a right to change these files and to change
version of this package. If you need to change the content of any
file mentioned above, you have to rename it. This package is
distributed in the hope that it will be useful, but WITHOUT ANY
WARRANTY.

\bigskip
Prague 08/16/2001 \hfill  the author: Petr Ol\v s\'ak
\bigskip

\sec History
%%%%%%%%%%%%

\noindent
8/16/2001 --- the first version introduced 

\noindent
10/24/2002 --- some petty changes realized in documentation, 
considering the changes in the version OFS Oct.~2002.
The command "\addcmd" has been added. The optional
parameters "(<Variant>)" in the command "\loadtextfam"
were enabled.

\bigskip
\noindent
12/29/2002: the English documentation in the file "ofsdoc-e.tex" is
written. This is more or less the translation of the Czech original
documentation from the file "ofsdoc.tex". Sorry for my poor English.

I want to say a word of thanks to Mat\v ej Cepl (www.ceplovi.cz)
who has made a proofreading of this English version.

\bigskip
\noindent
2/10/2004 --- All modifications consider only OFS for plain:

\def\bod{\hangindent=1.2em \hangafter=1 \noindent + }

\bod Handling the macro declarations dependent on encoding has been
     improved as well as the exception declarations and registering of the
     encoding for a chosen family.
     See~\beglink{3.5}section~3.5\endlink{} and \beglink{3.9}section~3.9\endlink.

\bod Added the possibility to redefine the family (by the additional loading
     the declaring file, that modifies the properties of the default families).

\bod Declaration of CMRoman, CMSans and CMTypewriter for 8t encoding
     by means of EC fonts.

\bod Added the support for extended encoding 8c.

\bod Modified the declaration of the mathematical encoding called PS.
     "\int", "\sum" and "\prod" produces greater characters
     in the display mode.

\bod Defined a macro "\ofshexbox" and "\ofshexboxdef".

\bod A new version of the interactive macro "ofstest.tex".

\bod Introduced the directory "\examples/". There will be continuously added
     the exaples of OFS usage there. 

\noindent
3/12/2004 --- OFS for plain:

\bod Encoding files are read directly from the "\loadtextfam".
     That means the possibility to place the mapping metrics into the
     encoding files by means of the command "\registerenc"
     (I will use this ability for LANG).

\bod "\characterdef" respects predefined macro an does not redefines it.

\bod "\showfonts" reimplemented. It spares the memory and time, when
     operates with the large font lists. Warning: if you have used macros,
     that plays upon the internal macro "\listfamilies", this will not
     operate any more. Since this version is not the obsolete "ofscatal.tex"
     functional too. Instead of it is possible to use "ofstest.tex".

\bod "ofstest.tex" modified. It operates with the newly imlemented list of
     families "\ofslistfamilies".

\noindent
4/2/2004 --- "\plaincatcodes" added before reading the files
"ofs-<encoding>.tex".

\bod "\safelet" and "\protectreading" introduced.

\bod The space behind the <character> in the "\accentdef" is optional.

\bod Added the option "\loadmathfam <family>[-<version>/]". The
     declaration for mathematical encodings "CM", "PS", "AMS",
     "TX", "PX", "MT" has been extended and remade.

\bod English documentation upgraded. Thaks to Tom\'a\v s Kom\'arek.


\sec Index of macros defined in OFS
%%%%%%%%%%%%%%%%%%%%%%%%%%%%%%%%%%%

This index includes only the pointers to the pages where the macro is
introduced (not only mentioned). The macro name is written in the
margin on that place. The index is generated after first run of
\TeX{}.

The internal or auxiliary macros in "ofs.tex" are listed here too
but they are not mentioned in the text, thus they have no pointer to
the page in this index. Only a short comment is appended here. The
version of OFS ({\smallbf PLAIN} or {\smallbf LATEX}) is listed near
the each macro name in this index.

\bigskip
\immediate\closeout\indout
\input \jobname.ind

\begingroup
\let\tt=\smalltt \let\it=\smallit \let\rm=\smallrm \rm
\baselineskip=10.3pt

\def\inr #1 #2 #3;{\noindent{\tt
  \expandafter \ifx \csname ind:#2\endcsname \relax
     \back#1%
  \else
     \beglink{#2}\back#1\endlink
  \fi}\quad
  #3 \quad\pg[#2]\par}
\def\\#1 #2;{\inr #1 #1 #2;}
\def\pl{{\sevenrm (PLAIN)}}
\def\la{{\sevenrm (LATEX)}}
\def\oba{{\sevenrm (PLAIN+LATEX)}}

\\accentabove \pl\vadjust{\nobreak};
\par\nobreak
\\accentbelow \pl;
\\accentdef \pl;
\\accentdel \pl;
\\accentdefori \unskip, {\tt\back accentdelori} \pl~internal,
    stores the default macro purport;
\\accentnodef \pl~internal, {\tt\back def\back <macro>\#1%
   \char`\{\back printaccent\char`\{<macro>\char`\}\char`\{\#1\char`\}\char`\}};
\\addcmd \oba;
\inr bf rm \oba;
\inr bi rm \oba;
\\bifam \pl~internal, math family for BoldItalic;
\\calculatemetricfile \pl~internal, defines {\tt\back metricfile}
                       by fontsize;
\\catcodesloop \pl~internal, sets the character categ. according to a given num.;
\\characterdef \pl;
\\characterdel \pl;
\\characterdefori \unskip, {\tt\back characterdelori} \pl~internal,
    stores default macro purport;
\\characternodef \pl~internal, {\tt\back def\back <macro>%
   \char`\{\back printcharacter\char`\{<macro>\char`\}\char`\}};
\\currentfamily \pl;
\\currentfomenc \pl~internal, the name of last used math. encoding;
\\currentvariant \pl;
\\declaredfamily \pl~internal, contains the name of the last declared family;
\\defaultextraenc \pl;
\\defaultmathchars \pl;
\\defaultmathfonts \pl;
\\defpttotmpa \pl~internal, does {\tt\back def\back tmpa\char`\{pt\char`\}},
    if the unit is omited;
\\detailfontmessages \pl;
\\displayfontmessages \pl;
\\displaymessage \pl~internal;
\\docharacterdef \pl~internal, for use of {\tt\back characterdef} macro;
\\doextchar \pl~internal, for use of {\tt\back extchar} macro;
\\dosafemathdef \pl~internal, for use of {\tt\back safemath*def} macros;
\\donumbercharacterdef \pl~internal, for use of {\tt\back characterdef} macro;
\\endOFSmacro \pl~internal, the reading of {\tt[<file>, ...]}
    after {\tt\back input ofs};
\\expandaction \pl;
\\extchar \oba;
\\extraenc \pl;
\\extrafont \pl;
\\fontdef \oba;
\\fontloadmessage \pl~internal, the tracing of {\tt\back font} primitives;
\\fontmessage \pl~internal, three values: nothing,
                     {\tt\back wlog} or {\tt\back displaymessage};
\\fontprefix \pl~internal, two values: nothing or {\tt\back global};
\\fontusage \oba;
\\fosize \pl;
\\fomenc \pl;
\\fotenc \pl;
\\fragilecommand \pl;
\\fragilecommand! \pl;
\\hex \pl;
\\ifknownfam \oba;
\\isunitpresent \pl~internal, solves the case of empty unit;
\inr it rm \oba;
\\knownchar pl;
\\knownfam pl;
\\lastfam pl;
\\lccodes \unskip, {\tt\back lccodesloop} \pl;
\\loadCMboldmath \pl;
\\loadCMnormalmath \pl;
\\loadingenc \pl;
\\loadmathfam \pl;
\\loadPSboldmath \pl;
\\loadPSnormalmath \pl;
\\loadtextfam \pl;
\\logfontmessages \pl;
\\mathaccentdef \pl;
\\mathchars \pl;
\\mathcharsback \pl;
\\mathencdef \pl;
\\mathencread \pl;
\\mathfonts \pl;
\\mathversion \pl;
\\metricfile \pl~internal, the metric name if {\tt\back font}
primitive is used;
\\metrictmpa \pl~internal, expands to the metric of {\tt\back tmpa} font;
\\modifydef \pl;
\\modifyenc \pl;
\\modifylist \pl;
\\modifyread \pl;
\\newfamily \pl~auxiliary, the given family name;
\\newmathfam \pl;
\\newmodifylist \pl;
\\newvariant \pl;
\\nofontmessages \pl;
\\noindexsize \pl;
\\noPT \pl~internal, removes {\tt pt} from~{\tt\back the<dimen>} [TBN, pg.~80];
\\ofsaddenctolist \pl~internal, adds parameter into the list {\tt\back newmodifylist};
\\OFSdeclarefamily \la;
\\ofsdeclarefamily \pl;
\\OFSextraencoding \la;
\\OFSfamily \la;
\\OFSfamilydefault \la;
\\ofshexbox \pl;
\\ofshexboxdef \pl;
\\ofsinput \pl~internal, reads file with {\tt\back globaldefs=1}, ignores endlinechars;
\\ofslistfamilies \pl~internal, family list for {\tt\back showfonts};
\\ofslistfamily \pl~internal, envokes the family in {\tt\back listfamilies};
\\ofslistvariants \pl~internal, text for listing of variants to the log;
\\ofslisttext \pl~intern�, envokes the text in {\tt\back listfamilies};
\\ofsloadfont \pl~internal, loads one font;
\\ofsloadfontori \pl~internal, loads one font;
\\ofsmeaning \pl~internal, removes the word  {\tt letter}/{\tt character}
  from {\tt \back meaning};
\\ofsmessageheader \pl~internal, header of the messages;
\\OFSnormalvariants \la;
\\OFSprocessoptions \la;
\\OFSputfamlist \la;
\\ofsputfamlist \pl;
\\ofsremovefromlist \pl~internal, erases the family from the list;
\\OFSversion \oba~internal, date and version of OFS;
\\orifosize \pl~auxiliary, saves {\tt\back fosize} value;
\\origTeX \pl~internal, original definition of \TeX{} logo;
\\oriloadfam \pl~auxiliary, saves {\tt loadtextfam} value;
\\pickmathfont \pl;
\\plaincatcodes \pl~internal, sets the chars categories according to plain\TeX{};
\\printaccent \pl;
\\printaccentwarn \pl;
\\printcharacter \pl;
\\printcharacterwarn \pl;
\\processOFSoption \pl~internal, for use of {\back endOFSmacro};
\\protectreading \pl;
\\readfamvariant \pl~internal, checks if the variant is given;
\\readfirsttoken \pl~internal, returns the first token of text separed
by {\tt:\back end};
\\readfosize \pl~internal, inserts the {\tt\back fosize} value to
                   {\tt\back dimen0};
\\readmag \pl~internal, calculates {\tt\back fosize} if
    {\tt mag}<decimal number> is given;
\\readOFSoptions \pl~internal, for use of {\tt\back endOFSmacro};
\\readothertokens \pl~internal, returns the second and other tokens to
{\tt:\back end};
\\readsixdigits \pl~internal, for rounding algorithm;
\\registeredfam \pl;
\\registerECfont \pl;
\\registerECTTfont \pl;
\\registerenc \pl;
\\registertfm \pl;
\\restorefontid \pl~internal, restores font name, see {\tt\back savefontid};
\\rm \oba;
\\runmodifylist \pl;
\\safelet \pl;
\\safeletwarn \pl;
\\safemathaccentdef \pl;
\\safemathchardef \pl;
\\savefontid \pl~internal, saves font name for Overfull messages;
\\savetokenname \pl~internal, similar to {\tt\back string} without backslash;
\\scriptfosize \pl;
\\scriptscriptfosize \pl;
\\separeofsvariant \pl~internal, separes the "(<Variant>)" in {\tt\back loadtextfam};
\\setextrafont \oba;
\\setCMmathchars \pl;
\\setfonts \oba;
\\setfontshook \pl;
\\setfontsOK \pl;
\\setfontfamily \pl~internal, {\tt\back setfonts},
         if variant is not given;
\\setfosize \pl~internal, calculate the value of {\tt\back fosize};
\\setmath \pl;
\\setPSmathchars \pl;
\\setsimplemath \pl;
\\setsinglefont \pl~internal, {\tt\back setfonts},
         if the variant is given;
\\setsinglefontname \pl~internal, removes the possible "at<dimen>" from <metric>;
\\sgfamily \pl~internal, the information about the family name;
\\sgvariant \pl~internal, the information about the variant;
\\showfonts \oba;
\\singlefont \pl~internal;
\\singlefontname \pl~internal, removes the {\tt at<dimen>} from metric name;
\\skipfirststep \pl;
\\slantcorrection \pl~internal, for use of
     {\tt\back accentabove} and {\tt\back accentbelow};
\\storeofsvariant \pl~internal, separes optional parameter of {\tt\back loadtextfam};
\\switchdeftodel \pl~internal, switches {\tt\back accent/characterdef}
    with~{\tt\back *del};
\\tenbi \pl;
\inr tenbf tenrm \pl;
\inr tenit tenrm \pl;
\\tenrm \pl;
\\testOFSoptions \pl~internal, for use of {\tt\back endOFSmacro};
\\testtfmsize \pl~internal, for use of {\tt\back registertfm} macro;
\\testtfmsizeat \pl~internal, the auxiliary value of {\tt\back testtfmsize};
\\testtfmsizescaled \pl~internal, the auxiliary value of {\tt\back testtfmsize};
\\textfosize \pl;
\\tryloadenc \pl~internal, loads encoding files;
\\tmpa \oba~temporary;
\\tmpb \oba~temporary;
\\tmpc \pl~temporary;
\\warnmissingfont \pl~internal, message about missing variant;
\\warnM \unskip, {\tt\back warnF}, {\tt\back warnT}, {\tt\back warnI},
\pl~internal, message about missing variant;
\\warnunregistered \pl~internal, print out, disabled encoding of a chosen family;

\endgroup

\sec References
%%%%%%%%%%%%%%%

\noindent \hangindent=2\parindent
[TBN] Petr Ol\v s\'ak {\it \TeX{}book naruby}, Konvoj
1.~ed. 1997 (ISBN 80-7302-007-6),
2.~ed. 2001 (ISBN 80-85615-64-9), Brno, 466 pages.
Czech language.
PDF version of this book is free available on
{\tt http://math.feld.cvut.cz/olsak/tbn.html}.


\end










