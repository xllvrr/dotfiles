\documentclass[a4paper,10pt]{article}

%% Au moins 1 argument obligatoire pour babel [french, english, ...]  
%% L'encodage par defaut pour inputenc est utf8, mais vous pouvez utiliser l'option [latin1]
%% vous pouvez ensuite utiliser \rsEncodage{} pour vérifier l'encodage latin1ou utf8
%% Exemples d'options (nb: si vous changez les options, compilez 2 fois de suite):
%\usepackage[latin1, french]{facture-belge-simple-sans-tva} % latin1, français
%\usepackage[french, english]{facture-belge-simple-sans-tva} % utf8, français, anglais
\usepackage[french]{facture-belge-simple-sans-tva} % utf8, français

\begin{document}


%%%%%%%%%%%%%%%%%%%%%%%%%%%%%%%%%%%/o\____
% **** %%%%%%%%%%%%%%%%%%%%%%%%%%%%%%%%%%%\__
%  **** %%% Note de frais, 1ère partie %%%%%%|==>
% **** %%%%%%%%%%%%%%%%%%%%%%%%%%%%%%%%%%|°°/
%%%%%%%%%%%%%%%%%%%%%%%%%%%%%%%%%%%%%%/
%%%%____________%%%%___________%%%%___\__________

%% pour vérifier l'encodage UTF-8 ou latin1
%\rsEncodage{}

%%%%%%%%%%%%%%%%%%%%%
%% Unité monétaire %%
%%%%%%%%%%%%%%%%%%%%%

%% permet de choisir l'unité monétaire
%% \euro (defaut), \pounds (livre anglaise) \$ (dollar américain) \textyen (yen japonais)
%% Decomenter ci-dessous pour voir les unités
% \${} \pounds{} \euro{} \textyen{}
\rsChoisirUniteMonetaire{\euro}
%% affiche l'unité monétaire choisie
%\rsuniteMonetaire{}

%%%%%%%%%%%%%%%%%%%%%%%%%%%%%%%%%%
%% Aération de la note de frais %%
%%%%%%%%%%%%%%%%%%%%%%%%%%%%%%%%%%

%% permet d'aérer la page verticalement; le paramètre est une mesure LaTeX
%% TeX comprend six unités de mesure :
%% + pt point = 0,35 mm
%% + mm millimètre
%% + ex correspond à la hauteur d'un x dans la fonte courante
%% + em correspond à la largeur d'un m dans la fonte courante
%% + cm centimètre
%% + in pouce = 2,54 cm
%\rsAerationVerticale{0.3cm}
%\rsAerationVerticale{1.5cm}
\rsAerationVerticale{1cm}


%%%%%%%%%%%%%%%%%%%%%%%%%%%%%%%%%/o\____
% **** %%%%%%%%%%%%%%%%%%%%%%%%%%%%%%%%%\__
%  **** %%% Note de frais, 2e partie %%%%%%|==>
% **** %%%%%%%%%%%%%%%%%%%%%%%%%%%%%%%%|°°/
%%%%%%%%%%%%%%%%%%%%%%%%%%%%%%%%%%%%/
%%____________%%%%___________%%%%___\__________

%%%%%%%%%%%%%%%%%%%%%%%%%%8%%%%%%%%%%%%%%%%%%%%%%%%%%%%%%%%%%%%%%%%%%%%%%%%%%%%%%%%%%%
%% On récolte des informations nécessaires sur le créancier (qui entre la note)     %%
%% et le client (qui la paie). Nom, adresse, mois + année note                      %%
%%%%%%%%%%%%%%%%%%%%%%%%%%%%%%%%%%%%%%%%%%%%%%%%%%%%%%%%%%%%%%%%%%%%%%%%%%%%%%%%%%%%%%
%% Le total en chiffre sera calculé automatiquement, pas celui en lettres, bien sur %%
%%%%%%%%%%%%%%%%%%%%%%%%%%%%%%%%%%%%%%%%%%%%%%%%%%%%%%%%%%%%%%%%%%%%%%%%%%%%%%%%%%%%%%

%%%%%%%%%%%%%%%%%%%%%%%%%%%%%%%%%%%%%%%%%%%%%%%%%%%%%%%%%%%%%%%%%%%%%%%
%% créancier (qui entre la note de frais)
%% prenom nom societe 
\rsIdentificationCreancier{Jules \textsc{Creancier}}{asbl \textsc{La Créance}}

%% adresse créancier {rue no}{codpost ville}{pays}{email}{téléphone}
%% format email: \href{mailto:user@domain.tld}{user@domain.tld}
\rsAdresseCreancier{52, rue Delarue}{4321 \textsc{Brux-aile}}{\textsc{Belgique}}{\href{mailto:user@domain.tld}{user@domain.tld}}{+32 2 123 45 67}
%%%%%%%%%%%%%%%%%%%%%%%%%%%%%%%%%%%%%%%%%%%%%%%%%%%%%%%%%%%%%%%%%%%%%%%

%%%%%%%%%%%%%%%%%%%%%%%%%%%%%%%%%%%%%%%%%%%%%%%%%%%%%%%%%%%%%%%%
%%  client (qui paye la note de frais)
%% Prenom Nom, societe, civilite
%% typographie française des civilités, voir la documentation 
%% facture-belge-simple-sans-tva-doc.pdf 
\rsIdentificationClient{Léon \textsc{Client}}{\textsc{SA La Cliance}}{M. }

% adresse client {rue no}{codpost ville}{pays}
\rsAdresseClient{DelarueStraat 25}{4321 \textsc{Brux-aile}}{\textsc{Belgique}}
%%%%%%%%%%%%%%%%%%%%%%%%%%%%%%%%%%%%%%%%%%%%%%%%%%%%%%%%%%%%%%%%

%%%%%%%%%%%%%%%%%%%%%%%%%%%%%
%% mois annee note de frais
\rsMoisAnneeNote{octobre 1952}
%%%%%%%%%%%%%%%%%%%%%%%%%%%%%

%%%%%%%%%%%%%%%%%%%%%%%%%%%%%%%%%%%%%%
%% numéro de compte créancier (IBAN)
\rsCompteEnBanqueCreancier{BE01 1234 5678 9012}
%%%%%%%%%%%%%%%%%%%%%%%%%%%%%


%%%%%%%%%%%%%%%%%%%%%%%%%%%%%%%%%/o\____
% **** %%%%%%%%%%%%%%%%%%%%%%%%%%%%%%%%%\__
%  **** %%% Note de frais, 3e partie %%%%%%|==>
% **** %%%%%%%%%%%%%%%%%%%%%%%%%%%%%%%%|°°/
%%%%%%%%%%%%%%%%%%%%%%%%%%%%%%%%%%%%/
%%____________%%%%___________%%%%___\__________

%%%%%%%%%%%%%%%%%%%%%%%%%%%%%%%%%%%%%%%%%%%%%%%%%%%%%%%%%%%%%%%%%%%%%%%%
%% Et maintenant, on commnence à construire le document note de frais %%
%%%%%%%%%%%%%%%%%%%%%%%%%%%%%%%%%%%%%%%%%%%%%%%%%%%%%%%%%%%%%%%%%%%%%%%%

%%%%%%%%%%%%%%%%%%%%%%%%%
%% Aération du texte   %%
\vspace{\rsespaceVertical}
%%%%%%%%%%%%%%%%%%%%%%%%%

%%%%%%%%%%%%%%%%%%%%%%%%%%%%%%%%%%%%%%%%%%%%%%%%%%%%%%%%%%%%%%%%%%%%%%%%%%%%%%%%%%%%%%%
%% prenom nom, societe, adresse, pays, email, téléphone créancier (qui entre la note) 
%% Seuls les champs non vides sont pris en considération.
\rsConstruitAdresseCreancier{}

%%%%%%%%%%%%%%%%%%%%%%%%%
%% Aération du texte   %%
\vspace{\rsespaceVertical}
%%%%%%%%%%%%%%%%%%%%%%%%%

%%%%%%%%%%%%%%%%%%%%%%%%%%%%%%%%%%%%%%%%%%%%%%%%%%%%%%%%%%%
%% Titre de la note, affiche: Note de frais de mois année
\rsConstruitTitreEtDateNote{}

%%%%%%%%%%%%%%%%%%%%%%%%%
%% Aération du texte   %%
\vspace{\rsespaceVertical}
%%%%%%%%%%%%%%%%%%%%%%%%%


%%%%%%%%%%%%%%%%%%%%%%%%%%%%%%%%%%%%%%%%%%%%%%%%%%%%%%%%%%%%%%%%%
%% prenom nom, societe, adresse, pays client (qui paie la note)
%% Seuls les champs non vides sont pris en considération.
\rsConstruitAdresseClient{}

%%%%%%%%%%%%%%%%%%%%%%%%%
%% Aération du texte   %%
\vspace{\rsespaceVertical}
%%%%%%%%%%%%%%%%%%%%%%%%%

%%%%%%%%%%%%%%%%%%%%%%%%%%%%%%%%%%%%%
%% Tableau des items à rembourser  %%
%%%%%%%%%%%%%%%%%%%%%%%%%%%%%%%%%%%%%

%%%%%%%%%%%%%%%%%%%%%%%%%%%%%%%%%%%%%%%%%%%%%%%%%%%%%%%%%%%%%%%%%%%%%%%%%%%%%%%%%%%%%%%%%%
%%  On ouvre l'entête du tableau des items à rembourser; à n'utiliser qu'une seule fois.
\rsEnteteTableauItemsARembourser{}% Ce commentaire bloque un blanc indésirable.
%% ATTENTION: retirer le commentaire ci-dessus déstabilisera l'affichage !!!
%%%%%%%%%%%%%%%%%%%%%%%%%%%%%%%%%%%%%%%%%%%%%%%%%%%%%%%%%%%%%%%%%%%%%%%%%%%%%%%%%%%%%%%%%%

%%%%%%%%%%%%%%%%%%%%%%%%%%%%%%%%%%%%%%%%%%%%%%%%%%%%%%%%%%%%%%%%%%%%%%%%%%%%%%%%%%%%
%% Entrée de lignes d'items à rembourser, dans l'ordre 
%% no piece, date, nature, montant ttc, moyen de paiement.
%% le total en chiffre est calculé automatiquement
%% les montants DOIVENT être en cents: 10€25 = 1025
%% \rsLigneTableauItemsARembourser{numero}{JJ/MM/AAAA}{Nature}{Montant TTC en cents}{Moyen}

\rsLigneTableauItemsARembourser{1}{04/10/1952}{Gateau d'anniversaire}{2520}{Carte bancaire}
\rsLigneTableauItemsARembourser{2}{04/10/1952}{Bougie d'anniversaire}{500}{Carte bancaire}
\rsLigneTableauItemsARembourser{3}{06/10/1952}{Bicarbonate de soude}{125}{Espèces}


%%%%%%%%%%%%%%%%%%%%%%%%%%%%%%%%%%%%%%%%%%%%%%%%%%%%%%%%%%%%%%%%%%%%%%%%%%%%%%%%%%%%%%%%%
%%  On ferme le pied du tableau des items à rembourser; le total des frais sera calculé 
%% automatiquement; à n'utiliser qu'une seule fois.
\rsPiedTableauItemsARembourser{}
%%%%%%%%%%%%%%%%%%%%%%%%%%%%%%%%%%%%%%%%%%%%%%%%%%%%%%%%%%%%%%%%%%%%%%%%%%%%%%%%%%%%%%%%%


%%%%%%%%%%%%%%%%%%%%%%%%%
%% Texte libre, DEBUT. %%
%%%%%%%%%%%%%%%%%%%%%%%%%


%%%%%%%%%%%%%%%%%%%%%%%%%
%% Variables utiles:
%% \rsprenomNomCreancier{}, \rssocieteCreancier{}, \rsprenomNomClient{}
%% \rssocieteClient{}, \rsciviliteClient{}, \rstotalEnChiffres{}
%%%%%%%%%%%%%%%%%%%%%%%%%%
%% A propos des totaux:
%% rsTotalEnChiffres sera construit automatiquement, mais:
%% NOTEZ BIEN: 
%% 1) il est impossible de placer la variable \rstotalEnChiffres{} avant le tableau
%% des items à rembourser, car elle est construite dynamiquement en même temps que ce tableau 
%% pour obtenir le total chiffres. 
%% 2) si vous souhaitez un total en lettres, il s'agit d'un affichage, vous devez l'écrire vous même.
%% !!! ATTENTION QU'IL CORRESPONDE AU TOTAL EN CHIFFRES !!!
%% CECI EST SOUS VOTRE RESPONSABILITE DE VERIFIER ET D'ECRIRE LE MONTANT CORRECT
\rsTotalEnLettres{trente et un euros virgule quarante-cinq}
%% Décommenter pour vérifier les correspondances chiffres - lettres, commentez pour me cacher.
\textbf{\begin{center} Vérifiez chiffres = lettres, puis commentez-moi pour me faire disparaître\\ \rstotalEnChiffres{}~\rsuniteMonetaire{} = \rstotalEnLettres{} \end{center}}

Je soussigné, \rsprenomNomCreancier{}, déclare qu'il m'est dû par \rsciviliteClient{}\rsprenomNomClient{} de la société \rssocieteClient{}, la somme de \rstotalEnChiffres{}~\rsuniteMonetaire{} (\rstotalEnLettres{}) pour les dépenses énoncées dans le tableau ci dessus.

%%%%%%%%%%%%%%%%%%%%%%%
%% Texte libre, FIN. %%
%%%%%%%%%%%%%%%%%%%%%%%

%%%%%%%%%%%%%%%%%%%%%%%%%%%%%%%%%%%%%%%%%%%%%%%%%%%%%%%%%%%
%% Construit l'injonction à payer; 
%% la somme en chiffres et l'IBAN seront ajoutés automatiquement
%% les autres arguments sont nécessaires: lieu et date de la rédaction de la note, et le 3e
%% Certains clients, par facilité, acceptent l'image d'une signature en lieu et place de la 
%% signature écrite. Si le 3e parametre = oui et qu'un fichier signature.png existe, il sera 
%% affiché, sinon la place sera pour une signature manuscrite.
%% \rsConstruitInjonctionAPayer{lieu_redaction_note}{date_redaction_note}{oui/non} 
\rsConstruitInjonctionAPayer{\textsc{Brux-aile}}{10 octobre 1952}{oui}


%%%%%%%%%%%%%%%%%%%%%%%%%
%% Aération du texte   %%
\vspace{\rsespaceVertical}
%%%%%%%%%%%%%%%%%%%%%%%%%


%%%%%%%%%%%%%%%%%%%%%%%%%
%% Texte libre, DEBUT. %%
%%%%%%%%%%%%%%%%%%%%%%%%%

Les pièces justificatives, numérotées suivant le tableau, sont jointes à la présente note.

%%%%%%%%%%%%%%%%%%%%%%%
%% Texte libre, FIN. %%
%%%%%%%%%%%%%%%%%%%%%%%


%%%%%%%%%%%%%%%%%%%%%%%%%%%%%%%%%/o\____
% **** %%%%%%%%%%%%%%%%%%%%%%%%%%%%%%%%%\__
%  **** %%% Note de frais, 4e partie %%%%%%|==>
% **** %%%%%%%%%%%%%%%%%%%%%%%%%%%%%%%%|°°/
%%%%%%%%%%%%%%%%%%%%%%%%%%%%%%%%%%%%/
%%____________%%%%___________%%%%___\__________

%%%%%%%%%%%%%%%%%%%%%%%%%%%%%%%%%%%
%% Ajoute les crédits en footer. %%
%%%%%%%%%%%%%%%%%%%%%%%%%%%%%%%%%%%
\rsCredit{}

%%%%%%%%%%%%%%%%%%%%%%%%%%%%%%%%%%%%%%%%%%%%%%
%% Ci-dessous, les lignes du haut et du bas %%
%%%%%%%%%%%%%%%%%%%%%%%%%%%%%%%%%%%%%%%%%%%%%%

\begin{figure}[t]
\begin{center}
\includegraphics[scale=0.3]{line_haut.png}
\end{center}
\end{figure}


\begin{figure}[b]
\begin{center}
\includegraphics[scale=0.3]{line_bas.png}
\end{center}
\end{figure}

\end{document}
