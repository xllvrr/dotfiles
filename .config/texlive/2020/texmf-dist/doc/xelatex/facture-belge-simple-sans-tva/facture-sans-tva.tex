
%%%%%%%%%%%%%%%%%%%%%%%%%%%%%%%
%% Définition des commandes  %%
%%%%%%%%%%%%%%%%%%%%%%%%%%%%%%%%%%%%%%%%%%%%%%%%%%%%%%%%%%%%%%%%%%%%%%%%%%%%%
%% Toutes les commandes sont obligatoirement sous la forme \rsQuelqueChose %%
%%%%%%%%%%%%%%%%%%%%%%%%%%%%%%%%%%%%%%%%%%%%%%%%%%%%%%%%%%%%%%%%%%%%%%%%%%%%%


%%%%%%%%%%%%%%%%%%%%%%%
%% Facture seulement %%
%%%%%%%%%%%%%%%%%%%%%%%%%%%%%%%%%%%%%%%%%%%
%% \rsNoDate Numéro et date de la facture

\newcommand{\rsNoDate}[2]{
\begin{center} 
\textcolor{grisfonce}{{\Huge Facture \no #1}\\ {\large  du #2}}
\end{center}
}

%%%%%%%%%%%%%%%%%%%%%%%
%% Facture seulement %%
%%%%%%%%%%%%%%%%%%%%%%%%%%%%%%%%%%%%%%%%%%%%%%%%%%%%%%
%% les adresses expédition, facturation et livraison

%%%%%%%%%%%%%%%%%%%%%%%
%% Facture seulement %%
%% \rsEnteteTableauAdresses ouvre l'entête du tableau des adresses; 
%% a n'utiliser qu'une seule fois dans le document principal.
\newcommand{\rsEnteteTableauAdresses}{
\noindent
\begin{tabular}{p{0.3\textwidth}p{0.3\textwidth}p{0.3\textwidth}}
\hline\\
\textcolor{grisfonce} {\emph{Expédition}} & \textcolor{grisfonce}{\emph{Facturation}} & \textcolor{grisfonce}{\emph{Livraison}} \\
}

%%%%%%%%%%%%%%%%%%%%%%%
%% Facture seulement %%
%% \rsLigneTableauAdresses{}{}{} entrée d'une ligne d'adresse, 
%% dans l'ordre expédition, facturation, livraison
\newcommand{\rsLigneTableauAdresses}[3]{#1 & #2 & #3 \\}

%%%%%%%%%%%%%%%%%%%%%%%
%% Facture seulement %%
%% \rsPiedTableauAdresses ferme le pied du tableau des adresses; 
%% à n'utiliser qu'une seule fois dans le document principal.
\newcommand{\rsPiedTableauAdresses}{
\hline\\
\end{tabular}
}


%%%%%%%%%%%%%%%%%%%%%%%
%% Facture seulement %%
%%%%%%%%%%%%%%%%%%%%%%%%%%%%%%%%%%%%%%%%%%%%%%%%%%%%%%%%%%%%%%%%%%%%%%%%%%%%%%%%%%%%%%%%%%%%
%% \compteBancaireEtDateLimiteDePaiement: le n° de compte en banque et la date de paiement

\newcommand{\rsCompteBancaireEtDateLimiteDePaiement}[2]{
\begin{flushright}
A payer au compte bancaire \emph{#1} avant le #2.\\
\end{flushright}
}
