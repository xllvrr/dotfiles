% در صورت تمایل به تغییر لوگوی بسم‌ الله، لوگوی دلخواه خود را با اسم logo در پوشه figures قرار دهید و سپس با استفاده از دستور
% زیر پهنای آن را با عددی بین 0 و 1 مشخص کنید
%\besmwidth{.7}
% پردیس، دانشکده، آموزشکده و یا پژوهشکده  خود را وارد کنید

\campus{پردیس علوم}
\faculty{دانشکدۀ علوم ریاضی}
% گروه آموزشی خود را وارد کنید
\department{گروه علوم کامپیوتر}
% نام رشته تحصیلی خود را وارد کنید
\subject{علوم کامپیوتر}
% گرایش خود را وارد کنید
\field{علوم کامپیوتر-محاسبات علمی}
% عنوان پایان‌نامه/رساله را وارد کنید
\title{نمونه پایان‌نامه و راهنمای استفاده از کلاس \lr{yazd-thesis} برای پایان‌نامه‌های دانشگاه یزد}
% نام استاد(ان) راهنما را وارد کنید
\firstsupervisor{دکتر محمد فرشی}
%\secondsupervisor{استاد راهنمای دوم}
% نام استاد(دان) مشاور را وارد کنید. چنانچه استاد مشاور ندارید، دستور(های) پایین را غیرفعال کنید.
\firstadvisor{دکتر احمد ...}
%\secondadvisor{استاد مشاور دوم}
% نام پژوهشگر را وارد کنید
\name{ابوالقاسم }
% نام خانوادگی پژوهشگر را وارد کنید
\surname{احمدی}
% تاریخ پایان‌نامه را وارد کنید
\thesisdate{بهمن‌ماه ۱۳۹۳}
\credit{۶}
\defensedate{۱۸/۱۰/۱۳۹۳}
% در صورت کامنت کردن سه دستور زیر، جای خالی به جای آن‌ها در فرم صورت‌جلسه ایجاد می‌شود
\grade{۲۰}
\letgrade{بیست}
\degree{عالی}
% داور یا صاحبنظر داخلی اول
\firstinternalreferee{دکتر علی}
% داور یا صاحبنظر داخلی دوم
\secondinternalreferee{دکتر رضا}
% داور یا صاحبنظر خارجی اول
\firstexternalreferee{دکتر تقی}
% داور یا صاحبنظر خارجی دوم
\secondexternalreferee{دکتر بیژن}
% ناظر جلسه دفاع
\viewer{دکتر احمد}
%%%%%%%%%%%%%%%%%%%%%%%%%%%%%%%%%%%%

\totext{%
\noindent{\Large\bfseries تقدیم به }\\
\vspace*{1em}
\begin{center}
\large\bfseries
پدر و مادر عزیزم
\end{center}
\begin{center}
و همه کسانی که درست اندیشیدن را به من آموختند.
%%%%دکتر علی شریعتی
\end{center}
}
%%%%%%%%%%%%%%%%%%%%%%%%%%%%%%%%%%%%
\ack{%
\subsection*{سپاس‌گزاری}
سپاس خداوند یکتای عزتمندی که رحمت و دانش او در سراسر گیتی گسترده شده، آسمان‌ها و زمین همه از آن اوست و  علم و دانش حقیقی را بر هر که بخواهد موهبت می‌فرماید. رحمت و لطف او را بی‌نهایت سپاس می‌گویم چرا که فهم و درک مطالب این پژوهش را بر من ارزانی داشت و مرا به این اصل رساند که علم و ایمان دو بال یک پروازند. توفیق تلاش به‌ من داد و هر بار که خطا کردم فرصتی دوباره، تا با امید، تلاشی تازه را آغاز کنم و به خواست او به نتیجه‌ی مطلوب نائل آیم. به‌راستی که همه چیز از آن اوست و همه‌چیز به خواست اوست. 
%%%% در صورت استفاده از دستور زیر، تاریخ و امضای شما، به طور خودکار درج می‌شود.
%\signature 
}
%%%%%%%%%%%%%%%%%%%%%%%%%%%%%%%%%%%%

\faabstract{%
\subsection*{چکیده}
\thispagestyle{empty}
با توجه به لزوم یکسان‌سازی پایان‌نامه‌های دانشگاه و همچنین جهت راحتی کار دانشجویان تحصیلات
تکمیلی در تایپ پایان‌نامه و با توجه به وجود نرم افزار حروف‌چینی علمی \TeX\  که علاوه بر 
متن-باز و رایگان بودن، از قابلیت‌های بالایی در حوزه نشر برخوردار است، با همکاری آقای
وحید دامن‌افشان اقدام به تهیه یک قالب
برای پایان‌نامه‌های کارشناسی ارشد و دکتری دانشگاه یزد شده است. 

این قالب که با بستۀ \XePersian\ 
کار می‌کند، به گونه‌ای تهیه شده است که قوانین مربوط به آیین‌نامه نگارشی دانشگاه یزد را رعایت کرده
و دانشجو لازم نیست در این خصوص کار خاصی انجام دهد. تنها کار وی، وارد کردن متن پایان‌نامه
و دریافت نتیجۀ خروجی مطابق با دستورالعمل نگارشی است. 

نرم‌افزار \TeX\ هرچند به دلیل قابلیت منحصر به فردش در نوشتن فرمول‌های ریاضی، به نرم‌افزار
تایپ متون ریاضی معروف است، اما قابلیت‌های دیگر آن به گونه‌ای است که در تمام رشته‌ها قابلیت
استفاده را دارد و در حال حاضر به عنوان مطرح‌ترین نرم‌افزار در حوزه نشر، به خصوص متون علمی
استفاده می‌شود. 
لذا استفاده از این امکان را به تمام دانشجویان توصیه می‌کنیم.

در این فایل سعی شده است ضمن معرفی مختصر نرم‌افزار \TeX، روش نصب نرم‌افزار و 
روش استفاده از قالب 
\lr{yazd-thesis} را که برای پایان‌نامه‌های دانشگاه یزد طراحی شده است را بیان کند. همین فایل به عنوان
نمونه‌ای از متن تایپ شده در قالب مربوطه نیز جهت استفاده در اختیار دانشجویان محترم قرار می‌گیرد.

به منظور دسترسی به آخرین نسخه، میتوانید به لینک زیر مراجعه کنید.

\centerline{\url{cs.yazd.ac.ir/forms/yazd-thesis.zip}}

 همچنین وبلاگ 
\url{http://yazd-thesis.blog.ir/} نیز برای بحث و تبادل نظر و اعلام اشکالات و ایرادات قالب
ایجاد شده است.

به امید این که این کار مورد قبول و استفاده دانشجویان دانشگاه یزد قرار بگیرد.

\centerline{محمد فرشی}
\centerline{بهمن‌ماه 1393}
 \vfill

\textcolor{red}{تاریخ ایجاد  فایل: \today}

}
\yazdtitle
