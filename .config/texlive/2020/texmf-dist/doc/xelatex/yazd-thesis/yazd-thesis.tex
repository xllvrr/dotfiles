% !TEX TS-program = XeLaTeX
% اگر قصد نوشتن رساله دکتری را دارید، در خط زیر، گزینه msc را پاک کنید. کلیه تنظیمات لازم به طور خودکار اعمال می‌شود.
% برای عدم نمایش فرم صورت‌جلسه، گزینه minutes را در خط زیر پاک کنید.
\documentclass[msc,hidelinks,minutes]{yazd-thesis}
\usepackage{xepersian}
\settextfont[Scale=1.32]{Yas}
\setlatintextfont[Scale=1.3]{Times New Roman} 
% فونت فرم صورت‌جلسه
\defpersianfont\minutesfont[Scale=1.1]{B Nazanin}
\setdigitfont[Scale=1.2]{Yas}
\DeclareMathSizes{10.95}{12}{7.3}{5.5}
\defpersianfont\ypfont[Scale=1]{Yas}
\deflatinfont\yefont[Scale=1]{Times New Roman} 
\SepMark{-}
\begin{document}
\baselineskip=1cm
% در صورت تمایل به تغییر لوگوی بسم‌ الله، لوگوی دلخواه خود را با اسم logo در پوشه figures قرار دهید و سپس با استفاده از دستور
% زیر پهنای آن را با عددی بین 0 و 1 مشخص کنید
%\besmwidth{.7}
% پردیس، دانشکده، آموزشکده و یا پژوهشکده  خود را وارد کنید

\campus{پردیس علوم}
\faculty{دانشکدۀ علوم ریاضی}
% گروه آموزشی خود را وارد کنید
\department{گروه علوم کامپیوتر}
% نام رشته تحصیلی خود را وارد کنید
\subject{علوم کامپیوتر}
% گرایش خود را وارد کنید
\field{علوم کامپیوتر-محاسبات علمی}
% عنوان پایان‌نامه/رساله را وارد کنید
\title{نمونه پایان‌نامه و راهنمای استفاده از کلاس \lr{yazd-thesis} برای پایان‌نامه‌های دانشگاه یزد}
% نام استاد(ان) راهنما را وارد کنید
\firstsupervisor{دکتر محمد فرشی}
%\secondsupervisor{استاد راهنمای دوم}
% نام استاد(دان) مشاور را وارد کنید. چنانچه استاد مشاور ندارید، دستور(های) پایین را غیرفعال کنید.
\firstadvisor{دکتر احمد ...}
%\secondadvisor{استاد مشاور دوم}
% نام پژوهشگر را وارد کنید
\name{ابوالقاسم }
% نام خانوادگی پژوهشگر را وارد کنید
\surname{احمدی}
% تاریخ پایان‌نامه را وارد کنید
\thesisdate{بهمن‌ماه ۱۳۹۳}
\credit{۶}
\defensedate{۱۸/۱۰/۱۳۹۳}
% در صورت کامنت کردن سه دستور زیر، جای خالی به جای آن‌ها در فرم صورت‌جلسه ایجاد می‌شود
\grade{۲۰}
\letgrade{بیست}
\degree{عالی}
% داور یا صاحبنظر داخلی اول
\firstinternalreferee{دکتر علی}
% داور یا صاحبنظر داخلی دوم
\secondinternalreferee{دکتر رضا}
% داور یا صاحبنظر خارجی اول
\firstexternalreferee{دکتر تقی}
% داور یا صاحبنظر خارجی دوم
\secondexternalreferee{دکتر بیژن}
% ناظر جلسه دفاع
\viewer{دکتر احمد}
%%%%%%%%%%%%%%%%%%%%%%%%%%%%%%%%%%%%

\totext{%
\noindent{\Large\bfseries تقدیم به }\\
\vspace*{1em}
\begin{center}
\large\bfseries
پدر و مادر عزیزم
\end{center}
\begin{center}
و همه کسانی که درست اندیشیدن را به من آموختند.
%%%%دکتر علی شریعتی
\end{center}
}
%%%%%%%%%%%%%%%%%%%%%%%%%%%%%%%%%%%%
\ack{%
\subsection*{سپاس‌گزاری}
سپاس خداوند یکتای عزتمندی که رحمت و دانش او در سراسر گیتی گسترده شده، آسمان‌ها و زمین همه از آن اوست و  علم و دانش حقیقی را بر هر که بخواهد موهبت می‌فرماید. رحمت و لطف او را بی‌نهایت سپاس می‌گویم چرا که فهم و درک مطالب این پژوهش را بر من ارزانی داشت و مرا به این اصل رساند که علم و ایمان دو بال یک پروازند. توفیق تلاش به‌ من داد و هر بار که خطا کردم فرصتی دوباره، تا با امید، تلاشی تازه را آغاز کنم و به خواست او به نتیجه‌ی مطلوب نائل آیم. به‌راستی که همه چیز از آن اوست و همه‌چیز به خواست اوست. 
%%%% در صورت استفاده از دستور زیر، تاریخ و امضای شما، به طور خودکار درج می‌شود.
%\signature 
}
%%%%%%%%%%%%%%%%%%%%%%%%%%%%%%%%%%%%

\faabstract{%
\subsection*{چکیده}
\thispagestyle{empty}
با توجه به لزوم یکسان‌سازی پایان‌نامه‌های دانشگاه و همچنین جهت راحتی کار دانشجویان تحصیلات
تکمیلی در تایپ پایان‌نامه و با توجه به وجود نرم افزار حروف‌چینی علمی \TeX\  که علاوه بر 
متن-باز و رایگان بودن، از قابلیت‌های بالایی در حوزه نشر برخوردار است، با همکاری آقای
وحید دامن‌افشان اقدام به تهیه یک قالب
برای پایان‌نامه‌های کارشناسی ارشد و دکتری دانشگاه یزد شده است. 

این قالب که با بستۀ \XePersian\ 
کار می‌کند، به گونه‌ای تهیه شده است که قوانین مربوط به آیین‌نامه نگارشی دانشگاه یزد را رعایت کرده
و دانشجو لازم نیست در این خصوص کار خاصی انجام دهد. تنها کار وی، وارد کردن متن پایان‌نامه
و دریافت نتیجۀ خروجی مطابق با دستورالعمل نگارشی است. 

نرم‌افزار \TeX\ هرچند به دلیل قابلیت منحصر به فردش در نوشتن فرمول‌های ریاضی، به نرم‌افزار
تایپ متون ریاضی معروف است، اما قابلیت‌های دیگر آن به گونه‌ای است که در تمام رشته‌ها قابلیت
استفاده را دارد و در حال حاضر به عنوان مطرح‌ترین نرم‌افزار در حوزه نشر، به خصوص متون علمی
استفاده می‌شود. 
لذا استفاده از این امکان را به تمام دانشجویان توصیه می‌کنیم.

در این فایل سعی شده است ضمن معرفی مختصر نرم‌افزار \TeX، روش نصب نرم‌افزار و 
روش استفاده از قالب 
\lr{yazd-thesis} را که برای پایان‌نامه‌های دانشگاه یزد طراحی شده است را بیان کند. همین فایل به عنوان
نمونه‌ای از متن تایپ شده در قالب مربوطه نیز جهت استفاده در اختیار دانشجویان محترم قرار می‌گیرد.

به منظور دسترسی به آخرین نسخه، میتوانید به لینک زیر مراجعه کنید.

\centerline{\url{cs.yazd.ac.ir/forms/yazd-thesis.zip}}

 همچنین وبلاگ 
\url{http://yazd-thesis.blog.ir/} نیز برای بحث و تبادل نظر و اعلام اشکالات و ایرادات قالب
ایجاد شده است.

به امید این که این کار مورد قبول و استفاده دانشجویان دانشگاه یزد قرار بگیرد.

\centerline{محمد فرشی}
\centerline{بهمن‌ماه 1393}
 \vfill

\textcolor{red}{تاریخ ایجاد  فایل: \today}

}
\yazdtitle

\frontmatter
\tableofcontents
\listofsymbols%[3em]
\mainmatter
\include{chapter1}
\chapter{راهنمای نصب  {\LaTeX}} % شامل \XePersian\  و  \lr{Farsi\TeX}
%\chapter{ثثث}
\section{مقدمه}
نرم‌افزار حروفچینی \TeX یکی از نرم‌افزارهای معروف حروفچینی متون علمی است
که در سطح وسیعی جهت حروفچینی  مجلات و کتب استفاده می‌شود. در این متن مختصر بر آنیم
که راهنمای سریعی برای نصب و استفاده از آن بیان کنیم با این امید که کاربران با پیگیری آن
به راحتی  بتوانند آن را نصب و استفاده نمایند.

قبل از این لازم است جهت واضح شدن شکل عملکرد این نرم افزار، اطلاعاتی در مورد آن داشته
باشیم که در ادامه به آن پرداخته می‌شود.

نرم افزار حروفچینی \TeX یک نرم افزار مجانی است که به صورت خط فرمانی کار می‌کند، به این
معنی که متن مورد نظر در یک فایل نوشته شده و سپس این فایل از طریق دستورات خط
فرمان به نرم افزار حروفچین \TeX داده می‌شود. این نرم افزار فایل داده شده را خوانده و بر
مبنای آن متن حروفچینی شده را به صورت یک فایل (مثلا PDF) ارائه می‌کند.

دستورات خط فرمان متعددی برای استفاده از این نرم افزار حروفچین وجود دارد که از مهمترین 
آنها می‌توان به latex، pdflatex و xelatex اشاره کرد. معمولاً ما این بخش از
نرم افزار حروفچین را موتور \TeX می‌نامیم. این خاصیت، اولین متمایز کنندۀ این نرم‌افزار
از سایر نرم‌افزارها نظیر Office  است زیرا در Office شما نتیجه نهایی را همزمان با تایپ
 می‌بینید ولی در این نرم‌افزار باید فایل را به حروفچین بدهید تا خودش شکل خروجی را آماده
 کند. عملاً به همین دلیل نیز آن را نرم‌افزار حروفچین می‌نامند، مشابه این که شما متن خام 
 خود را به یک فرد حروفچین می‌دهید تا با شکل دهی آن در قالب صفحات، آن را برای چاپ
 آماده کند.
 
 پس متن خام باید در یک ویرایشگر تایپ شده و سپس فایل حاصل (که پسوند آن .tex است)
 به برنامۀ حروفچین
 با استفاده از خط فرمان داده شود. ویرایشگرهایی وجود دارند که امکان وارد کردن متن خام
 و به طور همزمان، امکان دادن فایل به موتور \TeX و نشان دادن نتیجۀ حروفچینی را دارند. 
 اما تمام آنها بر مبنای همان دستورات خط فرمان عمل می‌کنند و هیچکدام به تنهایی و بدون
 دسترسی به یک موتور \TeX نمی‌توانند خروجی تولید کنند. البته هیچ وابستگی بین
 ویرایشگر و فایل تولید شده توسط آن وجود ندارد و یک فایل توسط هر کدام می‌تواند 
 تولید یا ویرایش شود یا فایل ایجاد شده توسط  یک ویرایشگر، در دیگری تغییر یابد.
 از معروف‌ترین این ویرایشگرها می‌توان به WinEdit، Texmaker
و  Notepad++  اشاره کرد.

برای حروفچینی فایل، می‌توان از طریق خط فرمان به صورت زیر عمل کرد. در ویندوز
وارد \lr{Command Prompt } شوید و به محل قرار گرفتن فایل مربوطه (همان فایل با پیوند
.tex) بروید. بسته به کاربرد خود  و شکل خروجی مورد نظر 
یکی از دستورات زیر را بزنید تا فایل خروجی مربوطه ایجاد شود. به جای filename  نام
فایل .tex گذاشته شود. 


\begin{center}
\begin{tabular}{|c|c|}
\hline
برای خروجی .dvi با فایل ورودی انگلیسی & \lr{latex filename}\\  \hline
برای خروجی .pdf با فایل ورودی انگلیسی & \lr{pdflatex filename}\\ \hline
برای خروجی .pdf با فایل ورودی فارسی یا  انگلیسی & \lr{xelatex filename}\\ \hline
\end{tabular}
\end{center}

\textcolor{red}{توجه:} دقت کنید که نام فایل یا فولدرهایی که فایل در آن قرار دارد فارسی
نباشد یا بین نام آنها فاصله وجود نداشته باشد. در صورت عدم رعایت این موضوع، در برخی
مواقع اجرا با مشکل روبرو می‌شود.

فایل آماده شده خام، شامل دستوراتی است که قسمتهای مختلف متن نظیر عنوان فصل و بخش
و سایر موارد را مشخص می‌کند. اگر این دستورات درست استفاده نشده باشند، حروفچین در زمان
حروفچینی خطا می‌دهد که پیام خطا شامل شماره خطی است که در آن خطا اتفاق افتاده است.
لذا، در این موارد باید مشابه خطاگیری از یک برنامۀ کامپیوتری، نسبت به رفع خطا اقدام کرد.
توجه کنید که وجود خطا ممکن است متن را به صورتی به غیر از آنچه مورد نظر است حروفچینی
کند و اگر تعداد خطاها زیاد باشد ممکن است قسمت یا کل متن را حروفچینی نکند و خروجی
نداشته باشد یا خروجی حاصل ناقص باشد.
 
\section{نصب موتور اصلی \TeX}\footnote{فایل‌های مورد استفاده علاوه بر آدرس‌های ذکر شده در اف تی پی دانشگاه یزد به آدرس  
%\begin{verbatim}
\href{ftp://ftp.yazd.ac.ir:8621/Mathematic/TeX}{\lr{ftp://ftp.yazd.ac.ir:8621/Mathematic/TeX}}
%\end{verbatim}
نیز وجود دارد. کاربران متصل به شبکه دانشگاه یزد می‌توانند بدون  نیاز به اتصال به اکانتینگ، این فایل‌ها را با سرعت بالا دانلود نمایند.}
توزیع‌های مختلفی برای موتور \TeX وجود دارد که در اینجا به نصب دو توزیع
معروف و مجانی آن به نام‌های TeXLive  و MikTeX می‌پردازیم. تاکید می‌شود که این توزیع‌ها با هم سازگار هستند، به این
معنی که فایل آماده شده روی تمام توزیع‌های موتور \TeX \ کار می‌کند. لذا مهم نیست کدام
توزیع را برای نصب انتخاب کنید. با نصب هر کدام از این دو توزیع، به طور اتوماتیک
بسته \XePersian \  نصب می‌شود و نیاز به هیچ کار اضافی نیست. فقط لازم است
که فونت‌های فارسی استفاده شده در متون فارسی روی سیستم عامل نصب شده باشد. لذا
تنها کار اضافی این است که مجموعه فونت‌های جمع آوری شده در فایل زیر روی سیستم عامل
نصب شود. توصیه می‌شود حتی اگر فونت‌ها را روی کامپیوتر خود دارید، دوباره آنها را با استفاده
از فونت‌های فایل زیر رونویسی کنید. این کار از بسیاری مشکلات بعدی جلوگیری می‌کند.
\begin{latin}

\href{http://bayanbox.ir/id/4609192605141061595}{Part 1: http://bayanbox.ir/id/4609192605141061595}

\href{http://bayanbox.ir/id/5468937351173971771}{Part 2: http://bayanbox.ir/id/5468937351173971771}

\href{http://bayanbox.ir/id/4133277893427051503}{Part 3: http://bayanbox.ir/id/4133277893427051503}
\end{latin}

البته توصیه پدیدآورندگان بسته \XePersian \ که جهت تولید متون فارسی در \TeX \ 
این بسته را ارائه کرده‌اند، استفاده از  TeXLive  است. 
%\pagebreak
\subsection{نصب \lr{TeXLive 2015}}
%\footnote{با توجه به ارائه نسخه 2015 از TeXLive، این
%نسخه از طریق وبسایت اعلام شده در این قسمت وجود دارد. اما فعلاً این نگارش مشکلاتی با \XePersian دارد. و لذا  نسخه 2014 این نرم افزار را که در اف تی پی دانشگاه یزد وجود دارد دانلود و نصب نمایید. 
%مشکل
%را در \href{http://qa.parsilatex.com/7838/%D9%85%D8%B4%DA%A9%D9%84%D8%A7%D8%AA-texlive-2015-%D9%88-%D8%A7%D8%B1%D8%AA%D9%82%D8%A7-%D8%A8%D8%B3%D8%AA%D9%87-%D9%87%D8%A7-%D8%AF%D8%B1-texlive-2014?show=7838#q7838}{این لینک} ببینید.
%}
سایت‌های معروف به CTAN، سایت‌هایی هستند که  وظیفه توزیع نسخه‌های مختلف
مجانی موتور \TeX را انجام می‌دهند. یکی از این وبسایت‌ها در  دانشگاه یزد 
قرار دارد که در آدرس زیر در دسترس است.\footnote{این وبسایت به همت آقای
مهندس فاطمی از مرکز اطلاع‌رسانی و خدمات رایانه‌ای دانشگاه یزد ایجاد شده است که
جا دارد از ایشان در این خصوص تشکر کرد.}

\centerline{\href{http://ctan.yazd.ac.ir/}{http://ctan.yazd.ac.ir/}}

این سایت به صورت روزانه به روز رسانی می‌شود. می‌توان از این سایت در هر 
لحظه آخرین نگارش‌های نرم افزارهای مربوطه را دانلود کرد. لازم به ذکر است که در صورت اتصال
به شبکه دانشگاه یزد، برای دسترسی به این سایت نیازی به استفاده از اکانتینگ و 
اتصال به اینترنت نیست، بلکه این سایت از طریق شبکه داخلی دانشگاه در دسترس است.

برای نصب TeXLive  مراحل زیر را انجام دهید:
\begin{enumerate}
\item وارد سایت \href{http://ctan.yazd.ac.ir/}{http://ctan.yazd.ac.ir/}
 شوید و در پایین صفحه روی \lr{TeX Live}
 کلیک کنید.
 \item روی مسیر Images کلیک کنید و از فولدر باز شده فایل با نام 
 	\lr{texlive2015-20150523.iso} را دانلود کنید. دقت کنید که 8 شماره آخر فایل ممکن 
	است مختلف باشد زیرا نشان‌دهنده تاریخ ایجاد فایل است. دقت کنید که حجم این فایل
	حدود \lr{2.7} گیگا بایت است.
 \item پس از دانلود کامل، آن را با نرم افزار WinRaR  باز کنید و در  پوشه‌ای به نام
 \lr{TeXLive2015} فایل را Extract کنید.
 \item وارد این پوشه شوید و برنامه 
 %install-tl
\lr{install-tl-windows}
 را اجرا کنید. ادامه روند مشابه نصب سایر نرم افزارها 
 است. روند نصب بسته به سرعت کامپیوتر شما ممکن است تا 1 ساعت طول بکشد.
 تصویر پنجره‌های نصب به صورت زیر است (به ترتیب از چپ به راست و از بالا به پایین)
\latin
\begin{figure}[htb]
\includegraphics[width=0.45\textwidth]{TexLive2015-1}\hfill
\includegraphics[width=0.45\textwidth]{TexLive2015-2}
\vspace*{4mm}

\includegraphics[width=0.45\textwidth]{TexLive2015-3}\hfill
\includegraphics[width=0.45\textwidth]{TexLive2015-4}
\vspace*{4mm}

\includegraphics[width=0.45\textwidth]{TexLive2015-5}\hfill
\includegraphics[width=0.45\textwidth]{TexLive2015-6}
\vspace*{4mm}
%
%\includegraphics[width=0.45\textwidth]{install-6}
\persian
\caption{پنجره‌های نصب \lr{TeXLive 2015} (ترتیب از چپ به راست)}
\end{figure}
\persian

 \item پس از پایان نصب، موتور \TeX آماده استفاده است. اگر قصد استفاده از \XePersian
 \ دارید، فقط لازم است فونت‌های مربوطه را که در بالا لینک آن آمده است را نصب کنید.
\end{enumerate}
بهتر است بعد از نصب؛ بسته های این نرم افزار را با روش زیر به روز رسانی کنید.
\subsubsection{بروزرسانی بسته‌های \lr{TeXLive 2015}}
دقت کنید که برای بروزرسانی شما باید به اینترنت متصل باشید زیرا بروزرسانی با استفاده از اینترنت انجام می‌شود.
\begin{enumerate}
\item ابتدا در قسمت برنامه‌ها، برنامه \lr{TeX Live manager} را اجرا کنید.
\item مطابق شکل زیر، مسیر به روزرسانی را \url{ctan.yazd.ac.ir} انتخاب کنید. انتخاب هر مسیر دیگر اشکالی ندارد ولی روی سرعت گرفتن فایل‌ها تاثیر دارد. 
\item ابتدا مطابق شکل، بسته مشخص شده را به روزرسانی کنید. پس از بروز رسانی این بسته، برنامه بسته می‌شود و لازم است دو مرحله قبل را دوباره تکرار کنید.
\item حال روی \lr{Updtate all installed} کلیک کنید. به روزرسانی نیز مشابه نصب مدت زمانی که به سرعت کامپیوتر و سرعت اینترنت شما وابسته است طول می‌کشد.
\end{enumerate}
%\endinput
\latin
\begin{figure}[h]
\includegraphics[width=0.4\textwidth]{TexLive2015-Update-1}\hfill
\includegraphics[width=0.4\textwidth]{TexLive2015-Update-2}
\vspace*{4mm}

\includegraphics[width=0.4\textwidth]{TexLive2015-Update-3}\hfill
\includegraphics[width=0.4\textwidth]{TexLive2015-Update-4}
\vspace*{4mm}

%\includegraphics[width=0.4\textwidth]{TexLive2015-Update-5}\hfill
%\includegraphics[width=0.4\textwidth]{TexLive2015-Update-6}
%\vspace*{4mm}

%\includegraphics[width=0.45\textwidth]{TexLive2015-3}\hfill
%\includegraphics[width=0.45\textwidth]{TexLive2015-4}
%\vspace*{4mm}

%\includegraphics[width=0.45\textwidth]{TexLive2015-5}\hfill
%\includegraphics[width=0.45\textwidth]{TexLive2015-6}
%\vspace*{4mm}
%
%\includegraphics[width=0.45\textwidth]{install-6}
\persian
\caption{پنجره‌های بروزرسانی \lr{TeXLive 2015} (ترتیب از چپ به راست)}
\end{figure}
\begin{figure}[t]
%\includegraphics[width=0.45\textwidth]{TexLive2015-1}\hfill
%\includegraphics[width=0.45\textwidth]{TexLive2015-2}
%\vspace*{4mm}
%

\includegraphics[width=0.4\textwidth]{TexLive2015-Update-5}\hfill
\includegraphics[width=0.4\textwidth]{TexLive2015-Update-6}
\vspace*{4mm}

\includegraphics[width=0.45\textwidth]{TexLive2015-Update-7}
\persian
\caption{ادامه پنجره‌های بروزرسانی  \lr{TeXLive 2015}  (ترتیب از چپ به راست)}
\end{figure}

\persian
\newpage 
\pagebreak
\subsection{نصب  \lr{Miktex 2.9}   به طور کامل (\XePersian)}
\baselineskip=1cm
\begin{enumerate}
\item برای دانلود فایل‌های لازم از یکی از روش‌های زیر استفاده کنید:
\begin{itemize}
\item اگر به شبکه دانشگاه یزد دسترسی دارید (به شبکه داخلی دانشگاه متصل هستید) بدون نیاز به اتصال
به اکانتینگ، از لینک زیر فایل مربوطه را دانلود کنید:

\centerline{\lr{ftp://ftp.yazd.ac.ir:8621/Mathematic/TeX/}}

توجه کنید که این فایل لزوماً آخرین نگارش نرم افزار نیست ولی برای اجرا مشکلی ندارد. پس از دانلود
فایل فشرده را باز کنید.

\item دانلود فایل \lr{setup-2.9.4503.exe} از مسیر زیر و اجرای آن و انتخاب گزینۀ دانلود برای دانلود فایل‌های لازم.

\lr{http://ctan.yazd.ac.ir/systems/win32/miktex/setup/}

دقت کنید که شماره پایانی ممکن است در نگارش‌های جدیدتر متفاوت باشد.
این نرم افزار تمام فایل‌های لازم را دانلود و در مسیری که مشخص کرده‌اید ذخیره می‌کند.
برای انتخاب محل از \lr{ctan.yazd.ac.ir} استفاده کنید.
\item دانلود تمام بسته‌ها مستقیماً از مسیر زیر:\\
\lr{http://ctan.yazd.ac.ir/systems/win32/miktex/tm/packages/}
\end{itemize}
\item   از پوشه مربوطه، فایل \lr{setup-2.9.4503.exe}  را اجرا کنید.  (شکل 1 را ببینید\footnote{شکل‌ها مربوط
به \lr{MikTeX2.8} است ولی مشابه شکل‌های اجرا است. }).
\item در پنجرۀ باز شده \lr{Accept the Miktex ...} را تیک زده و روی \lr{Next} کلیک کنید.
\item \label{complete} در پنجره بعدی \lr{Complete Miktex} را انتخاب کنید و روی \lr{Next} کلیک کنید.

\item  در پنجره بعدی \lr{Anyone who use ...}  را انتخاب کنید و روی \lr{Next} کلیک کنید.
\item  در پنجره بعدی  مسیر مورد نظر برای نصب را انتخاب کنید و روی \lr{Next} کلیک کنید. دقت کنید که تمام نام مسیر باید به 
انگلیسی باشد وگرنه در اجرای برنامه مشکل ایجاد خواهد شد. نصب برنامه به فضای هارد تقریبا 2 گیگا بایت نیاز دارد و روند نصب
بسته به سرعت کامپیوتر شما ممکن است تا 2 ساعت طول بکشد.
\item  پس از اتمام نصب، فونت‌های موجود در پوشه \lr{Xepersian-fonts} را روی ویندوز نصب نمایید.
توصیه می‌شود در صورت موجود بودن فونت‌ها، آنها رونویسی شوند.
\item در این مرحله \XePersian نصب شده و قابل استفاده است.
\item  جهت ویرایش متون خود باید از ادیتورهای پشتیبانی کننده یونیکد استفاده نمایید. نوع ادیتور نقشی در فرآیند حروفچینی ندارد. در ادامه
نصب و استفاده از \lr{Notepad++ } آمده است.
\end{enumerate}
\latin
\begin{figure}
\includegraphics[width=0.45\textwidth]{Setup}\hfill
\includegraphics[width=0.45\textwidth]{install-1}
\vspace*{4mm}

\includegraphics[width=0.45\textwidth]{install-2}\hfill
\includegraphics[width=0.45\textwidth]{install-3}
\vspace*{4mm}

\includegraphics[width=0.45\textwidth]{install-4}\hfill
\includegraphics[width=0.45\textwidth]{install-5}
\vspace*{4mm}

\includegraphics[width=0.45\textwidth]{install-6}
\persian
\caption{پنجره‌های نصب \lr{MikTeX 2.9} (ترتیب از چپ به راست)}
\end{figure}
\persian
\section{ناسازگاری فایل‌های قبلی آماده شده با زیپرشن }
پس از نصب \lr{MikTeX2.9} با اجرای فایل‌های قبلی ممکن است با پیغام خطای زیر مواجه شوید:
\latin
\begin{verbatim}
! Undefined control sequence.
\SetMathCode #1#2#3#4->\Umathcode 
                       #1="\mathchar@type #2 \csname sym#3\endcsn...
\end{verbatim}
\persian
برای رفع خطا دستورات زیر را در خط بعد از دستور $\backslash$ documentclass بگذارید:
\latin
\begin{verbatim}
\makeatletter
\@ifundefined{Umathcode}{\let\Umathcode\XeTeXmathcode}{}
\@ifundefined{Umathchardef}{\let\Umathchardef\XeTeXmathchardef}{}
\makeatother 
\end{verbatim}
\persian
\section{نصب Notepad++}
\baselineskip=1cm
ادیتور \lr{Notepad++ }  به دلیل قابلیت فارسی نویسی و همچنین از راست به چپ نویسی و امکان اجرای دستورات خط فرمان در ادیتور،
انتخاب مناسبی برای نوشتن متون  است. برای فعال کردن قابلیت اجرای دستورات خط فرمان با استفاده از کلید \lr{F6}، پس از نصب نرم افزار
\lr{Notepad++ }، فایل \lr{NppExec\_030\_dll\_Unicode.zip}
را در پوشه plugins  از مسیری که \lr{Notepad++ }  در آن نصب شده است باز کنید. حال با زدن کلید \lr{F6}  در ادیتور، پنجره اجرای دستور باز می‌شود.

نمونه دستوری که می‌توانید وارد کنید به صورت زیر است:

\begin{latin}
\begin{verbatim}
NPP_SAVE
cd $(CURRENT_DIRECTORY)
xelatex $(NAME_PART)
\end{verbatim}
\end{latin}

برای تایپ از راست به چپ کلیدهای \lr{Alt+CTRL+R}  را بزنید و برای از چپ به راست نویسی کلیدهای \lr{Alt+Ctrl+L}  را بزنید.

برای نیم فاصله، کلید استاندارد \lr{Ctrl+SHift+2}  است که در این ادیتور به دلیل استفاده از این ترکیب برای کار دیگری عمل نمی‌کند.
برای عمل کردن آن باید این ترکیب کلید را از ادیتور حذف کنید. برای این منظور از منوی \lr{settings -> Shortcut Mapper}
 در برگه \lr{Main Menu}  در ردیف حدودا 80  این ترکیب را پیدا کرده و به چیز دیگری (مثلا \lr{CTRL+Shift+T}) عوض کنید.

 پس از این کار ترکیب  \lr{Ctrl+SHift+2} برای نیم فاصله (وقتی زبان فارسی باشد) کار می‌کند.
 
 توجه: برای تهیه فایل مقاله یا کتاب با \XePersian، باید از کد \lr{UTF8}  برای کدگذاری فایل استفاده شود. برای انتخاب در ادیتور، از منوی
 Encoding  گزینه مورد نظر انتخاب شود.
 \section{کلاس پایان‌نامه دانشگاه یزد}
 با توجه به تنظیمات خاص پایان‌نامه‌های دانشگاههای مختلف و وقت قابل ملاحظه‌ای که انجام این تنظیمات از دانشجویان می‌گیرد، کلاسی بر مبنای تنظیمات پایان نامه‌های دانشگاه یزد ایجاد شده است و از طریق وبسایت زیر در دسترس دانشجویان عزیز می‌باشد.
 
 \centerline{\href{http://yazd-thesis.blog.ir/}{http://yazd-thesis.blog.ir/}}
 
 در این کلاس کلیه تنظیمات مربوط به پایان نامه لحاظ شده و نیاز به انجام تنظیمات اضافی ندارد. توضیحات استفاده از این کلاس و فایل نمونه در وبسایت فوق آمده است.
 %\pagebreak
 \section{تبدیل فایل‌های Word  به \LaTeX\   و برعکس}
 یک نرم افزار قوی برای تبدیل بین Word  و \LaTeX، نرم‌افزار \href{http://www.grindeq.com/}{GrindEQ}  است.
 این نرم‌افزار مجانی نیست ولی تا 10 فایل را برای شما تبدیل می‌کند. برای انجام تبدیل لازم است نرم‌افزار \lr{Office 2010}  را نصب و سپس
 دو فایل مربوط به تبدیل بین Word  و \LaTeX  (در پوشه Convert وجود دارد) را نصب نمایید.
 برای نسخه‌های بعدی \lr{Office}، آخرین نگارش نرم‌افزار را از وبسایت فوق دریافت و نصب نمایید. نرم‌افزار منطبق بر آفیس را (از حیث 32 بیتی یا 64 بیتی بودن) را نصب کنید وگرنه عملکرد تبدیل مناسب نخواهد بود.
 \subsection{تبدیل Word به \LaTeX}
 فایل خود را در Word باز کنید. سپس از منوی \lr{File}،
 \lr{Save As}  را انتخاب کنید و در قسمت \lr{Save as Type}، نوع \lr{LaTeX[GrindEQ]}
 را انتخاب کنید.  از پنجره باز شده گزینه‌های مناسب را انتخاب و فایل را ذخیره کنید. فایل ذخیره شده با فرمت \LaTeX \  است.
 
 اگر فایل Word شما \textcolor{red}{\bf فارسی} است، باید از پنجره باز شده، در قسمت encoding، 
 گزینۀ UTF8 یا Unicode  را انتخاب کنید. (شکل 3 را ببینید.) در فایل
 \LaTeX \ ایجاد شده نیز باید دستور \verb+ \usepackage{xepersian}+  و دستورات مربوط به فونت متن اضافه شود.
 
 همچنین در قسمتهای انگلیسی باید دستور \verb+\latin+  قبل از متن انگلیسی و \verb+\persian+ بعد از متن انگلیسی قرار گیرد.
 
 \subsection{تبدیل \LaTeX \ به Word}
  در Word   فایل \LaTeX  را باز کنید (فایل با پسوند \lr{.tex}). از پنجره باز شده مشابه قبلی، گزینه‌های مناسب را انتخاب کنید. فایل  به فرمت 
  Word تبدیل شده و باز می‌شود و می‌توانید آن را ذخیره کنید. امکان انتخاب فونت و سایر خواص در پنجره باز شده هنگام تبدیل ممکن است.
  
  مشابه قبل، اگر فایل \LaTeX \ شما \textcolor{red}{\bf فارسی} است، باید از پنجره باز شده، در قسمت encoding، 
 گزینۀ \lr{UTF8} یا Unicode  را انتخاب کنید. (شکل 3 را ببینید.)  
\latin
\begin{figure}
\begin{center}
\includegraphics[width=0.45\textwidth]{Word2Latex} %\hfill
\includegraphics[width=0.45\textwidth]{Latex2Word}
\end{center}
\persian
\caption{پنجره تبدیل Word  به \LaTeX\  (سمت  چپ ) و \LaTeX\  به Word (سمت راست).}
\end{figure}

  \persian 

{\bf توجه:} با توجه به این که امکان تبدیل فایل‌های   \lr{Farsi\TeX} به یونیکد در زیر بیان شده است، می‌توان پس از تبدیل
این فایل‌ها به یونیکد، آنها را با استفاده از ابزار فوق به Word  نیز تبدیل کرد. البته این مورد از نظر کیفیت انجام آزمایش نشده است.
 %\pagebreak 
\section{نصب  \lr{Farsi\TeX}}
\baselineskip=1cm
\begin{enumerate}
\item ابتدا \lr{MikTeX2.9} را نصب کنید.\footnote{امکان نصب فارسی تک روی
TeXLive نیز باید ممکن باشد ولی برای آن فعلا دستورالعملی نیست. می‌توانید دستورات معادل
این دستورات را در TeXLive اجرا کنید تا نتیجه را ببینید. با توجه به به روز رسانی نشدن فارسی تک
و همچنین مشکلات مربوط به ادیتور آن با ویندوزهای جدید و همچنین با توجه به این که فایل‌های
قدیمی نوشته شده در فارسی تک را به راحتی می‌توان به زیپرشن تبدیل کرد، به مقوله 
نصب فارسی تک زیاد پرداخته
نشده است.}
 در ادامه، فرض می‌کنیم این نرم افزار در مسیر 
\lr{C:$\backslash$miktex2.9} 
 نصب شده است.
در صورتی که برنامه در مسیر دیگری نصب شده است مسیر جدید جایگزین مسیر فوق در ادامه روند گردد.
\item فایل  \lr{farsitex-1.0-alpha-3} را در  مسیر \lr{C:$\backslash$miktex2.9}   باز  (unzip)  کنید. 
(فایل‌های این قسمت در پوشه FarsiTeX موجودند).
\item  ادیتور فارسی تک در مسیر \lr{Editor1}   را در مسیر \lr{C:$\backslash$miktex2.9$\backslash$miktex$\backslash$bin} 
نصب کنید. دقت کنید که معمولا bin  آخر توسط برنامه گذاشته می‌شود.
\item  فایل فشرده \lr{ftexed10-840121}  را که در مسیر \lr{Editor2} قرار دارد روی مسیر  \lr{C:$\backslash$miktex2.9$\backslash$miktex$\backslash$bin} 
باز کنید. در اینجا سوال پرسیده می شود که فایل وجود دارد و آیا رونویسی شود که بله را انتخاب کنید.
\item  در قسمت \lr{Start $\Longrightarrow$ Run} دستور \lr{mo$\_$admin} را اجرا کنید.
\item  در پنجره باز شده، روی \lr{Refresh FNDB} کلیک کنید و منتظر بمانید تا کار انجام شود.
\item  در همین پنجره، روی منوی Formats کلیک کنید و روی New کلیک کنید و جدول را مطابق اطلاعات زیر به طور دقیق تکمیل کنید.
در تکمیل دقیق اطلاعات این قسمت دقت کنید وگرنه فارسی تک اجرا نخواهد شد.
\begin{latin}
{\bf Format key:} {\Large farsitex}\\
{\bf Format name:} {\Large farsitex}\\
{\bf Compiler:} {\Large pdftex}\\
{\bf Input file name:} {\Large farsitex.ini}\\
{\bf Output file name:} {\Large farsitex.efmt}\\
{\bf Preloaded Format:} \\
{\bf Description:} {\Large FarsiTeX }\\
\end{latin}
 قسمت مربوط به \lr{Preloaded Format} خالی گذاشته شود. 
\item پس از وارد کردن اطلاعات و زدن OK عبارت FarsiTeX در جدول سمت چپ ظاهر می‌شود. روی آن کلیک کرده
و سپس روی Build کلیک کنید. پس از اتمام کار پنجره را ببندید.
\item  در مسیر 
\lr{C:$\backslash$miktex2.9$\backslash$tex$\backslash$latex209$\backslash$base} 
اسامی فایل‌های \lr{x.sty} را به \lr{x209.sty} (فقط برای فایل‌های article book، و report ) تغییر دهید. به عنوان مثال \lr{article.sty} به \lr{article209.sty} تغییر کند. با این کار
در فایل‌های فارسی‌تک نیز باید به جای \lr{article} در دستور $\backslash$documentstyle به \lr{article209} تبدیل شود.
\item  در قسمت \lr{Start $\Longrightarrow$ Run} دستور \lr{mo$\_$admin} را اجرا کنید. (بله! تکراری است ولی باید تکرار شود.)
\item  در پنجره باز شده، روی \lr{Refresh FNDB} کلیک کنید و منتظر بمانید تا کار انجام شود.
\item  در صورت تمایل به نصب فونت لوتوس، فایل lotusfont.zip  را در مسیر  \lr{C:$\backslash$miltex2.9}
باز کنید و سپس در قسمت  \lr{Start $\Longrightarrow$ Run} دستور \lr{initexmf\ \    -u} را اجرا کنید.
\item  در قسمت \lr{Start $\Longrightarrow$ Run} دستور \lr{mo$\_$admin} را اجرا کنید. (بله! تکراری است ولی باید تکرار شود.)
\item  در پنجره باز شده، روی \lr{Refresh FNDB} کلیک کنید و منتظر بمانید تا کار انجام شود.

%\item  جهت ویرایش متون خود باید از ادیتورهای پشتیبانی کننده یونیکد استفاده نمایید. نوع ادیتور نقشی در فرآیند حروفچینی ندارد.
%\item  جهت ویرایش متون خود باید از ادیتورهای پشتیبانی کننده یونیکد استفاده نمایید. نوع ادیتور نقشی در فرآیند حروفچینی ندارد.
\end{enumerate}
\section{تبدیل فایل های فارسی تک به زیپرشن}

نرم‌افزار تبدیل فایل‌های فارسی تک به زیپرشن توسط آقای دکتر واحدی آماده شده است. این نرم‌افزار با زبان Python نوشته
شده و برای اجرا لازم است آن را روی کامپیوتر نصب نمایید. جزئیات اجرا از راهنمای زیپرشن در زیر آمده است.
\begin{center}
\includegraphics[width=\textwidth]{farsitex2xepersian-1.jpg} 
\includegraphics[width=\textwidth]{farsitex2xepersian-2.jpg} 
%\includegraphics[width=\textwidth]{farsitex2xepersian-p1} 
%\includegraphics[width=\textwidth]{farsitex2xepersian-p2} 
\end{center}

انتخاب دیگر برای تبدیل، بازکردن فایل با استفاده از \lr{bidi-TeXMaker} است. این نرم افزار دارای منوی \lr{Import FTX File(s)} در منوی \lr{File} است. با باز کردن فایل‌های با پسوند \lr{.FTX}، فایل تبدیل  به یونیکد می‌شود.
\section{جزئیات فارسی نویسی در \lr{IPE Drawing}}
نرم‌افزار مجانی \lr{Ipe Drawing} که یک ابزار قوی برای رسم اشکال است و بر مبنای \TeX\ کار می‌کند 
نسخه جدید خود را منتشر کرد. این نسخه شامل فایل باینری برای سیستم‌های ویندوزی نیز هست. این نگارش 
جدید را علاوه بر وبسایت آن به آدرس \url{http://ipe.otfried.org/}
  می‌توانید از لینک زیر نیز دریافت کنید. این ابزار امکان درج فرمول‌های ریاضی و متون فارسی و همچنین درج تصاویر 
  با فرمت‌های bmp  و jpg  را نیز می‌دهد. لازم به ذکر است که برای اجرای این نرم‌افزار، حتما باید یکی از نگارش‌های TeX
  (نظیر \lr{MikTeX}  یا \lr{TexLive}) روی کامپیوتر شما نصب باشد. تصاویر تولید شده توسط این
  نرم افزار کاملاً برداری \lr{(vector)} است و حجم آن نیز پایین است.

  برای نصب نرم افزار کافی است فایل زیر را دانلود و آن را در مسیری باز کنید. سپس در
  مسیر bin از مسیر ایجاد شده فایل ipe را اجرا کنید. 
\href{http://bayanbox.ir/download/6365189354713376578/ipe-7.1.8-win.zip}{  دریافت \lr{Ipe Drawing 7.1.8-Win}}

  در خصوص طریقه فارسی نویسی در IPE، روش آن را که توسط آقای دکتر واحدی معرفی شده است به شرح زیر است:

پس از وارد شدن در \lr{IPE}، دستورات زیر را در منوی \lr{Edit-> Document properties}  در قسمت
\lr{Latex preamble } اضافه شود:

{\tiny
\latin
\begin{verbatim}
\usepackage[utf8]{inputenc}
\usepackage[LAE,LFE,OT1]{fontenc}
\usepackage[arabic,farsi,english]{babel}
\newcommand{\unichar}[1]{%
\ifnum#1="0621\hamza\fi%
\ifnum#1="0622\alefmadda\fi%
\ifnum#1="0623\alefhamza\fi%
\ifnum#1="0624\wawhamza\fi%
\ifnum#1="0625\aleflowerhamza\fi%
\ifnum#1="0626\yahamza\fi%
\ifnum#1="0627\alef\fi%
\ifnum#1="0628\baa\fi%
\ifnum#1="067E\peh\fi%
\ifnum#1="0629\T\fi%   %taa marbuuta
\ifnum#1="062A\taa\fi%
\ifnum#1="062B\thaa\fi%
\ifnum#1="062C\jeem\fi%
\ifnum#1="0679\tcheh\fi%  
\ifnum#1="062D\Haa\fi%
\ifnum#1="062E\kha\fi%
\ifnum#1="062F\dal\fi%
\ifnum#1="0630\dhal\fi%
\ifnum#1="0631\ra\fi%
\ifnum#1="0632\zay\fi%
\ifnum#1="0633\seen\fi%
\ifnum#1="0634\sheen\fi%
\ifnum#1="0635\sad\fi%
\ifnum#1="0636\dad\fi%
\ifnum#1="0637\Ta\fi%
\ifnum#1="0638\za\fi%
\ifnum#1="0639\ayn\fi%
\ifnum#1="063A\ghayn\fi%
\ifnum#1="0698\jeh\fi%
\ifnum#1="0640\keshchar\fi%
\ifnum#1="0641\fa\fi%
\ifnum#1="0642\qaf\fi%
\ifnum#1="06A9\farsikaf\fi%
\ifnum#1="0643\kaf\fi%
\ifnum#1="06AF\gaf\fi%
\ifnum#1="0644\lam\fi%
\ifnum#1="0645\meem\fi%
\ifnum#1="0646\nun\fi%
\ifnum#1="0647\ha\fi%
\ifnum#1="0648\waw\fi%
\ifnum#1="06CC\farsiya\fi%
\ifnum#1="064A\ya\fi%
\ifnum#1="0649\alefmaqsura\fi%
\ifnum#1="064B\nasb\fi%
\ifnum#1="064C\raff\fi%
\ifnum#1="064D\jarr\fi%
\ifnum#1="064E\fatha\fi%
\ifnum#1="064F\damma\fi%
\ifnum#1="0650\kasra\fi%
\ifnum#1="0651\shadda\fi%
\ifnum#1="0652\sukun\fi%
\ifnum#1="200c\ZWNJ\fi%
\ifnum#1="0649\tatweel\fi%
}
\TOCLanguage{farsi}
\farsimathdigits
\end{verbatim}
}
\persian

\baselineskip=1cm

نمونه فایل ساده ایجاد شده در لینک زیر است.
اگر در کپی و پیست کردن دستورات فوق ناموفق بودید، می‌توانی فایل نمونه زیر را در
Ipe باز کرده و سپس این دستورات را در محلی که در فوق اشاره شده کپی و سپس در محل
مورد نظر خود بچسبانید. راه حل دیگر استفاده از همین فایل، و حذف شکلهای موجود در آن و سپس
رسم شکل مورد نظر خودتان است.
\href{http://bayanbox.ir/id/8287021624570840892?info}{دریافت فایل نمونه PDF}
\section{منابع آموزشی و فایل‌های نمونه}
برای فایل‌های آموزشی و فایل نمونه، به لینک زیر مراجعه کنید:

\url{http://cs.yazd.ac.ir/farshi/LaTeX/LaTeX.html}


\include{chapter3}
\chapter{ نمونه‌ای از یک فصل}
\section{مدل‌های حرکت}\label{sec-model-motion}
بسته به کاربرد، حرکت مجموعه نقاط در فضای دلخواه از راه‌های مختلف نمایش داده می‌شود. حرکت می‌تواند به صورت صریح با توابع چندجمله‌ای، به صورت ضمنی با معادلات دیفرانسیلی، یا به صورت آماری با مدل‌های احتمالی نمایش داده شود. همان‌طورکه بعداً دیده خواهد شد، در مسائل وابسته به حرکت که نقاط مسئله به‌طور پیوسته در حال حرکت هستند، لازم است تا وضعیت نقاط در هر زمان با مجموعه‌ای از شرط‌های جبری مشخص شود. علاوه بر این نیاز است تا مسیر حرکت نقاط به‌گونه‌ای تعریف شود که تعداد دفعاتی که یک شرط جبری ممکن است نامعتبر ‌شود ثابت باشد. بدین منظور در این پایان‌نامه، فرض می‌شود که مسیر حرکت نقاط در صفحه به صورت  \textbf{حرکت‌های شبه‌جبری}\LTRfootnote{Pseudo algebraic motions} است. برای تعریف این نوع حرکت‌ها لازم است توابع شبه‌جبری تعریف شوند، که از تعریف زانگ\LTRfootnote{Zhang} مطرح شده در
 %\cite{z-kmps-00} 
 استفاده می‌شود. 

\begin{definition}\label{def-psdo-algb-func}
توابع پیوسته‌ی یک متغیره‌ی $f_{1}(x), f_{2}(x), ... , f_{m}(x)$، توابع شبه‌جبری از درجه‌ی حداکثر $s$ هستند، هرگاه برای هر تابع چندجمله‌ای $m$ متغیره‌ی $g$  از درجه‌ی $s_{1}$،  تابع  $$h(x) = g(f_{1}(x), f_{2}(x), ... , f_{m}(x))$$  متحد صفر باشد یا حداکثر $s \times s_{1}$  ریشه داشته باشد.
 \end{definition}
  برای نمونه، توابع چندجمله‌ای یا منطقی با درجه‌ی ثابت توابع شبه‌جبری هستند. یک مجموعه از نقاط دارای حرکت‌های شبه‌جبری از زمان هستند هرگاه مسیر حرکت آن‌ها با توابع شبه‌جبری از زمان توصیف شود. در ادامه، تعریف \ref{def-psdo-algb-func} با شرح یک مثال توضیح داده می‌شود. 
 
مسئله‌ی یک بعدی زیر که ترتیب $n$ نقطه‌ی در حال حرکت روی محور $x$ ها را گزارش می‌کند، درنظر بگیرید. هر نقطه‌ی $p_{i}$  دارای مسیر حرکت پیوسته‌ای از زمان است که با $f_{i}(t)$  نشان داده می‌شود و $t$ به زمان اشاره می‌کند. مقدار هر نقطه‌ی $p_{i}$  در زمان $t$، یعنی مولفه‌ی $x$ نقطه، با $v_{i}(t)$ نشان داده می‌شود. فرض کنید مجموعه توابع حرکت نقاط، یعنی $f_{1}(t), f_{2}(t), ... ، f_{n}(t)$،  توابع شبه‌جبری از درجه‌ی $s$ باشند و شرط‌های جبری برای تعیین موقعیت نقاط متحرک از نوع مقایسه تعریف شوند. اگر برای نشان دادن ترتیب هر دو نقطه‌ی متوالی $p_{i}$ و $p_{j}$ روی محور $x$ در زمان $t$، به‌طوری‌که $ v_{j}(t) > v_{i}(t)$ باشد، از شرط جبری $ v_{j}(t) - v_{i}(t) > 0$  استفاده شود، آنگاه با توجه به تعریف \mbox{\ref{def-psdo-algb-func}}، می‌توان تابع دو متغیره‌ی خطی $g(f_{j}, f_{i})(t) =  f_{j}(t) - f_{i}(t)$ (که طبق تعریف دارای $s_{1} = 1$ و $m = 2$ است)، را به عنوان تابع $h(t)$ درنظر گرفت و به این نتیجه رسید که تابع $h(t) = g(f_{j}, f_{i})(t)$ یا متحد صفر است یا دارای حداکثر $s \times 1$  ریشه است. این بدان معنی است که شرط جبری $ v_{j}(t) - v_{i}(t) > 0$ با گذشت زمان حداکثر به تعداد $s$ بار صفر می‌شود و متعاقباً تغییر علامت می‌دهد (تغییر ترتیب نقاط).
 
در تحلیل مسائلی که بعداً مطرح خواهند شد، بسیار اهمیت دارد که تعداد دفعاتی که شرط جبری مربوط به یک نقطه با گذشت زمان صفر می‌شود ثابت باشد. از آن‌جایی که در بیش‌تر مسائل شرط‌های جبری که برای تعیین وضعیت نقاط معرفی می‌شوند، چندجمله‌ای‌هایی از درجه‌ی کم، تعریف شده روی یک تعداد ثابت از نقاط هستند، فرض شبه‌جبری بودن حرکت نقاط کافی است تا برقراری این موضوع را تضمین کند. برای نمونه، در مثال بالا شرط‌های جبری (مقایسه‌ی دو مقدار) از درجه‌ی یک و تعریف شده روی دو نقطه هستند، بنابراین با فرض شبه‌جبری بودن حرکت نقاط، شرط جبری مربوط به یک نقطه همان تابع $g$ مطرح شده در تعریف \ref{def-psdo-algb-func} می‌شود که حداکثر به تعداد $s \times 1$  بار صفر خواهد شد ( $s$ یک عدد ثابت که درجه‌ی توابع شبه‌جبری حرکت را نشان می‌دهد). 

 \section{دنباله‌ی داونپورت-شینزل}\label{sec-DS-sequence}
 \begin{definition}
\label{def-DS-sequ}
یک $(n, s)$ دنباله‌ی داونپورت-شینزل\LTRfootnote{Davenport-Schinzel sequence}، که $n$ و $s$ اعداد صحیح مثبت هستند، یک دنباله‌ی ساخته شده از $n$ نماد با این خواص است که هیچ دو نماد مجاور در دنباله یکسان نیستند و این‌که برای هر دو نماد مجزای $a$ و $b$  حداکثر $s$  تناوب از آن‌ها در دنباله وجود دارد.
 \end{definition}
در تعریف بالا، منظور از تناوب $a$ و $b$ این است که نماد $b$ بعد از نماد $a$ و نماد $a$ بعد از نماد $b$ در دنباله ظاهر شود، ولی نه الزاماً مجاور به هم (کنار هم). مثلاً دنباله‌ی زیر، تشکیل شده از نمادهای $a, b, c, d$ را در نظر بگیرید: $$\underline{a}c\underline{b}dbc\underline{a}cd
\underline{b}\underline{a}dcdcadc\underline{b}$$ تعداد تناوب‌های $a$ و $b$ در دنباله، یعنی مجموع تعداد دفعاتی که $b$ بعد از $a$  و $a$ بعد از $b$ در دنباله ظاهر شده است، برابر با 5 است. با توجه به تعریف $(n, s)$ دنباله‌ی داونپورت-شینزل، می‌توان دریافت که با یک $n$ و $s$ معلوم (داده شده)، بسته به مقدارهای $n$ و $s$،  دنباله‌‌های داونپورت-شینزل متعددی می‌توان یافت، اما همگی دارای طول‌های متناهی هستند؛ زیرا با توجه به تعریف، امکان وجود دو نماد مجاور یکسان در دنباله نیست و نیز تعداد تناوب‌های هر دو نماد مجزا در دنباله به تعداد حداکثر $s$ محدود شده است. مثلاً برای $n = 3$ و $s = 2$ و نمادهای $a, b, c$،  طول  $(3, 2)$ دنباله‌های داونپورت-شینزل ممکن، حداکثر 5 است؛ زیرا هر دنباله با طول 6 یا بیش‌تر متشکل از این نمادها، یا حداقل برای یک جفت نماد مجزا دارای بیش از 2 تناوب خواهد بود یا دو نماد مجاور یکسان خواهد داشت و در نتیجه، شرایط تعریف یک $(3, 2)$ دنباله‌ی داونپورت-شینزل را نخواهد داشت، به عنوان مثال دنباله‌ی $abcbab$ که 3 تناوب از $a$ و $b$ را دارد. بنابراین طول طولانی‌ترین $(n, s)$ دنباله‌ی داونپورت-شینزل قابل تعریف خواهد بود 
و با $\lambda_{s}(n)$ نشان داده می‌شود. در ادامه به بیان اهمیت و کاربرد این دنباله‌ها در تحلیل مسائلی‌ مهم در هندسه‌ی محاسباتی پرداخته می‌شود.

اگر $\rbrace$ $f_{i}$ $ \lbrace$ $\cal F =$
  یک مجموعه از توابع باشد،  \textbf{ پوشش پایینی}\LTRfootnote{Lower envelope}  برای مجموعه $\cal F$  برابر تابع $\min f_{i}(x)$ است که با  $\Gamma(\cal F)$ نشان داده می‌شود.  به طور مشابه  $\max f_{i}(x)$ به عنوان \textbf{پوشش بالایی}\LTRfootnote{Upper envelope}  تعریف می‌شود. پیچیدگی $\Gamma(\cal F)$ نیز برابر با تعداد دفعاتی که تابع موجود در $\Gamma(\cal F)$ عوض می‌شود، یعنی تعداد نقاط شکست $\Gamma(\cal F)$ تعریف می‌شود. 

\begin{figure}[h]
\begin{center}
\includegraphics[width=0.5\textwidth]{DSsequence}
\end{center}
\caption{پوشش پایینی یک مجموعه از توابع متناظر با دنباله‌ای از نمادها}
\label{fig-DSsequence}
\end{figure}
اگر $\cal F$ مجموعه‌ای از $n$ تابع چندجمله‌ای با درجه‌ی $s$ باشد. با توجه به این‌که هر دو تابع چندجمله‌ای از درجه‌ی $s$ حداکثر در $s$ نقطه با یکدیگر برخورد می‌کنند (این بدان معنی است که حداکثر $s$ تناوب از هر دو تابع مجزا از $\cal F$  در $\Gamma(\cal F)$ وجود خواهد داشت) و نیز وجود دو تابع یکسان مجاور به هم در $\Gamma(\cal F)$ هم امکان ندارد، به‌راحتی می‌توان نتیجه گرفت که $\Gamma(\cal F)$ متناظر با یک $(n, s)$ دنباله‌ی داونپورت-شینزل است و پیچیدگی آن نیز، برابر با طول دنباله‌ی داونپورت-شینزل متناظر با آن خواهد شد. شکل \ref{fig-DSsequence} را ببینید. پس می‌توان گفت که پیچیدگی $\Gamma(\cal F)$ از مرتبه‌ی $O(\lambda_{s}(n))$ است. تمام نتایج به‌طور مشابه برای پیچیدگی پوشش بالایی نیز برقرار است. اگر دامنه‌ی تعریف $\Gamma(\cal F)$ در نظر گرفته شود و هر تابع چندجمله‌ای از $\cal F$  روی قسمتی از این دامنه (یک بازه مشخص از دامنه) تعریف شود، یعنی نمودار هر تابع تکه‌ای از نمودار آن تابع با دامنه نامحدود باشد، آنگاه پیچیدگی $ \Gamma(\cal F)$ برابر $O(\lambda_{s+2}(n))$ می‌شود \cite{sp-dssga-95}.
برای یک ثابت $s\geq 3$، $\lambda_{s}(n)$ یک تابع ابرخطی است اما خیلی آهسته رشد می‌کند. در ادامه قضیه‌ای بیان می‌شود که فرمول‌های مربوط به محاسبه  $\lambda_{s}(n)$ را بیان می‌کند. تابع $\alpha(n)$  به معکوس تابع آکرمان\LTRfootnote{Ackermann function} اشاره می‌کند \cite{sp-dssga-95}.


از آن جایی که $\alpha(n)$ تابعی است که بی‌نهایت آهسته رشد می‌کند (تقریبا برای مقادیر عملی و منطقاً بزرگ $n$ مقدار ثابت است)، بنابراین برای یک مقدار ثابت $s$، $\lambda_{s}(n)$ تقریبا خطی از $n$  است.   
%%%%%%%%%%%%%%%%%%%%%
 %%%%%%%%%%%%%%%%%%%
\section{پوشش‌های هندسی روی مجموعه نقاط}\label{sec-t-spann}
\subsection{شبکه‌های هندسی}
مجموعه‌ی $P$ شامل $n$ نقطه در فضای $\mathbb{R}^d$ را درنظر بگیرید، یک \textbf{شبکه‌ی متصل‌کننده‌ی نقاط} $P$\LTRfootnote{A network connecting the points of P}، یک گراف ${\cal G} = (P, E)$ با مجموعه رأس‌های $P$ و مجموعه یال‌های $E \subseteq P\times P$  است، به‌طوری‌که هر دو نقطه‌ی $p, q \in P$ با یک مسیر در $\cal G$ به‌هم متصل می‌شوند. یک \textbf{شبکه‌ی هندسی}\LTRfootnote{Geometric network} یا یک \textbf{گراف اقلیدسی}\LTRfootnote{Euclidean graph}، یک گراف وزن‌دار ${\cal G}$ است که رأس‌ها متناظر با نقاط در فضای اقلیدسی و وزن روی یال‌ها متناظر با فاصله‌‌ی اقلیدسی بین نقاط انتهایی آن یال است. شبکه‌های هندسی در واقع تعداد زیادی از شبکه‌های حقیقی موجود، مانند شبکه راه‌ها، شبکه مخابرات و غیره را مدل می‌کنند.
%\nocite{ns-gsn-07}

برای طراحی یک شبکه برای مجموعه‌ی مشخصی از نقاط، چندین معیار کیفی در نظر گرفته می‌شود. در زیر تعدادی از مهم‌ترین معیارهای کیفی برای ارزیابی شبکه‌های هندسی بیان شده است.
 \begin{enumerate}
\item
\textbf{اندازه}\LTRfootnote{Size}، به‌عنوان تعداد یال‌های شبکه تعریف می‌شود. در حالت کلی ترجیح داده می‌شود که شبکه‌ها تا جای ممکن اندازه‌ی کوچکی (خطی از تعداد نقاط) داشته باشند. 
\item
\textbf{وزن}\LTRfootnote{Weight}، به‌عنوان مجموع وزن یال‌های شبکه تعریف می‌شود. ازآن‌جایی‌که هر شبکه باید تمام نقاط را به‌هم وصل کند، درنتیجه وزن آن از پایین با وزن درخت پوشای کمینه کران‌دار می‌شود. وزن یک معیار خوب برای سنجش هزینه‌ی ساخت شبکه است. بنابراین، اغلب شبکه‌هایی با وزن کم مورد نظر هستند.
\item
\textbf{ضریب کشش}\LTRfootnote{Stretch factor}یا \textbf{تاخیر}\LTRfootnote{Dilation}  برای دو نقطه‌ی داده شده، برابر با نسبت کوتاه‌ترین مسیر (مسیر با وزن مینیمم) بین دو نقطه در شبکه، به فاصله‌ی آن دو نقطه بر اساس متر تعریف شده برای آن شبکه است (مثلاً این فاصله در متر اقلیدسی خط مستقیم متصل کننده‌ی دو نقطه است). ضریب کشش یک شبکه به‌عنوان بیش‌ترین ضریب کشش برای هر جفت از نقاط مجزا در شبکه تعریف می‌شود. در بسیاری از موارد، نیاز است که ضریب کشش شبکه با یک ثابت کوچک محدود شود (که حداقل باید یک باشد). شبکه‌ها با ضریب کشش حداکثر $t$، $t$-پوشش‌ها\LTRfootnote{t-Spanners} نامیده می‌شوند.
\item
\textbf{درجه}\LTRfootnote{Degree}،  بیش‌ترین تعداد یال‌های مجاور به هر نقطه در شبکه می‌باشد و اغلب نیاز است که با یک ثابت کوچک محدود شود. درجه‌ی محدود یک شبکه، به اندازه‌ی کوچک آن شبکه اشاره می‌کند، اما برعکس این مطلب لزوماً درست نیست.
\end{enumerate}
در حالت کلی، در زمان طراحی یک شبکه، آن‌چه که اهمیت زیادی دارد، اعمال ترکیبی از این معیارهای کیفی بر روی شبکه است و در زمان تحلیل شبکه نیز ویژگی‌های شبکه، نسبت به این معیارها سنجیده می‌شود. یکی از مسائل مهم در این زمینه، مطالعه‌ی شبکه‌هایی با ضریب کشش کم است (در ترکیب با دیگر ویژگی‌ها). ازجمله، در بسیاری از کاربردها مانند شبکه‌ی راه‌ها لازم است یک ارتباط سریع (مستقیم) بین هر جفت از نقاط در $P$ برقرار باشد (یعنی شبکه یک گراف کامل باشد) ولی این نیاز به خاطر هزینه‌های بالا، قابل اجرا شدن نیست. بنابراین نیاز به مطالعه‌ی شبکه‌هایی با ضریب کشش کم، منجر به شکل‌گیری مفهوم پوشش‌های هندسی می‌شود. این پوشش‌ها در واقع یک ساختار برای شبکه‌ها، زمانی که ارتباطات کوتاه بین نقاط اهمیت دارند را فراهم می‌کنند.
%%%%%%%%%%%%%%%%%%%%%%%%%%%%%%%%
\subsection{$t$-پوشش‌های هندسی}
 \begin{definition}
\label{def-t-spanner}
مجموعه‌ی $P$ شامل $n$ نقطه در فضای $\mathbb{R}^d$ و $t \geq 1$ را یک عدد حقیقی درنظر بگیرید. یک $t$-\textbf{پوشش}\LTRfootnote{t-spanner} برای $P$، یک گراف بدون ‌جهت $\cal G$ با مجموعه رأس‌های $P$  است، به‌طوری‌که کوتاه‌ترین مسیر بین هر دو نقطه‌ی $p$ و $q$ از $P$،  در $\cal G$ که با نماد $d_{\cal G}(p, q)$ نشان داده می‌شود، این شرط را داشته باشد:
$$d_{\cal G}(p, q) \leq t \cdot ||pq||.$$ هر مسیری که این شرط را برآورده سازد یک $t$-\textbf{مسیر}\LTRfootnote{t-path} بین $p$ و $q$ نامیده می‌شود.
\end{definition}
 برای هر عدد حقیقی $t^{'}$ که $t^{'} > t$ است، اگر $\cal G$ یک $t$-پوشش برای مجموعه نقاط $P$  باشد، بدیهی است که $\cal G$ یک $t^{'}$-پوشش نیز برای $P$ است. این، منجر به تعریف زیر می‌شود:
 \begin{definition}[\textbf{ضریب کشش}]
\label{def-strtch-factor}
مجموعه‌ی $P$ شامل $n$ نقطه در فضای $\mathbb{R}^d$ و $\cal G$ را یک گراف اقلیدسی با مجموعه رأس‌های $P$ درنظر بگیرید. ضریب کشش $\cal G$، کوچک‌ترین عدد حقیقی $t$ است به‌طوری‌که $\cal G$ یک $t$-پوشش از $P$ باشد.
\end{definition}
گراف کامل یک 1-پوشش است، اما تعداد مربعی یال دارد. درحقیقت، اگر فرض شود که هیچ سه نقطه‌ای از $P$  روی یک خط قرار ندارند، سپس گراف کامل تنها 1-پوشش برای $P$ است. بنابراین، در حالت کلی $t$-پوشش‌ها برای $t$ های بزرگ‌تر از یک بررسی می‌شوند. اما از آن‌جایی که در بسیاری از کاربردها، پوشش‌هایی با ارتباطات سریع (ضریب کشش کم) مورد نیاز هستند. بنابراین پوشش‌هایی با ضریب کشش نزدیک به یک، مورد مطالعه قرار می‌گیرند، یعنی $t = 1 + \varepsilon$،  برای مقادیر $\varepsilon$ مثبت و کوچک ($0 < \varepsilon < 1$). به این ترتیب، می‌توان با انتخاب مقادیر خیلی کوچک برای $\varepsilon$، مقدار $t$ را به یک نزدیک کرد، یعنی $t$-پوشش را به یک $1$-پوشش (گراف کامل) نزدیک کرد.
 %%%%%%%%%%%%%%%%%%%%%%%%%

 \section{ یک پوشش هندسی وابسته به حرکت در صفحه}
در این فصل یک $1 + \varepsilon$-پوشش ساده با اندازه‌ی خطی، برای یک مجموعه از $n$ نقطه در صفحه معرفی می‌شود. این پوشش زمانی‌که نقاط آن حرکت می‌کنند، می‌تواند به‌صورتی کارا نگهداری شود. فرض می‌شود که مسیر حرکت نقاط با توابع شبه‌جبری از درجه‌ی حداکثر $s$ توصیف می‌شوند (بخش \ref{sec-model-motion} را ببینید). لازم به ذکر است که مطالب این فصل بر اساس مقاله‌ی آبام\LTRfootnote{Abam} و همکارانش \cite{abg-seks-10} تنظیم شده است. \section{کارهای مرتبط}
یک $1 + \varepsilon$-پوشش را درنظر بگیرید. برای جزئیات بیش‌تر در مورد پوشش‌ها می‌توانید بخش \ref{sec-t-spann} را ببینید. هنگامی که نقاط پوشش دارای حرکت پیوسته‌ای از زمان باشند، پوشش به‌طور پیوسته از زمان تغییر می‌کند، اما تنها در لحظه‌های مشخصی نیاز به به‌روزرسانی خواهد داشت، تا همچنان در تمام زمان‌ها یک $1 + \varepsilon$-پوشش باقی بماند. در فصل 2 بیان شد که این لحظات رویداد نامیده می‌شوند. از آن‌جایی که کارایی و پاسخ‌گویی، از مهم‌ترین معیارها برای ارزیابی کیفیت ساختارهای وابسته به حرکت هستند، هدف، طراحی پوششی است که تعداد رویدادها و نیز زمان پاسخ‌گویی آن، کم باشد. همان‌طورکه در فصل 2 بیان شد، در حالت ایده‌ال، تعداد رویدادها باید نزدیک به کم‌ترین تعداد رویدادهایی که برای نگهداری ساختار لازم است باشد. برای پوشش‌ها، این بدین معنی است که تعداد رویدادها باید $O(n^{2})$ باشد؛ زیرا مجموعه‌های $n$ تایی از نقاط که به‌صورت خطی در صفحه حرکت می‌کنند وجود دارند، که برای هر $1 + \varepsilon$-پوشش باید $\Omega(n^{2}/(1 + \varepsilon)^{2})$ رویداد پردازش شود.
% \cite{ggn-dsa-06}. 
زمان پاسخ‌گویی نیز در حالت ایده‌ال باید چندلگاریتمی باشد.

\subsection{تحلیل ضریب کشش و اندازه‌ی پوشش}  \label{subsec-span-ratio}
در این زیربخش ثابت خواهد ‌شد که اگر زاویه‌های $\phi$ و $\varphi$ به‌گونه‌ای انتخاب شوند که شرط‌های 
%\begin{latin}
\begin{equation}\label{eq-rd-first}
(\cos \phi - \sin \phi) \geq 1/(1 + \varepsilon) 
\end{equation} 
%\end{latin}
و
 \begin{equation}\label{eq-rd-sec}(\sin(\phi/2)/\sin(\varphi/2)) \geq 1 + 2/\varepsilon \end{equation}
 برقرار باشند، آنگاه ${\cal DDS}(P)$ یک $1 + \varepsilon$-پوشش برای مجموعه نقاط $P$ خواهد بود. این شرط‌ها می‌توانند با تنظیم زوایای $\phi$ و $\varphi$ به‌صورت  $ \phi = \arcsin \frac{\varepsilon}{2(1+\varepsilon)}$ و $ \varphi = 2\arcsin \frac{\varepsilon^{2}}{4(1+\varepsilon)(2+\varepsilon)}$  به‌دست آیند. همواره فرض می‌شود که $\varphi <\phi$.  
\begin{observation}
\label{obsr-phi}
اگر $ \phi = \arcsin \frac{\varepsilon}{2(1+\varepsilon)}$ برقرار باشد آنگاه $0 < \phi < \dfrac{\pi}{6}$.
\end{observation}
\begin{proof}
از آن‌جایی‌که، $\dfrac{\varepsilon}{2(1+\varepsilon)} < \dfrac{1}{2}$  و تابع $\arcsin$ در بازه $ (0, \pi/2)$ یک تابع صعودی است، بنابراین،
 $$0 < \phi = \arcsin \frac{\varepsilon}{2(1+\varepsilon)} < \arcsin \frac{1}{2} = \dfrac{\pi}{6}.$$
\end{proof}
\begin{observation}
\label{obsr-angles}
با انتخاب $ \phi = \arcsin \frac{\varepsilon}{2(1+\varepsilon)}$ و $\varphi = 2\arcsin \frac{\varepsilon^{2}}{4(1+\varepsilon)(2+\varepsilon)}$ دو شرط \ref{eq-rd-first} و \ref{eq-rd-sec} برقرار است.
\end{observation}
مشتق\index{مشتق} تابع $f$، تابعی است که با علامت $f'$ نشان داده می‌شود و مقدار آن در هر عدد $x$ واقع در دامنهٔ $f$
به صورت 
\begin{align}\label{bderi1}
f'(x)=\lim_{\Delta x\rightarrow 0}\dfrac{f(x+\Delta x) - f(x)}{\Delta x}
\end{align}
\ysymbol{$f'(x)$}{مشتق تابع $f$ در نقطهٔ $x$}
تعریف می‌شود؛ به شرطی که حد فوق وجود داشته باشد.

اگر $f$ روی بازهٔ $[a,b]$ هموار باشد، طول منحنی $y=f(x)$ از $a$ تا $b$ برابر است با
\begin{align}\label{10eq6}
L=\int_a^b \sqrt{1+\Big(\dfrac{dy}{dx}\Big)^2}dx=\int_a^b \sqrt{1+(f'(x))^2}dx.
\ysymbol{$L$}{طول منحنی}
\end{align}

اگر تابع $f$ در $x_1$ تعریف شده باشد، آنگاه مشتق راست $f$ در $x_1$ با $f'_{+}(x_1)$ 
\ysymbol{$f'_{+}(x)$}{مشتق راست  $f$ در نقطهٔ $x$}
نشان داده می‌شود و به صورت 
\begin{align}
f'_{+}(x_1)=\lim_{\Delta x\rightarrow 0^{+}}\dfrac{f(x_1+\Delta x) - f(x_1)}{\Delta x}
\end{align}
و یا به عبارت دیگر،
تعریف می‌شود؛ به شرطی که این حدود موجود باشند.

\begin{example}
نمونه مثال
\end{example}
\begin{proposition}
نمونه گزاره
\end{proposition}
\begin{corollary}
نمونه نتیجه
\end{corollary}

\begin{remark}
نمونه ملاحظه
\end{remark}



\include{chapter5}
\appendix
\include{appendix1}
\include{dicfa2en}
\include{dicen2fa}
\linespread{1.7}\selectfont
\bibliographystyle{acm-fa}
\bibliography{MyReferences}
\printindex
% در این فایل، عنوان پایان‌نامه، مشخصات خود و چکیده پایان‌نامه را به انگلیسی وارد کنید.
%%%%%%%%%%%%%%%%%%%%%%%%%%%%%%%%%%%%
\begin{latin}
\latinfaculty{Faculty of Sciences}
\latindepartment{Department of Mathematical Sciences}
\latinfield{Communications (System)}
\latintitle{English Title of the Thesis}
\firstlatinsupervisor{First Supervisor}
\secondlatinsupervisor{Second Supervisor}
\firstlatinadvisor{First Advisor}
\secondlatinadvisor{Second Advisor}
\latinname{Nasser}
\latinsurname{Dehghan}
\latinthesisdate{January 2015}

\enabstract{
\subsection*{Abstract}
\thispagestyle{empty}
\baselineskip=1cm

In this thesis, we study $1 + \varepsilon$-spanners for a set of $n$ points in the plane and in
 d-dimensional Euclidean space that can be maintained efficiently as the points move. The
 kinetic spanner in the plane has size $O(n/\varepsilon^{2})$. Assuming the trajectories of 
the points can be described by polynomials whose degrees are at most $s$,
the number of events is $O(n^{2}\beta(n))$ ($\beta(n)$ grows slower than logarithmic functions),
 and at each event the spanner can be updated in $O(1)$ time. The kinetic spanner in $\mathbb{R}^{d}$ has
 size $O(n/\varepsilon^{d-1})$ and maximum degree $O(\log^{d} n)$. Assuming that the
trajectories of the points can be described by bounded-degree polynomials, the number of
 events is  $O(n^{2}/\varepsilon^{d-1})$, and using a supporting data
structure of size $O((n/\varepsilon^{d-1})\log^{d} n)$, we can handle events in
 time $O(\log^{d+1} n)$. Moreover, the spanner can be updated in time $O(\log n/\varepsilon^{d-1})$ if
 the flight plan of a point changes. These spanners are the first kinetic spanners whose performance does not depend on
the spread of the point set.
}
\latinyazdtitle
\end{latin}

\end{document}