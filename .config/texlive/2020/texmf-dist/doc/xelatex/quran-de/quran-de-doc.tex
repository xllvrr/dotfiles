\documentclass[a4paper]{ltxdoc}

\usepackage{holtxdoc}
\usepackage{url}
\usepackage{booktabs}
\usepackage{listings}
\usepackage{tikz}
\hypersetup{%
  plainpages=false,%
  bookmarksnumbered,%
  pdftitle={The quran-de Package},%
  pdfkeywords={quran-de, surah, ayah, juz, hizb, ruku, manzil},%
  pdfauthor={Seiied-Mohammad-Javad Razavian},%
  baseurl={http://mirrors.ctan.org/macros/xetex/latex/quran-de/doc/quran-de-doc.pdf},%
}
\usepackage[all]{quran-de}
% because of definition of \XeTeX and \XeLaTeX symbols in bidi, 
% I should undef these macro that are also defined in holtxdoc package. 
\makeatletter
\bidi@undef\XeTeX
\bidi@undef\XeLaTeX
\makeatother
\usepackage{arabxetex}
\makeatletter
\bidi@BeforeBeginEnvironment{declcs}{\vspace*{-4mm}}
\makeatother

%%%%%%%%%%%%%%%%%%%%%%%%%%%%%%%%%%%%%%%%%%%%%

\def\boxcs#1{\leftline{\fbox{\mcs{#1}}}}
\def\mcs#1{\lr{\texttt{\textbackslash{}#1}}}
\def\tt#1{\lr{\texttt{#1}}}
\lstdefinestyle{BashInputStyle}{
  language=bash,
  basicstyle=\footnotesize\sffamily,
  frame=tb,
  columns=fullflexible,
  backgroundcolor=\color{gray!10},
}

% Define box and box title style
\tikzstyle{mybox} = [draw=black, fill=gray!20, very thick,
    rectangle, rounded corners, inner sep=10pt, inner ysep=20pt]
\tikzstyle{fancytitle} =[fill=gray, text=white]

\def\mx#1#2{\mybox{#1}{#2}{.45\textwidth}}
\def\mxf#1#2{\mybox{#1}{#2}{\textwidth}}

\def\mybox#1#2#3{
\begin{tikzpicture}
\node [mybox] (box){%
    \begin{minipage}[t]{#3}
        #2
    \end{minipage}
};
\node[fancytitle, anchor=west, right=10pt] at (box.north west) {\small \cs{#1}};
\node (hole) [anchor=north east, left=5pt ] at (box.north east) { \tikz\fill[very thick,white] (0,0) circle (12pt); };
\node[ ] at (hole.center) {\includegraphics[width=.05\textwidth]{quran.png}};
\end{tikzpicture}%
}

%%%%%%%%%%%%%%%%%%%%%%%%%%%%%%%%%%%%%%%%%%%%%
\title{\includegraphics[scale=.3]{quran.png}\\
The \xpackage{quran-de} Package\footnote{To one having devoted his life to the holy Quran}\\
}
\author{Seiied Mohammad Javad Razavian\\\xemail{javadr@gmail.com}}

\date{\qurandedate,  Version \qurandeversion\footnote{First release: December 1st, 2018}}

\begin{document}
\maketitle

\tableofcontents
%\newpage

\section{Introduction}
The \xpackage{quran} package provides several macros for
typesetting the whole or any parts of the holy Quran based on its popular divisions. That package also develops 
commands for some translations of the holy Quran including German, Engilsh, and Persian. 
Some people asked me to include the other translations into the \xpackage{quran} package, 
but because of some technical issues I decided to create new packages for other translations. 
One of these variations is \xpackage{quran-de} including all German translations of the holy Quran 
provided by \url{tanzil.net}, i.e. \emph{``Abu Rida Muhammad ibn Ahmad ibn Rassoul''},
\emph{``A. S. F. Bubenheim and N. Elyas''}, \emph{``Adel Theodor Khoury''}, 
and \emph{``Amir Zaidan.''}


\section{Loading The Package}
The package will be loaded in the ordinary way
\cs{usepackage[option]\{quran-de\}}.
After loading the package, it writes some information about itself to the
terminal and to the log file, too. The information is something like:

\begin{quote}
\begin{lstlisting}[style=BashInputStyle, language=tex, escapechar={|}]
Package: quran-de |\qurandedate| v|\qurandeversion| 
German translations extension to the quran package.
\end{lstlisting}
\end{quote}


\section{Options of The Package}\label{sec:qurantypesetting}
There are four options by the names of \xoption{aburida}, \xoption{bubenheim}, \xoption{khoury}, 
and \xoption{zaidan}.
\marginpar{\xoption{aburida}\smallskip}\marginpar{\xoption{bubenheim}\smallskip} 
\marginpar{\xoption{khoury}}\marginpar{\xoption{zaidan}\smallskip}\marginpar{\xoption{all}\smallskip}
If you pass any combinations of these options, you will be enabled to typeset these translations in a way 
that the \xpackage{quran} package does. 
There is also \xoption{all} option which loads all German translations provided by the \xpackage{quran-de} package.
The package loads \xoption{aburida} option by default.

This package is completely built on top of the \xpackage{quran} package, therefore you can pass every options
defined in that package to the \xpackage{quran-de} package. 

\section{Differences between \xpackage{quran} and \xpackage{quran-de}}
At first glance, \xpackage{quran-de}  offers all functionalities of \xpackage{quran}. Therefore, 
both packages are the same in this manner except one difference. The \xpackage{quran-de} extends the \xpackage{quran} 
by adding three other German translations, namely `bubenheim', `khoury', and `zaidan'. All German translations 
provided by the package are shown in table~\ref{tab:detrans}.

\begin{table}[!htbp]
\centering
\begin{tabular}{|c|l|c|}
    \toprule
    order & \multicolumn{1}{c|}{translator} & option's name \\\midrule
    1 & Abu Rida Muhammad ibn Ahmad ibn Rassoul & aburida \\
    2 & A. S. F. Bubenheim and N. Elyas & bubenheim \\
    3 & Adel Theodor Khoury & khoury \\
    4 & Amir Zaidan & zaidan \\
    \bottomrule
\end{tabular}
    \caption{German translations used in the package and their option's names}
    \label{tab:detrans}
\end{table}

\subsection{How to Change the Current German Translation}
    The \xpackage{quran-de} package uses aburida's translation by default. 
    The following macro could be used to change the default German translation. 
    \begin{declcs}{deSetTrans}\marg{index}
    \end{declcs}    
    The \meta{index} could be an integer or a name; both ``order'' and ``option's name'' 
    shown in Table~\ref{tab:detrans} are appliable. Both \cs{deSetTrans\{4\}} and \cs{deSetTrans\{zaidan\}}, for example,
    have the same effect. 
    
\subsection{How to Get the Name of Current German Translation}
    \begin{declcs}{deGetTrans}
    \end{declcs}    
    The above macro returns the name of current German translation, i.e.  one of ``aburida'', ``bubenheim'', ``khoury'', or ``zaidan''.


\subsection{How to Typeset the German Translation}
    It's completely similiar to the \xpackage{quran} package when \xoption{transde} option is enabled. 
    All the following macros are usable.
    
\begin{multicols}{2}    
    \begin{itemize}
        \item \cs{quransurahde}
        \item \cs{quranayahde}
        \item \cs{quranpagede}
        \item \cs{quranjuzde}
        \item \cs{quranhizbde}
        \item \cs{quranquarterde}
        \item \cs{quranrukude}
        \item \cs{quranmanzilde}
        \item \cs{qurantextde}
    \end{itemize}
\end{multicols}    

    \centerline{\mxf{quransurah*}
    {\begin{arab}\small\quransurah*\end{arab}}}
    \centerline{\mxf{deSetTrans\{aburida\}\textbackslash{}quransurahde*}
    {\setLTR\deSetTrans{aburida}\quransurahde*}}  
    \centerline{\mxf{deSetTrans\{bubenheim\}\textbackslash{}quransurahde*}
    {\setLTR\deSetTrans{bubenheim}\quransurahde*}}  
    \centerline{\mxf{deSetTrans\{khoury\}\textbackslash{}quransurahde*}
    {\setLTR\deSetTrans{khoury}\quransurahde*}}  
    \centerline{\mxf{deSetTrans\{zaidan\}\textbackslash{}quransurahde*}
    {\setLTR\deSetTrans{zaidan}\quransurahde*}}  

\end{document}

