\documentclass[12pt,a4paper]{article}
\input{listingcode}
%
\tikzset{mynode/.style={draw=blue,circle,inner sep=1pt,font=\tiny,anchor=south}
}
\newcommand*\mycirc[1]{%
\begin{tikzpicture}[baseline=(C.base)]
\node[draw,circle,fill=red!10,inner sep=1pt,minimum size=3ex](C) {#1};
\end{tikzpicture}}
\ybannernewstyle{test}{
	color = Aqua}
\begin{document}
\title{\begin{center}
\yBanner[contour=true][test]{\begin{minipage}{6cm}
\LARGE
\textarabic{الحزمة
\texttt{na-position}
}\\
\centerline{\texttt{version1.1}}
\end{minipage}  }
\end{center}}
\author{\mos{\tikz{
 \draw (0,0) node {\resizebox {17cm}{1cm}{\contour{black}{ \color{white}{\textarabic{ \yagding[ark]{76} لرسم جداول الوضعية النسبية بين منحنى و مستقيم
  }}}}}}}}

\date{
\color{red}{ \shadowbox{\yagding[ark]{76}{\large \emph{\sffamily تعديل : لعويجي وليد}}}\\
$14$/$06$/$2018$}}
\maketitle

\vspace{-1cm}
\hrulefill
\thispagestyle{empty}
\begin{center}
\scalebox{2.2}{
\begin{tikzpicture}
 \draw[decorate,decoration={text along path,text={|\color{red!50}|$na-position. package - na-position-$ ||}, text align={fit to path stretching spaces}}] (0.2\paperwidth , -0.11\paperheight ) circle (1.15);
\node (A) at (0.2\paperwidth , -0.11\paperheight )[red!50!black] {بو سليم};
\node (B) at (0.2\paperwidth , -0.09\paperheight )[green!50!black] { محمد};
\node (C) at (0.2\paperwidth , -0.125\paperheight )[green!50!black] { ناعم};
\draw[red!50] (0.2\paperwidth , -0.11\paperheight ) circle (0.8);
\draw[red!50] (0.2\paperwidth , -0.11\paperheight ) circle (1.3);
 \end{tikzpicture}}	
\end{center}
\pagestyle{empty}
\begin{tcolorbox}[breakable,enhanced jigsaw,title={ \tikz{
 \draw (0,0) node {\resizebox {!}{0.5cm}{\contour{black}{ \color{white}{\leter h}}}}}},
 colback=yellow!10!white,colframe=red!50!black,
  interior style={fill overzoom image=goldshade.png,fill image opacity=0.25},
  colbacktitle=yellow!10!white, coltitle=red!56!black,
  enlargepage flexible=\baselineskip,pad at break*=3mm,
  watermark color=blue!35!red!25,
  watermark text={\bfseries\Large \leter h},
  attach boxed title to top center={yshift=-0.25mm-\tcboxedtitleheight/2,yshifttext=2mm-\tcboxedtitleheight/2},
  boxed title style={enhanced,boxrule=0.5mm,
    frame code={ \path[tcb fill frame] ([xshift=-4mm]frame.west) -- (frame.north west)
    -- (frame.north east) -- ([xshift=4mm]frame.east)
    -- (frame.south east) -- (frame.south west) -- cycle; },
    interior code={ \path[tcb fill interior] ([xshift=-2mm]interior.west)
    -- (interior.north west) -- (interior.north east)
    -- ([xshift=2mm]interior.east) -- (interior.south east) -- (interior.south west)
    -- cycle;}  },
  drop fuzzy shadow]
\small \tableofcontents
\end{tcolorbox}
\newpage
\pagestyle{fancy}
\cfoot{}
\lhead{}
 \rhead{\resizebox {!}{0.8cm}{\naams{  
 \thepage}}}
\lfoot{\naams{  
 \thepage}}	
\onecolumn
\setcounter{page}{1}
\pagenumbering{arabic}
\begin{paperbox}{مقدمة}
{\color{green!50!black}\centerline{ \LARGE \bantise 0}}
\begin{arab}
\color{red!40!black}
lqd wfqnA alalh t`AlY l-'in^sA' .hzmT smmyt 
{\color{blue!50!black}\texttt{na-position}} , thtm brsm mxtlf jdAwl alw.d`yyT alnnsbyyT byn mn.hnY w mstqymh almqArb , 'aw byn mn.hnY w mamAssh. \\
'in rsm jdAwl alw.d`yyT alnnsbyyT yt.tlb brmjyAt mxtlfT m_tl \texttt{GeoGebra}
w .gyyrhA w_dlk qd yst.grq wqtA w jhdA m`tbrA , lkn h_dh al.hzmT stxt.sr lk alwqt bqdr kbyr fy rsm tlk aljdAwl w ytm _dlk bmjrrd ktAbT t`lymAt bsy.tT , snf.s.sl fyhA fymA b`d
.\\
al.hzmT 'an^s-'ahA al-'astA_dyn : nA`m m.hmmd w slym bw .\\ 
في هذا الإصدار الجديد لحزمة
{\color{blue!50!black}\texttt{na-position}}
قمنا باختصار بعص الأوامر في أمر واحد و أضفنا الحالات التي يكون التقاطع بين المنحنى والمستقيم في نقطتين و ثلاث نقط .
\end{arab}
\end{paperbox}
\begin{naam cadre}{نبذة عن الحزمة
 \hfill
الحزمة
\texttt{na-position}}
\begin{arab}
al.hzmT \texttt{na-position} t`tmd 'asAsA `lY al.hzmtyn \texttt{tkz-tab} w \texttt{listofitems}
, bm`nY 'Axr lky t`ml h_dh al.hzmT `lyk bt_tbyt al.hzmtyn \texttt{tkz-tab} w \texttt{listofitems}
 `lY \texttt{Live TeX}  , al.hzmT \texttt{na-position}
t`ml m` al.hzmT \texttt{polyglossia}  `nd alm`AljT b-- \texttt{XeLaTeX} 
\end{arab}
 \end{naam cadre}
\newpage
\def\plot{f}
\section{التعليمة \LR{$\rm\backslash posab$}}
\begin{boxlis}
\posab[*\textarabic{الطرف أول
 }*,*$\alpha$*, *\textarabic{الطرف ثاني
 }*](  *\tikz[overlay]\node[xshift=-0.2cm,yshift=0.1cm,draw,inner sep=1pt,circle,fill=red!10]{$1$};*,*\textcircled{$\pm$}*, *\tikz[overlay]\node[xshift=-0.02cm,yshift=0.1cm,draw,inner sep=1pt,circle,fill=red!10]{$2$};* ,*\textcircled{$\pm$}*, *\tikz[overlay]\node[xshift=-0.02cm,yshift=0.1cm,draw,inner sep=1pt,circle,fill=red!10]{$3$};* ){*\textarabic{إحداثيات نقطة التقاطع}*}
\end{boxlis}
\begin{arab}
\yagding[ifsymgeowide]{8}
al.trf al-'awwl y`ny al.trf al-'aysr fy mmjmw`T altt`ryf \\
\yagding[ifsymgeowide]{8}
al.trf al_tAny  y`ny al.trf al-'aymn  fy mmjmw`T altt`ryf \\
\yagding[ifsymgeowide]{8}
$\alpha$ fA.slT nq.tT alttqA.t`\\
\yagding[ifsymgeowide]{8}
'i^sArT \textcircled{$\pm$} almwjwdT byn qwsyn hy 'i^sArT alw.d`yyT .hsb twAjd 
$(C\sb{f})$ \\
bAlnnsbT 'ilY  $(\Delta)$ bm`nY 'i_dA kAn $(C\sb{f})$ fwq  $(\Delta)$ nktb $+$ \\
 w 'i_dA $(C\sb{f})$ t.ht  $(\Delta)$ nktb $-$
\\
\yagding[ifsymgeowide]{8} al.hA.dntyn al-'axyrtyn fy altt`lymT
nktb 'ism nq.tT alttqA.t` w 'i.hdA_tyAthA\\ 
\end{arab}

\begin{center}
\begin{tikzpicture}
\tkzTabInit[espcl=4]
{ /1,
$f(x)-y$ /1.5,
الوضعية
 /2}%
{ , ,}%
\tkzTabLine{ ,  ,  ,  , }
\node[draw,inner sep=2pt,circle,fill=red!20] at (Z11) {$1$} ;
\node[draw,inner sep=2pt,circle,] at (S11) {$\pm$} ;
\node[draw,inner sep=2pt,circle,fill=red!20] at (Z21) {$2$} ;
\node[draw,inner sep=2pt,circle,] at (S21) {$\pm$} ;
\node[draw,inner sep=2pt,circle,fill=red!20] at (Z31) {$3$} ;
\node[,inner sep=2pt,rectangle,fill=green!20] at (F0) {$x$} ;
\node[,inner sep=2pt,fill=green!20] at (F1) {$f(x)-y$} ;
\node[,inner sep=1pt,fill=blue!10,xshift=0.5cm] at (L1) {{\scriptsize \RL{الطرف الأول}}} ;
\node[,inner sep=1pt,fill=blue!10,xshift=0.1cm] at (L2) {$\alpha$} ;
\node[,inner sep=1pt,fill=blue!10!white,xshift=-0.4cm] at (L3) {{\scriptsize \RL{الطرف الثاني}}} ;
\end{tikzpicture}

\end{center}


%\subsection{حالة $D\sb{f}$ من الشكل $[a,b]$ أو $]-\infty;+\infty[$}
\begin{enumerate}[itemsep=0pt,label=\protect\mycirc{\arabic*}]
\item
عندما يكون الطرف الأول من المجال مفتوحا عند عدد حقيقي 
\textcolor{red}{$a$}
 نضع الرمز :
{\textcolor{blue}{ $\rm d$}}
، وعندما يكون 
$-\infty$\\
أو مغلقا نترك مكان الرقم 
\,
\tikz[overlay]\node[,yshift=0.1cm,draw,inner sep=1pt,circle,fill=red!10]{$1$};
\,
فراغا. 
\item
إذا كانت
\textcolor{red}{$\alpha$}
نقطة التقاطع نضع الرمز
{\textcolor{blue}{ $\rm z$}}
، بينما إذا كانت 
\textcolor{red}{$\alpha$}
قيمة ممنوعة 
 نضع الرمز
{\textcolor{blue}{ $\rm d$}}
.
\item
عندما يكون الطرف الثاني من المجال مفتوحا عند عدد حقيقي 
\textcolor{red}{$b$}
 نضع الرمز :
{\textcolor{blue}{ $\rm d$}}
،
وعندما يكون 
$+\infty$\\
أو مغلقا نترك مكان الرقم 
\, 
\tikz[overlay]\node[,yshift=0.1cm,draw,inner sep=1pt,circle,fill=red!10]{$3$};
\,
فراغا. 
\end{enumerate}
\newpage
\subsection{حالة التقاطع بين المنحنى و المستقيم في نقطة}
\begin{myboxe}{{\large مثال أول : $[a,b]$ ,
حيث 
$\alpha=1$}}
\posab[a,1,b](,-,z,+,){A\left( 1 ;f(1) \right) }
\end{myboxe}
\vspace{1.5cm}
\begin{myboxe}{{\large مثال ثاني : $]a,b]$ ,
حيث 
$\alpha=2$}}
\posab[a,2,b](d,-,z,+,){B\left( 2 ;f(2) \right) }
\end{myboxe}
%
%%
\newpage
\begin{myboxe}{{\large مثال ثالث : $]-\infty,b]$ ,
حيث 
$\alpha=2$}}
\posab[-\infty,2,b](,+,z,-,){B\left( 2 ;f(2) \right) }
\end{myboxe}
%
%%
\begin{myboxe}{{\large مثال رابع : $]-\infty,b[$ ,
حيث 
$\alpha=2$}}
\posab[-\infty,2,b](,-,z,+,d){B\left( 2 ;f(2) \right) }
\end{myboxe}
%
%%
\begin{myboxe}{{\large مثال خامس : $]-\infty,+\infty[$ ,
حيث 
$\alpha=2$}}
\posab[-\infty,-3,+\infty](,+,z,+,){C\left( -3 ;f(-3) \right) }
\end{myboxe}
%
%%
\subsection{حالة  عدم التقاطع بين المنحنى و المستقيم}
\begin{myboxe}{{\large مثال أول : $]-\infty,c[\cup ]c,+\infty[$ ,
حيث 
أنه لاتوجد نقطة تقاطع}}
\posab[-\infty,c,+\infty](,-,d,+,){}
\end{myboxe}
\vspace{1cm}
%%
\begin{myboxe}{{\large مثال ثاني : $[a,c[\cup ]c,b]$ ,
حيث 
أنه لاتوجد نقطة تقاطع}}
\posab[a,c,b](,-,d,-,){}
\end{myboxe}
\newpage
\section{التعليمة \LR{$\rm\backslash posad$}}
الشكل العام للتعليمة هو :
\begin{boxlis}
\posad[*\textarabic{الطرف أول
 }*,*$\alpha$*,*$\beta$*,*\textarabic{الطرف الثاني
 }*](  *\tikz[overlay]\node[xshift=-0.2cm,yshift=0.1cm,draw,inner sep=1pt,circle,fill=red!10]{$1$};*,*\textcircled{$\pm$}*, *\tikz[overlay]\node[xshift=-0.02cm,yshift=0.1cm,draw,inner sep=1pt,circle,fill=red!10]{$2$};* ,*\textcircled{$\pm$}*, *\tikz[overlay]\node[xshift=-0.02cm,yshift=0.1cm,draw,inner sep=1pt,circle,fill=red!10]{$3$};* ,*\textcircled{$\pm$}*, *\tikz[overlay]\node[xshift=-0.02cm,yshift=0.1cm,draw,inner sep=1pt,circle,fill=red!10]{$4$};* ){*\textarabic{نقطة التقاطع}*}
\end{boxlis}
\begin{arab}
\yagding[ifsymgeowide]{8}
al.trf al-'awwl y`ny al.trf al-'aysr fy mmjmw`T altt`ryf w hw $a$ fy .hAlT $D\sb{f}=[a,b[\cup ]b,c]$ w hw $-\infty$ \\ fy .hAlT $D\sb{f}=]-\infty;b[\cup ]b,+\infty[$  \\
\yagding[ifsymgeowide]{8}
alqymT alm.h_dwfT hy  $b$ fy .hAlT $D\sb{f}=[a,b[\cup ]b,c]$ 'aw $D\sb{f}=]-\infty;b[\cup ]b,+\infty[$ \\
\yagding[ifsymgeowide]{8}
al.trf al_tAny hw $c$ fy .hAlT $D\sb{f}=[a,b[\cup ]b,c]$ w hw $+\infty$ fy .hAlT 
$D\sb{f}=]-\infty;b[\cup ]b,+\infty[$\\
\yagding[ifsymgeowide]{8}
al-'i^sArAt $\pm$ hy al-'i^sArtyn $+$ 'aw $-$
.hsb w.d` almstqym $(\Delta)$ bAlnnsbT lilmn.hnY $(C\sb{f})$ , .hy_t $+$ `ndmA ykwn 
$(C\sb{f})$ fwq $(\Delta)$ w $-$ `ndmA ykwn 
$(C\sb{f})$ t.ht  $(\Delta)$\\
\end{arab}
\begin{center}
\begin{tikzpicture}
\tkzTabInit[espcl=2.5]
{ /1,
$f(x)-y$ /1.5,
الوضعية
 /2}%
{ , , ,}%
\tkzTabLine{ ,  ,  ,  , , , }
\node[draw,inner sep=2pt,circle,fill=red!20] at (Z11) {$1$} ;
\node[draw,inner sep=2pt,circle,] at (S11) {$\pm$} ;
\node[draw,inner sep=2pt,circle,fill=red!20] at (Z21) {$2$} ;
\node[draw,inner sep=2pt,circle,] at (S21) {$\pm$} ;
\node[draw,inner sep=2pt,circle,fill=red!20] at (Z31) {$3$} ;
\node[draw,inner sep=2pt,circle,] at (S31) {$\pm$} ;
\node[draw,inner sep=2pt,circle,fill=red!20] at (Z41) {$4$} ;
\node[,inner sep=2pt,rectangle,fill=green!20] at (F0) {$x$} ;
\node[,inner sep=2pt,fill=green!20] at (F1) {$f(x)-y$} ;
\node[,inner sep=1pt,fill=blue!10,xshift=0.5cm] at (L1) {{\scriptsize \RL{الطرف الأول}}} ;
\node[,inner sep=2pt,fill=blue!10,xshift=0.1cm] at (L2) {$\alpha$} ;
\node[inner sep=1pt,fill=blue!10,xshift=0.1cm] at (L3) {$\beta$} ;
\node[,inner sep=1pt,fill=blue!10!white,xshift=-0.4cm] at (L4) {{\scriptsize \RL{الطرف الثاني}}} ;
\end{tikzpicture}
\end{center}
\begin{enumerate}[itemsep=0pt,label=\protect\mycirc{\arabic*}]
\item
عندما يكون الطرف الأول من المجال مفتوحا عند عدد حقيقي 
\textcolor{red}{$a$}
 نضع الرمز :
{\textcolor{blue}{ $\rm d$}}
، وعندما يكون 
$-\infty$\\
أو مغلقا نترك مكان الرقم 
\,
\tikz[overlay]\node[,yshift=0.1cm,draw,inner sep=1pt,circle,fill=red!10]{$1$};
\,
فراغا. 
\item
إذا كانت
\textcolor{red}{$\alpha$}
 فاصلة نقطة التقاطع نضع الرمز
{\textcolor{blue}{ $\rm z$}}
، بينما إذا كانت 
\textcolor{red}{$\alpha$}
القيمة المحذوفة
 نضع الرمز
{\textcolor{blue}{ $\rm d$}}
.
\item
إذا كانت
\textcolor{red}{$\beta$}
 فاصلة نقطة التقاطع نضع الرمز
{\textcolor{blue}{ $\rm z$}}
، بينما إذا كانت 
\textcolor{red}{$\alpha$}
القيمة المحذوفة
 نضع الرمز
{\textcolor{blue}{ $\rm d$}}
\item
عندما يكون الطرف الثاني من المجال مفتوحا عند عدد حقيقي 
\textcolor{red}{$b$}
 نضع الرمز :
{\textcolor{blue}{ $\rm d$}}
،
وعندما يكون 
$+\infty$\\
أو مغلقا نترك مكان الرقم 
\, 
\tikz[overlay]\node[,yshift=0.1cm,draw,inner sep=1pt,circle,fill=red!10]{$3$};
\,
فراغا. 
\end{enumerate}

\subsection{حالة $D\sb{f}$ من الشكل $[a,b[\cup ]b,c]$ }
\begin{myboxe}{ \begin{arab}
fA.slT nq.tT alttqA.t` $\alpha$ mn almjAl  $[a,b[$ .
\end{arab}}
\posad[a,\alpha ,b,c](,-,z,+,d,-,){A(\alpha;f(\alpha))}
\end{myboxe}
\begin{boxe}{مثال}
\posad[-5,-2,1,5](,+,z,-,d,-,){A(-2;f(-2))}
\end{boxe}

\subsection{حالة $D\sb{f}$ من الشكل $]-\infty,b[\cup ]b,+\infty[$ }
\begin{myboxe}{ \begin{arab}
 fA.slT nq.tT alttqA.t` $k$ mn almjAl  $]-\infty,b[$ .
\end{arab}}
\posad[-\infty,k ,b,+\infty](,-,z,+,d,-,){A(k;f(k))} 
\end{myboxe}
%
\begin{boxe}{مثال }
\posad[-\infty,-1,2,+\infty](,+,z,-,d,-,){B(-1;f(-1))} 
\end{boxe}
%
\subsection{حالة $D\sb{f}$ من الشكل $[a,b[\cup ]b,c]$ }
\begin{myboxe}{ \begin{arab}
fA.slT nq.tT alttqA.t` $w$ mn almjAl  $]b,c]$.
\end{arab}}
\posad[a,b ,w,c](,+,d,+,z,-,){C(w;f(w))} 
\end{myboxe}
%
\begin{boxe}{مثال }
\posad[-4,-1,3,4](,+,z,-,d,-,){B(3;f(3))}
\end{boxe}
%
\subsection{حالة $D\sb{f}$ من الشكل $]-\infty;b[\cup ]b,+\infty[$ }
%
\begin{myboxe}{ \begin{arab}
  fA.slT nq.tT alttqA.t` $m$ mn almjAl  $]b;+\infty[$ 
\end{arab}}
\posad[-\infty,b ,m,+\infty](,-,d,-,z,+,){C(m;f(m))}
\end{myboxe}
%
\begin{boxe}{مثال }
\posad[-\infty,-1,3,+\infty](,+,d,-,z,-,){B(3;f(3))}
\end{boxe}
\newpage
\begin{boxe}{مثال }
\posad[-\infty,-1,3,6](,+,d,-,z,-,d){B(3;f(3))}
\end{boxe}
%
\subsection{حالة $D\sb{f}$ من الشكل  $]-\infty;a[ \cup ]b;+\infty[$ أو $[c,a[\cup ]b,d]$ حيث لاتوجد نقطة تقاطع}
\begin{arab}
\yagding[ifsymgeowide]{8}
alqymT almamnw`T al-'awlY hy  $a$
(نضع الرمز  
\verb#d#
)
 w alqymT almamnw`T al_tAnyyT hy $b$ 
(نضع الرمز  
\verb#d#
) 
، وبين القيمتين الممنوعتين نضع الرمز
\verb#h#
لأنه لا يوجد عناصر مشتركة بين المجالين
 fy .hAlT 
$D\sb{f}=]-\infty;a[\cup ]b ;+\infty[$ 'aw 
$D\sb{f}=[c,a[\cup ]b,d]$ 
\end{arab}
أي تصبح التعليمة من الشكل :
 \begin{boxlis}
\posad[*\textarabic{الطرف أول}*,*a*,*b*,*\textarabic{الطرف الثاني}*](  *\tikz[overlay]\node[xshift=-0.2cm,yshift=0.1cm,draw,inner sep=1pt,circle,fill=red!10]{$1$};*,*\textcircled{$\pm$}*,*\textbf{d}*,*\textbf{h}*,*\textbf{d}*,*\textcircled{$\pm$}*, *\tikz[overlay]\node[xshift=-0.02cm,yshift=0.1cm,draw,inner sep=1pt,circle,fill=red!10]{$2$};* ){ }
\end{boxlis}
%
\begin{boxe}{مثال أول}
\posad[-\infty,1,2,+\infty](,+,d,h,d,+,){ }
\end{boxe}
\begin{boxe}{مثال ثاني}
\posad[-\infty,a,b,+\infty](,-,d,h,d,-,){ }
\end{boxe}
\begin{boxe}{مثال ثالث}
\posad[-\infty,\alpha,\theta,+\infty](,-,d,h,d,+,){ }
\end{boxe}
\begin{boxe}{مثال رابع}
\posad[-\infty,\dfrac{1}{2},\sqrt{2},+\infty](,+,d,h,d,-,){ }
\end{boxe}
%
\subsection{حالة $D\sb{f}$ من الشكل  $]-\infty;a] \cup [b;+\infty[$ أو $[c,a]\cup [b,d]$ حيث لا توجد نقطة تقاطع}
\begin{arab}
\yagding[ifsymgeowide]{8}
alqymT al-'awlY hy $a$ w alqymT al_tAnyyT hy $b$ 'i_dA kAnt $D\sb{f}=]-\infty;a]\cup [b;+\infty[$ 'aw $D\sb{f}=[c,a]\cup [b,d]$
\end{arab}
أي تصبح التعليمة من الشكل :
 \begin{boxlis}
\posad[*\textarabic{الطرف أول}*,*a*,*b*,*\textarabic{الطرف الثاني}*](  *\tikz[overlay]\node[xshift=-0.2cm,yshift=0.1cm,draw,inner sep=1pt,circle,fill=red!10]{$1$};*,*\textcircled{$\pm$}*,,*\textbf{h}*,,*\textcircled{$\pm$}*, *\tikz[overlay]\node[xshift=-0.02cm,yshift=0.1cm,draw,inner sep=1pt,circle,fill=red!10]{$2$};* ){ }
\end{boxlis}
%
\begin{boxe}{مثال أول}
\posad[-5,1,2,5](,+,,h,,-,){ }
\end{boxe}
\begin{boxe}{مثال ثاني}
\posad[-\infty,-2,3,+\infty](,-,,h,,+,){ }
\end{boxe}
\begin{boxe}{مثال ثالث}
\posad[1,2,4,+\infty](,-,,h,,-,){ }
\end{boxe}
\begin{boxe}{مثال رابع}
\posad[-\infty,2,3,5](,+,,h,,-,){ }
\end{boxe}
\section{التعليمة \LR{$\rm\backslash posba$}}
الشكل العام للتعليمة هو :
\begin{boxlis}
\posad[*\textarabic{الطرف الأول
 }*,*\textarabic{الطرف الثاني
 }*](  *\tikz[overlay]\node[xshift=-0.2cm,yshift=0.1cm,draw,inner sep=1pt,circle,fill=red!10]{$1$};*,*\textcircled{$\pm$}*, *\tikz[overlay]\node[xshift=-0.02cm,yshift=0.1cm,draw,inner sep=1pt,circle,fill=red!10]{$2$}; )
\end{boxlis}
\begin{arab}
\yagding[ifsymgeowide]{8}
al.trf al-'awwl y`ny al.trf al-'aysr fy mmjmw`T altt`ryf \\
\yagding[ifsymgeowide]{8}
al.trf al_tAny  y`ny al.trf al-'aymn  fy mmjmw`T altt`ryf \\
\yagding[ifsymgeowide]{8}
'i^sArT $\pm$ almwjwdT byn qwsyn hy 'i^sArT alw.d`yyT .hsb twAjd 
$(C\sb{f})$ bAlnnsbT 'ilY  $(\Delta)$ bm`nY 'i_dA kAn $(C\sb{f})$ fwq  $(\Delta)$ nktb $+$  w 'i_dA $(C\sb{f})$ t.ht  $(\Delta)$ nktb $-$
\end{arab}
\newpage
\subsection{حالة $D\sb{f}$ من الشكل $[a,b]$ أو $]-\infty;+\infty[$}
\begin{boxe}{مثال أول}
\posba[-\infty ,+ \infty](,+,)
\end{boxe}
\begin{boxe}{مثال ثاني}
\posba[a ,b](,-,)
\end{boxe}
\begin{boxe}{مثال ثالث}
\posba[-3 ,2](,+,)
\end{boxe}
\subsection{حالة $D\sb{f}$ من الشكل $]a,b]$ أو $]a;+\infty[$}
\begin{boxe}{مثال أول}
\posba[a,b](d,+,)
\end{boxe}
\begin{boxe}{مثال ثاني}
\posba[a ,+\infty](d,-,)
\end{boxe}
\begin{boxe}{مثال ثالث}
\posba[2,+\infty](d,+,)
\end{boxe}
\subsection{حالة $D\sb{f}$ من الشكل $[a,b[$ أو $]-\infty;b[$}
\begin{boxe}{مثال أول}
\posba[a,b](,+,d)
\end{boxe}
\begin{boxe}{مثال ثاني}
\posba[-\infty ,b](,-,d)
\end{boxe}
\begin{boxe}{مثال ثالث}
\posba[-\infty ,1](,+,d)
\end{boxe}
\newpage
\section{حالة وجود نقطتي تقاطع بين المنحنى و المستقيم}
\subsection{التعليمة $\rm\setminus\,posat$}
الشكل العام للتعليمة :
 \begin{boxlis}
\posat[*\textarabic{الطرف أول}*,*$\alpha$*,*$\beta$*,*\textarabic{الطرف الثاني}*](  *\tikz[overlay]\node[xshift=-0.2cm,yshift=0.1cm,draw,inner sep=1pt,circle,fill=red!10]{$1$};*,*\textcircled{$\pm$}*,z,*\textcircled{$\pm$}*,z,*\textcircled{$\pm$}*, *\tikz[overlay]\node[xshift=-0.02cm,yshift=0.1cm,draw,inner sep=1pt,circle,fill=red!10]{$2$};* )[*\textarabic{نقطة التقاطع الاولى}*,*\textarabic{نقطة التقاطع الثانية}*]
\end{boxlis}
\yagding[ifsymgeowide]{8}
\textcolor{red}{$\alpha$}
هي فاصلة نقطة التقاطع الأولى 
.\\
\yagding[ifsymgeowide]{8}
\textcolor{red}{$\beta$}
هي فاصلة نقطة التقاطع الثانية .
\subsection{حالة $D\sb{f}$ من الشكل $[a,b]$ أو
$]-\infty;+\infty[$}
\begin{boxe}{مثال أول}
\posat[a,-4,5,b](,-,z,+,z,-,)[A(-4;f(-4)),B(5;f(5))]
\end{boxe}
\newpage
\begin{boxe}{مثال ثاني}
\posat[-\infty,-1,2,+\infty](,+,z,-,z,-,)[A(-1;f(-1)),B(2;f(2))]
\end{boxe}
\section{التعليمة $\rm\setminus\,posaw$}
وتتضمن دراسة باقي حالات تقاطع المنحنى والمستقيم في نقطتين بالاضافة إلى حالة التقاطع في ثلاث نقط.\\
الشكل العام للتعليمة :
 \begin{boxlis}
\posaw[*\textarabic{الطرف أول}*,*$\alpha$*,*$\beta$*,*$\gamma$*,*\textarabic{الطرف الثاني}*](  *\tikz[overlay]\node[xshift=-0.2cm,yshift=0.1cm,draw,inner sep=1pt,circle,fill=red!10]{$1$};*,*\textcircled{$\pm$}*, *\tikz[overlay]\node[xshift=-0.02cm,yshift=0.1cm,draw,inner sep=1pt,circle,fill=red!10]{$2$};* ,*\textcircled{$\pm$}*, *\tikz[overlay]\node[xshift=-0.02cm,yshift=0.1cm,draw,inner sep=1pt,circle,fill=red!10]{$3$};* ,*\textcircled{$\pm$}*, *\tikz[overlay]\node[xshift=-0.02cm,yshift=0.1cm,draw,inner sep=1pt,circle,fill=red!10]{$4$};* )[*A*,*B*,*C*]
\end{boxlis}
\vspace{1cm}
\yagding[ifsymgeowide]{8}
\textcolor{red}{$\alpha$}
هي فاصلة نقطة التقاطع الأولى
\textcolor{red}{$A$} 
.\\
\yagding[ifsymgeowide]{8}
\textcolor{red}{$\beta$}
هي فاصلة نقطة التقاطع الثانية 
\textcolor{red}{$B$} 
.
\\
\yagding[ifsymgeowide]{8}
\textcolor{red}{$\gamma$}
هي فاصلة نقطة التقاطع الثالثة
\textcolor{red}{$C$} 
.
\subsection{حالة $D\sb{f}$ من الشكل $[a,c[\cup ]c,b]$}
\begin{myboxe}{فاصلتي نقطتي التقاطع 
$\alpha$
و
$\beta$
من المجال
$[a,c[$}
\posaw[a,\alpha,\beta,c,b](,+,z,-,z,+,d,+,)[A(\alpha;f(\alpha)),B(\beta;f(\beta)),]
\end{myboxe}
%
\begin{boxe}{مثال }
\posaw[-3,-2,1,2,3](,+,z,-,z,+,d,+,)[A(-2;f(-2)),B(1;f(1)) , ]
\end{boxe}
%
\begin{myboxe}{فاصلتي نقطتي التقاطع 
$\alpha$
و
$\beta$
من المجال
$]c,b]$}
\posaw[a,c,\alpha,\beta,b](,+,d,-,z,+,z,+,)[ , A(\alpha;f(\alpha)),B(\beta;f(\beta))]
\end{myboxe}
%
\begin{boxe}{مثال }
\posaw[-3,-2,1,2,3](,-,d,-,z,-,z,+,)[ , A(1;f(1)),B(2;f(2))]
\end{boxe}
%
\begin{myboxe}{فاصلة نقطة التقاطع  
$\alpha$
تنتمي إلى المجال
$[a,c[$
و
$\beta$
من المجال
$]c,b]$}
\posaw[a,\alpha,c,\beta,b](,-,z,-,d,+,z,-,)[A(\alpha;f(\alpha)),,B(\beta;f(\beta))]
\end{myboxe}
%
\begin{boxe}{مثال }
\posaw[-5,-3,0,3,5](,-,z,-,d,+,z,+,)[ , A(-3;f(-3)),,B(3;f(3))]
\end{boxe}
\subsection{حالة التقاطع في ثلاث نقط بين المنحنى والمستقيم}
\begin{myboxe}{التقاطع في ثلاثة نقاط
$\alpha$ , $\beta$ و
$\gamma$}
\posaw[a,\alpha,\beta,\gamma,b](,-,z,+,z,+,z,-,)[A(\alpha;f(\alpha)),B(\beta;f(\beta)),C(\gamma,f(\gamma))]
\end{myboxe}
%
\begin{boxe}{مثال }
\posaw[-5,-3,0,3,5](,-,z,+,z,-,z,+,)[A(-3;f(-3)),B(0;f(0)),C(3;f(3))]
\end{boxe}
%\newpage
\subsection{تغيير اسم المنحنى واسم المستقيم}
\begin{boxlis}
*\begin{flushright}
\textarabic{نضيف الأمر :
 }
\end{flushright}*
\def\Nplot{*\textarabic{اسم المنحنى
 }*}
\def\Nline{*\textarabic{إسم المستقيم
 }*}
\end{boxlis}
\begin{boxe}{مثال }
\def\Nplot{C\sb{\ell}}
\def\Nline{T}
\posba[1,2](,+,)
\end{boxe}
\subsection{تغيير اسم الدالة}
\begin{boxlis}
*\begin{flushright}
\textarabic{نضيف الأمر :
 }
\end{flushright}*
\def\plot{*\textarabic{اسم الدالة
 }*}
\end{boxlis}
\begin{boxe}{مثال }
\def\plot{g}
\posba[1,2](,-,)
\end{boxe}

\begin{paperbox}{خاتمة}
\color{red!40!black}{
في الأخير ، أقول أن حزمة 
\textbf{na-position}
تساعد على رسم أغلب حالات جداول الوضعية بين منحنى ومستقيمه المقارب أو مماسه\\
وقمنا في هذا الإصدار باضافة بعض الحالات مثل التقاطع في نقطتين و ثلاث نقط.\\
أتمنى أن تكون هذه الحزمة بداية لإنشاء حزمة أشمل تعطي كل الحالات الجداول مهما تغيرت مجموعة التعريف التي يمثلها المنحنى.\\
وما يسعني إلا أن أقدم شكري للأستاذين القديرين
\textbf{ناعم محمد }
و
 \textbf{سليم بو} 
 على فكرة إنشاء هذه الحزمة.\\
تقبلوا تحياتي "الأستاذ : \textbf{لعويجي وليد}".} 
\end{paperbox}
\end{document}