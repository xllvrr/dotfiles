\chapter{A Chapter}\label{chap_first}

\section{First Section}
This section has a theorem with its proof. 

\begin{theorem}\label{thm:earth}
    The Earth is round.
\end{theorem}

\begin{IEEEproof}
    Trivial.
\end{IEEEproof}

It also has a corollary.

\begin{corollary}
    The Earth is not flat.
\end{corollary}

\begin{IEEEproof}
    Follows from \Cref{thm:earth}.
\end{IEEEproof}
Note the use of the cleveref package here. 

We also cite an article~\cite{Blackwell1951} and another article \cite{knuth_notation}. 

\section{Second Section}
Lemmas are useful, too. 

\begin{lemma}\label{lem:roses}
    Roses are red. 
\end{lemma}

\begin{IEEEproof}
    Violets are blue.
\end{IEEEproof}

Sometimes, we need an example.

\begin{example}
    \Cref{tab:roses} demonstrates \Cref{lem:roses}. 
	\begin{table}[t]
			\centering
            \caption[Short caption]{Long caption (now it is)} \label{tab:roses}
	\begin{tabular}{cc}
        \toprule  color & flower \\  \midrule
        red    & rose \\ 
        blue   & violet \\
        yellow & daisy \\  \bottomrule
	\end{tabular} 	
	\end{table}		
\end{example}    

\lipsum[1-5]


% Making sure appendices are aptly named
\namedsubappendices
\begin{subappendices}

\section{First appendix}
An appendix. 

\section{Second appendix}
And another. 

 \end{subappendices}

