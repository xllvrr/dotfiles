% The author of this work is Daphne Parramon-Dhawan.
% This is a translation of the documentation of package font-change.
% This work was released under
% Creative Commons Attribution-Share Alike 3.0 Unported License
% on July 19, 2010. For details visit:
% http://creativecommons.org/licenses/by-sa/3.0/.
%
%%%%%%%%%   Fonts   %%%%%%%%%%
\font\webomints=WebOMintsGd at40pt
\font\titlefont=artemisiarg8a at40pt
\font\titleone=mdputr7t at18pt
\font\titletwo=mdputri7t at18pt
\font\sectionfont=rm-kurierh at18pt
\font\subsectionfont=rm-kurierb at14pt
\font\subsubsectionfont=rm-kurierb at12pt
\font\sansrm=rm-kurierl at10pt
\font\foliofont=mdbchr7t at9pt
\font\amstexfont=cmsy10 at7.5pt
\font\dev=dvng10 scaled \magstep1
\font\letterone=RoyalIn at36pt


%%%%%%%%%%   Packages   %%%%%%%%%%
%%% Eplain
\input eplain % Add Eplain before AmSTeX
\beginpackages
\usepackage{url}
\usepackage{color}
\usepackage{graphicx}
\endpackages
\enablehyperlinks[dvipdfm] % Enables hyperlinks using Eplain
%\hlopts{colormodel=named,color=Black} % Produces links in black color
%% Color
\definecolor{brown}{rgb}{.7,.2,.2}


%%% AmSTeX
\input amstex
\UseAMSsymbols

\catcode`@=12

%%% eps files
\input epsf  % Includes files like pictures and figures in eps format




%%%%%%%%%%  Page characteristics  %%%%%%%%%%
\magnification=1120
\parindent=20pt
\parfillskip=\parindent plus1fil
\everypar{\looseness=-1}
\headline{} \footline{}
\vsize=24truecm
\hoffset-2.9mm
\settabs 20 \columns
\exhyphenpenalty10000 % stops TeX to break words at dashes
\hyphenpenalty200



%%%%%%%%%%   Definitions   %%%%%%%%%%
\def\bs{\bigskip}%
\def\ms{\medskip}%
\def\sk{\smallskip}%
\def\cl{\centerline}%
\def\ii{\noindent}%
\def\pic#1#2#3{\bigskip\cl{\epsfxsize#1\epsfbox{#2.eps}}\par\cl{\eightrm #3}}%
\def\amstex{{\amstexfont  A\kern-.1667em\lower.5ex\hbox{M}\kern-.125em S}-\capstex}%
\def\xetex{X\lower0.5ex\hbox{\kern-0.11em\reflectbox{E}}\kern-0.165em\TeX}%
\def\latex{L\setbox0=\hbox{\sc A}\kern-0.5766\wd0\raise0.41ex\hbox{\sc A}\setbox1=\hbox{T}\kern-.177\wd1\TeX}%
\def\xelatex{X\lower0.5ex\hbox{\kern-0.11em\reflectbox{E}}\kern-0.13em\latex}%
\def\capstex{{\caps t\kern-.122em\lower.38ex\hbox{e}\kern-.11em x}}%
\def\capslatex{{\caps l\setbox0=\hbox{\sevencaps a}\kern-0.5888\wd0{\raise0.33ex\hbox{\sevencaps a}}\setbox1=\hbox{t}\kern-.2\wd1\capstex}}%
\def\capsxetex{{\caps x\lower0.38ex\hbox{\kern-0.1em\reflectbox{e}}\kern-0.13em\capstex}}%
\def\capsxelatex{{\caps x\lower0.38ex\hbox{\kern-0.1em\reflectbox{e}}\kern-0.122em\capslatex}}%
\def\fontss{\baselineskip=2.7ex plus0pt minus0pt%
\spaceskip=0.29em plus0.14em minus0.14em}%
\def\footnote#1{\numberedfootnote{\hskip-7mm\hbox to 20cm{\vtop{\hangindent\parindent\hangafter1\eightrm\fontss #1}}}}%
\def\quote#1{\sk\leftskip10mm{\sl\noindent #1}\sk \leftskip0mm\rightskip0mm}%
\def\emdash{\hbox{\kern0.15em---\kern0.15em}\relax}%
\def\endash{\hbox{\kern0.15em--\kern0.15em}\relax}%


\def\sample{\hrule\vbox{\noindent\vrule\NoBlackBoxes\vbox{\vskip2mm\leftskip7mm\rightskip7mm
\noindent{\bf Formule D'Euler} : La formule d'Euler, aussi connue sous le nom d'{\bf identit\'e d'Euler}, nous dit que
$$e^{\imath x} =\cos(x) + \imath \sin(x), $$
o\`u $\imath$~est {\sl l'unit\'e imaginaire}.

On peut \'etendre la formule d'Euler \`a une s\'erie
$$\eqalign {e^{\imath x}
&= \sum_{n=0}^{\infty} {(\imath x)^n\over{n!}}\cr
&= \sum_{n=0}^{\infty}{(-1)^{n}x^{2n}\over (2n)!} + \imath\sum_1^{\infty}{(-1)^{n-1}x^{2n-1}\over(2n-1)!}\cr
&= \cos(x) + \imath\sin(x).\cr}$$

\bigskip\bigskip
\noindent{\bf Th\'eor\`eme Int\'egral de Cauchy~:} Si $f(z)$ est analytique et ses d\'eriv\'ees partielles continues sur une r\'egion~$R$ simplement connexe,~alors
$$\oint_\gamma f(z)\,dz  = 0$$
pour tout lacet rectifiable~$\gamma$ contenu int\'egralement dans~$R$.\vskip2mm
}\vrule}\hrule\BlackBoxes\bigskip\bigskip
%\input mathcharacters
}%


\def\sampleansi{\hrule\vbox{\noindent\vrule\NoBlackBoxes\vbox{\vskip2mm\leftskip7mm\rightskip7mm
\noindent{\bf Formule D'Euler} : La formule d'Euler, aussi connue sous le nom d'{\bf identité d'Euler}, nous dit que
$$e^{\imath x} =\cos(x) + \imath \sin(x), $$
où $\imath$~est {\sl l'unité imaginaire}.

On peut étendre la formule d'Euler à une série
$$\eqalign {e^{\imath x}
&= \sum_{n=0}^{\infty} {(\imath x)^n\over{n!}}\cr
&= \sum_{n=0}^{\infty}{(-1)^{n}x^{2n}\over (2n)!} + \imath\sum_1^{\infty}{(-1)^{n-1}x^{2n-1}\over(2n-1)!}\cr
&= \cos(x) + \imath\sin(x).\cr}$$

\bigskip\bigskip
\noindent{\bf Théorème Intégral de Cauchy}: Si $f(z)$ est analytique et ses dérivées partielles continues sur une région~$R$ simplement connexe,~alors
$$\oint_\gamma f(z)\,dz  = 0$$
pour tout lacet rectifiable~$\gamma$ contenu intégralement dans~$R$.\vskip2mm
}\vrule}\hrule\BlackBoxes\bigskip\bigskip
%\input mathcharacters
}%


\def\samplebera{\hrule\vbox{\noindent\vrule\NoBlackBoxes\vbox{\vskip2mm\leftskip7mm\rightskip7mm
\noindent{\bf Formule D'Euler} : La formule d'Euler, aussi connue sous le nom d'{\bf identité d'Euler}, nous dit que
$$e^{\imath x} =\cos(x) + \imath \sin(x), $$
où $\imath$~est {\sl l'unité imaginaire}.

On peut étendre la formule d'Euler à une série
$$\eqalign {e^{\imath x}
&= \sum_{n=0}^{\infty} {(\imath x)^n\over{n!}}\cr
&= \sum_{n=0}^{\infty}{(-1)^{n}x^{2n}\over (2n)!} + \imath\sum_1^{\infty}{(-1)^{n-1}x^{2n-1}\over(2n-1)!}\cr
&= \cos(x) + \imath\sin(x).\cr}$$

\bigskip\bigskip
\noindent{\bf Th\'eor\`eme Int\'egral de Cauchy}: Si $f(z)$ est analytique et ses dérivées partielles continues sur une région~$R$ simplement connexe,~alors
$$\oint_\gamma f(z)\,dz  = 0$$
pour tout lacet rectifiable~$\gamma$ contenu intégralement dans~$R$.\vskip2mm
}\vrule}\hrule\BlackBoxes\bigskip\bigskip
%\input mathcharacters
}%

%%%%%%%%%%   Chapter, Section   %%%%%%%%%%
\newcount\sectionno\sectionno=0 % For sections
\newcount\subsectionno\subsectionno=0 % For subsections
\newcount\subsubsectionno\subsubsectionno=0 % For subsubsections
\definecolor{sectioncolor}{rgb}{0.22,0.38,0.62}
% NOTE: No two cross-references should be the same. This will cause chaos. In other words, no two chapters can have the same "label name"

\def\section#1#2{\newpage
{\definexref{#2}{#2}{}}
{\writetocentry{section}{\refs{#2}}}
{\special{pdf: outline 1 << /Title (#2)/F 0 /Dest [@thispage /FitH @ypos ]  >> }}
\subsectionno=0 % For subsections
\centerline{\textcolor{sectioncolor}{\sectionfont\fontss #1}}
\nopagebreak\bigskip\nopagebreak\noindent}

\def\subsection#1#2{
{\definexref{#2}{#2}{}}
{\writetocentry{subsection}{\kern5mm\refs{#2}}}
{\special{pdf: outline 2 << /Title (#2)/F 0 /Dest [@thispage /FitH @ypos ]  >> }}
\bigskip\bigskip\medskip\goodbreak
{\global\advance\subsectionno by 1
 \noindent{\textcolor{sectioncolor}{\subsectionfont\fontss #1}}}
\nopagebreak\medskip\nopagebreak\noindent}

\def\subsubsection#1{
{\definexref{#1}{#1}{}}
{\writetocentry{subsubsection}{\kern5mm\refs{#1}}}
{\special{pdf: outline 2 << /Title (#1)/F 0 /Dest [@thispage /FitH @ypos ]  >> }}
\bigskip\bigskip\medskip\goodbreak
{\global\advance\subsubsectionno by 1
 \noindent{\textcolor{sectioncolor}{\subsubsectionfont\fontss #1}}}
\nopagebreak\medskip\nopagebreak\noindent}
















\input font_charter
\fontss % font change and spacing
\frenchspacing










%%%%%%%%%%  Cover   %%%%%%%%%%
{\special{pdf: outline 1 << /Title (Cover)/F 0 /Dest [@thispage /FitH @ypos ]  >> }}
{\centerline{{\color{brown}\webomints\char'160}\hskip8mm\titlefont\color{sectioncolor} font-change\hskip8mm{\color{brown}\webomints\char'161}}\bs
\centerline{{\color{brown}\webomints\char'125\char'126}}
\bs
\centerline{Version \ 2015.2}
\bs\bs
{\color{sectioncolor}\titleone\fontss
\centerline{Macros utiles pour modifier les polices}
\centerline{de texte et de maths en \TeX}\kern3mm
\centerline{{\titleone 45}{\titletwo\fontss \ Variantes esth\'etiques}}}
\vskip2cm
\centerline{\color{brown}\webomints \char'063}
\vskip2cm
\centerline{\twelvebf\fontss Amit Raj Dhawan}\sk
\centerline{\href{mailto:amitrajdhawan@gmail.com}{\sansrm amitrajdhawan\@gmail.com}}\bs
\centerline{Traduit par {\twelvebf\fontss Daphne Parramon-Dhawan}}\ms
\centerline{\rm 2 Septembre 2015}

\vskip4cm

% Licence
\hrule\kern1pt\hrule\sk
\ii{\epsfxsize2cm\epsfbox{by-sa.eps}}
\vskip-7mm
\vbox{\leftskip3cm\parindent=0pt\eightrm\fontss Ce travail a \'et\'e publi\'e sous la licence \href{http://creativecommons.org/licenses/by-sa/3.0/}{\eightbf Creative Commons Attribution-Share Alike 3.0 Unported License}, le 19 Juillet 2010.

Vous \^etes donc libres de {\eightitbf Partager\/} (de copier, distribuer et/ou de transmettre ce travail) et de {\eightitbf Modifier\/} (d'adapter ce document) pourvu que vous suiviez les lignes directrices {\eightitbf d'Attribution\/} et de {\eightitbf Partage \`a l'Identique\/}. Pour le texte complet de la licence, vous pouvez aller sur le site~:
\href{http://creativecommons.org/licenses/by-sa/3.0/legalcode}{\eightrm http://creativecommons.org/licenses/by-sa/3.0/legalcode}.}\leftskip0cm
\sk\hrule\kern1pt\hrule

\BlackBoxes






%%%%%%%%%    Quote    %%%%%%%%%%%%
{%\input font_artemisia_euler

\fourteenit \fontss
\

\vskip1cm
\cl{\webomints\char'064}
\vskip2cm

\leftskip1cm \rightskip4cm

\raggedright
\ii Quand j'arrive \`a ma destination, plus que de r\'ealiser que j'ai atteint mon but, je suis occup\'e \`a me rem\'emorer les d\'etails de mon voyage. Et il m'appara\^\i t, encore et encore~: ``Le trajet n'est-il pas la vraie concr\'etisation de l'objectif ?'' De cette mani\`ere, m\^eme si je manque {\sixteencaps le} but, j'aurai au moins atteint {\sixteencaps un} but.


}
\leftskip0cm \rightskip0cm







%%%%%%%%%%   Contents   %%%%%%%%%%
\newpage\pageno=-3\footline{\centerline{\foliofont\folio}}
{\special{pdf: outline 1 << /Title (Contents)/F 0 /Dest [@thispage /FitH @ypos ]  >> }}
\cl{\textcolor{sectioncolor}{\sectionfont Table des Mati\`eres}}


{\parskip0pt\bs\bs
\readtocfile}



























%%%%%%%%%%   Introduction   %%%%%%%%%%
\section{Introduction}{Introduction}
\pageno=1


\hbox{\letterone \TeX\ }\vskip-14.5mm\ii \hangindent2.6cm\hangafter-3 typographie les documents dans les polices d'\'ecriture Computer Modern par d\'efaut.\footnote{La plupart des utilisateurs (j'en fait partie) utilise les termes {\eightit police}, {\eightit police de caract\`eres\/}, {\eightit fonte}, {\eightit police d'\'ecriture\/} ou encore {\eightit famille de polices...} comme synonymes. Dans ce manuel nous avons \'evit\'e ces distinctions.} Les polices Computer Modern de Knuth sont tr\`es \'el\'egantes mais de temps \`a autre nous cherchons tous un peu de changement. Bon nombre d'entre nous souhaiterait avoir un rendu de nos documents \capstex\ dans d'autres fontes que Computer Modern. Au niveau utilisateur, il est ais\'e de changer la police du {\sl mode texte} de \capstex\ (autrement dit la police texte), et il existe de nombreuses polices gratuites aux multiples styles de caract\`eres tels que {\rm romain}, {\bf gras}, {\it italique}, {\sl pench\'e}, {\itbf gras italique}, {\slbf gras pench\'e}, {\caps petites capitales}, {\capsbf petites capitales en gras}, etc. La difficult\'e est de changer les polices math\'ematiques dans les documents \capstex\. Ceci est principalement d\^u au manque de fontes math\'ematiques pour \capstex. Une autre raison est que changer de police en {\sl mode maths\/} n'est pas aussi simple que de la changer en {\sl mode texte}. Pour \capslatex\, beaucoup de packages peuvent servir \`a changer la police (texte et math) en une commande. Mais pour \capstex, je n'ai pas pu trouver de mani\`ere simple pour changer la fonte dans le document, \`a la fois pour le texte et l'\'ecriture math\'ematique. Le fait d'utiliser une police en {\sl mode texte\/} et une autre en {\sl mode maths\/} peut g\^acher le rendu du document. Il est bien s\^ur d\'esirable d'obtenir le texte et les \'ecritures math\'ematiques dans la m\^eme police; un texte en police New Century et des maths en Computer Modern ne vont pas bien ensemble. Certaines combinaisons, comme nous le verrons plus loin, marchent pourtant bien.

Etre en mesure de choisir entre plusieurs fontes est plut\^ot avantageux. Les polices Computer Modern rendent tr\`es bien sur papier, particuli\`erement sur les impressions jet d'encre, mais ont l'air relativement fines sur les \'ecrans d'ordinateurs (LCD) et dans une moindre mesure sur les impressions laser. Pour les diaporamas, la plupart des gens pr\'ef\`ere les caract\`eres sans s\'erif qui sont relativement plus ``lourds''. L'id\'ee de changer, en une seule commande, \`a la fois les polices math\'ematiques et la totalit\'e d'une famille de polices qui comprend des styles vari\'es comme le gras, l'italique, etc., a engendr\'e la motivation n\'ecessaire \`a l'accomplissement de cette t\^ache. Pour ce faire, j'ai \'ecrit 45~macros \capstex\ qui ordonnent \`a \capstex\ de typographier les documents dans les polices appel\'ees par ces macros. Tout au long de ce document, l'utilisation des 45 macros mentionn\'ees a \'et\'e expos\'ee. Chacune de ces macros change les fontes dans le document de mani\`ere globale, mais peut \'egalement \^etre utilis\'ee localement, par exemple \`a l'int\'erieur d'un groupe. D\'esormais, un document \capstex, normalement produit en Computer Modern, peut \^etre produit en 45 autres variantes. Ces fichiers macro sont facilement compr\'ehensibles et peuvent \^etre modifi\'es si besoin. Chaque macro a diff\'erents caract\`eres d\'eclar\'es en tailles 5, 6, 7, 8, 9, 10, 12, 14, 16, 18, et 20\,pt.

Afin de montrer nos 45~macros de changement de fonte en action, nous avons \'ecrit un texte \'echantillon 45 fois mais dans des polices diff\'erentes. Les polices/familles de police invoqu\'ees par ces macros ont presque tous les glyphes contenus dans la famille Computer Modern. En g\'en\'eral, ces polices contiennent plus de glyphes que Computer Modern. Pour voir tous les glyphes pr\'esents dans une police, vous pouvez utiliser \href{https://www.ctan.org/pkg/fontchart?lang=en}{l'outil en ligne} de Werner Lemberg. Dans quelques cas, par exemple pour la police Epigrafica normal~(\verbatim epigrafican8r|endverbatim), il manque des symboles importants comme $\Gamma$ and~$\Theta$. Notre macro prend cela en charge; l'utilisateur ne doit pas s'en soucier \`a moins qu'il ne demande \`a~\capstex\ quelque chose de tr\`es inhabituel.



\subsection{Utilisation}{Utilisation}Ces macros ont \'et\'e rassembl\'ees dans un package appel\'e {\color{brown}\verbatim font-change|endverbatim} qui est inclus dans les distributions {\caps m{\eightrm i}k}\capstex\ et \capstex~{\caps l{\eightrm ive}}. Le package peut aussi \^etre t\'el\'echarg\'e sur \href{http://www.ctan.org/tex-archive/macros/plain/contrib/font-change/}{\caps ctan}. Si l'installation \capstex\ contient d\'ej\`a le package {\color{brown}\verbatim font-change|endverbatim}, il peut \^etre utilis\'e d\`es \`a pr\'esent, pour mettre en forme n'importe quel document en police Charter par exemple. Il suffit d'\'ecrire la commande {\color{brown}\verbatim \input font_charter|endverbatim} dans le fichier source. Bien s\^ur, pour pouvoir utiliser les macros de font-change, l'installation \capstex\ doit contenir les polices en question. Au cas o\`u {\color{brown}\verbatim font-change|endverbatim} n'est pas install\'e sur le syst\`eme de l'utilisateur et celui-ci n'a pas envie de le faire, on peut t\'el\'echarger le package sur internet et suivre la proc\'edure ci-dessous. Pour conna\^\i tre toutes les options disponibles et voir les macros en action, veuillez lire la suite.

Supposons qu'on veuille typographier un document \capstex\ document en police Charter. Pour ce faire, il faut copier le fichier macro \capstex\ {\color{brown}\verbatim font_charter.tex|endverbatim} dans le dossier contenant le fichier source \capstex\. Apr\`es avoir ouvert ce fichier source dans l'\'editeur, il faudra y \'ecrire la commande {\color{brown}\verbatim \input font_charter|endverbatim}. Ceci changera la police du document en police Charter \`a partir de l'endroit o\`u la commande {\color{brown}\verbatim \input font_charter|endverbatim} a \'et\'e d\'eclar\'ee. Il est possible de d\'eclarer {\color{brown}\verbatim \input font_charter|endverbatim} dans un groupe ferm\'e~: ({\color{brown}\verbatim {\input font_charter ... }|endverbatim}) afin de changer la police de caract\`eres en Charter dans tout le groupe, pourvu qu'il n'y ait pas d'autre appel \`a font-change dans ce groupe ou dans un de ses sous-groupes.

Une autre mani\`ere d'utiliser les fichiers macros est de les mettre dans un dossier, nomm\'e par exemple ``font-change'', dans un endroit du disque (par exemple \`a la racine de ``C''), et ensuite d'invoquer ces fichiers dans le fichier source \capstex\. Pour utiliser la police Charter, il faudra \'ecrire la commande suivante (qui pr\'ecise juste la localisation du fichier dans l'arborescence syst\`eme) {\color{brown}\verbatim \input C:/font-change/font_charter|endverbatim}. Si les fichiers macros ont \'et\'e plac\'es dans un dossier dont le nom contient des espaces (par exemple ``font change''), il faudra alors \'ecrire naturellement la commande avec les espaces correspondants~: {\color{brown}\verbatim \input "C:/font change/font_charter"|endverbatim}.

Le changement complet de la police se fera \`a la taille \capstex\ par d\'efaut, \`a savoir (10\,pt), bien que l'on puisse utiliser les polices texte et maths \`a des tailles plus petites et plus grandes via de petites manipulations du fichier macro.

\goodbreak Les commandes de contr\^ole typographique \capstex\ basiques \sk
{\obeylines\leftskip1cm
{\color{brown}\verbatim\rm|endverbatim} \dots  {\rm romain}
{\color{brown}\verbatim\it|endverbatim} \dots  {\it italique}
{\color{brown}\verbatim\bf|endverbatim} \dots  {\bf gras}
{\color{brown}\verbatim\sl|endverbatim} \dots  {\sl pench\'e}
{\color{brown}\verbatim\tt|endverbatim} \dots  {\tt machine \`a \'ecrire}
\leftskip0cm}\sk

\ii gardent leur signification habituelle. Tous les fichiers macro que ce {\caps pdf} mentionne incluent les cinq options ci-dessus. De plus, la plupart des fichiers macro ont \'egalement d'autres options utiles. Ce sont:\sk
{\obeylines\leftskip1cm
{\color{brown}\verbatim\itbf|endverbatim} \dots  {\itbf gras italique}
{\color{brown}\verbatim\slbf|endverbatim} \dots  {\slbf gras pench\'e}
{\color{brown}\verbatim\caps|endverbatim} \dots  {\caps petites capitales}
{\color{brown}\verbatim\capsbf|endverbatim} \dots  {\capsbf petites capitales en gras}
\leftskip0cm}\ms

En {\sl mode texte}, les styles mentionn\'es ci-dessus peuvent \^etre utilis\'es en taille 5, 6, 7, 8, 9, 10, 12, 14, 16, 18, et 20\,pt. Ceci est obtenu en tapant la taille (nombre en anglais) en toutes lettres entre un backslash~($\backslash$) et le mot qui d\'eclare le style de caract\`eres \`a utiliser. Par exemple, si nous voulions typographier un texte en gras \`a~14\,pt nous n'aurions qu'\`a utiliser la commande de contr\^ole suivante {\color{brown}\verbatim \fourteenbf|endverbatim}.





\newpage\subsection{Exemple}{Exemple}Voici un exemple de fichier source \capstex~:

\bigskip\hrule\vbox{\noindent\vrule\NoBlackBoxes\vbox{\vskip2mm\leftskip7mm\rightskip7mm
{\obeylines\parindent=0pt\color{brown}\verbatim
\parindent=0pt
\input C:/font-change/font_cm
Voici la {\bf police Computer Modern}. La {\twelveslbf fonction Gamma\/}
est d\'efinie comme suit~:
$$\Gamma(z) \equiv \int_0^\infty t^{z-1} e^{-t} dt.$$

\input C:/font-change/font_charter
Voici la {\bf police Charter}. La {\twelveslbf fonction Gamma\/}
est d\'efinie comme suit~:
$$\Gamma(z) \equiv \int_0^\infty t^{z-1} e^{-t} dt.$$

{ % d\'ebut du groupe
\input C:/font-change/font_century
Voici la {\bf police New Century Schoolbook}. La {\twelveslbf fonction Gamma\/}
est d\'efinie comme suit~:
$$\Gamma(z) \equiv \int_0^\infty t^{z-1} e^{-t} dt.$$
} % fin du groupe

Nous revoil\`a en police Charter.|endverbatim}
\vskip2mm}\vrule}\hrule\BlackBoxes\bigskip\bigskip




\nopagebreak\ii qui nous donnera, apr\`es compilation~:\nopagebreak




\bigskip\bigskip\hrule\vbox{\noindent\vrule\NoBlackBoxes\vbox{\vskip2mm\leftskip7mm\rightskip7mm
{\parindent=0pt
\input font_cm \fontss
Voici la {\bf police Computer Modern}. La {\twelveslbf fonction Gamma\/} est d\'efinie comme suit~:
$$\Gamma(z) \equiv \int_0^\infty t^{z-1} e^{-t} dt.$$

\input font_charter \fontss
Voici la {\bf police Charter}. La {\twelveslbf fonction Gamma\/} est d\'efinie comme suit~:
$$\Gamma(z) \equiv \int_0^\infty t^{z-1} e^{-t} dt.$$

{\input font_century \fontss
Voici la {\bf police New Century}. La {\twelveslbf fonction Gamma\/} est d\'efinie comme suit~:
$$\Gamma(z) \equiv \int_0^\infty t^{z-1} e^{-t} dt.$$}

Nous revoil\`a en police Charter.}
\vskip2mm}\vrule}\hrule\BlackBoxes\bigskip\bigskip


\subsection{Symboles AMS}{Symboles AMS}Certaines polices d'\'ecriture, comme par exemple Kp-Fonts, supportent les symboles {\caps ams}. Les polices {\verbatim msam|endverbatim} et {\verbatim msbm|endverbatim} de la collection {\caps ams} contiennent ces symboles. Les lettres ajour\'ees~($\Bbb A, \Bbb B, \Bbb C, \Bbb R, \dots$) font partie des symboles {\caps ams}. Si vous utilisez \amstex, avec le style preprint (pr\'e-impression) ou que la commande \amstex\ {\color{brown}\verbatim \UseAMSsymbols|endverbatim} a d\'ej\`a \'et\'e d\'eclar\'ee, vous pourrez alors utiliser les symboles {\caps ams} avec certaines macros {\color{brown}\verbatim font-change|endverbatim} en d\'eclarant {\color{brown}\verbatim \UseAMSsymbols|endverbatim} {\bf \`a nouveau} apr\`es l'instruction d'appel \`a la macro. Nous allons voir un exemple de cette impl\'ementation dans un petit moment.

Si vous avez utilis\'e les instructions {\color{brown}\verbatim \loadmsam|endverbatim} ou {\color{brown}\verbatim \loadmsbm|endverbatim} de \amstex, vous pourrez les utiliser {\bf \`a nouveau} apr\`es avoir d\'eclar\'e la macro {\color{brown}\verbatim font-change|endverbatim} afin d'obtenir les r\'esultats d\'esir\'es. La s\'equence de contr\^ole {\color{brown}\verbatim \UseAMSsymbols|endverbatim} reprend les instructions {\color{brown}\verbatim \loadmsam|endverbatim} et {\color{brown}\verbatim \loadmsbm|endverbatim}.

Pour revenir aux polices {\caps ams} par d\'efaut ({\verbatim msam|endverbatim} et {\verbatim msbm|endverbatim}), il faudra entrer le fichier macro {\color{brown}\verbatim default-amssymbols.tex|endverbatim} en \'ecrivant la commande {\color{brown}\verbatim \input default-amssymbols|endverbatim} dans le fichier source. Ce petit fichier contient seulement ces deux d\'efinitions~:


\bigskip\hrule\vbox{\noindent\vrule\NoBlackBoxes\vbox{\vskip2mm\leftskip7mm\rightskip7mm
{\obeylines\parindent=0pt\color{brown}\verbatim
\def\loadmsam{\font\tenmsa=msam10 \font\sevenmsa=msam7 \font\fivemsa=msam5
\fam\msafam
\textfont\msafam=\tenmsa \scriptfont\msafam=\sevenmsa
\scriptscriptfont\msafam=\fivemsa \global\let\loadmsam\empty}%
\loadmsam
%
\def\loadmsbm{\font\tenmsb=msbm10 \font\sevenmsb=msbm7 \font\fivemsb=msbm5
\fam\msbfam
\textfont\msbfam=\tenmsb \scriptfont\msbfam=\sevenmsb
\scriptscriptfont\msbfam=\fivemsb \global\let\loadmsbm\empty}%
\loadmsbm
|endverbatim}
\vskip2mm}\vrule}\hrule\BlackBoxes\bigskip

Il sera pr\'ecis\'e plus loin pour chaque macro du package {\color{brown}\verbatim font-change|endverbatim} si celle-ci supporte les symboles {\caps ams}. Ci-dessous la pr\'esentation de ce qui a \'et\'e discut\'e (le caract\`ere en {\color{red}rouge} provient des symboles {\caps ams})~:

\bigskip\hrule\vbox{\noindent\vrule\NoBlackBoxes\vbox{\vskip2mm\leftskip7mm\rightskip7mm
{\obeylines\parindent=0pt\color{brown}\verbatim
\input amstex % Charge AmSTeX
\UseAMSsymbols % Invoque les symboles AMS
$$f:{\color{red}\Bbb R}^3\to R$$

\input font_kp % Invoque Kp-Fonts
\UseAMSsymbols % Utilise jkpsya et jkpsyb de Kp-Fonts \`a la place de msam et msbm des polices AMS
$$f:{\color{red}\Bbb R}\to R$$

\input default-amssymbols % Revient au d\'efaut
$$f:{\color{red}\Bbb R}^3\to R$$
|endverbatim}
\vskip2mm}\vrule}\hrule\BlackBoxes\bigskip

\nopagebreak\ii produit apr\`es compilation~:\nopagebreak

\bigskip\hrule\vbox{\noindent\vrule\NoBlackBoxes\vbox{\vskip2mm\leftskip7mm\rightskip7mm
{\input font_cm
$$f:{\color{red}\Bbb R}^3\to R$$
\input font_kp
\UseAMSsymbols
$$f:{\color{red}\Bbb R}^3\to R$$
\input default-amssymbols
$$f:{\color{red}\Bbb R}^3\to R$$}
\vskip2mm}\vrule}\hrule\BlackBoxes\bigskip



\subsection{Graisses disponibles}{Graisses disponibles}
Certaines macros de changement de police d'\'ecriture du package {\color{brown}\verbatim font-change|endverbatim} proposent des graisses l\'eg\`eres, moyennes et grasses. De nombreuses familles de police offrent la variante grasse des fontes math\'ematiques, mais nous n'avons pas toujours inclus certaines variantes qui ne fournissaient pas une police assez \'epaisse pour rendre le contraste. En tapant tout le texte en gras, si \`a certains endroits nous voulons mettre encore plus de gras, nous serons coinc\'es. La philosophie de {\color{brown}\verbatim font-change|endverbatim} dit que pour mettre en gras tout le texte et les maths, il faut trouver une police plus \'epaisse parmi la famille de polices utilis\'ee, encore plus grasse que le gras habituel. Les familles de fontes Kp-Fonts, Antykwa Toru\'nska, Iwona, et Kurier incluent de telles \'epaisseurs et font partie de {\color{brown}\verbatim font-change|endverbatim}. Par exemple, la macro {\color{brown}\verbatim font_kurier-bold|endverbatim}, qui utilise un style gras pour sa police normale (en maths et texte), prend une police poids lourd pour le style gras.



\subsection{Mises \`a jour et avertissement}{Mises a jour et avertissement}Les polices utilis\'ees dans ces 45 macros sont incluses dans  les distributions \href{http://miktex.org/}{{\caps m{\eightrm i}k}\capstex} et \href{http://www.tug.org/texlive/}{\capstex~{\caps l{\eightrm ive}}}. Toutes ces macros devraient marcher sans probl\`eme avec une installation compl\`ete de {{\caps m{\eightrm i}k}\capstex} (version 2.9.4503 test\'ee). Les macros devraient aussi fonctionner avec \capstex~{\caps l{\eightrm ive}}~2014, mais \capstex~{\caps l{\eightrm ive}}~2013 ne contient pas les mises \`a jour r\'ecentes de police, et donc plusieurs macros de la nouvelle version de {\color{brown}\verbatim font-change|endverbatim} pourraient ne pas fonctionner avec \capstex~{\caps l{\eightrm ive}} 2013 ou ant\'erieur. Mais cela ne devrait pas \^etre un gros probl\`eme puisque le disque d'installation de \capstex~{\caps l{\eightrm ive}} 2013 contient une version plus ancienne de {\color{brown}\verbatim font-change|endverbatim}, qui a les anciens noms de polices. De nombreuses macros de {\color{brown}\verbatim font-change|endverbatim} utilisent la police {\verbatim inconsolata|endverbatim} pour les caract\`eres machine \`a \'ecrire. La police s'appelait {\verbatim rm-inconsolata|endverbatim} en version 2010.1 de {\color{brown}\verbatim font-change|endverbatim}. La nouvelle version de {\verbatim inconsolata|endverbatim}, qui a \'et\'e mise \`a jour en {\caps m{\eightrm i}k}\capstex}~2.9.4503, ne contient aucune police appel\'ee {\verbatim rm-inconsolata|endverbatim}. Ainsi, en {\color{brown}\verbatim font-change|endverbatim}~(version 2013.1), nous avons choisi une autre police {\verbatim inconsolata|endverbatim} nomm\'ee {\verbatim ly1-zi4r-1|endverbatim}, qui est identique, ou tout du moins, semble identique, \`a {\verbatim rm-inconsolata|endverbatim}. Il y a eu aussi d'autres changements dans les noms des polices, par exemple pour les polices Libertine. Si des probl\`emes de police manquante sont rencontr\'es en utilisant {\color{brown}\verbatim font-change|endverbatim}, avec une installation compl\`ete de {{\caps m{\eightrm i}k}\capstex} ou de TeXLive, il est recommand\'e d'utiliser un version soit plus ancienne, soit plus r\'ecente de {\color{brown}\verbatim font-change|endverbatim}.

\footline{\centerline{\foliofont\folio}}


Ces 45 macros de changement de police ont bien fonctionn\'e avec plain~\capstex, ainsi qu'avec une combinaison de plain~\capstex\ et d'autres formats bas\'es sur plain~\capstex, par exemple~  \href{http://www.tex.ac.uk/tex-archive/help/Catalogue/entries/amstex.html}{\amstex} et~\href{http://www.tex.ac.uk/tex-archive/help/Catalogue/entries/eplain.html}{eplain}. Les macros marchent parfaitement avec \href{http://www.tex.ac.uk/tex-archive/help/Catalogue/entries/pdftex.html}{pdf\capstex} et \href{http://scripts.sil.org/cms/scripts/page.php?site_id=nrsi&id=xetex}{\capsxetex} \'egalement. Veuillez noter que ces macros ne fonctionnent pas avec \href{http://www.tex.ac.uk/tex-archive/help/Catalogue/entries/latex.html}{\capslatex}, pdf\capslatex\hfuzz2pt, ou \capsxelatex.

\hfuzz1pt
La composition d'un texte en anglais ou en fran\c cais avec des math\'ematiques ne devrait pas poser probl\`eme, \`a part peut-\^etre si des lettres comme \l~ sont utilis\'ees, surtout avec des caract\`eres en {\slbf gras pench\'e\/} ou {\caps petites capitales}. Ce sont des questions de glyphes manquants et d'encodage. Dans la police courante (Charter, romain r\'egulier, {\verbatim mdbchr7t|endverbatim}), {\color{brown}\verbatim\l|endverbatim} produit \l, {\color{brown}\verbatim\slbf \l|endverbatim} produit {\slbf \l}, mais {\color{brown}\verbatim\caps\l|endverbatim} produit~{\caps \l}.





Les polices sans s\'erif ne proposent pas de caract\`eres {\it italiques} mais seulement {\sl pench\'es\/}. Pour rendre les fichiers macros de {\color{brown}\verbatim font-change|endverbatim} plus coh\'erents, les commandes pour l'italique et le pench\'e (par exemple~: {\color{brown}\verbatim\it|endverbatim} et {\color{brown}\verbatim\sl|endverbatim}), produisent des caract\`eres {\sl pench\'es} dans le cas des polices sans s\'erif et de celles ne disposant pas de glyphes distincts pour l'italique et le pench\'e. Nous montrerons plus loin des exemples de changement des polices texte et maths de \capstex\ utilisant les macros de {\color{brown}\verbatim font-change|endverbatim}. Toutes les polices utilis\'ees dans ces macros sont aussi list\'ees dans ce document.


Ces macros ont \'et\'e con\c{c}ues \`a l'origine pour des utilisateurs de langue anglaise. Compte tenu des capacit\'es de \TeX, elles peuvent \^etre utiles pour composer en d'autres langages \'egalement, mais certaines polices contenues dans {\color{brown}\verbatim font-change|endverbatim} peuvent ne pas fonctionner pour toutes les langues (\`a cause du rendu des accents et caract\`eres sp\'eciaux). Nous avons remarqu\'e que pour ce document, qui est en langue franç\c{c}aise, les accents ne sont pas, ou mal, plac\'es sur certaines fontes de type pench\'e ou machine \`a \'ecrire. Cela concerne les polices Arev (page 38), Epigrafica (pages 40 \`a 41), Bera (pages 43 \`a 45), Artemisia (page 46), et Libertine (pages 47 \`a 49).


Nous esp\'erons que ces macros fonctionnent bien, sans probl\`eme de compatibilit\'e, mais nous ne pouvons rien promettre. Il n'y a pas de garantie. Si l'utilisateur trouve un d\'efaut, ou pour toute autre remarque, merci de m'envoyer un email.

















%%%%%%%%%%%%%%%%%%%%%%%%%%%%%%%%
%%%%%%%%%%   Macros   %%%%%%%%%%
%%%%%%%%%%%%%%%%%%%%%%%%%%%%%%%


\section{\sixteenbf\fontss Charter}{Charter}
\sample
\ii La police Charter est d\'eclar\'ee en entrant l'instruction {\color{brown}\verbatim\input font_charter|endverbatim}. Cette famille de polices utilise des fontes de la famille  \href{http://www.tex.ac.uk/tex-archive/help/Catalogue/entries/mathdesign-charter.html}{mdbch}, correspondant aux polices texte \href{http://www.tex.ac.uk/tex-archive/help/Catalogue/entries/charter.html}{Bitstream Charter}. Cette famille fait partie du projet \href{http://www.tex.ac.uk/tex-archive/help/Catalogue/entries/mathdesign.html}{MathDesign} de Paul Pichaureau. Les caract\`eres \href{http://new.myfonts.com/fonts/bitstream/charter-bt-pro/}{Charter} ont \'et\'e con\c cus \`a l'origine par Matthew Carter pour Bitstream Inc.\ en 1987. Des d\'etails sur cette macro sont donn\'es dans le tableau ci-dessous.
\bs
\hfil{Affectation de fonte pour la macro {\color{brown}\verbatim font_charter|endverbatim}}\hfil

{\parindent=0pt\settabs4\columns\hfil\vbox{\hrule\hbox{\vrule\hbox{\vbox{\kern1pt\hrule\NoBlackBoxes		 \eightrm\fontss																 \+\hfil	 \textcolor{blue}{ Style}	 &\strut\vrule\strut\hfil	 \textcolor{blue}{ Nom de la fonte}	 &\strut\vrule\kern1pt\vrule\strut\hfil	 \textcolor{blue}{ Style}	 &\strut\vrule\strut\hfil	 \textcolor{blue}{ Nom de la fonte}	 &\cr	\hrule
\+\hfil	\eightrm Romain	&\strut\vrule\strut\hfil	 mdbchr7t	 &\strut\vrule\kern1pt\vrule\strut\hfil	 \eightbf Gras	 &\strut\vrule\strut\hfil	 mdbchb7t	 &\cr	\hrule
\+\hfil	\eighti Maths italique	&\strut\vrule\strut\hfil	 mdbchri7m	 &\strut\vrule\kern1pt\vrule\strut\hfil	 \eighttt Machine \`a \'ecrire	 &\strut\vrule\strut\hfil	 ly1-zi4r-1	&\cr	\hrule
\+\hfil	\eightrm Symboles maths	&\strut\vrule\strut\hfil	 md-chr7y	 &\strut\vrule\kern1pt\vrule\strut\hfil	 \eightitbf Gras italique	 &\strut\vrule\strut\hfil	 mdbchbi7t	&\cr	\hrule
\+\hfil	\eightrm Extension maths	&\strut\vrule\strut\hfil	 mdbchr7v	 &\strut\vrule\kern1pt\vrule\strut\hfil	 \eightslbf Gras pench\'e	 &\strut\vrule\strut\hfil	 mdbchbo7t	&\cr	\hrule
\+\hfil	\eightit Italique	&\strut\vrule\strut\hfil	 mdbchri7t	 &\strut\vrule\kern1pt\vrule\strut\hfil	 \eightcaps Petites capitales	 &\strut\vrule\strut\hfil	 mdbchrfc8t	&\cr	 \hrule
\+\hfil	\eightsl Pench\'e	&\strut\vrule\strut\hfil	 mdbchro7t	 &\strut\vrule\kern1pt\vrule\strut\hfil	 \eightcapsbf Petites capitales en gras	 &\strut\vrule\strut\hfil	 mdbchbfc8t	&\cr	\hrule
}\vrule}}\hrule}\hfil}									

\BlackBoxes					










\input font_utopia   \fontss
\section{\sixteenbf\fontss Utopia}{Utopia}
\sample
\ii La police Utopia est d\'eclar\'ee en entrant l'instruction {\color{brown}\verbatim\input font_utopia|endverbatim}. Cette famille de polices utilise pour la plupart les fontes de la famille \href{http://www.tex.ac.uk/tex-archive/help/Catalogue/entries/mathdesign-utopia.html}{mdput}, ce qui correspond aux polices de texte \href{http://www.tex.ac.uk/tex-archive/help/Catalogue/entries/utopia.html}{Adobe Utopia}. Cette famille fait parte du projet \href{http://www.tex.ac.uk/tex-archive/help/Catalogue/entries/mathdesign.html}{MathDesign}~de Paul Pichaureau. Elle est tr\`es compl\`ete et inclut les polices math\'ematiques \'egalemement. Pour des raisons d'espacement interlettre, la macro {\color{brown}\verbatim font_utopia.tex|endverbatim} utilise la fonte maths italique et la fonte symboles math\'ematique du package \href{http://www.ctan.org/tex-archive/fonts/fourier-GUT/}{fourier} de Michel Bovani. La \href{http://new.myfonts.com/fonts/adobe/utopia/}{police Utopia} a \'et\'e con\c cue \`a l'origine par Robert Slimbach pour Adobe en~1989.

Les fontes maths italique~(\verbatim mdputri7m|endverbatim) et symboles math\'ematiques~(\verbatim md-utr7y|endverbatim) de la famille \href{http://www.tex.ac.uk/tex-archive/help/Catalogue/entries/mathdesign-utopia.html}{mdput} peuvent \^etre aussi utilis\'ees. Des d\'etails sur cette macro sont donn\'es dans le tableau ci-dessous.
\bs
\hfil{Affectation de fonte pour la macro {\color{brown}\verbatim font_utopia|endverbatim}}\hfil

{\parindent=0pt\settabs4\columns\hfil\vbox{\hrule\hbox{\vrule\hbox{\vbox{\kern1pt\hrule\NoBlackBoxes		 \eightrm\fontss																 \+\hfil	 \textcolor{blue}{ Style}	 &\strut\vrule\strut\hfil	 \textcolor{blue}{ Nom de la fonte}	 &\strut\vrule\kern1pt\vrule\strut\hfil	 \textcolor{blue}{ Style}	 &\strut\vrule\strut\hfil	\textcolor{blue}{ Nom de la fonte}	 &\cr	\hrule
\+\hfil	\eightrm Romain	&\strut\vrule\strut\hfil	 mdputr7t	 &\strut\vrule\kern1pt\vrule\strut\hfil	\eightbf Gras	 &\strut\vrule\strut\hfil	 mdputb7t	&\cr	 \hrule
\+\hfil	\eighti Maths italique	&\strut\vrule\strut\hfil	 futmii	 &\strut\vrule\kern1pt\vrule\strut\hfil	\eighttt Machine \`a \'ecrire	 &\strut\vrule\strut\hfil	 ly1-zi4r-1	 &\cr	\hrule
\+\hfil	\eightrm Symboles maths	&\strut\vrule\strut\hfil	 futsy	 &\strut\vrule\kern1pt\vrule\strut\hfil	\eightitbf Gras italique	 &\strut\vrule\strut\hfil	 mdputbi7t	&\cr	\hrule
\+\hfil	\eightrm Extension maths	&\strut\vrule\strut\hfil	 mdputr7v	 &\strut\vrule\kern1pt\vrule\strut\hfil	 \eightslbf Gras pench\'e	 &\strut\vrule\strut\hfil	mdputbo7t	&\cr	\hrule
\+\hfil	\eightit Italique	&\strut\vrule\strut\hfil	 mdputri7t	 &\strut\vrule\kern1pt\vrule\strut\hfil	 \eightcaps Petites capitales	 &\strut\vrule\strut\hfil	 mdputrfc8t	&\cr	 \hrule
\+\hfil	\eightsl Pench\'e	&\strut\vrule\strut\hfil	 mdputro7t	 &\strut\vrule\kern1pt\vrule\strut\hfil	 \eightcapsbf Petites capitales en gras	 &\strut\vrule\strut\hfil	 mdputbfc8t	&\cr	\hrule
}\vrule}}\hrule}\hfil}								

\BlackBoxes								























\input font_century   \fontss
\section{\sixteenbf\fontss New Century Schoolbook}{New Century Schoolbook}
\sample
\ii La police New Century Schoolbook est d\'eclar\'ee en entrant l'instruction {\color{brown}\verbatim\input font_century|endverbatim}. Cette famille de polices utilise des fontes de la famille \href{http://www.tex.ac.uk/tex-archive/help/Catalogue/entries/tex-gyre-schola.html}{TeX Gyre Schola}, ce qui correspond aux polices de texte \href{http://store1.adobe.com/cfusion/store/html/index.cfm?store=OLS-US&event=displayFontPackage&code=1240}
{Adobe New Century Schoolbook}. La police \href{http://new.myfonts.com/fonts/adobe/new-century-schoolbook/}{Century Schoolbook} a \'et\'e cr\'ee par Morris Fuller Benton entre 1918 et~1921.

\BlackBoxes
La macro utilise l'italique math\'ematique~(fncmii) et les symboles maths~(fncsy) du package \href{http://www.tex.ac.uk/tex-archive/help/Catalogue/entries/fouriernc.html}{fouriernc} de Michael Zedler. Des d\'etails sur cette macro sont donn\'es dans le tableau ci-dessous.
\bs
\hfil{Affectation de fonte pour la macro {\color{brown}\verbatim font_century|endverbatim}}\hfil

{\parindent=0pt\settabs4\columns\hfil\vbox{\hrule\hbox{\vrule\hbox{\vbox{\kern1pt\hrule\NoBlackBoxes\eightrm\fontss
\+\hfil	\textcolor{blue}{ Style}	&\strut\vrule\strut\hfil	 \textcolor{blue}{ Nom de la fonte}	 &\strut\vrule\kern1pt\vrule\strut\hfil	 \textcolor{blue}{ Style}	 &\strut\vrule\strut\hfil	\textcolor{blue}{ Nom de la fonte}	 &\cr	\hrule
\+\hfil	\eightrm Romain	&\strut\vrule\strut\hfil	 rm-qcsr	 &\strut\vrule\kern1pt\vrule\strut\hfil	\eightbf Gras	 &\strut\vrule\strut\hfil	 rm-qcsb	&\cr	 \hrule
\+\hfil	\eighti Maths italique	&\strut\vrule\strut\hfil	 fncmii	 &\strut\vrule\kern1pt\vrule\strut\hfil	\eighttt Machine \`a \'ecrire	 &\strut\vrule\strut\hfil	 cmtt10	&\cr	 \hrule
\+\hfil	\eightrm Symboles maths	&\strut\vrule\strut\hfil	 fncsy	 &\strut\vrule\kern1pt\vrule\strut\hfil	\eightitbf Gras italique	 &\strut\vrule\strut\hfil	rm-qsbi	 &\cr	\hrule
\+\hfil	\eightrm Extension maths	&\strut\vrule\strut\hfil	 cmex10	 &\strut\vrule\kern1pt\vrule\strut\hfil	\eightslbf Gras pench\'e	 &\strut\vrule\strut\hfil	pncbo7t	 &\cr	\hrule
\+\hfil	\eightit Italique	&\strut\vrule\strut\hfil	 rm-qcsri	 &\strut\vrule\kern1pt\vrule\strut\hfil	 \eightcaps Petites capitales	 &\strut\vrule\strut\hfil	 rm-qcsr-sc	&\cr	 \hrule
\+\hfil	\eightsl Pench\'e	&\strut\vrule\strut\hfil	 pncro7t	 &\strut\vrule\kern1pt\vrule\strut\hfil	 \sevencapsbf Petites capitales en gras	 &\strut\vrule\strut\hfil	 rm-qcsb-sc	&\cr	\hrule
}\vrule}}\hrule}\hfil}								
					
\BlackBoxes								









\input font_palatino   \fontss
\UseAMSsymbols
\section{\sixteenbf\fontss Palatino}{Palatino}
\sample
\ii La police Palatino est d\'eclar\'ee en entrant l'instruction {\color{brown}\verbatim\input font_palatino|endverbatim}. Cette famille de polices utilise les fontes du package  \href{http://www.tex.ac.uk/tex-archive/help/Catalogue/entries/pxfonts.html}{pxfonts} de Young Ryu, correspondant aux polices texte \href{http://www.myfonts.com/fonts/urw/palladio/}
{{\caps urw++} Palladio} dessin\'ees par Herman Zapf. La police Palladio {\caps urw++} est bas\'ee sur la \href{http://new.myfonts.com/fonts/adobe/palatino/}{police Palatino} qui avait \'et\'e con\c cue \`a l'origine par Hermann Zapf pour la fonderie Stempel en 1950. Les polices de cette macro fournissent leurs propres symboles {\caps ams}. Des d\'etails sur cette macro sont donn\'es dans le tableau ci-dessous.
\bs
\hfil{Affectation de fonte pour la macro {\color{brown}\verbatim font_palatino|endverbatim}}\hfil

{\parindent=0pt\settabs4\columns\hfil\vbox{\hrule\hbox{\vrule\hbox{\vbox{\kern1pt\hrule\NoBlackBoxes		 \eightrm\fontss \+\hfil	 \textcolor{blue}{ Style}	 &\strut\vrule\strut\hfil	 \textcolor{blue}{ Nom de la fonte}	 &\strut\vrule\kern1pt\vrule\strut\hfil	 \textcolor{blue}{ Style}	 &\strut\vrule\strut\hfil	\textcolor{blue}{ Nom de la fonte}	 &\cr	\hrule
\+\hfil	\eightrm Romain	&\strut\vrule\strut\hfil	pxr	 &\strut\vrule\kern1pt\vrule\strut\hfil	\eightbf Gras	 &\strut\vrule\strut\hfil	 pxb	&\cr	\hrule
\+\hfil	\eighti Maths italique	&\strut\vrule\strut\hfil	pxmi	 &\strut\vrule\kern1pt\vrule\strut\hfil	\eighttt Machine \`a \'ecrire	 &\strut\vrule\strut\hfil	 cmtt10	&\cr	\hrule
\+\hfil	\eightrm Symboles maths	&\strut\vrule\strut\hfil	 pxsy	 &\strut\vrule\kern1pt\vrule\strut\hfil	\eightitbf Gras italique	 &\strut\vrule\strut\hfil	pxbi	 &\cr	\hrule
\+\hfil	\eightrm Extension maths	&\strut\vrule\strut\hfil	 pxex	 &\strut\vrule\kern1pt\vrule\strut\hfil	\eightslbf Gras pench\'e	 &\strut\vrule\strut\hfil	pxbsl	 &\cr	\hrule
\+\hfil	\eightit Italique	&\strut\vrule\strut\hfil	 pxi	 &\strut\vrule\kern1pt\vrule\strut\hfil	\eightcaps Petites capitales	 &\strut\vrule\strut\hfil	 pxsc	 &\cr	\hrule
\+\hfil	\eightsl Pench\'e	&\strut\vrule\strut\hfil	 pxsl	 &\strut\vrule\kern1pt\vrule\strut\hfil	 \eightcapsbf Petites capitales en gras	 &\strut\vrule\strut\hfil	 pxbsc	&\cr	\hrule
}\vrule}}\hrule}\hfil}	
							
\BlackBoxes					

\bs\ii Symboles {\caps ams} associ\'es~: \circledR \ \yen \ $\blacksquare \ \approxeq \ \eqslantgtr \ \curlyeqprec \ \curlyeqsucc \ \preccurlyeq \ \leqq \ \leqslant \ \lessgtr \ \nless \ \nleq \ \nleqslant \ \Bbb R \ \Bbb E \ \Bbb C \ \dots$








\input font_pagella  \fontss
\UseAMSsymbols
\section{\sixteenbf\fontss Pagella}{Pagella}
\sample
\ii La police Pagella est d\'eclar\'ee en entrant l'instruction {\color{brown}\verbatim\input font_pagella|endverbatim}. La plupart du texte est typographi\'e en utilisant des polices du package \href{http://www.gust.org.pl/projects/e-foundry/tex-gyre/pagella}{\capstex~Gyre Pagella}, et en utilisant le package de Diego Puga \href{http://www.tex.ac.uk/tex-archive/help/Catalogue/entries/mathpazo.html}{mathpazo} pour les math\'ematiques. Certains styles~(polices pench\'ees) et maths~(symboles {\caps ams}) proviennent de \href{http://www.tex.ac.uk/tex-archive/help/Catalogue/entries/pxfonts.html}{pxfonts} de Young Ryu (toutes ces polices correspondent aux polices texte \href{http://www.myfonts.com/fonts/urw/palladio/}{{\caps urw++} Palladio} cr\'e\'ees par Herman Zapf). La police {\caps urw++} Palladio font est bas\'ee sur la \href{http://new.myfonts.com/fonts/adobe/palatino/}{police Palatino} qui avait \'et\'e con\c cue \`a l'origine par Hermann Zapf pour la fonderie Stempel en 1950. On peut dire que les polices \capstex~Gyre \href{http://www.gust.org.pl/projects/e-foundry/tex-gyre/pagella}{Pagella} sont une version un peu plus raffin\'ee des polices Palatino; elles proposent \'egalement la ligature ff, ce qui manque dans les polices \href{http://www.tex.ac.uk/tex-archive/help/Catalogue/entries/pxfonts.html}{pxfonts} ou autres polices bas\'ees sur Palatino. Les polices de cette macro fournissent leurs propres symboles {\caps ams}. Des d\'etails sur cette macro sont donn\'es dans le tableau ci-dessous.
\bs
\hfil{Affectation de fonte pour la macro {\color{brown}\verbatim font_pagella|endverbatim}}\hfil

{\parindent=0pt\settabs4\columns\hfil\vbox{\hrule\hbox{\vrule\hbox{\vbox{\kern1pt\hrule\NoBlackBoxes		 \eightrm\fontss							
									
\+\hfil	\textcolor{blue}{ Style}	&\strut\vrule\strut\hfil	 \textcolor{blue}{ Nom de la fonte}	 &\strut\vrule\kern1pt\vrule\strut\hfil	 \textcolor{blue}{ Style}	 &\strut\vrule\strut\hfil	\textcolor{blue}{ Nom de la fonte}	&\cr	\hrule
\+\hfil	\eightrm Romain	&\strut\vrule\strut\hfil	 rm-qplr	 &\strut\vrule\kern1pt\vrule\strut\hfil	\eightbf Gras	 &\strut\vrule\strut\hfil	 rm-qplb	&\cr	 \hrule
\+\hfil	\eighti Maths italique	&\strut\vrule\strut\hfil	 zplmr7m	 &\strut\vrule\kern1pt\vrule\strut\hfil	\eighttt Machine \`a \'ecrire	 &\strut\vrule\strut\hfil	 cmtt10	&\cr	 \hrule
\+\hfil	\eightrm Symboles maths	&\strut\vrule\strut\hfil	 zplmr7y	 &\strut\vrule\kern1pt\vrule\strut\hfil	\eightitbf Gras italique	 &\strut\vrule\strut\hfil	rm-qplbi	 &\cr	\hrule
\+\hfil	\eightrm Extension maths	&\strut\vrule\strut\hfil	 zplmr7v	 &\strut\vrule\kern1pt\vrule\strut\hfil	\eightslbf Gras pench\'e	 &\strut\vrule\strut\hfil	pxbsl	 &\cr	\hrule
\+\hfil	\eightit Italique	&\strut\vrule\strut\hfil	 rm-qplri	 &\strut\vrule\kern1pt\vrule\strut\hfil	 \eightcaps Petites capitales	 &\strut\vrule\strut\hfil	rm-qplr-sc	 &\cr	\hrule
\+\hfil	\eightsl Pench\'e	&\strut\vrule\strut\hfil	 pxsl	 &\strut\vrule\kern1pt\vrule\strut\hfil	\eightcapsbf Petites capitales en gras	 &\strut\vrule\strut\hfil	rm-qplb-sc	 &\cr	\hrule
									
	}\vrule}}\hrule}\hfil}								
									
	\BlackBoxes								
			

\bs\ii Symboles {\caps ams} associ\'es~: \circledR \ \yen \ $\blacksquare \ \approxeq \ \eqslantgtr \ \curlyeqprec \ \curlyeqsucc \ \preccurlyeq \ \leqq \ \leqslant \ \lessgtr \ \nless \ \nleq \ \nleqslant \ \Bbb R \ \Bbb E \ \Bbb C \ \dots$
















\input font_times \fontss
\UseAMSsymbols
\section{\sixteenbf\fontss Times}{Times}
\sample
\ii La police Times est d\'eclar\'ee en entrant l'instruction {\color{brown}\verbatim\input font_times|endverbatim}. Cette famille de polices utilise des polices du package de Young Ryu \href{http://www.tex.ac.uk/tex-archive/help/Catalogue/entries/txfonts.html}{txfonts}, ce qui correspond aux polices de texte \href{http://new.myfonts.com/fonts/adobe/times/}{Adobe Times}. La police \href{http://new.myfonts.com/fonts/adobe/times/}{Times} a \'et\'e cr\'e\'ee en 1931 par Stanley Morison de Monotype Corp. Les polices de cette macro fournissent leurs propres symboles {\caps ams}. Des d\'etails sur cette macro sont donn\'es dans le tableau ci-dessous.
\bs
\hfil{Affectation de fonte pour la macro {\color{brown}\verbatim font_times|endverbatim}}\hfil

{\parindent=0pt\settabs4\columns\hfil\vbox{\hrule\hbox{\vrule\hbox{\vbox{\kern1pt\hrule\NoBlackBoxes		 \eightrm\fontss																 \+\hfil	 \textcolor{blue}{ Style}	 &\strut\vrule\strut\hfil	 \textcolor{blue}{ Nom de la fonte}	 &\strut\vrule\kern1pt\vrule\strut\hfil	 \textcolor{blue}{ Style}	 &\strut\vrule\strut\hfil	\textcolor{blue}{ Nom de la fonte}	 &\cr	\hrule
\+\hfil	\eightrm Romain	&\strut\vrule\strut\hfil	txr	 &\strut\vrule\kern1pt\vrule\strut\hfil	\eightbf Gras	 &\strut\vrule\strut\hfil	 txb	&\cr	\hrule
\+\hfil	\eighti Maths italique	&\strut\vrule\strut\hfil	txmi	 &\strut\vrule\kern1pt\vrule\strut\hfil	\eighttt Machine \`a \'ecrire	 &\strut\vrule\strut\hfil	 txtt	&\cr	\hrule
\+\hfil	\eightrm Symboles maths	&\strut\vrule\strut\hfil	 txsy	 &\strut\vrule\kern1pt\vrule\strut\hfil	\eightitbf Gras italique	 &\strut\vrule\strut\hfil	txbi	 &\cr	\hrule
\+\hfil	\eightrm Extension maths	&\strut\vrule\strut\hfil	 txex	 &\strut\vrule\kern1pt\vrule\strut\hfil	\eightslbf Gras pench\'e	 &\strut\vrule\strut\hfil	txbsl	 &\cr	\hrule
\+\hfil	\eightit Italique	&\strut\vrule\strut\hfil	 txi	 &\strut\vrule\kern1pt\vrule\strut\hfil	\eightcaps Petites capitales	 &\strut\vrule\strut\hfil	 txsc	 &\cr	\hrule
\+\hfil	\eightsl Pench\'e	&\strut\vrule\strut\hfil	 txsl	 &\strut\vrule\kern1pt\vrule\strut\hfil	 \eightcapsbf Petites capitales en gras	 &\strut\vrule\strut\hfil	 txbsc	&\cr	\hrule
}\vrule}}\hrule}\hfil}								

\BlackBoxes								

\bs\ii Symboles {\caps ams} associ\'es~: \circledR \ \yen \ $\blacksquare \ \approxeq \ \eqslantgtr \ \curlyeqprec \ \curlyeqsucc \ \preccurlyeq \ \leqq \ \leqslant \ \lessgtr \ \nless \ \nleq \ \nleqslant \ \Bbb R \ \Bbb E \ \Bbb C \ \dots$








\input font_bookman   \fontss
\section{\sixteenbf\fontss Bookman}{Bookman}
\sample
\ii La police Bookman est d\'eclar\'ee en entrant l'instruction {\color{brown}\verbatim\input font_bookman|endverbatim}. Cette famille de polices utilise des polices (\capstex\ Gyre) \href{http://www.tex.ac.uk/tex-archive/help/Catalogue/entries/tex-gyre-bonum.html}{bonum} de Jackowski et Nowacki, et du package  \href{http://www.tex.ac.uk/tex-archive/help/Catalogue/entries/kerkis.html}{kerkis} d'Antonis Tsolomitis; ces deux packages correspondent aux polices texte  \href{http://new.myfonts.com/fonts/adobe/itc-bookman/}{ITC Bookman}. Les symboles math\'ematiques et caract\`eres d'extension sont issus du package  \href{http://www.tex.ac.uk/tex-archive/help/Catalogue/entries/txfonts.html}{txfonts} de Young Ryu. La police \href{http://new.myfonts.com/fonts/adobe/itc-bookman/}{Bookman} a \'et\'e con\c cue \`a l'origine par Alexander Phemister en 1860, pour la fonderie Miller \& Richard (Ecosse). Des d\'etails sur cette macro sont donn\'es dans le tableau ci-dessous.
\bs
\hfil{Affectation de fonte pour la macro {\color{brown}\verbatim font_bookman|endverbatim}}\hfil

{\parindent=0pt\settabs4\columns\hfil\vbox{\hrule\hbox{\vrule\hbox{\vbox{\kern1pt\hrule\NoBlackBoxes		 \eightrm\fontss							 \+\hfil	 \textcolor{blue}{ Style}	 &\strut\vrule\strut\hfil	 \textcolor{blue}{ Nom de la fonte}	 &\strut\vrule\kern1pt\vrule\strut\hfil	 \textcolor{blue}{ Style}	 &\strut\vrule\strut\hfil	\textcolor{blue}{ Nom de la fonte}	 &\cr	\hrule
\+\hfil	\eightrm Romain	&\strut\vrule\strut\hfil	 rm-qbkr	 &\strut\vrule\kern1pt\vrule\strut\hfil	\eightbf Gras	 &\strut\vrule\strut\hfil	 rm-qbkb	&\cr	 \hrule
\+\hfil	\eighti Maths italique	&\strut\vrule\strut\hfil	 kmath8r	 &\strut\vrule\kern1pt\vrule\strut\hfil	\eighttt Machine \`a \'ecrire	 &\strut\vrule\strut\hfil	 txtt	&\cr	 \hrule
\+\hfil	\eightrm Symboles maths	&\strut\vrule\strut\hfil	 txsy	 &\strut\vrule\kern1pt\vrule\strut\hfil	\eightitbf Gras italique	 &\strut\vrule\strut\hfil	rm-qbkbi	 &\cr	\hrule
\+\hfil	\eightrm Extension maths	&\strut\vrule\strut\hfil	 txex	 &\strut\vrule\kern1pt\vrule\strut\hfil	\eightslbf Gras pench\'e	 &\strut\vrule\strut\hfil	pbkdo7t	 &\cr	\hrule
\+\hfil	\eightit Italique	&\strut\vrule\strut\hfil	 rm-qbkri	 &\strut\vrule\kern1pt\vrule\strut\hfil	 \eightcaps Petites capitales	 &\strut\vrule\strut\hfil	 rm-qbkr-sc	&\cr	 \hrule
\+\hfil	\eightsl Pench\'e	&\strut\vrule\strut\hfil	 pbklo7t	 &\strut\vrule\kern1pt\vrule\strut\hfil	 \sevencapsbf Petites capitales en gras	 &\strut\vrule\strut\hfil	 rm-qbkb-sc	&\cr	\hrule
}\vrule}}\hrule}\hfil}								

\BlackBoxes								
						













\input font_kp   \fontss
\UseAMSsymbols
\section{\sixteenbf\fontss Kp-Fonts}{Kp-Fonts}
\sample
\ii Kp-Fonts est d\'eclar\'ee en entrant l'instruction {\color{brown}\verbatim\input font_kp|endverbatim}. Cette famille de fontes utilise des polices de la famille \href{http://www.tex.ac.uk/tex-archive/help/Catalogue/entries/kpfonts.html}{Kp-Fonts} de Chris\-tophe \hfuzz4pt Caignaert. Les polices de cette macro fournissent leurs propres symboles {\caps ams}. Des d\'etails sur cette macro sont donn\'es dans le tableau ci-dessous.
\bs\hfuzz=1pt
\hfil{Affectation de fonte pour la macro {\color{brown}\verbatim font_kp|endverbatim}}\hfil
					
{\parindent=0pt\settabs4\columns\hfil\vbox{\hrule\hbox{\vrule\hbox{\vbox{\kern1pt\hrule\NoBlackBoxes		 \eightrm\fontss																 \+\hfil	 \textcolor{blue}{ Style}	 &\strut\vrule\strut\hfil	\textcolor{blue}{ Nom de la fonte}	 &\strut\vrule\kern1pt\vrule\strut\hfil	 \textcolor{blue}{ Style}	 &\strut\vrule\strut\hfil	\textcolor{blue}{ Nom de la fonte}	 &\cr	\hrule
\+\hfil	\eightrm Romain	&\strut\vrule\strut\hfil	 jkpmn7t	 &\strut\vrule\kern1pt\vrule\strut\hfil	\eightbf Gras	 &\strut\vrule\strut\hfil	 jkpbn7t	&\cr	 \hrule
\+\hfil	\eighti Maths italique	&\strut\vrule\strut\hfil	 jkpmi	 &\strut\vrule\kern1pt\vrule\strut\hfil	\eighttt Machine \`a \'ecrire	 &\strut\vrule\strut\hfil	 jkpttmn7t	&\cr	 \hrule
\+\hfil	\eightrm Symboles maths	&\strut\vrule\strut\hfil	 jkpsy	 &\strut\vrule\kern1pt\vrule\strut\hfil	\eightitbf Gras italique	 &\strut\vrule\strut\hfil	jkpbit7t	 &\cr	\hrule
\+\hfil	\eightrm Extension maths	&\strut\vrule\strut\hfil	 jkpex	 &\strut\vrule\kern1pt\vrule\strut\hfil	\eightslbf Gras pench\'e	 &\strut\vrule\strut\hfil	jkpbsl7t	 &\cr	\hrule
\+\hfil	\eightit Italique	&\strut\vrule\strut\hfil	 jkpmit7t	 &\strut\vrule\kern1pt\vrule\strut\hfil	 \eightcaps Petites capitales	 &\strut\vrule\strut\hfil	jkpmsc7t	 &\cr	\hrule
\+\hfil	\eightsl Pench\'e	&\strut\vrule\strut\hfil	 jkpmsl7t	 &\strut\vrule\kern1pt\vrule\strut\hfil	 \eightcapsbf Petites capitales en gras	 &\strut\vrule\strut\hfil	 jkpbsc7t	&\cr	\hrule
									
	}\vrule}}\hrule}\hfil}								
									
	\BlackBoxes								

\bs\ii Symboles {\caps ams} associ\'es~: \circledR \ \yen \ $\blacksquare \ \approxeq \ \eqslantgtr \ \curlyeqprec \ \curlyeqsucc \ \preccurlyeq \ \leqq \ \leqslant \ \lessgtr \ \nless \ \nleq \ \nleqslant \ \Bbb R \ \Bbb E \ \Bbb C \ \dots$






\input font_kp-light   \fontss
\UseAMSsymbols
\section{{\sixteenbf\fontss Kp}-{\sixteenslbf Light}}{Kp-Light}
\sample
\ii Le polices Kp-{\sl Light\/} sont d\'eclar\'ees en entrant l'instruction {\color{brown}\verbatim\input font_kp-light|endverbatim}. Cette famille de fontes utilise des polices de la famille \href{http://www.tex.ac.uk/tex-archive/help/Catalogue/entries/kpfonts.html}{Kp-Fonts} de Christophe Caignaert. C'est la version l\'eg\`ere de Kp-Fonts. La diff\'erence entre les versions moyenne~(normal) et l\'eg\`ere est visible dans la {\sl couleur} du texte et, bien s\^ur, lorsque l'on agrandi les caract\`eres. D'apr\`es les auteurs de Kp-Fonts, l'option {\sl l\'eg\`ere\/}, qui fait r\'ealiser des \'economies au niveau de l'impression, devrait mieux rendre imprim\'ee que sur \'ecran. Les polices de cette macro fournissent leurs propres symboles {\caps ams}. Des d\'etails sur cette macro sont donn\'es dans le tableau ci-dessous.
\bs
\hfil{Affectation de fonte pour la macro {\color{brown}\verbatim font_kp-light|endverbatim}}\hfil

{\parindent=0pt\settabs4\columns\hfil\vbox{\hrule\hbox{\vrule\hbox{\vbox{\kern1pt\hrule\NoBlackBoxes		 \eightrm\fontss															 \+\hfil	 \textcolor{blue}{ Style}	 &\strut\vrule\strut\hfil	\textcolor{blue}{ Nom de la fonte}	 &\strut\vrule\kern1pt\vrule\strut\hfil	 \textcolor{blue}{ Style}	 &\strut\vrule\strut\hfil	\textcolor{blue}{ Nom de la fonte}	 &\cr	\hrule
\+\hfil	\eightrm Romain	&\strut\vrule\strut\hfil	 jkplmn7t	 &\strut\vrule\kern1pt\vrule\strut\hfil	\eightbf Gras	 &\strut\vrule\strut\hfil	 jkplbn7t	&\cr	 \hrule
\+\hfil	\eighti Maths italique	&\strut\vrule\strut\hfil	 jkplmi	 &\strut\vrule\kern1pt\vrule\strut\hfil	\eighttt Machine \`a \'ecrire	 &\strut\vrule\strut\hfil	 jkpttmn7t	&\cr	 \hrule
\+\hfil	\eightrm Symboles maths	&\strut\vrule\strut\hfil	 jkplsy	 &\strut\vrule\kern1pt\vrule\strut\hfil	\eightitbf Gras italique	 &\strut\vrule\strut\hfil	 jkplbit7t	&\cr	\hrule
\+\hfil	\eightrm Extension maths	&\strut\vrule\strut\hfil	 jkpex	 &\strut\vrule\kern1pt\vrule\strut\hfil	\eightslbf Gras pench\'e	 &\strut\vrule\strut\hfil	 jkplbsl7t	&\cr	\hrule
\+\hfil	\eightit Italique	&\strut\vrule\strut\hfil	 jkplmit7t	 &\strut\vrule\kern1pt\vrule\strut\hfil	 \eightcaps Petites capitales	 &\strut\vrule\strut\hfil	jkplmsc7t	 &\cr	\hrule
\+\hfil	\eightsl Pench\'e	&\strut\vrule\strut\hfil	 jkplmsl7t	 &\strut\vrule\kern1pt\vrule\strut\hfil	 \eightcapsbf Petites capitales en gras	 &\strut\vrule\strut\hfil	 jkplbsc7t	&\cr	\hrule
									
	}\vrule}}\hrule}\hfil}								
									
	\BlackBoxes								

\bs\ii Symboles {\caps ams} associ\'es~: \circledR \ \yen \ $\blacksquare \ \approxeq \ \eqslantgtr \ \curlyeqprec \ \curlyeqsucc \ \preccurlyeq \ \leqq \ \leqslant \ \lessgtr \ \nless \ \nleq \ \nleqslant \ \Bbb R \ \Bbb E \ \Bbb C \ \dots$








\input font_antt   \fontss
\section{\sixteenbf\fontss Antykwa Toru\'nska}{Antykwa Torunska}
\sample
\ii La police Antykwa Toru\'nska est d\'eclar\'ee en entrant l'instruction {\color{brown}\verbatim\input font_antt|endverbatim}. Cette famille de fontes utilise des polices du package  \href{http://www.tex.ac.uk/tex-archive/help/Catalogue/entries/antt.html}{antt} de J.\;M.\;Nowacki, correspondant aux polices texte de Zygfryd Gardzielewski \href{http://nowacki.strefa.pl/torunska-e.html}{Antykwa Toru\'nska}. Zygfryd Gardzielewski a \'elabor\'e Antykwa Toru\'nska en 1960 pour la fonderie Grafmasz \`a Varsovie. On obtient un L barr\'e~(\Lstroke) avec la commande {\color{brown}\verbatim\Lstroke|endverbatim} et un l barr\'e (\lstroke) avec la commande {\color{brown}\verbatim\lstroke|endverbatim}. Pendant l'utilisation de cette macro, les commandes par d\'efaut de plain \capstex\ {\color{brown}\verbatim\L|endverbatim} ou {\color{brown}\verbatim\l|endverbatim} ne marchent pas. Des d\'etails sur cette macro sont donn\'es dans le tableau ci-dessous.
\bs
\hfil{Affectation de fonte pour la macro {\color{brown}\verbatim font_antt|endverbatim}}\hfil

{\parindent=0pt\settabs4\columns\hfil\vbox{\hrule\hbox{\vrule\hbox{\vbox{\kern1pt\hrule\NoBlackBoxes		 \eightrm\fontss							 \+\hfil	 \textcolor{blue}{ Style}	 &\strut\vrule\strut\hfil	 \textcolor{blue}{ Nom de la fonte}	 &\strut\vrule\kern1pt\vrule\strut\hfil	 \textcolor{blue}{ Style}	 &\strut\vrule\strut\hfil	\textcolor{blue}{ Nom de la fonte}	 &\cr	\hrule
\+\hfil	\eightrm Romain	&\strut\vrule\strut\hfil	 rm-anttr	 &\strut\vrule\kern1pt\vrule\strut\hfil	\eightbf Gras	 &\strut\vrule\strut\hfil	 rm-anttb	&\cr	 \hrule
\+\hfil	\eighti Maths italique	&\strut\vrule\strut\hfil	 mi-anttri	 &\strut\vrule\kern1pt\vrule\strut\hfil	\eighttt Machine \`a \'ecire	 &\strut\vrule\strut\hfil	 ly1-zi4r-1	 &\cr	\hrule
\+\hfil	\eightrm Symboles maths	&\strut\vrule\strut\hfil	 sy-anttrz	 &\strut\vrule\kern1pt\vrule\strut\hfil	 \eightitbf Gras italique	 &\strut\vrule\strut\hfil	 rm-anttbi	&\cr	\hrule
\+\hfil	\eightrm Extension maths	&\strut\vrule\strut\hfil	 ex-anttr	 &\strut\vrule\kern1pt\vrule\strut\hfil	 \eightslbf Gras pench\'e	 &\strut\vrule\strut\hfil	rm-anttbi	&\cr	\hrule
\+\hfil	\eightit Italique	&\strut\vrule\strut\hfil	 rm-anttri	 &\strut\vrule\kern1pt\vrule\strut\hfil	 \eightcaps Petites capitales	 &\strut\vrule\strut\hfil	 qx-anttrcap	 &\cr	\hrule
\+\hfil	\eightsl Pench\'e	&\strut\vrule\strut\hfil	 rm-anttri	 &\strut\vrule\kern1pt\vrule\strut\hfil	 \sevencapsbf Petites capitales en gras	 &\strut\vrule\strut\hfil	 rx-anttbcap	&\cr	\hrule
}\vrule}}\hrule}\hfil}								
									
\BlackBoxes								









\input font_antt-light   \fontss
\section{{\sixteenbf\fontss Antykwa Toru\'nska}-{\sixteenslbf Light}}{Antykwa Torunska-Light}
\sample
\ii La police Antykwa Toru\'nska-{\sl Light\/} est d\'eclar\'ee en entrant l'instruction {\color{brown}\verbatim\input font_antt-light|endverbatim}. Cette famille de fontes utilise les polices de graisses l\'eg\`ere et moyenne du package  \href{http://www.tex.ac.uk/tex-archive/help/Catalogue/entries/antt.html}{antt} de J.\;M.\;Nowacki, correspondant aux polices texte de Zygfryd Gardzielewski \href{http://nowacki.strefa.pl/torunska-e.html}{Antykwa Toru\'nska}. Zygfryd Gardzielewski a \'elabor\'e Antykwa Toru\'nska en 1960 pour la fonderie Grafmasz \`a Varsovie. On obtient un L barr\'e~(\Lstroke) avec la commande {\color{brown}\verbatim\Lstroke|endverbatim} et un l barr\'e (\lstroke) avec la commande {\color{brown}\verbatim\lstroke|endverbatim}. Pendant l'utilisation de cette macro, les commandes par d\'efaut de plain \capstex\ {\color{brown}\verbatim\L|endverbatim} ou {\color{brown}\verbatim\l|endverbatim} ne marchent pas. Des d\'etails sur cette macro sont donn\'es dans le tableau ci-dessous.
\bs
\hfil{Affectation de fonte pour la macro {\color{brown}\verbatim font_antt-light|endverbatim}}\hfil	
								
{\parindent=0pt\settabs4\columns\hfil\vbox{\hrule\hbox{\vrule\hbox{\vbox{\kern1pt\hrule\NoBlackBoxes		 \eightrm\fontss																
\+\hfil	\textcolor{blue}{ Style}	&\strut\vrule\strut\hfil	 \textcolor{blue}{ Nom de la fonte}	 &\strut\vrule\kern1pt\vrule\strut\hfil	 \textcolor{blue}{ Style}	 &\strut\vrule\strut\hfil	\textcolor{blue}{ Nom de la fonte}	 &\cr	\hrule
\+\hfil	\eightrm Romain	&\strut\vrule\strut\hfil	 rm-anttl	 &\strut\vrule\kern1pt\vrule\strut\hfil	\eightbf Gras	 &\strut\vrule\strut\hfil	 rm-anttm	&\cr	 \hrule
\+\hfil	\eighti Maths italique	&\strut\vrule\strut\hfil	 mi-anttli	 &\strut\vrule\kern1pt\vrule\strut\hfil	\eighttt Machine \`a \'ecire	 &\strut\vrule\strut\hfil	 ly1-zi4r-1	 &\cr	\hrule
\+\hfil	\eightrm Symboles maths	&\strut\vrule\strut\hfil	 sy-anttlz	 &\strut\vrule\kern1pt\vrule\strut\hfil	 \eightitbf Gras italique	 &\strut\vrule\strut\hfil	 rm-anttmi	&\cr	\hrule
\+\hfil	\eightrm Extension maths	&\strut\vrule\strut\hfil	 ex-anttl	 &\strut\vrule\kern1pt\vrule\strut\hfil	 \eightslbf Gras pench\'e	 &\strut\vrule\strut\hfil	rm-anttmi	&\cr	\hrule
\+\hfil	\eightit Italique	&\strut\vrule\strut\hfil	 rm-anttli	 &\strut\vrule\kern1pt\vrule\strut\hfil	 \eightcaps Petites capitales	 &\strut\vrule\strut\hfil	 qx-anttlcap	 &\cr	\hrule
\+\hfil	\eightsl Pench\'e	&\strut\vrule\strut\hfil	 rm-anttli	 &\strut\vrule\kern1pt\vrule\strut\hfil	 \eightcapsbf Petites capitales en gras	 &\strut\vrule\strut\hfil	 qx-anttmcap	&\cr	\hrule
									
	}\vrule}}\hrule}\hfil}								
									
	\BlackBoxes								








\input font_antt-medium   \fontss
\section{{\sixteenbf\fontss Antykwa Toru\'nska}-{\sixteenslbf Medium}}{Antykwa Torunska-Medium}
\sample
\ii La police Antykwa Toru\'nska-{\sl Medium} est d\'eclar\'ee en entrant l'instruction\break {\color{brown}\verbatim\input font_antt-medium|endverbatim}. Cette famille de fontes utilise les polices de graisses moyenne et gras du package \href{http://www.tex.ac.uk/tex-archive/help/Catalogue/entries/antt.html}{antt} de J.\;M.\;Nowacki, correspondant aux polices texte de Zygfryd Gardzielewski \href{http://nowacki.strefa.pl/torunska-e.html}{Antykwa Toru\'nska}. Zygfryd Gardzielewski a \'elabor\'e Antykwa Toru\'nska en 1960 pour la fonderie Grafmasz \`a Varsovie. On obtient un L barr\'e~(\Lstroke) avec la commande {\color{brown}\verbatim\Lstroke|endverbatim} et un l barr\'e (\lstroke) avec la commande {\color{brown}\verbatim\lstroke|endverbatim}. Pendant l'utilisation de cette macro, les commandes par d\'efaut de plain \capstex\ {\color{brown}\verbatim\L|endverbatim} ou {\color{brown}\verbatim\l|endverbatim} ne marchent pas. Des d\'etails sur cette macro sont donn\'es dans le tableau ci-dessous.
\bs
\hfil{Affectation de fonte pour la macro {\color{brown}\verbatim font_antt-medium|endverbatim}}\hfil	
								
{\parindent=0pt\settabs4\columns\hfil\vbox{\hrule\hbox{\vrule\hbox{\vbox{\kern1pt\hrule\NoBlackBoxes		 \eightrm\fontss							
									
\+\hfil	\textcolor{blue}{ Style}	&\strut\vrule\strut\hfil	 \textcolor{blue}{ Nom de la fonte}	 &\strut\vrule\kern1pt\vrule\strut\hfil	 \textcolor{blue}{ Style}	 &\strut\vrule\strut\hfil	\textcolor{blue}{ Nom de la fonte}	 &\cr	\hrule
\+\hfil	\eightrm Romain	&\strut\vrule\strut\hfil	 rm-anttm	 &\strut\vrule\kern1pt\vrule\strut\hfil	\eightbf Gras	 &\strut\vrule\strut\hfil	 rm-anttb	&\cr	 \hrule
\+\hfil	\eighti Maths italique	&\strut\vrule\strut\hfil	 mi-anttmi	 &\strut\vrule\kern1pt\vrule\strut\hfil	\eighttt Machine \`a \'ecire	 &\strut\vrule\strut\hfil	 ly1-zi4r-1	 &\cr	\hrule
\+\hfil	\eightrm Symboles maths	&\strut\vrule\strut\hfil	 sy-anttmz	 &\strut\vrule\kern1pt\vrule\strut\hfil	 \eightitbf Gras italique	 &\strut\vrule\strut\hfil	 rm-anttbi	&\cr	\hrule
\+\hfil	\eightrm Extension maths	&\strut\vrule\strut\hfil	 ex-anttm	 &\strut\vrule\kern1pt\vrule\strut\hfil	 \eightslbf Gras pench\'e	 &\strut\vrule\strut\hfil	rm-anttbi	&\cr	\hrule
\+\hfil	\eightit Italique	&\strut\vrule\strut\hfil	 rm-anttmi	 &\strut\vrule\kern1pt\vrule\strut\hfil	 \eightcaps Petites capitales	 &\strut\vrule\strut\hfil	 qx-anttmcap	 &\cr	\hrule
\+\hfil	\eightsl Pench\'e	&\strut\vrule\strut\hfil	 rm-anttmi	 &\strut\vrule\kern1pt\vrule\strut\hfil	 \sevencapsbf Petites capitales en gras	 &\strut\vrule\strut\hfil	 qx-anttbcap	&\cr	\hrule
									
	}\vrule}}\hrule}\hfil}								
									
	\BlackBoxes								













\input font_antt-condensed   \fontss
\section{{\sixteenbf\fontss Antykwa Toru\'nska}-{\sixteenslbf Condensed}}{Antykwa Torunska-Condensed}
\sample
\ii La police Antykwa Toru\'nska-{\sl Condensed\/} est d\'eclar\'ee en entrant l'instruction\break {\color{brown}\verbatim\input font_antt-condensed|endverbatim}. Cette famille de fontes utilise les polices de largeur condens\'ee et graisses normal et gras du package  \href{http://www.tex.ac.uk/tex-archive/help/Catalogue/entries/antt.html}{antt} de J.\;M.\;Nowacki, correspondant aux polices texte de Zygfryd Gardzielewski \href{http://nowacki.strefa.pl/torunska-e.html}{Antykwa Toru\'nska}. Zygfryd Gardzielewski a \'elabor\'e Antykwa Toru\'nska en 1960 pour la fonderie Grafmasz \`a Varsovie. On obtient un L barr\'e~(\Lstroke) avec la commande {\color{brown}\verbatim\Lstroke|endverbatim} et un l barr\'e (\lstroke) avec la commande {\color{brown}\verbatim\lstroke|endverbatim}. Pendant l'utilisation de cette macro, les commandes par d\'efaut de plain \capstex\ {\color{brown}\verbatim\L|endverbatim} ou {\color{brown}\verbatim\l|endverbatim} ne marchent pas. Des d\'etails sur cette macro sont donn\'es dans le tableau ci-dessous.
\bs
\hfil{Affectation de fonte pour la macro {\color{brown}\verbatim font_antt-condensed|endverbatim}}\hfil	
													
{\parindent=0pt\settabs4\columns\hfil\vbox{\hrule\hbox{\vrule\hbox{\vbox{\kern1pt\hrule\NoBlackBoxes		 \eightrm\fontss							
									
\+\hfil	\textcolor{blue}{ Style}	&\strut\vrule\strut\hfil	 \textcolor{blue}{ Nom de la fonte}	 &\strut\vrule\kern1pt\vrule\strut\hfil	 \textcolor{blue}{ Style}	 &\strut\vrule\strut\hfil	\textcolor{blue}{ Nom de la fonte}	 &\cr	\hrule
\+\hfil	\eightrm Romain	&\strut\vrule\strut\hfil	 rm-anttcr	 &\strut\vrule\kern1pt\vrule\strut\hfil	\eightbf Gras	 &\strut\vrule\strut\hfil	 rm-anttcb	&\cr	 \hrule
\+\hfil	\eighti Maths italique	&\strut\vrule\strut\hfil	 mi-anttcri	 &\strut\vrule\kern1pt\vrule\strut\hfil	\eighttt Machine \`a \'ecire	 &\strut\vrule\strut\hfil	 ly1-zi4r-1	 &\cr	\hrule
\+\hfil	\eightrm Symboles maths	&\strut\vrule\strut\hfil	 sy-anttcrz	 &\strut\vrule\kern1pt\vrule\strut\hfil	 \eightitbf Gras italique	 &\strut\vrule\strut\hfil	 rm-anttcbi	&\cr	\hrule
\+\hfil	\eightrm Extension maths	&\strut\vrule\strut\hfil	 ex-anttcr	 &\strut\vrule\kern1pt\vrule\strut\hfil	 \eightslbf Gras pench\'e	 &\strut\vrule\strut\hfil	rm-anttcbi	&\cr	\hrule
\+\hfil	\eightit Italique	&\strut\vrule\strut\hfil	 rm-anttcri	 &\strut\vrule\kern1pt\vrule\strut\hfil	 \eightcaps Petites capitales	 &\strut\vrule\strut\hfil	 qx-anttcrcap	 &\cr	\hrule
\+\hfil	\eightsl Pench\'e	&\strut\vrule\strut\hfil	 rm-anttcri	 &\strut\vrule\kern1pt\vrule\strut\hfil	 \eightcapsbf Petites capitales en gras	 &\strut\vrule\strut\hfil	 qx-anttcbcap	&\cr	\hrule
									
	}\vrule}}\hrule}\hfil}								
									
	\BlackBoxes								













\input font_antt-condensed-light   \fontss
\section{{\sixteenbf\fontss Antykwa Toru\'nska}-{\sixteenslbf Condensed Light}}{Antykwa Torunska-Condensed Light}
\sample
\ii La police Antykwa Toru\'nska-{\sl Condensed Light\/} est d\'eclar\'ee en entrant l'instruction\break {\color{brown}\verbatim\input font_antt-condensed-light|endverbatim}. Cette famille de fontes utilise les polices de largeur condens\'ee et graisses l\'eg\`ere et moyenne du package \href{http://www.tex.ac.uk/tex-archive/help/Catalogue/entries/antt.html}{antt} de J.\;M.\;Nowacki, correspondant aux polices texte de Zygfryd Gardzielewski \href{http://nowacki.strefa.pl/torunska-e.html}{Antykwa Toru\'nska}. Zygfryd Gardzielewski a \'elabor\'e Antykwa Toru\'nska en 1960 pour la fonderie Grafmasz \`a Varsovie. On obtient un L barr\'e~(\Lstroke) avec la commande {\color{brown}\verbatim\Lstroke|endverbatim} et un l barr\'e (\lstroke) avec la commande {\color{brown}\verbatim\lstroke|endverbatim}. Pendant l'utilisation de cette macro, les commandes par d\'efaut de plain \capstex\ {\color{brown}\verbatim\L|endverbatim} ou {\color{brown}\verbatim\l|endverbatim} ne marchent pas. Des d\'etails sur cette macro sont donn\'es dans le tableau ci-dessous.
\bs
\hfil{Affectation de fonte pour la macro {\color{brown}\verbatim font_antt-condensed-light|endverbatim}}\hfil

			% Antykwa Toru?ska-Light						
{\parindent=0pt\settabs4\columns\hfil\vbox{\hrule\hbox{\vrule\hbox{\vbox{\kern1pt\hrule\NoBlackBoxes		 \eightrm\fontss							
									
\+\hfil	\textcolor{blue}{ Style}	&\strut\vrule\strut\hfil	 \textcolor{blue}{ Nom de la fonte}	 &\strut\vrule\kern1pt\vrule\strut\hfil	 \textcolor{blue}{ Style}	 &\strut\vrule\strut\hfil	\textcolor{blue}{ Nom de la fonte}	 &\cr	\hrule
\+\hfil	\eightrm Romain	&\strut\vrule\strut\hfil	 rm-anttcl	 &\strut\vrule\kern1pt\vrule\strut\hfil	\eightbf Gras	 &\strut\vrule\strut\hfil	 rm-anttcm	&\cr	 \hrule
\+\hfil	\eighti Maths italique	&\strut\vrule\strut\hfil	 mi-anttcli	 &\strut\vrule\kern1pt\vrule\strut\hfil	\eighttt Machine \`a \'ecire	 &\strut\vrule\strut\hfil	 ly1-zi4r-1	 &\cr	\hrule
\+\hfil	\eightrm Symboles maths	&\strut\vrule\strut\hfil	 sy-anttclz	 &\strut\vrule\kern1pt\vrule\strut\hfil	 \eightitbf Gras italique	 &\strut\vrule\strut\hfil	 rm-anttcmi	&\cr	\hrule
\+\hfil	\eightrm Extension maths	&\strut\vrule\strut\hfil	 ex-anttcl	 &\strut\vrule\kern1pt\vrule\strut\hfil	 \eightslbf Gras pench\'e	 &\strut\vrule\strut\hfil	rm-anttcmi	&\cr	\hrule
\+\hfil	\eightit Italique	&\strut\vrule\strut\hfil	 rm-anttcli	 &\strut\vrule\kern1pt\vrule\strut\hfil	 \eightcaps Petites capitales	 &\strut\vrule\strut\hfil	 qx-anttclcap	 &\cr	\hrule
\+\hfil	\eightsl Pench\'e	&\strut\vrule\strut\hfil	 rm-anttcli	 &\strut\vrule\kern1pt\vrule\strut\hfil	 \eightcapsbf Petites capitales en gras	 &\strut\vrule\strut\hfil	 qx-anttcmcap	&\cr	\hrule
									
	}\vrule}}\hrule}\hfil}								
									
	\BlackBoxes								


















\input font_antt-condensed-medium   \fontss
\section{{\sixteenbf\fontss Antykwa Toru\'nska}-{\sixteenslbf Condensed Medium}}{Antykwa Torunska-Condensed Medium}
\sample
\ii La police Antykwa Toru\'nska-{\sl Condensed Medium\/} peut \^etre uilis\'ee dans les documents  \capstex\ apr\`es avoir tap\'e l'instruction {\color{brown}\verbatim\input font_antt-condensed-medium|endverbatim}. Cette famille de fontes utilise les polices de largeur condens\'ee et graisses moyenne et gras du package \href{http://www.tex.ac.uk/tex-archive/help/Catalogue/entries/antt.html}{antt} de J.\;M.\;Nowacki, correspondant aux polices texte de Zygfryd Gardzielewski \href{http://nowacki.strefa.pl/torunska-e.html}{Antykwa Toru\'nska}. Zygfryd Gardzielewski a \'elabor\'e Antykwa Toru\'nska en 1960 pour la fonderie Grafmasz \`a Varsovie. On obtient un L barr\'e~(\Lstroke) avec la commande {\color{brown}\verbatim\Lstroke|endverbatim} et un l barr\'e (\lstroke) avec la commande {\color{brown}\verbatim\lstroke|endverbatim}. Pendant l'utilisation de cette macro, les commandes par d\'efaut de plain \capstex\ {\color{brown}\verbatim\L|endverbatim} ou {\color{brown}\verbatim\l|endverbatim} ne marchent pas. Des d\'etails sur cette macro sont donn\'es dans le tableau ci-dessous.
\bs
\hfil{Affectation de fonte pour la macro {\color{brown}\verbatim font_antt-condensed-medium|endverbatim}}\hfil	

							
{\parindent=0pt\settabs4\columns\hfil\vbox{\hrule\hbox{\vrule\hbox{\vbox{\kern1pt\hrule\NoBlackBoxes		 \eightrm\fontss							
									
\+\hfil	\textcolor{blue}{ Style}	&\strut\vrule\strut\hfil	 \textcolor{blue}{ Nom de la fonte}	 &\strut\vrule\kern1pt\vrule\strut\hfil	 \textcolor{blue}{ Style}	 &\strut\vrule\strut\hfil	\textcolor{blue}{ Nom de la fonte}	 &\cr	\hrule
\+\hfil	\eightrm Romain	&\strut\vrule\strut\hfil	 rm-anttcm	 &\strut\vrule\kern1pt\vrule\strut\hfil	\eightbf Gras	 &\strut\vrule\strut\hfil	 rm-anttcb	&\cr	 \hrule
\+\hfil	\eighti Maths italique	&\strut\vrule\strut\hfil	 mi-anttcmi	 &\strut\vrule\kern1pt\vrule\strut\hfil	\eighttt Machine \`a \'ecire	 &\strut\vrule\strut\hfil	 ly1-zi4r-1	 &\cr	\hrule
\+\hfil	\eightrm Symboles maths	&\strut\vrule\strut\hfil	 sy-anttcmz	 &\strut\vrule\kern1pt\vrule\strut\hfil	 \eightitbf Gras italique	 &\strut\vrule\strut\hfil	 rm-anttcbi	&\cr	\hrule
\+\hfil	\eightrm Extension maths	&\strut\vrule\strut\hfil	 ex-anttcm	 &\strut\vrule\kern1pt\vrule\strut\hfil	 \eightslbf Gras pench\'e	 &\strut\vrule\strut\hfil	rm-anttcbi	&\cr	\hrule
\+\hfil	\eightit Italique	&\strut\vrule\strut\hfil	 rm-anttcmi	 &\strut\vrule\kern1pt\vrule\strut\hfil	 \eightcaps Petites capitales	 &\strut\vrule\strut\hfil	 qx-anttcmcap	 &\cr	\hrule
\+\hfil	\eightsl Pench\'e	&\strut\vrule\strut\hfil	 rm-anttcmi	 &\strut\vrule\kern1pt\vrule\strut\hfil	 \eightcapsbf Petites capitales en gras	 &\strut\vrule\strut\hfil	 qx-anttcbcap	&\cr	\hrule
									
	}\vrule}}\hrule}\hfil}								
									
	\BlackBoxes								















\input font_iwona   \fontss
\section{\sixteenbf\fontss Iwona}{Iwona}
\sample
\ii La police Iwona est d\'eclar\'ee en entrant l'instruction {\color{brown}\verbatim\input font_iwona|endverbatim}. Cette famille de fontes utilise des polices du package \href{http://www.tex.ac.uk/tex-archive/help/Catalogue/entries/iwona.html}{iwona} de J.\;M. Nowacki, correspondant aux polices texte de Ma\lstroke{}gorzata Budyta. On obtient un L barr\'e~(\Lstroke) avec la commande {\color{brown}\verbatim\Lstroke|endverbatim} et un l barr\'e (\lstroke) avec la commande {\color{brown}\verbatim\lstroke|endverbatim}. Pendant l'utilisation de cette macro, les commandes par d\'efaut de plain \capstex\ {\color{brown}\verbatim\L|endverbatim} ou {\color{brown}\verbatim\l|endverbatim} ne marchent pas. Des d\'etails sur cette macro sont donn\'es dans le tableau ci-dessous.
\bs
\hfil{Affectation de fonte pour la macro {\color{brown}\verbatim font_iwona|endverbatim}}\hfil

{\parindent=0pt\settabs4\columns\hfil\vbox{\hrule\hbox{\vrule\hbox{\vbox{\kern1pt\hrule\NoBlackBoxes		 \eightrm\fontss							 \+\hfil	 \textcolor{blue}{ Style}	 &\strut\vrule\strut\hfil	 \textcolor{blue}{ Nom de la fonte}	 &\strut\vrule\kern1pt\vrule\strut\hfil	 \textcolor{blue}{ Style}	 &\strut\vrule\strut\hfil	 \textcolor{blue}{ Nom de la fonte}	 &\cr	\hrule
\+\hfil	\eightrm Romain	&\strut\vrule\strut\hfil	 rm-iwonar	 &\strut\vrule\kern1pt\vrule\strut\hfil	 \eightbf Gras	 &\strut\vrule\strut\hfil	 rm-iwonab	 &\cr	\hrule
\+\hfil	\eighti Maths italique	&\strut\vrule\strut\hfil	 mi-iwonari	 &\strut\vrule\kern1pt\vrule\strut\hfil	 \eighttt Machine \`a \'ecire	 &\strut\vrule\strut\hfil	 ly1-zi4r-1	&\cr	\hrule
\+\hfil	\eightrm Symboles maths	&\strut\vrule\strut\hfil	 sy-iwonarz	 &\strut\vrule\kern1pt\vrule\strut\hfil	 \eightitbf Gras italique	 &\strut\vrule\strut\hfil	 rm-iwonabi	&\cr	\hrule
\+\hfil	\eightrm Extension maths	&\strut\vrule\strut\hfil	 ex-iwonar	 &\strut\vrule\kern1pt\vrule\strut\hfil	 \eightslbf Gras pench\'e	 &\strut\vrule\strut\hfil	 rm-iwonabi	&\cr	\hrule
\+\hfil	\eightit Italique	&\strut\vrule\strut\hfil	 rm-iwonari	 &\strut\vrule\kern1pt\vrule\strut\hfil	 \eightcaps Petites capitales	 &\strut\vrule\strut\hfil	 qx-iwonarcap	 &\cr	\hrule
\+\hfil	\eightsl Pench\'e	&\strut\vrule\strut\hfil	 rm-iwonari	 &\strut\vrule\kern1pt\vrule\strut\hfil	 \eightcapsbf Petites capitales en gras	 &\strut\vrule\strut\hfil	 qx-iwonabcap	&\cr	\hrule
}\vrule}}\hrule}\hfil}								
									
\BlackBoxes	










\input font_iwona-light   \fontss
\section{\sixteenbf\fontss Iwona-{\sixteenslbf Light}}{Iwona-Light}
\sample
\ii La police Iwona-{\sl Light\/} est d\'eclar\'ee en entrant l'instruction {\color{brown}\verbatim\input font_iwona-light|endverbatim}. Cette famille de fontes utilise les polices Iwona de graisses l\'eg\`ere et gras du package \href{http://www.tex.ac.uk/tex-archive/help/Catalogue/entries/iwona.html}{iwona} de J.\;M. Nowacki, correspondant aux polices texte de Ma\lstroke{}gorzata Budyta. On obtient un L barr\'e~(\Lstroke) avec la commande {\color{brown}\verbatim\Lstroke|endverbatim} et un l barr\'e (\lstroke) avec la commande {\color{brown}\verbatim\lstroke|endverbatim}. Pendant l'utilisation de cette macro, les commandes par d\'efaut de plain \capstex\ {\color{brown}\verbatim\L|endverbatim} ou {\color{brown}\verbatim\l|endverbatim} ne marchent pas. Des d\'etails sur cette macro sont donn\'es dans le tableau ci-dessous.
\bs
\hfil{Affectation de fonte pour la macro {\color{brown}\verbatim font_iwona-light|endverbatim}}\hfil

{\parindent=0pt\settabs4\columns\hfil\vbox{\hrule\hbox{\vrule\hbox{\vbox{\kern1pt\hrule\NoBlackBoxes		 \eightrm\fontss							
									
\+\hfil	\textcolor{blue}{ Style}	&\strut\vrule\strut\hfil	 \textcolor{blue}{ Nom de la fonte}	 &\strut\vrule\kern1pt\vrule\strut\hfil	 \textcolor{blue}{ Style}	 &\strut\vrule\strut\hfil	\textcolor{blue}{ Nom de la fonte}	&\cr	\hrule
\+\hfil	\eightrm Romain	&\strut\vrule\strut\hfil	 rm-iwonal	 &\strut\vrule\kern1pt\vrule\strut\hfil	\eightbf Gras	 &\strut\vrule\strut\hfil	 rm-iwonam	&\cr	 \hrule
\+\hfil	\eighti Maths italique	&\strut\vrule\strut\hfil	 mi-iwonali	 &\strut\vrule\kern1pt\vrule\strut\hfil	\eighttt Machine \`a \'ecire	 &\strut\vrule\strut\hfil	 ly1-zi4r-1	 &\cr	\hrule
\+\hfil	\eightrm Symboles maths	&\strut\vrule\strut\hfil	 sy-iwonalz	 &\strut\vrule\kern1pt\vrule\strut\hfil	 \eightitbf Gras italique	 &\strut\vrule\strut\hfil	 rm-iwonami	&\cr	\hrule
\+\hfil	\eightrm Extension maths	&\strut\vrule\strut\hfil	 ex-iwonal	 &\strut\vrule\kern1pt\vrule\strut\hfil	 \eightslbf Gras pench\'e	 &\strut\vrule\strut\hfil	rm-iwonami	&\cr	\hrule
\+\hfil	\eightit Italique	&\strut\vrule\strut\hfil	 rm-iwonali	 &\strut\vrule\kern1pt\vrule\strut\hfil	 \eightcaps Petites capitales	 &\strut\vrule\strut\hfil	 qx-iwonalcap	 &\cr	\hrule
\+\hfil	\eightsl Pench\'e	&\strut\vrule\strut\hfil	 rm-iwonali	 &\strut\vrule\kern1pt\vrule\strut\hfil	 \eightcapsbf Petites capitales en gras	 &\strut\vrule\strut\hfil	 qx-iwonamcap	&\cr	\hrule
									
	}\vrule}}\hrule}\hfil}								
									
	\BlackBoxes								










\input font_iwona-medium   \fontss
\section{\sixteenbf\fontss Iwona-{\sixteenslbf Medium}}{Iwona-Medium}
\sample
\ii La police Iwona-{\sl Medium\/} est d\'eclar\'ee en entrant l'instruction {\color{brown}\verbatim\input font_iwona-medium|endverbatim}. Cette famille de fontes utilise les polices Iwona de graisses moyenne et gras du package \href{http://www.tex.ac.uk/tex-archive/help/Catalogue/entries/iwona.html}{iwona} de J.\;M. Nowacki, correspondant aux polices texte de Ma\lstroke{}gorzata Budyta. On obtient un L barr\'e~(\Lstroke) avec la commande {\color{brown}\verbatim\Lstroke|endverbatim} et un l barr\'e (\lstroke) avec la commande {\color{brown}\verbatim\lstroke|endverbatim}. Pendant l'utilisation de cette macro, les commandes par d\'efaut de plain \capstex\ {\color{brown}\verbatim\L|endverbatim} ou {\color{brown}\verbatim\l|endverbatim} ne marchent pas. Des d\'etails sur cette macro sont donn\'es dans le tableau ci-dessous.
\bs
\hfil{Affectation de fonte pour la macro {\color{brown}\verbatim font_iwona-medium|endverbatim}}\hfil

{\parindent=0pt\settabs4\columns\hfil\vbox{\hrule\hbox{\vrule\hbox{\vbox{\kern1pt\hrule\NoBlackBoxes		 \eightrm\fontss							
									
\+\hfil	\textcolor{blue}{ Style}	&\strut\vrule\strut\hfil	 \textcolor{blue}{ Nom de la fonte}	 &\strut\vrule\kern1pt\vrule\strut\hfil	 \textcolor{blue}{ Style}	 &\strut\vrule\strut\hfil	\textcolor{blue}{ Nom de la fonte}	&\cr	\hrule
\+\hfil	\eightrm Romain	&\strut\vrule\strut\hfil	 rm-iwonam	 &\strut\vrule\kern1pt\vrule\strut\hfil	\eightbf Gras	 &\strut\vrule\strut\hfil	 rm-iwonah	&\cr	 \hrule
\+\hfil	\eighti Maths italique	&\strut\vrule\strut\hfil	 mi-iwonami	 &\strut\vrule\kern1pt\vrule\strut\hfil	\eighttt Machine \`a \'ecire	 &\strut\vrule\strut\hfil	 ly1-zi4r-1	 &\cr	\hrule
\+\hfil	\eightrm Symboles maths	&\strut\vrule\strut\hfil	 sy-iwonamz	 &\strut\vrule\kern1pt\vrule\strut\hfil	 \eightitbf Gras italique	 &\strut\vrule\strut\hfil	 rm-iwonahi	&\cr	\hrule
\+\hfil	\eightrm Extension maths	&\strut\vrule\strut\hfil	 ex-iwonam	 &\strut\vrule\kern1pt\vrule\strut\hfil	 \eightslbf Gras pench\'e	 &\strut\vrule\strut\hfil	rm-iwonahi	&\cr	\hrule
\+\hfil	\eightit Italique	&\strut\vrule\strut\hfil	 rm-iwonami	 &\strut\vrule\kern1pt\vrule\strut\hfil	 \eightcaps Petites capitales	 &\strut\vrule\strut\hfil	 qx-iwonamcap	 &\cr	\hrule
\+\hfil	\eightsl Pench\'e	&\strut\vrule\strut\hfil	 rm-iwonami	 &\strut\vrule\kern1pt\vrule\strut\hfil	 \eightcapsbf Petites capitales en gras	 &\strut\vrule\strut\hfil	 qx-iwonahcap	&\cr	\hrule
									
	}\vrule}}\hrule}\hfil}								
									
	\BlackBoxes								






\input font_iwona-bold   \fontss
\section{\sixteenbf\fontss Iwona-{\sixteenslbf Bold}}{Iwona-Bold}
\sample
\ii La police Iwona-{\sl Bold\/} est d\'eclar\'ee en entrant l'instruction {\color{brown}\verbatim\input font_iwona-bold|endverbatim}. Cette famille de fontes utilise les fontes grasses Iwona du package \href{http://www.tex.ac.uk/tex-archive/help/Catalogue/entries/iwona.html}{iwona} de J.\;M. Nowacki, correspondant aux polices texte de Ma\lstroke{}gorzata Budyta. On obtient un L barr\'e~(\Lstroke) avec la commande {\color{brown}\verbatim\Lstroke|endverbatim} et un l barr\'e (\lstroke) avec la commande {\color{brown}\verbatim\lstroke|endverbatim}. Pendant l'utilisation de cette macro, les commandes par d\'efaut de plain \capstex\ {\color{brown}\verbatim\L|endverbatim} ou {\color{brown}\verbatim\l|endverbatim} ne marchent pas. Des d\'etails sur cette macro sont donn\'es dans le tableau ci-dessous.
\bs
\hfil{Affectation de fonte pour la macro {\color{brown}\verbatim font_iwona-medium|endverbatim}}\hfil

{\parindent=0pt\settabs4\columns\hfil\vbox{\hrule\hbox{\vrule\hbox{\vbox{\kern1pt\hrule\NoBlackBoxes		 \eightrm\fontss							
									
\+\hfil	\textcolor{blue}{ Style}	&\strut\vrule\strut\hfil	 \textcolor{blue}{ Nom de la fonte}	 &\strut\vrule\kern1pt\vrule\strut\hfil	 \textcolor{blue}{ Style}	 &\strut\vrule\strut\hfil	\textcolor{blue}{ Nom de la fonte}	&\cr	\hrule
\+\hfil	\eightrm Romain	&\strut\vrule\strut\hfil	 rm-iwonab	 &\strut\vrule\kern1pt\vrule\strut\hfil	\eightbf Gras	 &\strut\vrule\strut\hfil	 rm-iwonah	&\cr	 \hrule
\+\hfil	\eighti Maths italique	&\strut\vrule\strut\hfil	 mi-iwonabi	 &\strut\vrule\kern1pt\vrule\strut\hfil	\eighttt Machine \`a \'ecire	 &\strut\vrule\strut\hfil	 ly1-zi4r-1	 &\cr	\hrule
\+\hfil	\eightrm Symboles maths	&\strut\vrule\strut\hfil	 sy-iwonabz	 &\strut\vrule\kern1pt\vrule\strut\hfil	 \eightitbf Gras italique	 &\strut\vrule\strut\hfil	 rm-iwonahi	&\cr	\hrule
\+\hfil	\eightrm Extension maths	&\strut\vrule\strut\hfil	 ex-iwonab	 &\strut\vrule\kern1pt\vrule\strut\hfil	 \eightslbf Gras pench\'e	 &\strut\vrule\strut\hfil	rm-iwonahi	&\cr	\hrule
\+\hfil	\eightit Italique	&\strut\vrule\strut\hfil	 rm-iwonabi	 &\strut\vrule\kern1pt\vrule\strut\hfil	 \eightcaps Petites capitales	 &\strut\vrule\strut\hfil	 qx-iwonabcap	 &\cr	\hrule
\+\hfil	\eightsl Pench\'e	&\strut\vrule\strut\hfil	 rm-iwonabi	 &\strut\vrule\kern1pt\vrule\strut\hfil	 \eightcapsbf Petites capitales en gras	 &\strut\vrule\strut\hfil	 qx-iwonahcap	&\cr	\hrule
									
	}\vrule}}\hrule}\hfil}								
									
	\BlackBoxes								








\input font_iwona-condensed   \fontss
\section{\sixteenbf\fontss Iwona-{\sixteenslbf Condensed}}{Iwona-Condensed}
\sample
\ii La police Iwona-{\sl Condensed\/} est d\'eclar\'ee en entrant l'instruction {\color{brown}\verbatim\input font_iwona-condensed|endverbatim}. Cette famille de fontes utilise les polices Iwona en largeur condens\'ee et styles gras du package \href{http://www.tex.ac.uk/tex-archive/help/Catalogue/entries/iwona.html}{iwona} de J.\;M. Nowacki, correspondant aux polices texte de Ma\lstroke{}gorzata Budyta. On obtient un L barr\'e~(\Lstroke) avec la commande {\color{brown}\verbatim\Lstroke|endverbatim} et un l barr\'e (\lstroke) avec la commande {\color{brown}\verbatim\lstroke|endverbatim}. Pendant l'utilisation de cette macro, les commandes par d\'efaut de plain \capstex\ {\color{brown}\verbatim\L|endverbatim} ou {\color{brown}\verbatim\l|endverbatim} ne marchent pas. Des d\'etails sur cette macro sont donn\'es dans le tableau ci-dessous.
\bs
\hfil{Affectation de fonte pour la macro {\color{brown}\verbatim font_iwona-condensed|endverbatim}}\hfil

{\parindent=0pt\settabs4\columns\hfil\vbox{\hrule\hbox{\vrule\hbox{\vbox{\kern1pt\hrule\NoBlackBoxes		 \eightrm\fontss							
									
\+\hfil	\textcolor{blue}{ Style}	&\strut\vrule\strut\hfil	 \textcolor{blue}{ Nom de la fonte}	 &\strut\vrule\kern1pt\vrule\strut\hfil	 \textcolor{blue}{ Style}	 &\strut\vrule\strut\hfil	\textcolor{blue}{ Nom de la fonte}	&\cr	\hrule
\+\hfil	\eightrm Romain	&\strut\vrule\strut\hfil	 rm-iwonacr	 &\strut\vrule\kern1pt\vrule\strut\hfil	\eightbf Gras	 &\strut\vrule\strut\hfil	 rm-iwonacb	&\cr	 \hrule
\+\hfil	\eighti Maths italique	&\strut\vrule\strut\hfil	 mi-iwonacri	 &\strut\vrule\kern1pt\vrule\strut\hfil	\eighttt Machine \`a \'ecire	 &\strut\vrule\strut\hfil	 ly1-zi4r-1	 &\cr	\hrule
\+\hfil	\eightrm Symboles maths	&\strut\vrule\strut\hfil	 sy-iwonacrz	 &\strut\vrule\kern1pt\vrule\strut\hfil	 \eightitbf Gras italique	 &\strut\vrule\strut\hfil	 rm-iwonacbi	&\cr	\hrule
\+\hfil	\eightrm Extension maths	&\strut\vrule\strut\hfil	 ex-iwonacr	 &\strut\vrule\kern1pt\vrule\strut\hfil	 \eightslbf Gras pench\'e	 &\strut\vrule\strut\hfil	rm-iwonacbi	&\cr	\hrule
\+\hfil	\eightit Italique	&\strut\vrule\strut\hfil	 rm-iwonacri	 &\strut\vrule\kern1pt\vrule\strut\hfil	 \eightcaps Petites capitales	 &\strut\vrule\strut\hfil	 qx-iwonacrcap	 &\cr	\hrule
\+\hfil	\eightsl Pench\'e	&\strut\vrule\strut\hfil	 rm-iwonacri	 &\strut\vrule\kern1pt\vrule\strut\hfil	 \eightcapsbf Petites capitales en gras	 &\strut\vrule\strut\hfil	 qx-iwonacbcap	&\cr	\hrule
									
	}\vrule}}\hrule}\hfil}								
									
	\BlackBoxes								







\input font_iwona-condensed-light   \fontss
\section{\sixteenbf\fontss Iwona-{\sixteenslbf Condensed Light}}{Iwona-Condensed Light}
\sample
\ii La police Iwona-{\sl Condensed Light\/} est d\'eclar\'ee en entrant l'instruction {\color{brown}\verbatim\input font_iwona-condensed-light|endverbatim}. Cette famille de fontes utilise les polices Iwona en largeur condens\'ee et graisses moyenne et l\'eg\`ere du package \href{http://www.tex.ac.uk/tex-archive/help/Catalogue/entries/iwona.html}{iwona} de J.\;M. Nowacki, correspondant aux polices texte de Ma\lstroke{}gorzata Budyta. On obtient un L barr\'e~(\Lstroke) avec la commande {\color{brown}\verbatim\Lstroke|endverbatim} et un l barr\'e (\lstroke) avec la commande {\color{brown}\verbatim\lstroke|endverbatim}. Pendant l'utilisation de cette macro, les commandes par d\'efaut de plain \capstex\ {\color{brown}\verbatim\L|endverbatim} ou {\color{brown}\verbatim\l|endverbatim} ne marchent pas. Des d\'etails sur cette macro sont donn\'es dans le tableau ci-dessous.
\bs
\hfil{Affectation de fonte pour la macro {\color{brown}\verbatim font_iwona-condensed-light|endverbatim}}\hfil

{\parindent=0pt\settabs4\columns\hfil\vbox{\hrule\hbox{\vrule\hbox{\vbox{\kern1pt\hrule\NoBlackBoxes		 \eightrm\fontss							
									
\+\hfil	\textcolor{blue}{ Style}	&\strut\vrule\strut\hfil	 \textcolor{blue}{ Nom de la fonte}	 &\strut\vrule\kern1pt\vrule\strut\hfil	 \textcolor{blue}{ Style}	 &\strut\vrule\strut\hfil	\textcolor{blue}{ Nom de la fonte}	&\cr	\hrule
\+\hfil	\eightrm Romain	&\strut\vrule\strut\hfil	 rm-iwonacl	 &\strut\vrule\kern1pt\vrule\strut\hfil	\eightbf Gras	 &\strut\vrule\strut\hfil	 rm-iwonacm	&\cr	 \hrule
\+\hfil	\eighti Maths italique	&\strut\vrule\strut\hfil	 mi-iwonacli	 &\strut\vrule\kern1pt\vrule\strut\hfil	\eighttt Machine \`a \'ecire	 &\strut\vrule\strut\hfil	 ly1-zi4r-1	 &\cr	\hrule
\+\hfil	\eightrm Symboles maths	&\strut\vrule\strut\hfil	 sy-iwonaclz	 &\strut\vrule\kern1pt\vrule\strut\hfil	 \eightitbf Gras italique	 &\strut\vrule\strut\hfil	 rm-iwonacmi	&\cr	\hrule
\+\hfil	\eightrm Extension maths	&\strut\vrule\strut\hfil	 ex-iwonacl	 &\strut\vrule\kern1pt\vrule\strut\hfil	 \eightslbf Gras pench\'e	 &\strut\vrule\strut\hfil	rm-iwonacmi	&\cr	\hrule
\+\hfil	\eightit Italique	&\strut\vrule\strut\hfil	 rm-iwonacli	 &\strut\vrule\kern1pt\vrule\strut\hfil	 \eightcaps Petites capitales	 &\strut\vrule\strut\hfil	 qx-iwonaclcap	 &\cr	\hrule
\+\hfil	\eightsl Pench\'e	&\strut\vrule\strut\hfil	 rm-iwonacli	 &\strut\vrule\kern1pt\vrule\strut\hfil	 \eightcapsbf Petites capitales en gras	 &\strut\vrule\strut\hfil	 qx-iwonacmcap	&\cr	\hrule
									
	}\vrule}}\hrule}\hfil}								
									
	\BlackBoxes								
									








\input font_iwona-condensed-medium  \fontss
\section{\sixteenbf\fontss Iwona-{\sixteenslbf Condensed Medium}}{Iwona-Condensed Medium}
\sample
\ii La police Iwona-{\sl Condensed Medium\/} est d\'eclar\'ee en entrant l'instruction\newline {\color{brown}\verbatim\input font_iwona-condensed-medium|endverbatim}. Cette famille de fontes utilise les polices Iwona en largeur condens\'ee et graisses moyenne et gras du package \href{http://www.tex.ac.uk/tex-archive/help/Catalogue/entries/iwona.html}{iwona} de J.\;M. Nowacki, correspondant aux polices texte de Ma\lstroke{}gorzata Budyta. On obtient un L barr\'e~(\Lstroke) avec la commande {\color{brown}\verbatim\Lstroke|endverbatim} et un l barr\'e (\lstroke) avec la commande {\color{brown}\verbatim\lstroke|endverbatim}. Pendant l'utilisation de cette macro, les commandes par d\'efaut de plain \capstex\ {\color{brown}\verbatim\L|endverbatim} ou {\color{brown}\verbatim\l|endverbatim} ne marchent pas. Des d\'etails sur cette macro sont donn\'es dans le tableau ci-dessous.
\bs
\hfil{Affectation de fonte pour la macro {\color{brown}\verbatim font_iwona-condensed-medium|endverbatim}}\hfil

{\parindent=0pt\settabs4\columns\hfil\vbox{\hrule\hbox{\vrule\hbox{\vbox{\kern1pt\hrule\NoBlackBoxes		 \eightrm\fontss							
									
\+\hfil	\textcolor{blue}{ Style}	&\strut\vrule\strut\hfil	 \textcolor{blue}{ Nom de la fonte}	 &\strut\vrule\kern1pt\vrule\strut\hfil	 \textcolor{blue}{ Style}	 &\strut\vrule\strut\hfil	\textcolor{blue}{ Nom de la fonte}	&\cr	\hrule
\+\hfil	\eightrm Romain	&\strut\vrule\strut\hfil	 rm-iwonacm	 &\strut\vrule\kern1pt\vrule\strut\hfil	\eightbf Gras	 &\strut\vrule\strut\hfil	 rm-iwonach	&\cr	 \hrule
\+\hfil	\eighti Maths italique	&\strut\vrule\strut\hfil	 mi-iwonacmi	 &\strut\vrule\kern1pt\vrule\strut\hfil	\eighttt Machine \`a \'ecire	 &\strut\vrule\strut\hfil	 ly1-zi4r-1	 &\cr	\hrule
\+\hfil	\eightrm Symboles maths	&\strut\vrule\strut\hfil	 sy-iwonacmz	 &\strut\vrule\kern1pt\vrule\strut\hfil	 \eightitbf Gras italique	 &\strut\vrule\strut\hfil	 rm-iwonachi	&\cr	\hrule
\+\hfil	\eightrm Extension maths	&\strut\vrule\strut\hfil	 ex-iwonacm	 &\strut\vrule\kern1pt\vrule\strut\hfil	 \eightslbf Gras pench\'e	 &\strut\vrule\strut\hfil	rm-iwonachi	&\cr	\hrule
\+\hfil	\eightit Italique	&\strut\vrule\strut\hfil	 rm-iwonacmi	 &\strut\vrule\kern1pt\vrule\strut\hfil	 \eightcaps Petites capitales	 &\strut\vrule\strut\hfil	 qx-iwonacmcap	 &\cr	\hrule
\+\hfil	\eightsl Pench\'e	&\strut\vrule\strut\hfil	 rm-iwonacmi	 &\strut\vrule\kern1pt\vrule\strut\hfil	 \eightcapsbf Petites capitales en gras	 &\strut\vrule\strut\hfil	 qx-iwonachcap	&\cr	\hrule
									
	}\vrule}}\hrule}\hfil}								
									
	\BlackBoxes								







\input font_iwona-condensed-bold  \fontss
\section{\sixteenbf\fontss Iwona-{\sixteenslbf Condensed Bold}}{Iwona-Condensed Bold}
\sample
\ii La police Iwona-{\sl Condensed Bold\/} est d\'eclar\'ee en entrant l'instruction {\color{brown}\verbatim\input font_iwona-condensed-bold|endverbatim}. Cette famille de fontes utilise les polices Iwona en largeur condens\'ee et styles gras du package \href{http://www.tex.ac.uk/tex-archive/help/Catalogue/entries/iwona.html}{iwona} de J.\;M. Nowacki, correspondant aux polices texte de Ma\lstroke{}gorzata Budyta. On obtient un L barr\'e~(\Lstroke) avec la commande {\color{brown}\verbatim\Lstroke|endverbatim} et un l barr\'e (\lstroke) avec la commande {\color{brown}\verbatim\lstroke|endverbatim}. Pendant l'utilisation de cette macro, les commandes par d\'efaut de plain \capstex\ {\color{brown}\verbatim\L|endverbatim} ou {\color{brown}\verbatim\l|endverbatim} ne marchent pas. Des d\'etails sur cette macro sont donn\'es dans le tableau ci-dessous.
\bs
\hfil{Affectation de fonte pour la macro {\color{brown}\verbatim font_iwona-condensed-bold|endverbatim}}\hfil

{\parindent=0pt\settabs4\columns\hfil\vbox{\hrule\hbox{\vrule\hbox{\vbox{\kern1pt\hrule\NoBlackBoxes		 \eightrm\fontss							
									
\+\hfil	\textcolor{blue}{ Style}	&\strut\vrule\strut\hfil	 \textcolor{blue}{ Nom de la fonte}	 &\strut\vrule\kern1pt\vrule\strut\hfil	 \textcolor{blue}{ Style}	 &\strut\vrule\strut\hfil	\textcolor{blue}{ Nom de la fonte}	&\cr	\hrule
\+\hfil	\eightrm Romain	&\strut\vrule\strut\hfil	 rm-iwonacb	 &\strut\vrule\kern1pt\vrule\strut\hfil	\eightbf Gras	 &\strut\vrule\strut\hfil	 rm-iwonach	&\cr	 \hrule
\+\hfil	\eighti Maths italique	&\strut\vrule\strut\hfil	 mi-iwonacbi	 &\strut\vrule\kern1pt\vrule\strut\hfil	\eighttt Machine \`a \'ecire	 &\strut\vrule\strut\hfil	 ly1-zi4r-1	 &\cr	\hrule
\+\hfil	\eightrm Symboles maths	&\strut\vrule\strut\hfil	 sy-iwonacbz	 &\strut\vrule\kern1pt\vrule\strut\hfil	 \eightitbf Gras italique	 &\strut\vrule\strut\hfil	 rm-iwonachi	&\cr	\hrule
\+\hfil	\eightrm Extension maths	&\strut\vrule\strut\hfil	 ex-iwonacb	 &\strut\vrule\kern1pt\vrule\strut\hfil	 \eightslbf Gras pench\'e	 &\strut\vrule\strut\hfil	rm-iwonachi	&\cr	\hrule
\+\hfil	\eightit Italique	&\strut\vrule\strut\hfil	 rm-iwonacbi	 &\strut\vrule\kern1pt\vrule\strut\hfil	 \eightcaps Petites capitales	 &\strut\vrule\strut\hfil	 qx-iwonacbcap	 &\cr	\hrule
\+\hfil	\eightsl Pench\'e	&\strut\vrule\strut\hfil	 rm-iwonacbi	 &\strut\vrule\kern1pt\vrule\strut\hfil	 \eightcapsbf Petites capitales en gras	 &\strut\vrule\strut\hfil	 qx-iwonachcap	&\cr	\hrule
									
	}\vrule}}\hrule}\hfil}								
									
	\BlackBoxes								




\input font_kurier  \fontss
\section{\sixteenbf\fontss Kurier}{Kurier}
\sample
\ii La police Kurier est d\'eclar\'ee en entrant l'instruction {\color{brown}\verbatim\input font_kurier|endverbatim}. Cette famille de fontes utilise des polices du package \href{http://www.tex.ac.uk/tex-archive/help/Catalogue/entries/kurier.html}{kurier} de J.\;M.\ Nowacki, correspondant aux polices texte de Ma\lstroke{}gorzata Budyta. La police Kurier est tr\`es similaire \`a la police Iwona; Kurier est un peu plus \'etendue et comporte des pi\`eges d'encre ({\it ink traps}). On obtient un L barr\'e~(\Lstroke) avec la commande {\color{brown}\verbatim\Lstroke|endverbatim} et un l barr\'e (\lstroke) avec la commande {\color{brown}\verbatim\lstroke|endverbatim}. Pendant l'utilisation de cette macro, les commandes par d\'efaut de plain \capstex\ {\color{brown}\verbatim\L|endverbatim} ou {\color{brown}\verbatim\l|endverbatim} ne marchent pas. Des d\'etails sur cette macro sont donn\'es dans le tableau ci-dessous.
\bs
\hfil{Affectation de fonte pour la macro {\color{brown}\verbatim font_kurier|endverbatim}}\hfil

{\parindent=0pt\settabs4\columns\hfil\vbox{\hrule\hbox{\vrule\hbox{\vbox{\kern1pt\hrule\NoBlackBoxes		 \eightrm\fontss							 \+\hfil	 \textcolor{blue}{ Style}	 &\strut\vrule\strut\hfil	 \textcolor{blue}{ Nom de la fonte}	 &\strut\vrule\kern1pt\vrule\strut\hfil	 \textcolor{blue}{ Style}	 &\strut\vrule\strut\hfil	 \textcolor{blue}{ Nom de la fonte}	 &\cr	\hrule
\+\hfil	\eightrm Romain	&\strut\vrule\strut\hfil	 rm-kurierr	 &\strut\vrule\kern1pt\vrule\strut\hfil	 \eightbf Gras	 &\strut\vrule\strut\hfil	 rm-kurierb	 &\cr	\hrule
\+\hfil	\eighti Maths italique	&\strut\vrule\strut\hfil	 mi-kurierri	 &\strut\vrule\kern1pt\vrule\strut\hfil	 \eighttt Machine \`a \'ecire	 &\strut\vrule\strut\hfil	 ly1-zi4r-1	&\cr	\hrule
\+\hfil	\eightrm Symboles maths	&\strut\vrule\strut\hfil	 sy-kurierrz	 &\strut\vrule\kern1pt\vrule\strut\hfil	 \eightitbf Gras italique	 &\strut\vrule\strut\hfil	 rm-kurierbi	&\cr	\hrule
\+\hfil	\eightrm Extension maths	&\strut\vrule\strut\hfil	 ex-kurierr	 &\strut\vrule\kern1pt\vrule\strut\hfil	 \eightslbf Gras pench\'e	 &\strut\vrule\strut\hfil	 rm-kurierbi	&\cr	\hrule
\+\hfil	\eightit Italique	&\strut\vrule\strut\hfil	 rm-kurierri	 &\strut\vrule\kern1pt\vrule\strut\hfil	 \eightcaps Petites capitales	 &\strut\vrule\strut\hfil	 qx-kurierrcap	 &\cr	\hrule
\+\hfil	\eightsl Pench\'e	&\strut\vrule\strut\hfil	 rm-kurierri	 &\strut\vrule\kern1pt\vrule\strut\hfil	 \eightcapsbf Petites capitales en gras	 &\strut\vrule\strut\hfil	 qx-kurierbcap	&\cr	\hrule
}\vrule}}\hrule}\hfil}								
									
\BlackBoxes								
									














\input font_kurier-light   \fontss
\section{\sixteenbf\fontss Kurier-{\sixteenslbf Light}}{Kurier-Light}
\sample
\ii La police Kurier-{\sl Light\/} est d\'eclar\'ee en entrant l'instruction {\color{brown}\verbatim\input font_kurier-light|endverbatim}. Cette famille de fontes utilise les polices Kurier en graisses l\'eg\`ere et moyenne du package \href{http://www.tex.ac.uk/tex-archive/help/Catalogue/entries/kurier.html}{kurier} de J.\;M.\;Nowacki, correspondant aux polices texte de Ma\lstroke{}gorzata Budyta. La police Kurier est tr\`es similaire \`a la police Iwona; Kurier est un peu plus \'etendue et comporte des pi\`eges d'encre ({\it ink traps}). On obtient un L barr\'e~(\Lstroke) avec la commande {\color{brown}\verbatim\Lstroke|endverbatim} et un l barr\'e (\lstroke) avec la commande {\color{brown}\verbatim\lstroke|endverbatim}. Pendant l'utilisation de cette macro, les commandes par d\'efaut de plain \capstex\ {\color{brown}\verbatim\L|endverbatim} ou {\color{brown}\verbatim\l|endverbatim} ne marchent pas. Des d\'etails sur cette macro sont donn\'es dans le tableau ci-dessous.
\bs
\hfil{Affectation de fonte pour la macro {\color{brown}\verbatim font_kurier-light|endverbatim}}\hfil

{\parindent=0pt\settabs4\columns\hfil\vbox{\hrule\hbox{\vrule\hbox{\vbox{\kern1pt\hrule\NoBlackBoxes		 \eightrm\fontss							
									
\+\hfil	\textcolor{blue}{ Style}	&\strut\vrule\strut\hfil	 \textcolor{blue}{ Nom de la fonte}	 &\strut\vrule\kern1pt\vrule\strut\hfil	 \textcolor{blue}{ Style}	 &\strut\vrule\strut\hfil	\textcolor{blue}{ Nom de la fonte}	&\cr	\hrule
\+\hfil	\eightrm Romain	&\strut\vrule\strut\hfil	 rm-kurierl	 &\strut\vrule\kern1pt\vrule\strut\hfil	\eightbf Gras	 &\strut\vrule\strut\hfil	 rm-kurierm	&\cr	 \hrule
\+\hfil	\eighti Maths italique	&\strut\vrule\strut\hfil	 mi-kurierli	 &\strut\vrule\kern1pt\vrule\strut\hfil	\eighttt Machine \`a \'ecire	 &\strut\vrule\strut\hfil	 ly1-zi4r-1	 &\cr	\hrule
\+\hfil	\eightrm Symboles maths	&\strut\vrule\strut\hfil	 sy-kurierlz	 &\strut\vrule\kern1pt\vrule\strut\hfil	 \eightitbf Gras italique	 &\strut\vrule\strut\hfil	 rm-kuriermi	&\cr	\hrule
\+\hfil	\eightrm Extension maths	&\strut\vrule\strut\hfil	 ex-kurierl	 &\strut\vrule\kern1pt\vrule\strut\hfil	 \eightslbf Gras pench\'e	 &\strut\vrule\strut\hfil	rm-kuriermi	&\cr	\hrule
\+\hfil	\eightit Italique	&\strut\vrule\strut\hfil	 rm-kurierli	 &\strut\vrule\kern1pt\vrule\strut\hfil	 \eightcaps Petites capitales	 &\strut\vrule\strut\hfil	 qx-kurierlcap	 &\cr	\hrule
\+\hfil	\eightsl Pench\'e	&\strut\vrule\strut\hfil	 rm-kurierli	 &\strut\vrule\kern1pt\vrule\strut\hfil	 \eightcapsbf Petites capitales en gras	 &\strut\vrule\strut\hfil	 qx-kuriermcap	&\cr	\hrule
									
	}\vrule}}\hrule}\hfil}								
									
	\BlackBoxes								










\input font_kurier-medium   \fontss
\section{\sixteenbf\fontss Kurier-{\sixteenslbf Medium}}{Kurier-Medium}
\sample
\ii La police Kurier-{\sl Medium\/} est d\'eclar\'ee en entrant l'instruction {\color{brown}\verbatim\input font_kurier-medium|endverbatim}. Cette famille de fontes utilise les polices Kurier en graisses moyenne et gras du package \href{http://www.tex.ac.uk/tex-archive/help/Catalogue/entries/kurier.html}{kurier} de J.\;M.\;Nowacki, correspondant aux polices texte de Ma\lstroke{}gorzata Budyta. La police Kurier est tr\`es similaire \`a la police Iwona; Kurier est un peu plus \'etendue et comporte des pi\`eges d'encre ({\it ink traps}). On obtient un L barr\'e~(\Lstroke) avec la commande {\color{brown}\verbatim\Lstroke|endverbatim} et un l barr\'e (\lstroke) avec la commande {\color{brown}\verbatim\lstroke|endverbatim}. Pendant l'utilisation de cette macro, les commandes par d\'efaut de plain \capstex\ {\color{brown}\verbatim\L|endverbatim} ou {\color{brown}\verbatim\l|endverbatim} ne marchent pas. Des d\'etails sur cette macro sont donn\'es dans le tableau ci-dessous.
\bs
\hfil{Affectation de fonte pour la macro {\color{brown}\verbatim font_kurier-medium|endverbatim}}\hfil

{\parindent=0pt\settabs4\columns\hfil\vbox{\hrule\hbox{\vrule\hbox{\vbox{\kern1pt\hrule\NoBlackBoxes		 \eightrm\fontss							
									
\+\hfil	\textcolor{blue}{ Style}	&\strut\vrule\strut\hfil	 \textcolor{blue}{ Nom de la fonte}	 &\strut\vrule\kern1pt\vrule\strut\hfil	 \textcolor{blue}{ Style}	 &\strut\vrule\strut\hfil	\textcolor{blue}{ Nom de la fonte}	&\cr	\hrule
\+\hfil	\eightrm Romain	&\strut\vrule\strut\hfil	 rm-kurierm	 &\strut\vrule\kern1pt\vrule\strut\hfil	\eightbf Gras	 &\strut\vrule\strut\hfil	 rm-kurierh	&\cr	 \hrule
\+\hfil	\eighti Maths italique	&\strut\vrule\strut\hfil	 mi-kuriermi	 &\strut\vrule\kern1pt\vrule\strut\hfil	\eighttt Machine \`a \'ecire	 &\strut\vrule\strut\hfil	 ly1-zi4r-1	 &\cr	\hrule
\+\hfil	\eightrm Symboles maths	&\strut\vrule\strut\hfil	 sy-kuriermz	 &\strut\vrule\kern1pt\vrule\strut\hfil	 \eightitbf Gras italique	 &\strut\vrule\strut\hfil	 rm-kurierhi	&\cr	\hrule
\+\hfil	\eightrm Extension maths	&\strut\vrule\strut\hfil	 ex-kurierm	 &\strut\vrule\kern1pt\vrule\strut\hfil	 \eightslbf Gras pench\'e	 &\strut\vrule\strut\hfil	rm-kurierhi	&\cr	\hrule
\+\hfil	\eightit Italique	&\strut\vrule\strut\hfil	 rm-kuriermi	 &\strut\vrule\kern1pt\vrule\strut\hfil	 \eightcaps Petites capitales	 &\strut\vrule\strut\hfil	 qx-kuriermcap	 &\cr	\hrule
\+\hfil	\eightsl Pench\'e	&\strut\vrule\strut\hfil	 rm-kuriermi	 &\strut\vrule\kern1pt\vrule\strut\hfil	 \eightcapsbf Petites capitales en gras	 &\strut\vrule\strut\hfil	 qx-kurierhcap	&\cr	\hrule
									
	}\vrule}}\hrule}\hfil}								
									
	\BlackBoxes								






\input font_kurier-bold   \fontss
\section{\sixteenbf\fontss Kurier-{\sixteenslbf Bold}}{Kurier-Bold}
\sample
\ii La police Kurier-{\sl Bold\/} est d\'eclar\'ee en entrant l'instruction {\color{brown}\verbatim\input font_kurier-bold|endverbatim}. Cette famille de fontes utilise les polices Kurier en styles gras du package \href{http://www.tex.ac.uk/tex-archive/help/Catalogue/entries/kurier.html}{kurier} de J.\;M.\;Nowacki, correspondant aux polices texte de Ma\lstroke{}gorzata Budyta. La police Kurier est tr\`es similaire \`a la police Iwona; Kurier est un peu plus \'etendue et comporte des pi\`eges d'encre ({\it ink traps}). On obtient un L barr\'e~(\Lstroke) avec la commande {\color{brown}\verbatim\Lstroke|endverbatim} et un l barr\'e (\lstroke) avec la commande {\color{brown}\verbatim\lstroke|endverbatim}. Pendant l'utilisation de cette macro, les commandes par d\'efaut de plain \capstex\ {\color{brown}\verbatim\L|endverbatim} ou {\color{brown}\verbatim\l|endverbatim} ne marchent pas. Des d\'etails sur cette macro sont donn\'es dans le tableau ci-dessous.
\bs
\hfil{Affectation de fonte pour la macro {\color{brown}\verbatim font_kurier-medium|endverbatim}}\hfil

{\parindent=0pt\settabs4\columns\hfil\vbox{\hrule\hbox{\vrule\hbox{\vbox{\kern1pt\hrule\NoBlackBoxes		 \eightrm\fontss							
									
\+\hfil	\textcolor{blue}{ Style}	&\strut\vrule\strut\hfil	 \textcolor{blue}{ Nom de la fonte}	 &\strut\vrule\kern1pt\vrule\strut\hfil	 \textcolor{blue}{ Style}	 &\strut\vrule\strut\hfil	\textcolor{blue}{ Nom de la fonte}	&\cr	\hrule
\+\hfil	\eightrm Romain	&\strut\vrule\strut\hfil	 rm-kurierb	 &\strut\vrule\kern1pt\vrule\strut\hfil	\eightbf Gras	 &\strut\vrule\strut\hfil	 rm-kurierh	&\cr	 \hrule
\+\hfil	\eighti Maths italique	&\strut\vrule\strut\hfil	 mi-kurierbi	 &\strut\vrule\kern1pt\vrule\strut\hfil	\eighttt Machine \`a \'ecire	 &\strut\vrule\strut\hfil	 ly1-zi4r-1	 &\cr	\hrule
\+\hfil	\eightrm Symboles maths	&\strut\vrule\strut\hfil	 sy-kurierbz	 &\strut\vrule\kern1pt\vrule\strut\hfil	 \eightitbf Gras italique	 &\strut\vrule\strut\hfil	 rm-kurierhi	&\cr	\hrule
\+\hfil	\eightrm Extension maths	&\strut\vrule\strut\hfil	 ex-kurierb	 &\strut\vrule\kern1pt\vrule\strut\hfil	 \eightslbf Gras pench\'e	 &\strut\vrule\strut\hfil	rm-kurierhi	&\cr	\hrule
\+\hfil	\eightit Italique	&\strut\vrule\strut\hfil	 rm-kurierbi	 &\strut\vrule\kern1pt\vrule\strut\hfil	 \eightcaps Petites capitales	 &\strut\vrule\strut\hfil	 qx-kurierbcap	 &\cr	\hrule
\+\hfil	\eightsl Pench\'e	&\strut\vrule\strut\hfil	 rm-kurierbi	 &\strut\vrule\kern1pt\vrule\strut\hfil	 \eightcapsbf Petites capitales en gras	 &\strut\vrule\strut\hfil	 qx-kurierhcap	&\cr	\hrule
									
	}\vrule}}\hrule}\hfil}								
									
	\BlackBoxes								








\input font_kurier-condensed   \fontss
\section{\sixteenbf\fontss Kurier-{\sixteenslbf Condensed}}{Kurier-Condensed}
\sample
\ii La police Kurier-{\sl Condensed\/} est d\'eclar\'ee en entrant l'instruction {\color{brown}\verbatim\input font_kurier-condensed|endverbatim}. Cette famille de fontes utilise les polices Kurier de largeur condens\'ee en graisses normal et gras du package \href{http://www.tex.ac.uk/tex-archive/help/Catalogue/entries/kurier.html}{kurier} de J.\;M.\;Nowacki, correspondant aux polices texte de Ma\lstroke{}gorzata Budyta. La police Kurier est tr\`es similaire \`a la police Iwona; Kurier est un peu plus \'etendue et comporte des pi\`eges d'encre ({\it ink traps}). On obtient un L barr\'e~(\Lstroke) avec la commande {\color{brown}\verbatim\Lstroke|endverbatim} et un l barr\'e (\lstroke) avec la commande {\color{brown}\verbatim\lstroke|endverbatim}. Pendant l'utilisation de cette macro, les commandes par d\'efaut de plain \capstex\ {\color{brown}\verbatim\L|endverbatim} ou {\color{brown}\verbatim\l|endverbatim} ne marchent pas. Des d\'etails sur cette macro sont donn\'es dans le tableau ci-dessous.
\bs
\hfil{Affectation de fonte pour la macro {\color{brown}\verbatim font_kurier-condensed|endverbatim}}\hfil

{\parindent=0pt\settabs4\columns\hfil\vbox{\hrule\hbox{\vrule\hbox{\vbox{\kern1pt\hrule\NoBlackBoxes		 \eightrm\fontss							
									
\+\hfil	\textcolor{blue}{ Style}	&\strut\vrule\strut\hfil	 \textcolor{blue}{ Nom de la fonte}	 &\strut\vrule\kern1pt\vrule\strut\hfil	 \textcolor{blue}{ Style}	 &\strut\vrule\strut\hfil	\textcolor{blue}{ Nom de la fonte}	&\cr	\hrule
\+\hfil	\eightrm Romain	&\strut\vrule\strut\hfil	 rm-kuriercr	 &\strut\vrule\kern1pt\vrule\strut\hfil	\eightbf Gras	 &\strut\vrule\strut\hfil	 rm-kuriercb	 &\cr	\hrule
\+\hfil	\eighti Maths italique	&\strut\vrule\strut\hfil	 mi-kuriercri	 &\strut\vrule\kern1pt\vrule\strut\hfil	 \eighttt Machine \`a \'ecire	 &\strut\vrule\strut\hfil	 ly1-zi4r-1	&\cr	\hrule
\+\hfil	\eightrm Symboles maths	&\strut\vrule\strut\hfil	 sy-kuriercrz	 &\strut\vrule\kern1pt\vrule\strut\hfil	 \eightitbf Gras italique	 &\strut\vrule\strut\hfil	 rm-kuriercbi	&\cr	\hrule
\+\hfil	\eightrm Extension maths	&\strut\vrule\strut\hfil	 ex-kuriercr	 &\strut\vrule\kern1pt\vrule\strut\hfil	 \eightslbf Gras pench\'e	 &\strut\vrule\strut\hfil	rm-kuriercbi	&\cr	\hrule
\+\hfil	\eightit Italique	&\strut\vrule\strut\hfil	 rm-kuriercri	 &\strut\vrule\kern1pt\vrule\strut\hfil	 \eightcaps Petites capitales	 &\strut\vrule\strut\hfil	 qx-kuriercrcap	 &\cr	\hrule
\+\hfil	\eightsl Pench\'e	&\strut\vrule\strut\hfil	 rm-kuriercri	 &\strut\vrule\kern1pt\vrule\strut\hfil	 \eightcapsbf Petites capitales en gras	 &\strut\vrule\strut\hfil	 qx-kuriercbcap	&\cr	\hrule
									
	}\vrule}}\hrule}\hfil}								
									
	\BlackBoxes								







\input font_kurier-condensed-light   \fontss
\section{\sixteenbf\fontss Kurier-{\sixteenslbf Condensed Light}}{Kurier-Condensed Light}
\sample
\ii La police Kurier-{\sl Condensed Light\/} est d\'eclar\'ee en entrant l'instruction\newline {\color{brown}\verbatim\input font_kurier-condensed-light|endverbatim}. Cette famille de fontes utilise les polices Kurier de largeur condens\'ee en graisses l\'eg\`ere et moyenne du package \href{http://www.tex.ac.uk/tex-archive/help/Catalogue/entries/kurier.html}{kurier} de J.\;M.\;Nowacki, correspondant aux polices texte de Ma\lstroke{}gorzata Budyta. La police Kurier est tr\`es similaire \`a la police Iwona; Kurier est un peu plus \'etendue et comporte des pi\`eges d'encre ({\it ink traps}). On obtient un L barr\'e~(\Lstroke) avec la commande {\color{brown}\verbatim\Lstroke|endverbatim} et un l barr\'e (\lstroke) avec la commande {\color{brown}\verbatim\lstroke|endverbatim}. Pendant l'utilisation de cette macro, les commandes par d\'efaut de plain \capstex\ {\color{brown}\verbatim\L|endverbatim} ou {\color{brown}\verbatim\l|endverbatim} ne marchent pas. Des d\'etails sur cette macro sont donn\'es dans le tableau ci-dessous.
\bs
\hfil{Affectation de fonte pour la macro {\color{brown}\verbatim font_kurier-condensed-light|endverbatim}}\hfil

{\parindent=0pt\settabs4\columns\hfil\vbox{\hrule\hbox{\vrule\hbox{\vbox{\kern1pt\hrule\NoBlackBoxes		 \eightrm\fontss							
									
\+\hfil	\textcolor{blue}{ Style}	&\strut\vrule\strut\hfil	 \textcolor{blue}{ Nom de la fonte}	 &\strut\vrule\kern1pt\vrule\strut\hfil	 \textcolor{blue}{ Style}	 &\strut\vrule\strut\hfil	\textcolor{blue}{ Nom de la fonte}	&\cr	\hrule
\+\hfil	\eightrm Romain	&\strut\vrule\strut\hfil	 rm-kuriercl	 &\strut\vrule\kern1pt\vrule\strut\hfil	\eightbf Gras	 &\strut\vrule\strut\hfil	 rm-kuriercm	 &\cr	\hrule
\+\hfil	\eighti Maths italique	&\strut\vrule\strut\hfil	 mi-kuriercli	 &\strut\vrule\kern1pt\vrule\strut\hfil	 \eighttt Machine \`a \'ecire	 &\strut\vrule\strut\hfil	 ly1-zi4r-1	&\cr	\hrule
\+\hfil	\eightrm Symboles maths	&\strut\vrule\strut\hfil	 sy-kurierclz	 &\strut\vrule\kern1pt\vrule\strut\hfil	 \eightitbf Gras italique	 &\strut\vrule\strut\hfil	 rm-kuriercmi	&\cr	\hrule
\+\hfil	\eightrm Extension maths	&\strut\vrule\strut\hfil	 ex-kuriercl	 &\strut\vrule\kern1pt\vrule\strut\hfil	 \eightslbf Gras pench\'e	 &\strut\vrule\strut\hfil	rm-kuriercmi	&\cr	\hrule
\+\hfil	\eightit Italique	&\strut\vrule\strut\hfil	 rm-kuriercli	 &\strut\vrule\kern1pt\vrule\strut\hfil	 \eightcaps Petites capitales	 &\strut\vrule\strut\hfil	 qx-kurierclcap	 &\cr	\hrule
\+\hfil	\eightsl Pench\'e	&\strut\vrule\strut\hfil	 rm-kuriercli	 &\strut\vrule\kern1pt\vrule\strut\hfil	 \eightcapsbf Petites capitales en gras	 &\strut\vrule\strut\hfil	 qx-kuriercmcap	&\cr	\hrule
									
	}\vrule}}\hrule}\hfil}								
									
	\BlackBoxes								
									








\input font_kurier-condensed-medium  \fontss
\section{\sixteenbf\fontss Kurier-{\sixteenslbf Condensed Medium}}{Kurier-Condensed Medium}
\sample
\ii La police Kurier-{\sl Condensed Medium\/} est d\'eclar\'ee en entrant l'instruction\newline {\color{brown}\verbatim\input font_kurier-condensed-medium|endverbatim}. Cette famille de fontes utilise les polices Kurier de largeur condens\'ee en graisses moyenne et gras du package \href{http://www.tex.ac.uk/tex-archive/help/Catalogue/entries/kurier.html}{kurier} de J.\;M.\;Nowacki, correspondant aux polices texte de Ma\lstroke{}gorzata Budyta. La police Kurier est tr\`es similaire \`a la police Iwona; Kurier est un peu plus \'etendue et comporte des pi\`eges d'encre ({\it ink traps}). On obtient un L barr\'e~(\Lstroke) avec la commande {\color{brown}\verbatim\Lstroke|endverbatim} et un l barr\'e (\lstroke) avec la commande {\color{brown}\verbatim\lstroke|endverbatim}. Pendant l'utilisation de cette macro, les commandes par d\'efaut de plain \capstex\ {\color{brown}\verbatim\L|endverbatim} ou {\color{brown}\verbatim\l|endverbatim} ne marchent pas. Des d\'etails sur cette macro sont donn\'es dans le tableau ci-dessous.
\bs
\hfil{Affectation de fonte pour la macro {\color{brown}\verbatim font_kurier-condensed-medium|endverbatim}}\hfil

{\parindent=0pt\settabs4\columns\hfil\vbox{\hrule\hbox{\vrule\hbox{\vbox{\kern1pt\hrule\NoBlackBoxes		 \eightrm\fontss							
									
\+\hfil	\textcolor{blue}{ Style}	&\strut\vrule\strut\hfil	 \textcolor{blue}{ Nom de la fonte}	 &\strut\vrule\kern1pt\vrule\strut\hfil	 \textcolor{blue}{ Style}	 &\strut\vrule\strut\hfil	\textcolor{blue}{ Nom de la fonte}	&\cr	\hrule
\+\hfil	\eightrm Romain	&\strut\vrule\strut\hfil	 rm-kuriercm	 &\strut\vrule\kern1pt\vrule\strut\hfil	\eightbf Gras	 &\strut\vrule\strut\hfil	 rm-kurierch	 &\cr	\hrule
\+\hfil	\eighti Maths italique	&\strut\vrule\strut\hfil	 mi-kuriercmi	 &\strut\vrule\kern1pt\vrule\strut\hfil	 \eighttt Machine \`a \'ecire	 &\strut\vrule\strut\hfil	 ly1-zi4r-1	&\cr	\hrule
\+\hfil	\eightrm Symboles maths	&\strut\vrule\strut\hfil	 sy-kuriercmz	 &\strut\vrule\kern1pt\vrule\strut\hfil	 \eightitbf Gras italique	 &\strut\vrule\strut\hfil	 rm-kurierchi	&\cr	\hrule
\+\hfil	\eightrm Extension maths	&\strut\vrule\strut\hfil	 ex-kuriercm	 &\strut\vrule\kern1pt\vrule\strut\hfil	 \eightslbf Gras pench\'e	 &\strut\vrule\strut\hfil	rm-kurierchi	&\cr	\hrule
\+\hfil	\eightit Italique	&\strut\vrule\strut\hfil	 rm-kuriercmi	 &\strut\vrule\kern1pt\vrule\strut\hfil	 \eightcaps Petites capitales	 &\strut\vrule\strut\hfil	 qx-kuriercmcap	 &\cr	\hrule
\+\hfil	\eightsl Pench\'e	&\strut\vrule\strut\hfil	 rm-kuriercmi	 &\strut\vrule\kern1pt\vrule\strut\hfil	 \eightcapsbf Petites capitales en gras	 &\strut\vrule\strut\hfil	 qx-kurierchcap	&\cr	\hrule
									
	}\vrule}}\hrule}\hfil}								
									
	\BlackBoxes								







\input font_kurier-condensed-bold  \fontss
\section{\sixteenbf\fontss Kurier-{\sixteenslbf Condensed Bold}}{Kurier-Condensed Bold}
\sample
\ii La police Kurier-{\sl Condensed Bold\/} est d\'eclar\'ee en entrant l'instruction\newline {\color{brown}\verbatim\input font_kurier-condensed-bold|endverbatim}. Cette famille de fontes utilise les polices Kurier de largeur condens\'ee et styles gras du package \href{http://www.tex.ac.uk/tex-archive/help/Catalogue/entries/kurier.html}{kurier} de J.\;M.\;Nowacki, correspondant aux polices texte de Ma\lstroke{}gorzata Budyta. La police Kurier est tr\`es similaire \`a la police Iwona; Kurier est un peu plus \'etendue et comporte des pi\`eges d'encre ({\it ink traps}). On obtient un L barr\'e~(\Lstroke) avec la commande {\color{brown}\verbatim\Lstroke|endverbatim} et un l barr\'e (\lstroke) avec la commande {\color{brown}\verbatim\lstroke|endverbatim}. Pendant l'utilisation de cette macro, les commandes par d\'efaut de plain \capstex\ {\color{brown}\verbatim\L|endverbatim} ou {\color{brown}\verbatim\l|endverbatim} ne marchent pas. Des d\'etails sur cette macro sont donn\'es dans le tableau ci-dessous.
\bs
\hfil{Affectation de fonte pour la macro {\color{brown}\verbatim font_kurier-condensed-bold|endverbatim}}\hfil

{\parindent=0pt\settabs4\columns\hfil\vbox{\hrule\hbox{\vrule\hbox{\vbox{\kern1pt\hrule\NoBlackBoxes		 \eightrm\fontss							
									
\+\hfil	\textcolor{blue}{ Style}	&\strut\vrule\strut\hfil	 \textcolor{blue}{ Nom de la fonte}	 &\strut\vrule\kern1pt\vrule\strut\hfil	 \textcolor{blue}{ Style}	 &\strut\vrule\strut\hfil	\textcolor{blue}{ Nom de la fonte}	&\cr	\hrule
\+\hfil	\eightrm Romain	&\strut\vrule\strut\hfil	 rm-kuriercb	 &\strut\vrule\kern1pt\vrule\strut\hfil	\eightbf Gras	 &\strut\vrule\strut\hfil	 rm-kurierch	 &\cr	\hrule
\+\hfil	\eighti Maths italique	&\strut\vrule\strut\hfil	 mi-kuriercbi	 &\strut\vrule\kern1pt\vrule\strut\hfil	 \eighttt Machine \`a \'ecire	 &\strut\vrule\strut\hfil	 ly1-zi4r-1	&\cr	\hrule
\+\hfil	\eightrm Symboles maths	&\strut\vrule\strut\hfil	 sy-kuriercbz	 &\strut\vrule\kern1pt\vrule\strut\hfil	 \eightitbf Gras italique	 &\strut\vrule\strut\hfil	 rm-kurierchi	&\cr	\hrule
\+\hfil	\eightrm Extension maths	&\strut\vrule\strut\hfil	 ex-kuriercb	 &\strut\vrule\kern1pt\vrule\strut\hfil	 \eightslbf Gras pench\'e	 &\strut\vrule\strut\hfil	rm-kurierchi	&\cr	\hrule
\+\hfil	\eightit Italique	&\strut\vrule\strut\hfil	 rm-kuriercbi	 &\strut\vrule\kern1pt\vrule\strut\hfil	 \eightcaps Petites capitales	 &\strut\vrule\strut\hfil	 qx-kuriercbcap	 &\cr	\hrule
\+\hfil	\eightsl Pench\'e	&\strut\vrule\strut\hfil	 rm-kuriercbi	 &\strut\vrule\kern1pt\vrule\strut\hfil	 \eightcapsbf Petites capitales en gras	 &\strut\vrule\strut\hfil	 qx-kurierchcap	&\cr	\hrule
									
	}\vrule}}\hrule}\hfil}								
									
	\BlackBoxes								
















\input font_arev   \fontss
\section{\sixteenbf\fontss Arev}{Arev}
\sample
\ii La police Arev est d\'eclar\'ee en entrant l'instruction {\color{brown}\verbatim\input font_arev|endverbatim}. Cette famille de fontes utilise des polices du package  \href{http://www.tex.ac.uk/tex-archive/help/Catalogue/entries/arev.html}{arev} de S.\;G.\; Hartke, correspondant aux polices texte \href{http://www.fontsquirrel.com/fonts/Bitstream-Vera-Sans}{Bitstream Vera Sans}. La police \href{http://www.gnome.org/fonts/}{Bitstream Vera} a \'et\'e cr\'e\'ee par Jim Lyles. Des d\'etails sur cette macro sont donn\'es dans le tableau ci-dessous.
\bs
\hfil{Affectation de fonte pour la macro {\color{brown}\verbatim font_arev|endverbatim}}\hfil

{\parindent=0pt\settabs4\columns\hfil\vbox{\hrule\hbox{\vrule\hbox{\vbox{\kern1pt\hrule\NoBlackBoxes		 \eightrm\fontss							 \+\hfil	 \textcolor{blue}{ Style}	 &\strut\vrule\strut\hfil	 \textcolor{blue}{ Nom de la fonte}	 &\strut\vrule\kern1pt\vrule\strut\hfil	 \textcolor{blue}{ Style}	 &\strut\vrule\strut\hfil	 \textcolor{blue}{ Nom de la fonte}	 &\cr	\hrule
\+\hfil	\eightrm Romain	&\strut\vrule\strut\hfil	 zavmr7t	 &\strut\vrule\kern1pt\vrule\strut\hfil	 \eightbf Gras	 &\strut\vrule\strut\hfil	 zavmb7t	&\cr	 \hrule
\+\hfil	\eighti Maths italique	&\strut\vrule\strut\hfil	 zavmri7m	 &\strut\vrule\kern1pt\vrule\strut\hfil	 \eighttt Machine \`a \'ecire	 &\strut\vrule\strut\hfil	 fvmr8t	&\cr	\hrule
\+\hfil	\eightrm Symboles maths	&\strut\vrule\strut\hfil	 zavmr7y	 &\strut\vrule\kern1pt\vrule\strut\hfil	 \eightitbf Gras italique	 &\strut\vrule\strut\hfil	 favbi8t	 &\cr	\hrule
\+\hfil	\eightrm Extension maths	&\strut\vrule\strut\hfil	 ex-kurierr	 &\strut\vrule\kern1pt\vrule\strut\hfil	 \eightslbf Gras pench\'e	 &\strut\vrule\strut\hfil	 favbi8t	&\cr	\hrule
\+\hfil	\eightit Italique	&\strut\vrule\strut\hfil	 favri8t	 &\strut\vrule\kern1pt\vrule\strut\hfil	Petites capitales	 &\strut\vrule\strut\hfil	 \emdash  	 &\cr	 \hrule
\+\hfil	\eightsl Pench\'e	&\strut\vrule\strut\hfil	 favri8t	 &\strut\vrule\kern1pt\vrule\strut\hfil	Petites capitales en gras	 &\strut\vrule\strut\hfil	 \emdash  	 &\cr	 \hrule
}\vrule}}\hrule}\hfil}								
									
\BlackBoxes								
						
									
				
							
									
		



\input font_cmbright   \fontss
\UseAMSsymbols
\section{\sixteenbf\fontss Computer Modern Bright}{Computer Modern Bright}
\sample
\ii La police Computer Modern Bright est d\'eclar\'ee en entrant l'instruction {\color{brown}\verbatim\input font_cmbright|endverbatim}. Cette famille de fontes utilise des polices du package \href{http://www.tex.ac.uk/tex-archive/help/Catalogue/entries/cmbright.html}{cmbright} de Walter Schmidt, correspondant aux polices texte Computer Modern Sans Serif de Donald Knuth. Les polices Computer Modern Bright sont plus l\'eg\`eres que les polices Computer Modern Sans Serif. Les polices de cette macro fournissent leurs propres symboles {\eightrm AMS}. Des d\'etails sur cette macro sont donn\'es dans le tableau ci-dessous.
\bs
\hfil{Affectation de fonte pour la macro {\color{brown}\verbatim font_cmbright|endverbatim}}\hfil

{\parindent=0pt\settabs4\columns\hfil\vbox{\hrule\hbox{\vrule\hbox{\vbox{\kern1pt\hrule\NoBlackBoxes		 \eightrm\fontss							 \+\hfil	 \textcolor{blue}{ Style}	 &\strut\vrule\strut\hfil	 \textcolor{blue}{ Nom de la fonte}	 &\strut\vrule\kern1pt\vrule\strut\hfil	 \textcolor{blue}{ Style}	 &\strut\vrule\strut\hfil	 \textcolor{blue}{ Nom de la fonte}	 &\cr	\hrule
\+\hfil	\eightrm Romain	&\strut\vrule\strut\hfil	 cmbr10	 &\strut\vrule\kern1pt\vrule\strut\hfil	 \eightbf Gras	 &\strut\vrule\strut\hfil	 cmbrbx10	&\cr	 \hrule
\+\hfil	\eighti Maths italique	&\strut\vrule\strut\hfil	 cmbrmi10	 &\strut\vrule\kern1pt\vrule\strut\hfil	 \eighttt Machine \`a \'ecire	 &\strut\vrule\strut\hfil	 ly1-zi4r-1	&\cr	\hrule
\+\hfil	\eightrm Symboles maths	&\strut\vrule\strut\hfil	 cmbrsy10	 &\strut\vrule\kern1pt\vrule\strut\hfil	 \eightitbf Gras italique	 &\strut\vrule\strut\hfil	 rm-lmssbo10	&\cr	\hrule
\+\hfil	\eightrm Extension maths	&\strut\vrule\strut\hfil	 ex-kurierr	 &\strut\vrule\kern1pt\vrule\strut\hfil	 \eightslbf Gras pench\'e	 &\strut\vrule\strut\hfil	 rm-lmssbo10	&\cr	\hrule
\+\hfil	\eightit Italique	&\strut\vrule\strut\hfil	 cmbrsl10	 &\strut\vrule\kern1pt\vrule\strut\hfil	 Petites capitales	 &\strut\vrule\strut\hfil	 Non disponible  	 &\cr	\hrule
\+\hfil	\eightsl Pench\'e	&\strut\vrule\strut\hfil	 cmbrsl10	 &\strut\vrule\kern1pt\vrule\strut\hfil	 Petites capitales en gras	 &\strut\vrule\strut\hfil	 Non disponible  	&\cr	 \hrule
}\vrule}}\hrule}\hfil}								
									
\BlackBoxes								

\bs\ii Symboles {\eightrm AMS} associ\'es~: \circledR \ \yen \ $\blacksquare \ \approxeq \ \eqslantgtr \ \curlyeqprec \ \curlyeqsucc \ \preccurlyeq \ \leqq \ \leqslant \ \lessgtr \ \nless \ \nleq \ \nleqslant \ \Bbb R \ \Bbb E \ \Bbb C \ \dots$












\input font_epigrafica_euler   \fontss
\section{\sixteenbf\fontss Epigrafica avec Euler}{Epigrafica avec Euler}
\sample
\ii Cette macro nous permet de taper du texte dans la police Epigrafica et des maths dans la police Euler. La macro est d\'eclar\'ee en entrant l'instruction {\color{brown}\verbatim\input font_epigrafica_euler|endverbatim}. Celle-ci typographie le texte dans les fontes du package \href{http://www.tex.ac.uk/tex-archive/help/Catalogue/entries/epigrafica.html}{epigrafica} de Antonis Tsolomitis (bas\'e sur la police texte \href{http://new.myfonts.com/fonts/adobe/optima/}{Optima} d'Hermann Zapf) et les maths dans les fontes \href{http://www.tex.ac.uk/tex-archive/help/Catalogue/entries/eulervm.html}{Euler-VM} de Walter Schmidt (bas\'e sur la police Euler d'Hermann Zapf et la police CM de Knuth). Des d\'etails sur cette macro sont donn\'es dans le tableau ci-dessous.
\bs
\hfil{Affectation de fonte pour la macro {\color{brown}\verbatim font_epigrafica_euler|endverbatim}}\hfil

{\parindent=0pt\settabs4\columns\hfil\vbox{\hrule\hbox{\vrule\hbox{\vbox{\kern1pt\hrule\NoBlackBoxes		 \eightrm\fontss							 \+\hfil	 \textcolor{blue}{ Style}	 &\strut\vrule\strut\hfil	 \textcolor{blue}{ Nom de la fonte}	 &\strut\vrule\kern1pt\vrule\strut\hfil	 \textcolor{blue}{ Style}	 &\strut\vrule\strut\hfil	 \textcolor{blue}{ Nom de la fonte}	 &\cr	\hrule
\+\hfil	\eightrm Romain	&\strut\vrule\strut\hfil	 epigrafican8r	 &\strut\vrule\kern1pt\vrule\strut\hfil	 \eightbf Gras	 &\strut\vrule\strut\hfil	 epigraficab8r	&\cr	\hrule
\+\hfil	\eighti Maths italique	&\strut\vrule\strut\hfil	 eurm10	 &\strut\vrule\kern1pt\vrule\strut\hfil	 \eighttt Machine \`a \'ecire	 &\strut\vrule\strut\hfil	 ly1-zi4r-1	 &\cr	\hrule
\+\hfil	\eightrm Symboles maths	&\strut\vrule\strut\hfil	 cmsy10	 &\strut\vrule\kern1pt\vrule\strut\hfil	\eightitbf Gras italique	 &\strut\vrule\strut\hfil	 epigraficabi8r	&\cr	\hrule
\+\hfil	\eightrm Extension maths	&\strut\vrule\strut\hfil	 euex10	 &\strut\vrule\kern1pt\vrule\strut\hfil	\eightslbf Gras pench\'e	 &\strut\vrule\strut\hfil	 epigraficabi8r	&\cr	\hrule
\+\hfil	\eightit Italique	&\strut\vrule\strut\hfil	 epigraficai8r	 &\strut\vrule\kern1pt\vrule\strut\hfil	 \eightcaps Petites capitales	 &\strut\vrule\strut\hfil	 epigraficac8r	 &\cr	\hrule
\+\hfil	\eightsl Pench\'e	&\strut\vrule\strut\hfil	 epigraficai8r	 &\strut\vrule\kern1pt\vrule\strut\hfil	Petites capitales en gras	 &\strut\vrule\strut\hfil	Non disponible  	 &\cr	 \hrule
}\vrule}}\hrule}\hfil}								
									
\BlackBoxes								
									








\input font_epigrafica_palatino   \fontss
\section{\sixteenbf\fontss Epigrafica with Palatino}{Epigrafica avec Palatino}
\sample
\ii Cette macro nous permet de taper du texte dans la police Epigrafica et des maths dans la police PX. La macro est d\'eclar\'ee en entrant l'instruction {\color{brown}\verbatim\input font_epigrafica_palatino|endverbatim}. Celle-ci typographie le texte dans les fontes du package \href{http://www.tex.ac.uk/tex-archive/help/Catalogue/entries/epigrafica.html}{epigrafica} de Antonis Tsolomitis (bas\'e sur la police texte \href{http://new.myfonts.com/fonts/adobe/optima/}{Optima} d'Hermann Zapf) et les maths dans les fontes du package \href{http://www.tex.ac.uk/tex-archive/help/Catalogue/entries/pxfonts.html}{pxfonts} de Young Ryu (correspondant aux polices texte \href{http://www.adobe.com/type/browser/html/readmes/PalatinoStdReadMe.html#A2}
{Adobe Palatino}). Des d\'etails sur cette macro sont donn\'es dans le tableau ci-dessous.
\bs
\hfil{Affectation de fonte pour la macro {\color{brown}\verbatim font_epigrafica_palatino|endverbatim}}\hfil

{\parindent=0pt\settabs4\columns\hfil\vbox{\hrule\hbox{\vrule\hbox{\vbox{\kern1pt\hrule\NoBlackBoxes		 \eightrm\fontss							 \+\hfil	 \textcolor{blue}{ Style}	 &\strut\vrule\strut\hfil	 \textcolor{blue}{ Nom de la fonte}	 &\strut\vrule\kern1pt\vrule\strut\hfil	 \textcolor{blue}{ Style}	 &\strut\vrule\strut\hfil	 \textcolor{blue}{ Nom de la fonte}	 &\cr	\hrule
\+\hfil	\eightrm Romain	&\strut\vrule\strut\hfil	 epigrafican8r	 &\strut\vrule\kern1pt\vrule\strut\hfil	 \eightbf Gras	 &\strut\vrule\strut\hfil	 epigraficab8r	&\cr	\hrule
\+\hfil	\eighti Maths italique	&\strut\vrule\strut\hfil	pxmi	 &\strut\vrule\kern1pt\vrule\strut\hfil	 \eighttt Machine \`a \'ecire	 &\strut\vrule\strut\hfil	 ly1-zi4r-1	&\cr	 \hrule
\+\hfil	\eightrm Symboles maths	&\strut\vrule\strut\hfil	 pxsy	 &\strut\vrule\kern1pt\vrule\strut\hfil	\eightitbf Gras italique	 &\strut\vrule\strut\hfil	 epigraficabi8r	&\cr	\hrule
\+\hfil	\eightrm Extension maths	&\strut\vrule\strut\hfil	 pxex	 &\strut\vrule\kern1pt\vrule\strut\hfil	\eightslbf Gras pench\'e	 &\strut\vrule\strut\hfil	 epigraficabi8r	&\cr	\hrule
\+\hfil	\eightit Italique	&\strut\vrule\strut\hfil	 epigraficai8r	 &\strut\vrule\kern1pt\vrule\strut\hfil	 \eightcaps Petites capitales	 &\strut\vrule\strut\hfil	 epigraficac8r	 &\cr	\hrule
\+\hfil	\eightsl Pench\'e	&\strut\vrule\strut\hfil	 epigraficai8r	 &\strut\vrule\kern1pt\vrule\strut\hfil	Petites capitales en gras	 &\strut\vrule\strut\hfil	Non disponible  	 &\cr	 \hrule
}\vrule}}\hrule}\hfil}								
									
\BlackBoxes								
									






\input font_antp_euler   \fontss
\section{\sixteenbf\fontss Antykwa P\'o\char'252tawskiego avec Euler}{Antykwa Poltawskiego avec Euler}
\sample
\ii Cette macro nous permet de taper du texte dans la police Antykwa P\'o\lstroke{}tawskiego et des maths dans la police Euler. Elle est d\'eclar\'ee en entrant l'instruction {\color{brown}\verbatim\input font_antp_euler|endverbatim}. Celle-ci typographie le texte dans les fontes du package  \href{http://www.tex.ac.uk/tex-archive/help/Catalogue/entries/antp.html}{antp} de J.\;M.\;No\-wacki (bas\'e sur les polices de texte  \href{http://nowacki.strefa.pl/poltawski-e.html}{Antykwa P\'o\lstroke{}tawskiego} du typographe polonais Adam P\'o\lstroke{}tawski) et les math\'ematiques dans les fontes de Walter Schmidt \href{http://www.tex.ac.uk/tex-archive/help/Catalogue/entries/eulervm.html}{Euler-VM} (bas\'e sur la police Euler d'Hermann Zapf et la police CM de Knuth). On obtient un L barr\'e~(\Lstroke) avec la commande {\color{brown}\verbatim\Lstroke|endverbatim} et un l barr\'e (\lstroke) avec la commande {\color{brown}\verbatim\lstroke|endverbatim}. Pendant l'utilisation de cette macro, les commandes par d\'efaut de plain \capstex\ {\color{brown}\verbatim\L|endverbatim} ou {\color{brown}\verbatim\l|endverbatim} ne marchent pas. Des d\'etails sur cette macro sont donn\'es dans le tableau ci-dessous.
\bs
\hfil{Affectation de fonte pour la macro {\color{brown}\verbatim font_antp_euler|endverbatim}}\hfil

{\parindent=0pt\settabs4\columns\hfil\vbox{\hrule\hbox{\vrule\hbox{\vbox{\kern1pt\hrule\NoBlackBoxes		 \eightrm\fontss							 \+\hfil	 \textcolor{blue}{ Style}	 &\strut\vrule\strut\hfil	 \textcolor{blue}{ Nom de la fonte}	 &\strut\vrule\kern1pt\vrule\strut\hfil	 \textcolor{blue}{ Style}	 &\strut\vrule\strut\hfil	\textcolor{blue}{ Nom de la fonte}	 &\cr	\hrule
\+\hfil	\eightrm Romain	&\strut\vrule\strut\hfil	 rm-antpr10	 &\strut\vrule\kern1pt\vrule\strut\hfil	\eightbf Gras	 &\strut\vrule\strut\hfil	 rm-antpb10	&\cr	 \hrule
\+\hfil	\eighti Maths italique	&\strut\vrule\strut\hfil	 eurm10	 &\strut\vrule\kern1pt\vrule\strut\hfil	\eighttt Machine \`a \'ecire	 &\strut\vrule\strut\hfil	 ly1-zi4r-1	 &\cr	\hrule
\+\hfil	\eightrm Symboles maths	&\strut\vrule\strut\hfil	 cmsy10	 &\strut\vrule\kern1pt\vrule\strut\hfil	\eightitbf Gras italique	 &\strut\vrule\strut\hfil	rm-antpbi10	 &\cr	\hrule
\+\hfil	\eightrm Extension maths	&\strut\vrule\strut\hfil	 euex10	 &\strut\vrule\kern1pt\vrule\strut\hfil	\eightslbf Gras pench\'e	 &\strut\vrule\strut\hfil	rm-antpbi10	 &\cr	\hrule
\+\hfil	\eightit Italique	&\strut\vrule\strut\hfil	 rm-antpri10	 &\strut\vrule\kern1pt\vrule\strut\hfil	\eightcaps Petites capitales	 &\strut\vrule\strut\hfil	 rm-antpr10-sc  	 &\cr	 \hrule
\+\hfil	\eightsl Pench\'e	&\strut\vrule\strut\hfil	 rm-antpri10	 &\strut\vrule\kern1pt\vrule\strut\hfil\eightcapsbf	Petites capitales en gras	 &\strut\vrule\strut\hfil	rm-antpb10-sc  	 &\cr	 \hrule
}\vrule}}\hrule}\hfil}								
									
\BlackBoxes								









\input font_bera_concrete   \fontss
\section{\sixteenbf\fontss Bera Serif avec Concrete}{Bera Serif avec Concrete}
\samplebera
\ii Cette macro nous permet de taper du texte dans la police Bera serif et des maths avec Concrete. La macro est déclarée en entrant l'instruction {\color{brown}\verbatim\input font_bera_concrete|endverbatim}. Celle-ci typographie le texte dans les fontes Bera serif du package \href{http://www.tex.ac.uk/tex-archive/help/Catalogue/entries/bera.html}{bera} de Walter Schmidt (basé sur la police \href{http://www.fontsquirrel.com/fonts/Bitstream-Vera-Serif?q[term]=bitstream+vera&q[search_check]=Y}{Bitstream Vera serif} dessinée par Jim Lyles de Bitstream Inc.) et les parties mathématiques sont typographiées en utilisant le package \href{http://www.tex.ac.uk/tex-archive/help/Catalogue/entries/cc-pl.html}{cc-pl} de Jackowski, Ry\char'242ko et Bzyl (basé sur les polices de Knuth  \href{http://www.tex.ac.uk/tex-archive/help/Catalogue/entries/concrete.html}{Concrete Roman}). Des détails sur cette macro sont donnés dans le tableau ci-dessous.
\bs
\hfil{Affectation de fonte pour la macro {\color{brown}\verbatim font_bera_concrete|endverbatim}}\hfil

{\parindent=0pt\settabs4\columns\hfil\vbox{\hrule\hbox{\vrule\hbox{\vbox{\kern1pt\hrule\NoBlackBoxes		 \eightrm\fontss							 \+\hfil	 \textcolor{blue}{ Style}	 &\strut\vrule\strut\hfil	 \textcolor{blue}{ Nom de la fonte}	 &\strut\vrule\kern1pt\vrule\strut\hfil	 \textcolor{blue}{ Style}	 &\strut\vrule\strut\hfil	\textcolor{blue}{ Nom de la fonte}	 &\cr	\hrule
\+\hfil	\eightrm Romain	&\strut\vrule\strut\hfil	 fver8t	 &\strut\vrule\kern1pt\vrule\strut\hfil	\eightbf Gras	 &\strut\vrule\strut\hfil	 fveb8t	&\cr	 \hrule
\+\hfil	\eighti Maths italique	&\strut\vrule\strut\hfil	 pcmi10	 &\strut\vrule\kern1pt\vrule\strut\hfil	\eighttt Machine \`a \'ecire	 &\strut\vrule\strut\hfil	 fvmr8t	&\cr	 \hrule
\+\hfil	\eightrm Symboles maths	&\strut\vrule\strut\hfil	 cmsy10	 &\strut\vrule\kern1pt\vrule\strut\hfil	\eightitbf Gras italique	 &\strut\vrule\strut\hfil	fvebo8t	 &\cr	\hrule
\+\hfil	\eightrm Extension maths	&\strut\vrule\strut\hfil	 cmex10	 &\strut\vrule\kern1pt\vrule\strut\hfil	\eightslbf Gras pench\'e	 &\strut\vrule\strut\hfil	fvebo8t	 &\cr	\hrule
\+\hfil	\eightit Italique	&\strut\vrule\strut\hfil	 fvero8t	 &\strut\vrule\kern1pt\vrule\strut\hfil	Petites capitales	 &\strut\vrule\strut\hfil	 Non disponible  	 &\cr	 \hrule
\+\hfil	\eightsl Pench\'e	&\strut\vrule\strut\hfil	 fvero8t	 &\strut\vrule\kern1pt\vrule\strut\hfil	Petites capitales en gras	 &\strut\vrule\strut\hfil	 Non disponible  	 &\cr	 \hrule
}\vrule}}\hrule}\hfil}								
									
\BlackBoxes								









\input font_bera_euler   \fontss
\section{\sixteenbf\fontss Bera Serif avec Euler}{Bera Serif avec Euler}
\sample
\ii Cette macro nous permet de taper du texte dans la police Bera serif et des maths avec Euler. La macro est d\'eclar\'ee en entrant l'instruction {\color{brown}\verbatim\input font_bera_euler|endverbatim}. Celle-ci typographie le texte dans les fontes Bera serif du package \href{http://www.tex.ac.uk/tex-archive/help/Catalogue/entries/bera.html}{bera} de Walter Schmidt (bas\'e sur la police \href{http://www.fontsquirrel.com/fonts/Bitstream-Vera-Serif?q[term]=bitstream+vera&q[search_check]=Y}{Bitstream Vera serif} dessin\'ee par Jim Lyles de Bitstream Inc.) et les maths dans les fontes \href{http://www.tex.ac.uk/tex-archive/help/Catalogue/entries/eulervm.html}{Euler-VM} de Walter Schmidt (bas\'e sur la police Euler d'Hermann Zapf et la police CM de Knuth). Des d\'etails sur cette macro sont donn\'es dans le tableau ci-dessous.
\bs
\hfil{Affectation de fonte pour la macro {\color{brown}\verbatim font_bera_euler|endverbatim}}\hfil

{\parindent=0pt\settabs4\columns\hfil\vbox{\hrule\hbox{\vrule\hbox{\vbox{\kern1pt\hrule\NoBlackBoxes		 \eightrm\fontss							 \+\hfil	 \textcolor{blue}{ Style}	 &\strut\vrule\strut\hfil	 \textcolor{blue}{ Nom de la fonte}	 &\strut\vrule\kern1pt\vrule\strut\hfil	 \textcolor{blue}{ Style}	 &\strut\vrule\strut\hfil	\textcolor{blue}{ Nom de la fonte}	 &\cr	\hrule
\+\hfil	\eightrm Romain	&\strut\vrule\strut\hfil	 fver8t	 &\strut\vrule\kern1pt\vrule\strut\hfil	\eightbf Gras	 &\strut\vrule\strut\hfil	 fveb8t	&\cr	 \hrule
\+\hfil	\eighti Maths italique	&\strut\vrule\strut\hfil	 eurm10	 &\strut\vrule\kern1pt\vrule\strut\hfil	\eighttt Machine \`a \'ecire	 &\strut\vrule\strut\hfil	 fvmr8t	&\cr	 \hrule
\+\hfil	\eightrm Symboles maths	&\strut\vrule\strut\hfil	 cmsy10	 &\strut\vrule\kern1pt\vrule\strut\hfil	\eightitbf Gras italique	 &\strut\vrule\strut\hfil	fvebo8t	 &\cr	\hrule
\+\hfil	\eightrm Extension maths	&\strut\vrule\strut\hfil	 euex10	 &\strut\vrule\kern1pt\vrule\strut\hfil	\eightslbf Gras pench\'e	 &\strut\vrule\strut\hfil	fvebo8t	 &\cr	\hrule
\+\hfil	\eightit Italique	&\strut\vrule\strut\hfil	 fvero8t	 &\strut\vrule\kern1pt\vrule\strut\hfil	Petites capitales	 &\strut\vrule\strut\hfil	 Non disponible  	 &\cr	 \hrule
\+\hfil	\eightsl Pench\'e	&\strut\vrule\strut\hfil	 fvero8t	 &\strut\vrule\kern1pt\vrule\strut\hfil	Petites capitales en gras	 &\strut\vrule\strut\hfil	 Non disponible  	 &\cr	 \hrule
}\vrule}}\hrule}\hfil}								
									
\BlackBoxes								









\input font_bera_fnc   \fontss
\section{\sixteenbf\fontss Bera Serif avec Fouriernc}{Bera Serif avec Fouriernc}
\sample
\ii Cette macro nous permet de taper du texte dans la police Bera serif et des math\'ematiques avec Fouriernc (utilis\'ee \`a l'origine avec New Century). La macro est d\'eclar\'ee en entrant l'instruction {\color{brown}\verbatim\input font_bera_fnc|endverbatim}. Celle-ci typographie le texte dans les fontes Bera serif du package \href{http://www.tex.ac.uk/tex-archive/help/Catalogue/entries/bera.html}{bera} de Walter Schmidt (bas\'e sur la police \href{http://www.fontsquirrel.com/fonts/Bitstream-Vera-Serif?q[term]=bitstream+vera&q[search_check]=Y}{Bitstream Vera serif} dessin\'ee par Jim Lyles de Bitstream Inc.) et les maths dans les fontes \href{http://www.tex.ac.uk/tex-archive/help/Catalogue/entries/fouriernc.html}{fouriernc} de Michael Zedler. Des d\'etails sur cette macro sont donn\'es dans le tableau ci-dessous.
\bs
\hfil{Affectation de fonte pour la macro {\color{brown}\verbatim font_bera_fnc|endverbatim}}\hfil

{\parindent=0pt\settabs4\columns\hfil\vbox{\hrule\hbox{\vrule\hbox{\vbox{\kern1pt\hrule\NoBlackBoxes		 \eightrm\fontss							 \+\hfil	 \textcolor{blue}{ Style}	 &\strut\vrule\strut\hfil	 \textcolor{blue}{ Nom de la fonte}	 &\strut\vrule\kern1pt\vrule\strut\hfil	 \textcolor{blue}{ Style}	 &\strut\vrule\strut\hfil	\textcolor{blue}{ Nom de la fonte}	 &\cr	\hrule
\+\hfil	\eightrm Romain	&\strut\vrule\strut\hfil	 fver8t	 &\strut\vrule\kern1pt\vrule\strut\hfil	\eightbf Gras	 &\strut\vrule\strut\hfil	 fveb8t	&\cr	 \hrule
\+\hfil	\eighti Maths italique	&\strut\vrule\strut\hfil	 fncmii	 &\strut\vrule\kern1pt\vrule\strut\hfil	\eighttt Machine \`a \'ecire	 &\strut\vrule\strut\hfil	 fvmr8t	&\cr	 \hrule
\+\hfil	\eightrm Symboles maths	&\strut\vrule\strut\hfil	 fncsy	 &\strut\vrule\kern1pt\vrule\strut\hfil	\eightitbf Gras italique	 &\strut\vrule\strut\hfil	fvebo8t	 &\cr	\hrule
\+\hfil	\eightrm Extension maths	&\strut\vrule\strut\hfil	 cmex10	 &\strut\vrule\kern1pt\vrule\strut\hfil	\eightslbf Gras pench\'e	 &\strut\vrule\strut\hfil	fvebo8t	 &\cr	\hrule
\+\hfil	\eightit Italique	&\strut\vrule\strut\hfil	 fvero8t	 &\strut\vrule\kern1pt\vrule\strut\hfil	Petites capitales	 &\strut\vrule\strut\hfil	Non disponible  	 &\cr	 \hrule
\+\hfil	\eightsl Pench\'e	&\strut\vrule\strut\hfil	 fvero8t	 &\strut\vrule\kern1pt\vrule\strut\hfil	Petites capitales en gras	 &\strut\vrule\strut\hfil	 Non disponible  	 &\cr	 \hrule
}\vrule}}\hrule}\hfil}								
									
\BlackBoxes								






\input font_artemisia_euler   \fontss
\section{\sixteenbf\fontss Artemisia avec Euler}{Artemisia avec Euler}
\sampleansi
\ii Cette macro nous permet de taper du texte dans la police GFS Artemisia et des maths avec Euler. La macro est déclarée en entrant l'instruction {\color{brown}\verbatim\input font_artemisia_euler|endverbatim}. Celle-ci typographie le texte dans les fontes d'Antonis Tsolomitis, George D.\ Matthiopoulos et de The Greek Font Society \href{http://www.tex.ac.uk/tex-archive/help/Catalogue/entries/gfsartemisia.html}{GFS Artemisia fonts}, et les maths dans les fontes \href{http://www.tex.ac.uk/tex-archive/help/Catalogue/entries/eulervm.html}{Euler-VM} de Walter Schmidt (basé sur la police Euler d'Hermann Zapf et la police CM de Knuth). Des détails sur cette macro sont donnés dans le tableau ci-dessous.
\bs
\hfil{Affectation de fonte pour la macro {\color{brown}\verbatim font_artemisia_euler|endverbatim}}\hfil

{\parindent=0pt\settabs4\columns\hfil\vbox{\hrule\hbox{\vrule\hbox{\vbox{\kern1pt\hrule\NoBlackBoxes		 \eightrm\fontss							
									
\+\hfil	\textcolor{blue}{ Style}	&\strut\vrule\strut\hfil	 \textcolor{blue}{ Nom de la fonte}	 &\strut\vrule\kern1pt\vrule\strut\hfil	 \textcolor{blue}{ Style}	 &\strut\vrule\strut\hfil	\textcolor{blue}{ Nom de la fonte}	&\cr	\hrule
\+\hfil	\eightrm Romain	&\strut\vrule\strut\hfil	 artemisiarg8a	 &\strut\vrule\kern1pt\vrule\strut\hfil	 \eightbf Gras	 &\strut\vrule\strut\hfil	 artemisiab8a	&\cr	\hrule
\+\hfil	\eighti Maths italique	&\strut\vrule\strut\hfil	 zeurm10	 &\strut\vrule\kern1pt\vrule\strut\hfil	\eighttt Machine \`a \'ecire	 &\strut\vrule\strut\hfil	 ly1-zi4r-1	 &\cr	\hrule
\+\hfil	\eightrm Symboles maths	&\strut\vrule\strut\hfil	 zeusm10	 &\strut\vrule\kern1pt\vrule\strut\hfil	\eightitbf Gras italique	 &\strut\vrule\strut\hfil	 artemisiabi8a	&\cr	\hrule
\+\hfil	\eightrm Extension maths	&\strut\vrule\strut\hfil	 zeuex10	 &\strut\vrule\kern1pt\vrule\strut\hfil	\eightslbf Gras pench\'e	 &\strut\vrule\strut\hfil	 artemisiabo8a	&\cr	\hrule
\+\hfil	\eightit Italique	&\strut\vrule\strut\hfil	 artemisiai8a	 &\strut\vrule\kern1pt\vrule\strut\hfil	 \eightcaps Petites capitales &\strut\vrule\strut\hfil	 artemisiasc8a	 &\cr	\hrule
\+\hfil	\eightsl Pench\'e	&\strut\vrule\strut\hfil	 artemisiao8a	 &\strut\vrule\kern1pt\vrule\strut\hfil	Petites capitales en gras	 &\strut\vrule\strut\hfil	\emdash  	 &\cr	\hrule
									
	}\vrule}}\hrule}\hfil}								
									
	\BlackBoxes								
								











\input font_libertine_kp   \fontss
\UseAMSsymbols
\section{\sixteenbf\fontss Libertine avec Kp-Fonts}{Libertine avec Kp-Fonts}
\sampleansi
\ii Cette macro nous permet de taper du texte dans la police Linux-Libertine et des maths dans les polices Kp-Fonts. La macro est déclarée en entrant l'instruction {\color{brown}\verbatim\input font_libertine_kp|endverbatim}. Celle-ci typographie le texte dans les fontes \href{http://www.tex.ac.uk/tex-archive/help/Catalogue/entries/libertine.html}{Linux-Libertine} de Michael Niedermair   et les maths dans celles de Chris\-tophe Caignaert, \href{http://www.tex.ac.uk/tex-archive/help/Catalogue/entries/kpfonts.html}{Kp-Fonts}. Les polices de cette macro fournissent leurs propres symboles {\caps ams}. Des détails sur cette macro sont donnés dans le tableau ci-dessous.
\bs
\hfil{Affectation de fonte pour la macro {\color{brown}\verbatim font_libertine_kp|endverbatim}}\hfil
					
{\parindent=0pt\settabs4\columns\hfil\vbox{\hrule\hbox{\vrule\hbox{\vbox{\kern1pt\hrule\NoBlackBoxes		 \eightrm\fontss							
									
\+\hfil	\textcolor{blue}{ Style}	&\strut\vrule\strut\hfil	 \textcolor{blue}{ Nom de la fonte}	 &\strut\vrule\kern1pt\vrule\strut\hfil	 \textcolor{blue}{ Style}	 &\strut\vrule\strut\hfil	\textcolor{blue}{ Nom de la fonte}	&\cr	\hrule
\+\hfil	\eightrm Romain	&\strut\vrule\strut\hfil	 LinLibertineT-lf-ot1	 &\strut\vrule\kern1pt\vrule\strut\hfil	\eightbf Gras	 &\strut\vrule\strut\hfil	 LinLibertineTZ-lf-ot1	&\cr	 \hrule
\+\hfil	\eighti Maths italique	&\strut\vrule\strut\hfil	 jkpmi	 &\strut\vrule\kern1pt\vrule\strut\hfil	\eighttt Machine \`a \'ecire	 &\strut\vrule\strut\hfil	 ly1-zi4r-1	 &\cr	\hrule
\+\hfil	\eightrm Symboles maths	&\strut\vrule\strut\hfil	 jkpsy	 &\strut\vrule\kern1pt\vrule\strut\hfil	\eightitbf Gras italique	 &\strut\vrule\strut\hfil	LinLibertineTZI-lf-ot1	 &\cr	\hrule
\+\hfil	\eightrm Extension maths	&\strut\vrule\strut\hfil	 jkpex	 &\strut\vrule\kern1pt\vrule\strut\hfil	\eightslbf Gras pench\'e	 &\strut\vrule\strut\hfil	LinLibertineTZI-lf-ot1	 &\cr	\hrule
\+\hfil	\eightit Italique	&\strut\vrule\strut\hfil	 LinLibertineTI-lf-ot1	 &\strut\vrule\kern1pt\vrule\strut\hfil	 \eightcaps Petites capitales	 &\strut\vrule\strut\hfil	LinLibertineT-lf-sc-ot1	 &\cr	\hrule
\+\hfil	\eightsl Pench\'e	&\strut\vrule\strut\hfil	 LinLibertineTI-lf-ot1	 &\strut\vrule\kern1pt\vrule\strut\hfil	 \eightcapsbf Petites capitales en gras	 &\strut\vrule\strut\hfil	 LinLibertineTZ-lf-sc-ot1	&\cr	\hrule
									
	}\vrule}}\hrule}\hfil}								
									
	\BlackBoxes								

\bs\ii Symboles {\eightrm AMS} associ\'es~: \circledR \ \yen \ $\blacksquare \ \approxeq \ \eqslantgtr \ \curlyeqprec \ \curlyeqsucc \ \preccurlyeq \ \leqq \ \leqslant \ \lessgtr \ \nless \ \nleq \ \nleqslant \ \Bbb R \ \Bbb E \ \Bbb C \ \dots$






\input font_libertine_palatino   \fontss
\UseAMSsymbols
\section{\sixteenbf\fontss Libertine avec Palatino}{Libertine avec Palatino}
\sampleansi
\ii Cette macro nous permet de taper du texte dans la police Linux-Libertine et des maths dans les fontes PX. Elle est déclarée en entrant l'instruction {\color{brown}\verbatim\input font_libertine_palatino|endverbatim}.  Celle-ci typographie le texte dans les fontes \href{http://www.tex.ac.uk/tex-archive/help/Catalogue/entries/libertine.html}{Linux-Libertine} de Michael Niedermair et les maths dans celles de Young Ryu, \href{http://www.tex.ac.uk/tex-archive/help/Catalogue/entries/pxfonts.html}{pxfonts}, correspondant aux polices texte \href{http://www.myfonts.com/fonts/urw/palladio/}
{{\caps urw++} Palladio} dessinées par Herman Zapf. La police Palladio {\caps urw++} est basée sur la \href{http://new.myfonts.com/fonts/adobe/palatino/}{police Palatino} qui avait été conçue à l'origine par Hermann Zapf pour la fonderie Stempel en 1950. Les polices de cette macro fournissent leurs propres symboles {\caps ams}. Des détails sur cette macro sont donnés dans le tableau ci-dessous.
\bs
\hfil{Affectation de fonte pour la macro {\color{brown}\verbatim font_libertine_palatino|endverbatim}}\hfil
					
{\parindent=0pt\settabs4\columns\hfil\vbox{\hrule\hbox{\vrule\hbox{\vbox{\kern1pt\hrule\NoBlackBoxes		 \eightrm\fontss							
									
\+\hfil	\textcolor{blue}{ Style}	&\strut\vrule\strut\hfil	 \textcolor{blue}{ Nom de la fonte}	 &\strut\vrule\kern1pt\vrule\strut\hfil	 \textcolor{blue}{ Style}	 &\strut\vrule\strut\hfil	\textcolor{blue}{ Nom de la fonte}	&\cr	\hrule
\+\hfil	\eightrm Romain	&\strut\vrule\strut\hfil	 LinLibertineT-lf-ot1	 &\strut\vrule\kern1pt\vrule\strut\hfil	\eightbf Gras	 &\strut\vrule\strut\hfil	 LinLibertineTZ-lf-ot1	&\cr	 \hrule
\+\hfil	\eighti Maths italique	&\strut\vrule\strut\hfil	pxmi	 &\strut\vrule\kern1pt\vrule\strut\hfil	\eighttt Machine \`a \'ecire	 &\strut\vrule\strut\hfil	 ly1-zi4r-1	&\cr	 \hrule
\+\hfil	\eightrm Symboles maths	&\strut\vrule\strut\hfil	 pxsy	 &\strut\vrule\kern1pt\vrule\strut\hfil	\eightitbf Gras italique	 &\strut\vrule\strut\hfil	LinLibertineTZI-lf-ot1	 &\cr	\hrule
\+\hfil	\eightrm Extension maths	&\strut\vrule\strut\hfil	 pxex	 &\strut\vrule\kern1pt\vrule\strut\hfil	\eightslbf Gras pench\'e	 &\strut\vrule\strut\hfil	LinLibertineTZI-lf-ot1	 &\cr	\hrule
\+\hfil	\eightit Italique	&\strut\vrule\strut\hfil	 LinLibertineTI-lf-ot1	 &\strut\vrule\kern1pt\vrule\strut\hfil	 \eightcaps Petites capitales	 &\strut\vrule\strut\hfil	LinLibertineT-lf-sc-ot1	 &\cr	\hrule
\+\hfil	\eightsl Pench\'e	&\strut\vrule\strut\hfil	 LinLibertineTI-lf-ot1	 &\strut\vrule\kern1pt\vrule\strut\hfil	 \eightcapsbf Petites capitales en gras	 &\strut\vrule\strut\hfil	 LinLibertineTZ-lf-sc-ot1	&\cr	\hrule
									
	}\vrule}}\hrule}\hfil}								
									
	\BlackBoxes								

\bs\ii Symboles {\eightrm AMS} associ\'es~: \circledR \ \yen \ $\blacksquare \ \approxeq \ \eqslantgtr \ \curlyeqprec \ \curlyeqsucc \ \preccurlyeq \ \leqq \ \leqslant \ \lessgtr \ \nless \ \nleq \ \nleqslant \ \Bbb R \ \Bbb E \ \Bbb C \ \dots$







\input font_libertine_times   \fontss
\UseAMSsymbols
\section{\sixteenbf\fontss Libertine avec Times}{Libertine avec Times}
\sampleansi
\ii Cette macro nous permet de taper du texte dans la police Linux Libertine et des maths dans les fontes TX. La macro est déclarée en entrant l'instruction {\color{brown}\verbatim\input font_libertine_times|endverbatim}. Celle-ci typographie le texte dans les fontes \href{http://www.tex.ac.uk/tex-archive/help/Catalogue/entries/libertine.html}{Linux-Libertine} de Michael Niedermair et les maths dans celles de Young Ryu, \href{http://www.tex.ac.uk/tex-archive/help/Catalogue/entries/txfonts.html}{txfonts}, correspondant aux polices texte \href{http://new.myfonts.com/fonts/adobe/times/}{Adobe Times}. Les polices de cette macro fournissent leurs propres symboles {\caps ams}. Des détails sur cette macro sont donnés dans le tableau ci-dessous.
\bs
\hfil{Affectation de fonte pour la macro {\color{brown}\verbatim font_libertine_times|endverbatim}}\hfil

{\parindent=0pt\settabs4\columns\hfil\vbox{\hrule\hbox{\vrule\hbox{\vbox{\kern1pt\hrule\NoBlackBoxes		 \eightrm\fontss							
									
\+\hfil	\textcolor{blue}{ Style}	&\strut\vrule\strut\hfil	 \textcolor{blue}{ Nom de la fonte}	 &\strut\vrule\kern1pt\vrule\strut\hfil	 \textcolor{blue}{ Style}	 &\strut\vrule\strut\hfil	\textcolor{blue}{ Nom de la fonte}	&\cr	\hrule
\+\hfil	\eightrm Romain	&\strut\vrule\strut\hfil	 LinLibertineT-lf-ot1	 &\strut\vrule\kern1pt\vrule\strut\hfil	\eightbf Gras	 &\strut\vrule\strut\hfil	 LinLibertineTZ-lf-ot1	&\cr	 \hrule
\+\hfil	\eighti Maths italique	&\strut\vrule\strut\hfil	txmi	 &\strut\vrule\kern1pt\vrule\strut\hfil	\eighttt Machine \`a \'ecire	 &\strut\vrule\strut\hfil	 cmtt10	&\cr	 \hrule
\+\hfil	\eightrm Symboles maths	&\strut\vrule\strut\hfil	 txsy	 &\strut\vrule\kern1pt\vrule\strut\hfil	\eightitbf Gras italique	 &\strut\vrule\strut\hfil	LinLibertineTZI-lf-ot1	 &\cr	\hrule
\+\hfil	\eightrm Extension maths	&\strut\vrule\strut\hfil	 txex	 &\strut\vrule\kern1pt\vrule\strut\hfil	\eightslbf Gras pench\'e	 &\strut\vrule\strut\hfil	LinLibertineTZI-lf-ot1	 &\cr	\hrule
\+\hfil	\eightit Italique	&\strut\vrule\strut\hfil	 LinLibertineTI-lf-ot1	 &\strut\vrule\kern1pt\vrule\strut\hfil	 \eightcaps Petites capitales	 &\strut\vrule\strut\hfil	LinLibertineT-lf-sc-ot1	 &\cr	\hrule
\+\hfil	\eightsl Pench\'e	&\strut\vrule\strut\hfil	 LinLibertineTI-lf-ot1	 &\strut\vrule\kern1pt\vrule\strut\hfil	 \eightcapsbf Petites capitales en gras	 &\strut\vrule\strut\hfil	 LinLibertineTZ-lf-sc-ot1	&\cr	\hrule
									
	}\vrule}}\hrule}\hfil}								
									
	\BlackBoxes								

\bs\ii Symboles {\eightrm AMS} associ\'es~: \circledR \ \yen \ $\blacksquare \ \approxeq \ \eqslantgtr \ \curlyeqprec \ \curlyeqsucc \ \preccurlyeq \ \leqq \ \leqslant \ \lessgtr \ \nless \ \nleq \ \nleqslant \ \Bbb R \ \Bbb E \ \Bbb C \ \dots$























\input font_concrete   \fontss
\section{\fourteenrm Concrete}{Concrete}


{\hrule\vbox{\noindent\vrule\NoBlackBoxes\vbox{\vskip2mm\leftskip7mm\rightskip7mm
\noindent\underbar{Formule D'Euler}: La formule d'Euler, aussi connue sous le nom d'{identit\'e d'Euler}, nous dit que~:
$$e^{\imath x} =\cos(x) + \imath \sin(x), $$
o\`u $\imath$~est {\sl l'unit\'e imaginaire}.

On peut \'etendre la formule d'Euler \`a une s\'erie~:
$$\eqalign {e^{\imath x}
&= \sum_{n=0}^{\infty} {(\imath x)^n\over{n!}}\cr
&= \sum_{n=0}^{\infty}{(-1)^{n}x^{2n}\over (2n)!} + \imath\sum_1^{\infty}{(-1)^{n-1}x^{2n-1}\over(2n-1)!}\cr
&= \cos(x) + \imath\sin(x).\cr}$$

\bigskip\bigskip
\noindent\underbar{Th\'eor\`eme Int\'egral de Cauchy}: Si $f(z)$ est analytique et ses d\'eriv\'ees partielles continues sur une r\'egion~$R$ simplement connexe,~alors~:
$$\oint_\gamma f(z)\,dz  = 0$$
pour tout lacet rectifiable~$\gamma$ contenu int\'egralement dans~$R$.\vskip2mm
}\vrule}\hrule\BlackBoxes\bigskip\bigskip}

\ii Cette macro nous permet de taper du texte et des maths dans la police \href{http://www.tex.ac.uk/tex-archive/help/Catalogue/entries/concrete.html}{Concrete} de Donald Knuth. Cette macro est d\'eclar\'ee en entrant l'instruction {\color{brown}\verbatim\input font_concrete|endverbatim}. La macro utilise le package de Jackowski, Ry\'cko et Bzyl \href{http://www.tex.ac.uk/tex-archive/help/Catalogue/entries/cc-pl.html}{cc-pl}, qui est bas\'e sur la police de Knuth, \href{http://www.tex.ac.uk/tex-archive/help/Catalogue/entries/concrete.html}{Concrete Roman}. Des d\'etails sur cette macro sont donn\'es dans le tableau ci-dessous.
\bs
\hfil{Affectation de fonte pour la macro {\color{brown}\verbatim font_concrete|endverbatim}}\hfil

{\parindent=0pt\settabs4\columns\hfil\vbox{\hrule\hbox{\vrule\hbox{\vbox{\kern1pt\hrule\NoBlackBoxes		 \eightrm\fontss							 \+\hfil	 \textcolor{blue}{ Style}	 &\strut\vrule\strut\hfil	 \textcolor{blue}{ Nom de la fonte}	 &\strut\vrule\kern1pt\vrule\strut\hfil	 \textcolor{blue}{ Style}	 &\strut\vrule\strut\hfil	\textcolor{blue}{ Nom de la fonte}	 &\cr	\hrule
\+\hfil	\eightrm Romain	&\strut\vrule\strut\hfil	 pcr10	 &\strut\vrule\kern1pt\vrule\strut\hfil	Gras	 &\strut\vrule\strut\hfil	 Non disponible  	&\cr	 \hrule
\+\hfil	\eighti Maths italique	&\strut\vrule\strut\hfil	 pcmi10	 &\strut\vrule\kern1pt\vrule\strut\hfil	\eighttt Machine \`a \'ecire	 &\strut\vrule\strut\hfil	 cmtt10	&\cr	 \hrule
\+\hfil	\eightrm Symboles maths	&\strut\vrule\strut\hfil	 cmsy10	 &\strut\vrule\kern1pt\vrule\strut\hfil	Gras italique	 &\strut\vrule\strut\hfil	 Non disponible  	&\cr	 \hrule
\+\hfil	\eightrm Extension maths	&\strut\vrule\strut\hfil	 cmex10	 &\strut\vrule\kern1pt\vrule\strut\hfil	Gras pench\'e	 &\strut\vrule\strut\hfil	 Non disponible  	&\cr	 \hrule
\+\hfil	\eightit Italique	&\strut\vrule\strut\hfil	 pcti10	 &\strut\vrule\kern1pt\vrule\strut\hfil	\eightcaps Petites capitales	 &\strut\vrule\strut\hfil	 pccsc10	 &\cr	 \hrule
\+\hfil	\eightsl Pench\'e	&\strut\vrule\strut\hfil	 pcsl10	 &\strut\vrule\kern1pt\vrule\strut\hfil	Petites capitales en gras	 &\strut\vrule\strut\hfil	 Non disponible  	 &\cr	 \hrule
}\vrule}}\hrule}\hfil}								

\BlackBoxes								
									







\input font_cm   \fontss
\section{\sixteenbf\fontss Computer Modern}{Computer Modern}
\sample
\ii Cette macro nous permet de taper du texte dans la police Computer Modern (s\'erif). Bien que \capstex\ produit des documents par d\'efaut dans les polices Computer Modern de Donald Knuth, cette macro est fournie pour que l'utilisateur puisse utiliser les diff\'erentes tailles comme mentionn\'e dans ce document, et au cas o\`u la police principale de n'importe quel document \capstex\ document est autre que Computer Modern (ainsi, en utilisant cette macro, on peut changer la police en Computer Modern dans un groupe). La macro est d\'eclar\'ee en entrant l'instruction {\color{brown}\verbatim\input font_cm|endverbatim}. Des d\'etails sur cette macro sont donn\'es dans le tableau ci-dessous.
\bs
\hfil{Affectation de fonte pour la macro {\color{brown}\verbatim font_cm|endverbatim}}\hfil

{\parindent=0pt\settabs4\columns\hfil\vbox{\hrule\hbox{\vrule\hbox{\vbox{\kern1pt\hrule\NoBlackBoxes		 \eightrm\fontss							 \+\hfil	 \textcolor{blue}{ Style}	 &\strut\vrule\strut\hfil	 \textcolor{blue}{ Nom de la fonte}	 &\strut\vrule\kern1pt\vrule\strut\hfil	 \textcolor{blue}{ Style}	 &\strut\vrule\strut\hfil	\textcolor{blue}{ Nom de la fonte}	 &\cr	\hrule
\+\hfil	\eightrm Romain	&\strut\vrule\strut\hfil	 cmr10	 &\strut\vrule\kern1pt\vrule\strut\hfil	\eightbf Gras	 &\strut\vrule\strut\hfil	 cmbx10	&\cr	 \hrule
\+\hfil	\eighti Maths italique	&\strut\vrule\strut\hfil	 cmmi10	 &\strut\vrule\kern1pt\vrule\strut\hfil	\eighttt Machine \`a \'ecire	 &\strut\vrule\strut\hfil	 cmtt10	&\cr	 \hrule
\+\hfil	\eightrm Symboles maths	&\strut\vrule\strut\hfil	 cmsy10	 &\strut\vrule\kern1pt\vrule\strut\hfil	\eightitbf Gras italique	 &\strut\vrule\strut\hfil	cmbxti10	 &\cr	\hrule
\+\hfil	\eightrm Extension maths	&\strut\vrule\strut\hfil	 cmex10	 &\strut\vrule\kern1pt\vrule\strut\hfil	\eightslbf Gras pench\'e	 &\strut\vrule\strut\hfil	cmbxsl10	 &\cr	\hrule
\+\hfil	\eightit Italique	&\strut\vrule\strut\hfil	 cmti10	 &\strut\vrule\kern1pt\vrule\strut\hfil	\eightcaps Petites capitales	 &\strut\vrule\strut\hfil	 cmcsc10	 &\cr	 \hrule
\+\hfil	\eightsl Pench\'e	&\strut\vrule\strut\hfil	 cmsl10	 &\strut\vrule\kern1pt\vrule\strut\hfil	Petites capitales en gras	 &\strut\vrule\strut\hfil	 Non disponible	&\cr	 \hrule
}\vrule}}\hrule}\hfil}								
									
\BlackBoxes								









%%%%%%%%%%   Styles et Tailles   %%%%%%%%%%
\input font_charter \fontss
\section{Typefaces and Sizes}{Styles et Tailles}

\ii Sont montr\'es ci-dessous diff\'erents styles et tailles propos\'es par mes macros. \bs\hrule\kern1pt\hrule\bs


{\obeylines
\rightline{Romain}\nopagebreak
{\twentyrm \fontss Ce texte est en taille 20\,pt.}
{\eighteenrm \fontss Ce texte est en taille 18\,pt.}
{\sixteenrm \fontss Ce texte est en taille 16\,pt.}
{\fourteenrm \fontss Ce texte est en taille 14\,pt.}
{\twelverm \fontss Ce texte est en taille 12\,pt.}
{\rm \fontss Ce texte est en taille 10\,pt.}
{\ninerm \fontss Ce texte est en taille 9\,pt.}
{\eightrm \fontss Ce texte est en taille 8\,pt.}
{\sevenrm \fontss Ce texte est en taille 7\,pt.}
{\sixrm \fontss Ce texte est en taille 6\,pt.}
{\fiverm \fontss Ce texte est en taille 5\,pt.}
\

\rightline{Italique}\nopagebreak
{\twentyit \fontss Ce texte est en taille 20\,pt.}
{\eighteenit \fontss Ce texte est en taille 18\,pt.}
{\sixteenit \fontss Ce texte est en taille 16\,pt.}
{\fourteenit \fontss Ce texte est en taille 14\,pt.}
{\twelveit \fontss Ce texte est en taille 12\,pt.}
{\it \fontss Ce texte est en taille 10\,pt.}
{\nineit \fontss Ce texte est en taille 9\,pt.}
{\eightit \fontss Ce texte est en taille 8\,pt.}
{\sevenit \fontss Ce texte est en taille 7\,pt.}
{\sixit \fontss Ce texte est en taille 6\,pt.}
{\fiveit \fontss Ce texte est en taille 5\,pt.}
\

\rightline{Pench\'e}\nopagebreak
{\twentysl \fontss Ce texte est en taille 20\,pt.}
{\eighteensl \fontss Ce texte est en taille 18\,pt.}
{\sixteensl \fontss Ce texte est en taille 16\,pt.}
{\fourteensl \fontss Ce texte est en taille 14\,pt.}
{\twelvesl \fontss Ce texte est en taille 12\,pt.}
{\sl \fontss Ce texte est en taille 10\,pt.}
{\ninesl \fontss Ce texte est en taille 9\,pt.}
{\eightsl \fontss Ce texte est en taille 8\,pt.}
{\sevensl \fontss Ce texte est en taille 7\,pt.}
{\sixsl \fontss Ce texte est en taille 6\,pt.}
{\fivesl \fontss Ce texte est en taille 5\,pt.}
\

\newpage
\rightline{Gras}\nopagebreak
{\twentybf \fontss Ce texte est en taille 20\,pt.}
{\eighteenbf \fontss Ce texte est en taille 18\,pt.}
{\sixteenbf \fontss Ce texte est en taille 16\,pt.}
{\fourteenbf \fontss Ce texte est en taille 14\,pt.}
{\twelvebf \fontss Ce texte est en taille 12\,pt.}
{\bf \fontss Ce texte est en taille 10\,pt.}
{\ninebf \fontss Ce texte est en taille 9\,pt.}
{\eightbf \fontss Ce texte est en taille 8\,pt.}
{\sevenbf \fontss Ce texte est en taille 7\,pt.}
{\sixbf \fontss Ce texte est en taille 6\,pt.}
{\fivebf \fontss Ce texte est en taille 5\,pt.}
\

\rightline{Gras italique}\nopagebreak
{\twentyitbf \fontss Ce texte est en taille 20\,pt.}
{\eighteenitbf \fontss Ce texte est en taille 18\,pt.}
{\sixteenitbf \fontss Ce texte est en taille 16\,pt.}
{\fourteenitbf \fontss Ce texte est en taille 14\,pt.}
{\twelveitbf \fontss Ce texte est en taille 12\,pt.}
{\itbf \fontss Ce texte est en taille 10\,pt.}
{\nineitbf \fontss Ce texte est en taille 9\,pt.}
{\eightitbf \fontss Ce texte est en taille 8\,pt.}
{\sevenitbf \fontss Ce texte est en taille 7\,pt.}
{\sixitbf \fontss Ce texte est en taille 6\,pt.}
{\fiveitbf \fontss Ce texte est en taille 5\,pt.}
\

\rightline{Gras pench\'e}\nopagebreak
{\twentyslbf \fontss Ce texte est en taille 20\,pt.}
{\eighteenslbf \fontss Ce texte est en taille 18\,pt.}
{\sixteenslbf \fontss Ce texte est en taille 16\,pt.}
{\fourteenslbf \fontss Ce texte est en taille 14\,pt.}
{\twelveslbf \fontss Ce texte est en taille 12\,pt.}
{\slbf \fontss Ce texte est en taille 10\,pt.}
{\nineslbf \fontss Ce texte est en taille 9\,pt.}
{\eightslbf \fontss Ce texte est en taille 8\,pt.}
{\sevenslbf \fontss Ce texte est en taille 7\,pt.}
{\sixslbf \fontss Ce texte est en taille 6\,pt.}
{\fiveslbf \fontss Ce texte est en taille 5\,pt.}
\

\rightline{Petites capitales}\nopagebreak
{\twentycaps \fontss Ce texte est en taille 20\,pt.}
{\eighteencaps \fontss Ce texte est en taille 18\,pt.}
{\sixteencaps \fontss Ce texte est en taille 16\,pt.}
{\fourteencaps \fontss Ce texte est en taille 14\,pt.}
{\twelvecaps \fontss Ce texte est en taille 12\,pt.}
{\caps \fontss Ce texte est en taille 10\,pt.}
{\ninecaps \fontss Ce texte est en taille 9\,pt.}
{\eightcaps \fontss Ce texte est en taille 8\,pt.}
{\sevencaps \fontss Ce texte est en taille 7\,pt.}
{\sixcaps \fontss Ce texte est en taille 6\,pt.}
{\fivecaps \fontss Ce texte est en taille 5\,pt.}

\

\rightline{Petites capitales en gras}\nopagebreak
{\twentycapsbf \fontss Ce texte est en taille 20\,pt.}
{\eighteencapsbf \fontss Ce texte est en taille 18\,pt.}
{\sixteencapsbf \fontss Ce texte est en taille 16\,pt.}
{\fourteencapsbf \fontss Ce texte est en taille 14\,pt.}
{\twelvecapsbf \fontss Ce texte est en taille 112\,pt.}
{\capsbf \fontss Ce texte est en taille 10\,pt.}
{\ninecapsbf \fontss Ce texte est en taille 9\,pt.}
{\eightcapsbf \fontss Ce texte est en taille 8\,pt.}
{\sevencapsbf \fontss Ce texte est en taille 7\,pt.}
{\sixcapsbf \fontss Ce texte est en taille 6\,pt.}
{\fivecapsbf \fontss Ce texte est en taille 5\,pt.}
}














%%%%%%%%%%   Espacement Interligne et Intermot   %%%%%%%%%%
\section{Espacement Interligne et Intermot}{Espacement Interligne et Intermot}

\ii Les caract\`eres \'etant chers au style typographique, leur arrangement l'est aussi. Bien s\^ur, la valeur de la signification et de l'utilit\'e du texte, qui tient m\^eme si les phrases ont \'et\'e gribouill\'ees, n'a pas de comparaison, mais il est de bon ton d'arranger un texte d'une jolie mani\`ere. Cette partie traite de deux caract\'eristiques pro\'eminentes dans la composition de texte, l'espacement interligne et intermot.

Si l'on change la police texte \capstex, l'espacement interligne et intermot n'est pas modifi\'e en cons\'equence. Ce n'est pas un gros probl\`eme si on d\'eclare une nouvelle police \`a la m\^eme taille que la pr\'ec\'edente. Mais si la nouvelle police est d\'eclar\'ee \`a une taille consid\'erablement inf\'erieure ou sup\'erieure, la composition ne sera probablement pas (esth\'etiquement) \'el\'egante.

Examinons ces probl\`emes d'espacement en commen\c cant par un exemple. Ensuite, une solution ``acceptable'' pour ce probl\`eme sera pr\'esent\'ee. Celle-ci n'est pas parfaite, mais elle est pratique et c'est un compromis tol\'erable. Puis nous nous dirigerons vers les aspects plus th\'eoriques de l'espacement. La discussion, plut\^ot br\`eve, pourra agir comme point de d\'epart dans la r\'e-\'evaluation de ``probl\`emes d'espace''. En ce qui concerne l'espacement des mots, le meilleur guide est l'exp\-\'erience. En essayant de justifier du texte~(12\,pt) dans des colonnes triples sur une page A4, on s'expose certainement \`a quelques difficult\'es. Plus la colonne est \'etroite, et plus la justification est s\'ev\`ere. Nous n'allons pas d\'elib\'erer sur des histoires de microtypographie (une approche distinctive qui traite beaucoup de probl\`emes d'espacement et qui peut \^etre utilis\'ee avec~pdf\capstex\ ). Les lecteurs curieux pourront se r\'ef\'erer \`a ces trois ouvrages~:~\cite{zapf_microtypography}, \cite{thanh_microtypographic}, et~\cite{text_justification}.


\subsection{Exemple}{Exemple}Un \'echantillon de fichier source \capstex\, montr\'e ci-dessous\dots

\bigskip\hrule\vbox{\parindent=0pt\vrule\NoBlackBoxes\vbox{\vskip2mm\leftskip7mm\rightskip7mm
{\obeylines\parindent=0pt\color{brown}\verbatim
\parindent=0pt
\input font_century % la taille de la police est de 10pt
Les espacements interligne et intermot sont d'importants param\`etres de composition. Un texte compos\'e avec de jolis caract\`eres mais avec un ``mauvais'' espacement interligne et intermot ne pla\^\i t pas \`a l'\oe il. Attention \`a l'espace entre les lignes d'un paragraphe, et entre les mots d'une ligne.
\medskip

\sixrm % modifie la taille de police en 6pt
Les espacements interligne et intermot sont d'importants param\`etres de composition. Un texte compos\'e avec de jolis caract\`eres mais avec un ``mauvais'' espacement interligne et intermot ne pla\^\i t pas \`a l'\oe il. Attention \`a l'espace entre les lignes d'un paragraphe, et entre les mots d'une ligne.
\medskip

\eighteenrm % modifie la taille de police en 18pt
Les espacements interligne et intermot sont d'importants param\`etres de composition. Un texte compos\'e avec de jolis caract\`eres mais avec un ``mauvais'' espacement interligne et intermot ne pla\^\i t pas \`a l'\oe il. Attention \`a l'espace entre les lignes d'un paragraphe, et entre les mots d'une ligne.|endverbatim}
\vskip2mm}\vrule}\hrule\BlackBoxes\bigskip

\nopagebreak\ii produira apr\`es compilation le r\'esultat suivant~: \nopagebreak

\bigskip\hrule\vbox{\noindent\vrule\NoBlackBoxes\vbox{\vskip2mm\leftskip7mm\rightskip7mm
{\parindent=0pt
\input font_century
% la taille de la police est de 10pt
Les espacements interligne et intermot sont d'importants param\`etres de composition. Un texte compos\'e avec de jolis caract\`eres mais avec un ``mauvais'' espacement interligne et intermot ne pla\^\i t pas \`a l'\oe il. Attention \`a l'espace entre les lignes d'un paragraphe, et entre les mots d'une ligne.
\medskip
\sixrm % modifie la taille de police en 6pt
Les espacements interligne et intermot sont d'importants param\`etres de composition. Un texte compos\'e avec de jolis caract\`eres mais avec un ``mauvais'' espacement interligne et intermot ne pla\^\i t pas \`a l'\oe il. Attention \`a l'espace entre les lignes d'un paragraphe, et entre les mots d'une ligne.
\medskip
\eighteenrm % modifie la taille de police en 18pt
Les espacements interligne et intermot sont d'importants param\`etres de composition. Un texte compos\'e avec de jolis caract\`eres mais avec un ``mauvais'' espacement interligne et intermot ne pla\^\i t pas \`a l'\oe il. Attention \`a l'espace entre les lignes d'un paragraphe, et entre les mots d'une ligne.}
\vskip2mm}\vrule}\hrule\BlackBoxes\bigskip\bigskip


\ii Au final, on remarque que les espacements interligne et intermot sont ad\'equats lorsque la taille de la police est de 10\,pt. Dans un texte en 6\,pt, l'espace interligne est trop important et l'espace intermot plus grand que n\'ecessaire. Dans un texte en 18\,pt, \`a la fois l'espace interligne et l'espace intermot sont moindres. C'est parce que \capstex\ fonctionne encore avec les valeurs d'espacement par d\'efaut, qui sont d\'eclar\'ees pour des tailles de police de 10\,pt. Pour y rem\'edier, \capstex\ offre deux primitives de contr\^ole tr\`es utiles~(\cite{knuth_texbook}, pages~76 et 78), qui sont~:\ms
{\color{brown}\verbatim \spaceskip|endverbatim} pour contr\^oler l'espace intermot, et

{\color{brown}\verbatim \baselineskip|endverbatim} pour contr\^oler l'espace interligne.




\definexref{solution}{solution}{}
\subsection{Une Solution Ais\'ee}{Une Solution Aisee}Ici je pr\'esente une technique que j'utilise pour r\'esoudre des probl\`emes d'espacement quand j'utilise diff\'erentes polices \`a diff\'erentes tailles. Ecrivons une nouvelle d\'efinition appel\'ee~{\color{brown}\verbatim \fontspacing|endverbatim}.\ms

\ii {\color{brown}\verbatim\def\fontspacing{\baselineskip=2.8ex plus0pt minus0pt
             \spaceskip=0.333333em plus0.122222em minus0.0999999em}|endverbatim}\sk
             \def\fontspacing{\baselineskip=2.8ex plus0pt minus0pt
             \spaceskip=0.333333em plus0.122222em minus0.0999999em}

\ii Les unit\'es {\sl ex\/} et {\sl em\/} sont relatives~(\cite{knuth_texbook}, page~60). Cela rend notre d\'efinition plus g\'en\'erale.\ms

{\sl em\/} est la largeur d'un cadratin (caract\`ere carr\'e dont la chasse et le corps ont la m\^eme
valeur) dans la police courante,

{\sl ex\/} est la hauteur d'x (hauteur des bas de casse) de la police courante.\sk

\ii D\'eclarer {\color{brown}\verbatim \fontspacing|endverbatim} va fixer notre espace interligne \`a 2.8ex~(=~12.05553\,pt si la police {\verbatim cmr10|endverbatim} \`a 10\,pt est utilis\'ee), sans aucune {\sl \'etirabilit\'e\/}~(donn\'e apr\`es {\sl plus\/}) ou {\sl contractibilit\'e}~(donn\'e apr\`es {\sl minus\/}). {\color{brown}\verbatim \fontspacing|endverbatim} fixe \'egalement l'espace intermot \`a 0.333333\,em, avec 0.122222\,em d'\'etirabilit\'e et 0.0999999\,em de contractibilit\'e autoris\'ees. Pour la police {\verbatim cmr10|endverbatim}, ces valeurs (d\'efaut) sont 3.33333\,pt, 1.66666\,pt, et 1.11111\,pt, respectivement.

Essayons d'utiliser {\color{brown}\verbatim \fontspacing|endverbatim} dans l'exemple donn\'e au d\'ebut de ce chapitre. Ci-dessous un exemple de source \capstex~:


\newpage

\bigskip\hrule\vbox{\noindent\vrule\NoBlackBoxes\vbox{\vskip2mm\leftskip7mm\rightskip7mm
{\obeylines\parindent=0pt\color{brown}
\verbatim
\parindent=0pt
\input font_century % the font size is 10pt
\fontspacing % \baselineskip and \spaceskip are set accordingly
Les espacements interligne et intermot sont d'importants param\`etres de composition. Un texte compos\'e avec de jolis caract\`eres mais avec un ``mauvais'' espacement interligne et intermot ne pla\^\i t pas \`a l'\oe il. Attention \`a l'espace entre les lignes d'un paragraphe, et entre les mots d'une ligne.
\medskip
\sixrm % changes the font size to 6pt
\fontspacing % \baselineskip and \spaceskip are set accordingly
Les espacements interligne et intermot sont d'importants param\`etres de composition. Un texte compos\'e avec de jolis caract\`eres mais avec un ``mauvais'' espacement interligne et intermot ne pla\^\i t pas \`a l'\oe il. Attention \`a l'espace entre les lignes d'un paragraphe, et entre les mots d'une ligne.
\medskip
\eighteenrm % changes the font size to 18pt
\fontspacing % \baselineskip and \spaceskip are set accordingly
Les espacements interligne et intermot sont d'importants param\`etres de composition. Un texte compos\'e avec de jolis caract\`eres mais avec un ``mauvais'' espacement interligne et intermot ne pla\^\i t pas \`a l'\oe il. Attention \`a l'espace entre les lignes d'un paragraphe, et entre les mots d'une ligne.|endverbatim}
\vskip2mm}\vrule}\hrule\BlackBoxes\bigskip


\nopagebreak\ii \nopagebreak\ii produira apr\`es compilation le r\'esultat suivant~:\nopagebreak


\bigskip\hrule\vbox{\noindent\vrule\NoBlackBoxes\vbox{\vskip2mm\leftskip7mm\rightskip7mm
{\parindent=0pt
\input font_century % the font size is 10pt
\fontspacing % \baselineskip and \spaceskip are set accordingly
Les espacements interligne et intermot sont d'importants param\`etres de composition. Un texte compos\'e avec de jolis caract\`eres mais avec un ``mauvais'' espacement interligne et intermot ne pla\^\i t pas \`a l'\oe il. Attention \`a l'espace entre les lignes d'un paragraphe, et entre les mots d'une ligne.
\medskip
\sixrm % changes the font size to 6pt
\fontspacing % \baselineskip and \spaceskip are set accordingly
Les espacements interligne et intermot sont d'importants param\`etres de composition. Un texte compos\'e avec de jolis caract\`eres mais avec un ``mauvais'' espacement interligne et intermot ne pla\^\i t pas \`a l'\oe il. Attention \`a l'espace entre les lignes d'un paragraphe, et entre les mots d'une ligne.
\medskip
\eighteenrm % changes the font size to 18pt
\fontspacing % \baselineskip and \spaceskip are set accordingly
Les espacements interligne et intermot sont d'importants param\`etres de composition. Un texte compos\'e avec de jolis caract\`eres mais avec un ``mauvais'' espacement interligne et intermot ne pla\^\i t pas \`a l'\oe il. Attention \`a l'espace entre les lignes d'un paragraphe, et entre les mots d'une ligne.}
\vskip2mm}\vrule}\hrule\BlackBoxes\bigskip\bigskip

En utilisant les primitives de contr\^ole {\color{brown}\verbatim \spaceskip|endverbatim} et {\color{brown}\verbatim \baselineskip|endverbatim}, qui peuvent \^etre d\'eclar\'ees presque n'importe o\`u, nous obtenons l'espacement d\'esir\'e. Pour plus de d\'etails sur l'espacement, vous pouvez vous r\'ef\'erer \`a~\cite{knuth_texbook}.


\subsection{Espacement id\'eal ?}{Espacement ideal ?}C'est un fait bien connu que les espacements interligne et intermot sont des aspects essentiels d'une bonne typographie. On appelle aussi l'espace interligne {\sl interlignage}, {\sl espace ligne}, ou encore {\sl espace interlin\'eaire}. L'espace intermot est aussi connu sous le nom d'{\sl espace mot}, ou {\sl interlettrage}. Quelles sont les ``meilleures'' valeurs pour les espaces interligne et intermot ? Bien s\^ur, il n'existe pas de r\'eponse en une ligne \`a cette question. C'est subjectif; ce qui est optimal pour l'un para\^\i tra mauvais pour l'autre.

On peut d\'ej\`a noter que l'espacement est certainement d\'ependent de la taille de la fonte. Des fontes de taille plus \'elev\'ees demandent un espacement diff\'erent de celles de tailles moyennes ou de tailles plus petites. Aussi, l'espacement (interligne et intermot) n'est pas directement ou inversement proportionnel \`a la taille des caract\`eres, bien que cela peuve servir de bonne approximation; dans notre \ref{solution} nous avons utilis\'e ce concept de proportionnalit\'e. Diff\'erents styles de caract\`eres auront besoin d'espacements diff\'erents. Le support de repr\'esentation influe aussi les valeurs d'espacement (un texte sur du papier est bien diff\'erent d'un texte sur un \'ecran d'ordinateur ou d'un diaporama projet\'e).  Les exigences d'espacement varient si le texte est sur une seule ligne et sera lu en un coup d'\oe il (tels que les listes de noms), ou si la lecture est continue (tel que ce paragraphe).

Recentrons la discussion en ne consid\'erant que le cas le plus courant, c'est-\`a-dire le texte normal; celui des livres, romans et magazines. Dans ce cas, le texte est con\c cu pour une lecture continue. M\^eme dans ce cas, pour une police donn\'ee, les exigences en terme d'espacement d\'ependent de la largeur du texte. Un texte de largeur 15\,cm devra \^etre typographi\'e avec diff\'erents param\`etres d'espacement qu'un texte de seulement 6\,cm de largeur, par exemple pour une colonne dans une page multicolonne. Mais ceci est pour une autre fois. Pour l'instant, nous allons nous concentrer sur le cas g\'en\'eral (du texte normal et continu, principalement en taille 10 \`a 14\,pt), et ainsi traiter s\'epar\'ement de l'espace interligne et de l'espace intermot.

\subsubsection{Espace Intermot}
Commen\c cons avec les r\`egles de composition de texte de \href{http://www.linotype.com/794/inhonorofthe100thbirthdayofjantschichold.html}{Jan Tschichold}, qui font partie des
\href{http://openlibrary.org/books/OL19449256M/Penguin_composition_rules.}{R\`egles de composition des livres Penguin Books}, compilation des id\'ees de Tschichold. On peut les trouver \href{http://ronin-group.org/misc_etext_tschichold.html}{ici}. Il est mentionn\'e que, pour la composition d'un texte~:
\sk{\sl
\itemitem{1.}Toute composition de texte devra avoir des mots aussi rapproch\'es que possible. Comme r\`egle, l'espacement devrait \^etre un espace moyen ou l'\'epaisseur du i dans la taille de texte utilis\'ee.\sk
\itemitem{2.}De larges espaces devront \^etre strictement \'evit\'es. Les mots seront coup\'es librement lorsque n\'ecessaire afin d'\'eviter de grands espaces, puisque la c\'esure fait moins de d\'eg\^ats dans l'apparence de la page que trop d'espace entre les mots.\sk
\itemitem{3.}Toute ponctuation majeure du texte (point, virgule, point-virgule et double point) devra \^etre suivie par le m\^eme espacement que celui utilis\'e sur le reste de la ligne.\ms}

\ii Il n'y a pas de r\`egles du jeu rigides. \href{http://www.typotheque.com/authors/robert_bringhurst}{Robert Bringhurst} \'ecrit, dans son livre influent~(\cite{elements_typographic}):
\quote{Pour un texte normal dans une taille normale, la valeur typique pour l'espace intermot est un quart d'em, ce qu'on peut \'ecrire M/4. Un quart d'em est g\'en\'eralement identique \`a, ou l\'eg\`erement plus que, la largeur de la lettre t.}

\ii L'espace intermot optimal~(sans \'etirement ou contraction) dans la police r\'eguli\`ere \capstex\ par d\'efaut (\hfuzz4pt{\verbatim cmr10|endverbatim} en 10\,pt) est de 3.33333\,pt. La largeur de la lettre i de {\verbatim cmr10|endverbatim} en 10\,pt est de 2.77779\,pt et celle de la lettre t est de 3.8889\,pt. Un quart d'em de {\verbatim cmr10|endverbatim} en 10\,pt est de 2.5\,pt. Une petite manipulation de l'espace intermot, de son \'etirabilit\'e ou contractibilit\'e, peuvent conduire \`a des changements plut\^ot apparents.

\hfuzz1pt

Id\'ealement, l'espacement intermot devrait \^etre constant tout au long du texte, ce qui est impossible d'obtenir avec un texte justifi\'e. L'\'etirement et la contraction de l'espace intermot ainsi que la c\'esure des mots ont leurs limites. Certains seraient d'accord avec Tschichold et opteraient pour plus de c\'esures et un espace intermot moins flexible pour maintenir une meilleure coloration de page, alors que d'autres diraient qu'un nombre de c\'esures excessif emp\^eche de lire correctement, et mettraient ainsi en place un espacement intermot plus large et flexible, ce qui pourrait conduire \`a des rivi\`eres. Au fil des ans, l'espace intermot s'est \'elargi, ou peut-\^etre qu'il est trop d\'ependent du langage, ou encore le manque de papier \'etait un probl\`eme par le pass\'e~: comparez l'espace intermot dans \href{http://burton.byu.edu/Bible\%20Site/Gutenberg.htm}{La Bible de Gutenberg}, connue pour son excellente typographie, et le livre de Knuth~\cite{knuth_texbook}, ouvrage typographi\'e avec \'el\'egance.

Dans ce livre~\cite{knuth_texbook}, Knuth demande \`a \capstex\ de mettre plus d'espace apr\`es les points, les virgules, les points-virgules, les doubles points, les points d'interrogation et d'exclamation. Plain~\capstex\ le ferait par d\'efaut sauf si la commande {\color{brown}\verbatim \frenchspacing|endverbatim} \'etait donn\'ee. Tschichold pr\'econise au contraire de ne pas donner autant d'espace suppl\'ementaire. Dans ce document, j'ai utilis\'e la commande {\color{brown}\verbatim \frenchspacing|endverbatim} car le texte mis en forme semblait ainsi avoir une coloration uniforme sans blocs blancs ou rivi\`eres. Mais quand j'\'ecris un rapport scientifique ou une th\`ese, qui contiennent des math\'ematiques, symboles, variables, etc., je pr\'ef\`ere la mani\`ere de Knuth, qui donne plus d'espace apr\`es la ponctuation (je pense que cela rend le texte plus lisible et plus facile \`a comprendre). Pour des colonnes multiples avec une taille de texte normale sur du A4 ou une feuille taille lettre, mon exp\'erience sugg\`ere que l'espace suppl\'ementaire apr\`es la ponctuation conduit \`a des rivi\`eres et des blocs blancs.

Diff\'erentes polices n\'ecessitent diff\'erents espaces intermot. Bitstream Charter, la police actuel\-le\-ment utilis\'ee, peut supporter un espacement intermot moindre et plus rigide (et donne m\^eme un meilleur rendu) que la police Computer Modern. Pour une typographie ``au top'', il faudra modifier l'espacement intermot en fonction de l'usage de telle ou telle police.

Le livre ~\cite{elements_typographic} mentionne une valeur ``raisonnable'' de l'espace intermot, accompagn\'ee de valeurs \'etir\'ees et r\'etr\'ecies. Traduit en langage \capstex, cela devient~: {\color{brown}\verbatim \spaceskip=0.25em plus0.08em minus0.05em|endverbatim}. Essayez-les pour voir les diff\'erents effets. Cela remplit-il la page de bo\^\i tes noires ? Quel effet sur la c\'esure ? Et si l'on utilise des colonnes multiples ?

Notre discussion sur l'espace intermot se conclut avec la phrase~: {\sl Il n'y a pas de param\`etres id\'eaux ou parfaits ou meilleurs que d'autres pour l'espacement intermot}. Nous sommes juges de notre propre travail, et raffiner sa capacit\'e de jugement vient avec l'exp\'erience, donc allons voir ce qui suit.

\subsubsection{Espace Interligne}
Il est normalement facile de g\'erer l'espace interligne, sauf si la ligne est veuve (club line). Pour du texte de taille normale, l'espace interligne a g\'en\'eralement 0\endash 4\,pt de plus que la taille des caract\`eres en points. Ce document est principalement en police {\verbatim mdbchr7t|endverbatim} en 10\,pt, avec un espace interligne de {\tt\the\baselineskip}, et, pour le grossissement global, j'ai utilis\'e {\verbatim \magnification=1100|endverbatim}.

Sur une page, disons A4, et pour n'importe quelle police, on peut accepter d'avoir un espace interligne plus restreint lorsqu'on utilise des colonnes multiples. Le regard ne perd pas la ligne et il est ais\'e de sauter \`a la ligne suivante lorsque la largeur de la colonne est moindre, par exemple environ~6\,cm. Plusieurs d\'efis tels que la composition de grilles ou les lignes veuves ponctuent le domaine de l'espacement interligne, mais nous n'allons pas en parler ici. Quant \`a l'instruction \TeX\ {\color{brown}\verbatim \baselineskip|endverbatim}, nous en avons d\'ej\`a touch\'e un mot, et pour en savoir plus, vous pouvez vous r\'ef\'erer \`a \cite{knuth_texbook} et~\cite{against_widows}.



%%%%%%%%%%   Remerciements   %%%%%%%%%%
\section{Remerciements}{Remerciements}

\bs

\ii Je remercie Donald~E.\;Knuth du fond du c\oe ur de nous avoir donn\'e \capstex, le meilleur programme de mise en forme \`a ce jour, et qui a r\'eussi \`a l'\'epreuve du temps.  J'\'eprouve \'egalement de la gratitude envers nos chers utilisateurs \capstex\ qui ont contribu\'e \`a l'am\'elioration de \capstex\ en donnant plus de latitude \`a ses fonctionalit\'es libres. Les concepteurs des polices et packages que j'ai utilis\'es font partie de ces chers utilisateurs. Je suis reconnaissant envers Petr Habala pour m'avoir initi\'e \`a \capstex; \`a ma famille pour leur amour et soutien indispensable. Je remercie ma femme, Daphne, pour avoir particip\'e aux discussions sur la typographie et pour sa compr\'ehension. Il y a tellement de facteurs que je suis incapable de prendre en compte qu'\`a la fin, mais par dessus tout, je dis~: ``Merci~{\dev~:}.''











%%%%%%%%%%   R\'ef\'erences   %%%%%%%%%%
\section{R\'ef\'erences\raise3mm\hbox{\bf 2}}{References}

\nocite{habala_amstex}\nocite{knuth_texbook}\nocite{fonts_tex_latex}\nocite{elements_typographic}

\bibliography{C:/bib}

\bibliographystyle{ieeetran}


\vskip7.5cm\hrule width 5.70truecm\kern2mm\eightrm\fontss
\hbox{ \raise1.2mm\hbox{\ \ \ 2}\ \;  Le lecteur a \'et\'e renvoy\'e \`a la plupart des r\'ef\'erences (non list\'ees sur cette page) par des hyperliens}
\hbox{\ \ \ \ \ \ \ \ donn\'es dans ce document {\caps pdf}.}





















\special{pdf: docinfo << /Author (Daphne Parramon-Dhawan)
/Title (Macros de changement des polices Texte et Maths en TeX)
/Creator(XeTeX: Bas\'e sur TeX---Le G\'enie de Knuth)
/Subject(Polices Texte et Maths en TeX)
/Keywords(TeX, gratuit, polices, fontes, typographie, fonte math\'ematique, texte, polices math\'ematiques, font-change, Charter, Utopia, Century, Palatino, Bookman, Times, Euler, Bera, Arev, Vera, Iwona, Kurier, Kp-Fonts, Antykwa Torunska, Libertine, Epigrafica, Computer Modern Bright, CM Bright, Computer Modern, Concrete, macro, macros)>>}

\special{pdf: docview <</PageMode /UseOutlines>> }















\bye







