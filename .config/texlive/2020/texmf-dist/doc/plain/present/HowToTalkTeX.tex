%
%
% HOW TO DO A TALK IN TeX
%
% Author: Matthias Meister
% Version 2.2.1, February 2019
%
% This file is part of a work distributed under the LaTeX Project Public License
%

\input present 		   % Load presentation specific macros
\input present-transitions % Load macros for transitions.

\setcolour\textcolour
\setbgcolour\backgroundcolour
\overfullrule=0pt
%\WideAspect
\vskip20mm plus 5mm minus 5mm
\line{\pagelink{NextPage}{\hfil\titlefont How To Do A Talk In \TeX\hfil}}
\vskip 5mm plus 2mm minus 1mm
\line{\hfil\font\subtitlefont=cmssbx10 at 15pt\subtitlefont One Of Many Solutions\hfil}
\vskip 5mm plus 2mm minus 1mm
\line{\hfil\linkbarfont Version 2.2.1\hfil}
\vskip 20mm plus 2mm minus 1mm
\line{\hfil\normalfont Matthias Meister\hfil}\target{title}
\vskip 5mm plus 2mm minus 1mm
\line{\hfil\normalfont February 2019\hfil}
\NewSlide
\pageno=1% Don't want to count the titlepage
\normalfont
\def\LinkBar{} 	% LinkBar is left part of the footline, and currently is empty
%%
%%
%%
\SlideTitle{\hfill Motivation}%
\SlideFoot 	% "Activates" the footline, titlepage had an empty footline
\vglue7mm
\leftskip5mm
\item{$\bullet$}Make it possible to prepare presentations in \TeX
\vskip3mm
\item{$\bullet$}Without having to learn lots of extra syntax
\vskip3mm
\item{$\bullet$}By using simple macros that can easily be adapted to one's needs, maybe for each presentation prepared
\vskip3mm
\item{$\bullet$}Without restricting the possibilities that \TeX\ offers
\NewSlide
\SlideTitle{\hfil Requirements}
\font\cdfo=cmtt12
\def\codecolour{1.0 0.0 1.0}
\def\cf{\setcolour\codecolour\cdfo}
\leftskip0pt
You need the program {\cf pdftex}, which probably is included in your \TeX-distribution.
\vskip3mm
You need the file {\cf present.tex}, which is to be {\cf\char'134input} at the beginning of your source code.
The file can for example be obtained from \weblink{http://www.ctan.org}{ctan.org}.
\vskip3mm
You furthermore need a pdf-viewer with fullscreen display capabilities, e.g. {\cf xpdf}.
\vskip5pt
\att{Warning: Some pdf-viewers do not handle links in a document properly.}
%%
%%
%%
\NewSlide
\SlideTitle{\hfil Basic Examples}
You can type text as usual, inline equations $a^2+b^2=c^2$, displayed equations
$$
\exp(z)=\sum_{n=0}^\infty {z^n\over n!}=\lim_{n\to\infty}\Big(1+{z\over n}\Big)^n
$$
and tables
\vskip0.5\baselineskip
\moveright2cm\vbox{\offinterlineskip
\halign{\vrule width 0pt height12pt depth3pt\kern5pt # &\vrule\hfil\kern5pt #\kern5pt\hfil 
&\vrule\hfil\kern5pt #\kern5pt\hfil\cr
& $x$ & $\Delta x$\cr
\noalign{\hrule}
A & 1.03 & 0.07 \cr
\noalign{\hrule}
B & 2.05 & 0.06 \cr
}}\vskip5pt
So you probably can use \TeX\ in the ordinary fashion; just run {\cf pdftex} on your source instead of {\cf tex}.
\NewSlide
You can include images easily:\vskip20pt plus 3pt minus 10pt
\hbox{\image[width3cm]{Stern.png}\hskip5mm plus 2mm minus 2mm A pixel image (png)}
\vskip10pt\hrule
\hbox{A vector image (pdf)\hskip5mm plus 2mm minus 2mm\image[width3cm]{Sagnac.pdf}}
\NewSlide
%%%
%%%
%%%
\SlideTitle{\hfil Ordinary stuff}
You type text, equations, and so on as usual in \TeX. Keep in mind, though, that the effective paper size is 
rather small (12cm wide, 9cm high in standard configuration).
\vskip10pt plus 5pt minus 10pt
The paper size is small, because you rely on the fullscreen mode of your pdf-viewer to blow the slide up to full 
screen size, thus also automatically enlarging the fonts.
\vskip10pt plus 5pt minus 10pt
Your macros should work (unless there is a collision of names), because what you are doing is creating an ordinary
pdf-file with {\cf pdftex}, for viewing it with a pdf-viewer.
\vskip10pt plus 5pt minus 10pt
You start a new slide with {\cf\char'134 NewSlide}. If you use {\cf\char'134 NewFrame}, it has almost the same effect, only
the slide number doesn't get increased.
\NewSlide
%%%
%%%
%%%
\SlideTitle{\hfil Fonts}
{\cf present.tex} defines the following fonts:\vskip5pt
{\openup2pt
\halign{#\hfil\kern5pt&#\hfil\kern5pt&#\hfil\cr
Font Command & Purpose & Default Value\cr
{\cf\char'134 titlefont} & presentation title & {\cf cmssbx10 at 20pt}\cr
{\cf\char'134 slidetitlefont} & title of a slide & {\cf cmssbx10}\cr
{\cf\char'134 normalfont} & ordinary text & {\cf cmss12}\cr
{\cf\char'134 linkbarfont} & text in {\cf\char'134 LinkBar} & {\cf cmss8}\cr
{\cf\char'134 it} & italic text & {\cf cmti12}\cr
{\cf\char'134 rm} & roman text in math & {\cf cmr}\cr
}}
\vskip10pt
Of course you can define further font commands or redefine the existing ones.
The default fonts are used in this presentation, so you see what they look like.
\NewSlide
%%%
%%%
%%%
\SlideTitle{\hfil Colours}
Colours are defined as control sequences, e.g.
\vskip5pt
{\cf\char'134def\char'134mycolour\char'173}{\it R G B\/}{\cf\char'175}
\vskip5pt
where {\it R G B} are the red, green, and blue values of the colour defined, with
$0\leq R, G, B\leq 1$.
\vskip5pt
Colours are used in these ways:
\vskip5pt
{\cf\char'134setcolour\char'134mycolour} 

to set the colour of following text, grouping is respected
\vskip5pt
{\cf\char'134coloured\char'134mycolour\char'173}{\it Stuff\/}{\cf\char'175} 

to set {\it Stuff} in colour {\cf\char'134mycolour}
\vskip5pt
{\cf\char'134setbgcolour\char'134mycolour}

to set the colour of the background
\NewSlide
%%%
%%%
%%%
The following colours are
defined in {\cf present.tex}:
\vskip5pt
{\cf \char'134backgroundcolour}, for the slide background 
\vskip5pt
{\cf \char'134textcolour}, for the text, and
\vskip5pt
{\cf \char'134attentioncolour}, an \att{attention} colour. 

You draw attention to \att{Stuff} by saying
{\cf\char'134att\char'173}{\it Stuff\/}{\cf\char'175}.
\vskip5pt
If a predefined colour is redefined, {\cf\char'134setcolour}
must also be used for the change to take effect for text. 
\NewSlide
%%%
%%%
%%%
\SlideTitle{\hfil Paper Size}
The paper size is determined by the dimensions 
\vskip3pt {\cf\char'134 pdfpagewidth} and
{\cf\char'134 pdfpageheight}. 
\vskip10pt
If you say {\cf\char'134 StandardAspect}, you get a page 120mm wide and 90mm high.
\vskip10pt
If you say {\cf\char'134 WideAspect}, you get a page 144mm wide and 90mm high.
\NewSlide
%%%
%%%
%%%
\SlideTitle{\vbox to 0pt{\kern-5mm\image[width10mm]{Stern.png}\vss}\hfil Images}
Images can be included with
\vskip5pt
{\cf\char'134 image[}{\it dimensions}{\cf]\char'173}{\it filename}{\cf\char'175}
\vskip5pt
where {\it dimensions} are {\cf height}, {\cf depth}, and {\cf width}, familiar from \TeX.
\vskip5pt
\hbox{%
\image[width3mm]{Stern.png}
\image[width6mm]{Stern.png}
\image[width9mm]{Stern.png}
\image[height9mm]{Stern.png}
\image[height3mm depth6mm]{Stern.png}
\image[height9mm width4mm]{Stern.png}
\image[width5mm]{Sagnac.pdf}\image[width10mm height4mm]{Sagnac.pdf}
\image[width15mm depth 5mm]{Sagnac.pdf}
}
\vskip5pt
If only {\cf width} is given, the image is scaled with the aspect ratio preserved.\target{Images}

Image files need to be in a format that can be handled by the pdf-viewer and by {\cf pdftex}.
E.g. pdf, png, jpg should work.
\NewSlide
%%%
%%%
%%%
\SlideTitle{\hfil Slide Structure}
Each slide has a headline, a body, and a footline.
\vskip5pt
The headline holds the slide title, which is set by 
\vskip5pt
{\cf\char'134 SlideTitle\char'173}{\it TITLE\/}{\cf\char'175}
\vskip5pt
The footline is defined by 
{\cf\char'134 SlideFoot}, of which several versions are contained in {\cf present.tex}. Uncomment the one you want,
or define further ones.
\vskip5pt
The version used here shows the number of the current slide and the total number of slides in the centre (the latter is
provided by the {\cf\char'134 LP} macro), and the {\cf\char'134 PageBar} on the right.
\vskip5pt
The {\cf\char'134 PageBar}-symbols \PageBar move to the previous or following page, or move back and forth in the 
page history.
\NewSlide
%%%
%%%
%%%
The page history is relevant if cross-references are used in the presentation. These will be discussed later.
\vskip5pt
\phantom{With cross-references or links, also the {\cf\char'134 LinkBar} can be useful. In the {\cf\char'134 SlideFoot}-version
used, it is shown on the left side of the footline, but currently is defined to be empty.}
\vskip5pt
\phantom{The {\cf\char'134 LinkBar} is intended to hold links to various places in the presentation you might want to jump to.}
\vskip5pt
\phantom{This slide has been shown incrementally by a copy-paste procedure when writing the source (have a look at it). 
It is straightforward, very flexible, one might occasionally lose track, and it may be tedious to introduce
changes later. A further possibility is shown on the next slides.}
\NewFrame
The page history is relevant if cross-references are used in the presentation. These will be discussed later.
\vskip5pt
With cross-references or links, also the {\cf\char'134 LinkBar} can be useful. In the {\cf\char'134 SlideFoot}-version
used, it is shown on the left side of the footline, but currently is defined to be empty.
\vskip5pt
\phantom{The {\cf\char'134 LinkBar} is intended to hold links to various places in the presentation you might want to jump to.}
\vskip5pt
\phantom{This slide has been shown incrementally by a copy-paste procedure when writing the source (have a look at it). 
It is straightforward, very flexible, one might occasionally lose track, and it may be tedious to introduce
changes later. A further possibility is shown on the next slides.}
\NewFrame
The page history is relevant if cross-references are used in the presentation. These will be discussed later.
\vskip5pt
With cross-references or links, also the {\cf\char'134 LinkBar} can be useful. In the {\cf\char'134 SlideFoot}-version
used, it is shown on the left side of the footline, but currently is defined to be empty.
\vskip5pt
The {\cf\char'134 LinkBar} is intended to hold links to various places in the presentation you might want to jump to.
\vskip5pt
\phantom{This slide has been shown incrementally by a copy-paste procedure when writing the source (have a look at it). 
It is straightforward, very flexible, one might occasionally lose track, and it may be tedious to introduce
changes later. A further possibility is shown on the next slides.}
\NewFrame
The page history is relevant if cross-references are used in the presentation. These will be discussed later.
\vskip5pt
With cross-references or links, also the {\cf\char'134 LinkBar} can be useful. In the {\cf\char'134 SlideFoot}-version
used, it is shown on the left side of the footline, but currently is defined to be empty.
\vskip5pt
The {\cf\char'134 LinkBar} is intended to hold links to various places in the presentation you might want to jump to.
\vskip5pt
This slide has been shown incrementally by a copy-paste procedure when writing the source (have a look at it). 
It is straightforward, very flexible, one might occasionally lose track, and it may be tedious to introduce
changes later. A further possibility is shown on the next slides.
\NewSlide
\def\hl{\setcolour\attentioncolour}
%%%
%%%
%%%
\Frames5{%
\SlideTitle{\hfill {\onframe1\hl Slides} {\onframe2\hl Step} {\onframe3\hl by} {\onframe4\hl Step}}
A slide can be shown incrementally using
\vskip1pt
\beforeframe2\hide{\afterframe1\cf
\char'134 NewSlide\par
\char'134 Frames\char'173}{\it num}{\afterframe1\cf\char'175\char'173}%
{\it CONTENT}
{\afterframe1\cf\char'175\char'134 endFrames\par
\char'134NewSlide}
\vskip1pt
\beforeframe3\hide By the above, the {\it CONTENT} of the slide, enclosed between 
{\afterframe2\cf\char'134begingroup} \dots{\afterframe2\cf\char'134endgroup}, 
is repeated {\it num} times, with a 
{\afterframe2\cf\char'134NewFrame} between subsequent repetitions. 
\vskip1pt
\beforeframe4\hide Due to this grouping, definitions are local to the body of a frame.
{\afterframe3\cf\char'134global} is used in the replacement texts of {\afterframe3\cf\char'134SlideTitle} 
and {\afterframe3\cf\char'134SlideFoot} to allow frame-dependent changes here, too.  
\vskip1pt
\beforeframe5\hide  
{\onframe5\cf\char'134hide}, defined as {\onframe5\cf\char'134setcolour\char'134backgroundcolour}, can be used
to hide parts of the slide on a monochromatic background.}
\endFrames
%%%
%%%
%%%
\NewSlide
\Frames5{%
Incremental showing or modification of content is achieved by
\vskip1pt
\beforeframe2\hide{\afterframe1\cf\char'134 onframe\char'173}{\it fn}{\afterframe1\cf\char'175\char'173}{\it arg}%
{\afterframe1\cf\char'175}\par
which evaluates to {\it arg} on frame number {\it fn} and to {\afterframe1\cf\char'134relax} otherwise (frames are counted from 1).
\vskip1pt
\beforeframe3\hide{\afterframe2\cf\char'134 afterframe\char'173}{\it fn}{\afterframe2\cf\char'175\char'173}%
{\it arg}{\afterframe2\cf\char'175}\par
which evaluates to {\it arg} after frame {\it fn} and to {\afterframe2\cf\char'134relax} before and on frame {\it fn}.
\vskip1pt
\beforeframe4\hide{\afterframe3\cf\char'134 beforeframe\char'173}{\it fn}{\afterframe3\cf\char'175\char'173}{\it arg}%
{\afterframe3\cf\char'175}\par
which evaluates to {\it arg} before frame {\it fn} and to {\afterframe3\cf\char'134relax} on and after frame {\it fn}.
\vskip1pt
\beforeframe5\hide If {\it arg} is a single token, no braces are necessary. The same applies if {\it fn} or {\it num}
is a single digit number.}%
\endFrames
\NewSlide
%%%
%%%
%%%
\SlideTitle{\hfil Links}
{\cf\char'134 target\char'173}{\it NAME\/}{\cf\char'175} creates a target named {\it NAME} for a link at the position in 
the presentation where it is used.
\vskip5pt
{\cf\char'134 link\char'173}{\it NAME\/}{\cf\char'175\char'173}{\it Stuff\/}{\cf\char'175}
turns {\it Stuff} into a link to the target named {\it NAME}.
\vskip5pt
{\cf\char'134 weblink\char'173}{\it URI\/}{\cf\char'175\char'173}{\it Stuff\/}{\cf\char'175}
makes {\it Stuff} a link to the specified {\it URI}.
\vskip5pt
{\cf\char'134 filelink\char'173}{\it file}{\cf\char'175\char'173}{\it filedest}{\cf\char'175\char'173}{\it Stuff\/}{\cf\char'175}
makes {\it Stuff} a link to destination {\it filedest} in {\it file}.

{\it filedest} for example can be {\cf [}{\it page} {\cf/Fit]}, with {\it page} the page number (starting at 0).
\vskip10pt
A further possibility can be found \link{Further}{here (click)}.
\def\LinkBar{%
\link{Images}{\linkbarfont Images}
\link{Further}{\linkbarfont Area}
\link{title}{\linkbarfont Title}
}
\NewSlide
%%%
%%%
%%%
{\cf\char'134 linkarea\char'173}{\it dest}{\cf\char'175\char'173}{\it rect}{\cf\char'175\char'173}{\it border}{\cf
\char'175\char'173}{\it colour}{\cf\char'175}\target{Further}
\vskip5pt
Creates a rectangular area which is a link to target {\it dest}; {\it rect} consists of 
four space-separated numbers for lower left and upper right corner, {\it border} is the border width, and {\it colour} is the 
border colour, specified as three space-separated values for red, green, and blue, all between 0 and 1.
\vskip5pt
\linkarea{title}{150 30 180 60}{2}{0.8 0 0}
\image[width 2cm]{Stern.png}
\vskip5pt
Notice also that {\cf\char'134 LinkBar} has been redefined (on the previous slide already)
\linkarea{Images}{10 100 50 120}{2}{0.8 0 1}
\NewSlide
%%%
%%%
%%%
\SlideTitle{\hfill Transition Effects}
\UseTrans
\SetTrans{Fade}
\SetDuration{2}
\def\satt{\setcolour\attentioncolour}
\def\sumcol{0 0 0}
\def\sscol{\setcolour\sumcol}
\Frames{6}{%
Transition effects for a presentation 
can be provided by the macros in {\cf present-transitions.tex}
to be {\cf\char'134input} after {\cf present.tex}. 

\beforeframe2\hide
Transitions are activated with {\afterframe1\cf\char'134UseTrans} and deactivated with
{\afterframe1\cf\char'134NoTrans}.

\beforeframe3\hide
The type of a transition effect is selected with {\afterframe2\cf\char'134SetTrans\char'173}{\it TYPE\/}{\afterframe2\cf\char'175},
the duration of a transition effect 
with {\afterframe2\cf\char'134SetDuration\char'173}{\it num}{\afterframe2\cf\char'175}.

\beforeframe4\hide
Possible values for {\it TYPE} are {\afterframe3\cf Split},
{\afterframe3\cf Blinds},
{\afterframe3\cf Box},
{\afterframe3\cf Wipe},
{\afterframe3\cf Dissolve},
{\afterframe3\cf Glitter},
{\afterframe3\cf R},
{\afterframe3\cf Fly},
{\afterframe3\cf Push},
{\afterframe3\cf Cover},
{\afterframe3\cf Uncover},
{\afterframe3\cf Fade}.
See the PDF Reference for details.
\beforeframe5\hide
{\it num} specifies the duration of the transition effect in seconds.

\beforeframe6\hide{\afterframe5\satt Only 
some transition effects may be supported by your pdf-viewer}.}%
\endFrames
\NewSlide
\SetTrans{Push}
\Frames{15}{%
Compare
$$
\Gamma^{{\onframe3\satt a}}_{{\onframe4\satt b}{\onframe5\satt c}}=
{\textstyle{1\over2}}
g^{{\onframe3\satt a}{\afterframe1\sscol u}}
\Big(\partial_{\onframe4\satt b}
g_{{\afterframe1\sscol u}{\onframe5\satt c}}
+\partial_{\onframe5\satt c}g_{{\onframe4\satt b}{\afterframe1\sscol u}}
-\partial_{\afterframe1\sscol u}g_{{\onframe4\satt b}{\onframe5\satt c}}\Big)
$$
\beforeframe6\hide
\onframe{5}\NoTrans
with
$$
\Gamma^{{\onframe{8}\satt a}}_{{\onframe{9}\satt b}{\onframe{10}\satt c}}=
{\textstyle{1\over2}}
g^{{\onframe{8}\satt a}{\afterframe6\sscol u}}
\Big(\partial_{\onframe{9}\satt b}
g_{{\afterframe6\sscol u}{\onframe{10}\satt c}}
+\partial_{\onframe{10}\satt c}g_{{\onframe{9}\satt b}{\afterframe6\sscol u}}
-\partial_{\afterframe6\sscol u}g_{{\onframe{9}\satt b}{\onframe{10}\satt c}}\Big)
$$
\onframe{11}\UseTrans
\beforeframe{11}\hide\onframe{11}{\SetTrans{Dissolve}}%
Are transition effects an advantage didactically?
\vskip5pt
\beforeframe{12}\hide\onframe{12}{\SetTrans{Fade}}
Sometimes {\afterframe{12}\satt\onframe{12}\hide transition} effects may be {\afterframe{12}\satt\onframe{12}\hide helpful}.
\vskip3pt
Sometimes they {\beforeframe{14}\hide\afterframe{13}\satt distract}
the {\beforeframe{14}\hide\afterframe{13}\satt audience} 
from the {\beforeframe{14}\hide\afterframe{13}\satt content} of the slide.
\vskip3pt
\onframe{14}{\SetTrans{Dissolve}\SetDuration{5}}
\beforeframe{15}\hide Use links or {\onframe{15}\cf \char'134PageBar}-symbols for navigation in order to
see transition effects.
}\endFrames
\NewSlide
\NoTrans
%%%
%%%
%%%
\SlideTitle{\hfil Background Images}
\SetBackground{background.png}
You can set images for the background with 
\vskip5pt
{\cf\char'134 SetBackground\char'173}{\it filename}{\cf\char'175}
\vskip5pt
where
{\it filename} is the name of the image file.
\vskip5pt
The background image is unset (i.e. the background colour  will show again) with {\cf\char'134 UnsetBackground}.
\vskip5pt
The background image is scaled to width and height of the slide. It should have an appropriate aspect ratio.
\vskip5pt
\att{As you see some combinations of text colour and background image can cause visibility problems.}
\NewSlide
\UnsetBackground
%%%
%%%
%%%
\SlideTitle{\hfil Concluding Remarks}
PDF provides the possibility to include movies, and to launch applications (which could be a movie player) via clickable
elements. The usability for a presentation depends on how to manage the player together with a fullscreen display of the
slides. Movie support is not officially included in {\cf present.tex}.
\vskip5pt
If you prepare a pdf-image for a presentation which is to be shown on some different computer, it would be best to
embed fonts used in the image into the image file. For example, if you have a file {\cf image.eps}, you need to
convert it to pdf, in order to use it with {\cf pdftex}. You can embed the fonts by
\vskip5pt
{\cf  ps2pdf -dEPSCrop=true -dPDFA image.eps} 
\vskip5pt
A look at {\cf present.tex} is recommended.
\bye
